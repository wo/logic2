\documentclass{wobook2018}

% fonts:
% https://r2src.github.io/top10fonts/
% https://ctan.org/pkg/mlmodern?lang=en
  
\newif\ifcompilesolutions
\compilesolutionsfalse
\compilesolutionstrue

\definecolor{commentcol}{rgb}{0.0, 0.2, 0.1}
%\newcommand{\cmnt}[1]{{\small\color{commentcol}#1}}
\newcommand{\cmnt}[1]{{\iffalse #1 \fi}}

%\DeclareFontFamily{OT1}{pzc}{}
%\DeclareFontShape{OT1}{pzc}{m}{it}{<-> s * [1.15] pzcmi7t}{}
%\DeclareMathAlphabet{\mathpzc}{OT1}{pzc}{m}{it}
%\newcommand{\Fr}[1]{\ensuremath{\mathpzc{#1}}}
%\newcommand{\Sc}[1]{\ensuremath{\mathpzc{#1}}}
\let\polishL\L
\renewcommand{\L}{\ensuremath{\mathfrak{L}}\xspace}
\newcommand{\Cfr}{\ensuremath{\mathfrak{C}}\xspace}
\newcommand{\Mfr}{\ensuremath{M}\xspace}

% make box and diamond larger, and vdash (too small in newtxmath):
\usepackage{scalerel}
\let\oldbox\Box
\renewcommand\Box{\scaleobj{1.2}{\oldbox}}
\let\olddiamond\Diamond
\renewcommand\Diamond{\scaleobj{1.2}{\olddiamond}}
\let\oldvdash\vdash
\renewcommand\vdash{\scaleobj{1.3}{\oldvdash}}

% define inverted \models:
\makeatletter
\providecommand*{\invmodels}{%
  \mathrel{%
    \mathpalette\@invmodels\models
  }%
}
\newcommand*{\@invmodels}[2]{%
  \reflectbox{$\m@th#1#2$}%
}
\makeatother


% make \to smaller and with less spacing to the sides:
\let\oldto\to
\renewcommand\to{\scaleobj{0.95}{\,\oldto\,}}
\let\oldleftrightarrow\leftrightarrow
\renewcommand\leftrightarrow{\scaleobj{0.95}{\,\oldleftrightarrow\,}}

% define strictif and boxright:
\DeclareSymbolFont{symbolsC}{U}{txsyc}{m}{n}
\DeclareMathSymbol{\strictif}{\mathrel}{symbolsC}{74}
\DeclareMathSymbol{\boxright}{\mathrel}{symbolsC}{128}

% define boxes and diamonds for specialist logics:
\DeclareMathOperator{\Kn}{\mathsf{K}{}}
\DeclareMathOperator{\Ind}{\mathsf{Ind}{}}
\DeclareMathOperator{\Mi}{\mathsf{M}}
\DeclareMathOperator{\Po}{\mathsf{P}}
\DeclareMathOperator{\Bel}{\mathsf{B}}
\DeclareMathOperator{\EKn}{\mathsf{E}}
\DeclareMathOperator{\CKn}{\mathsf{C}}
\DeclareMathOperator{\tP}{\mathsf{P}}
\DeclareMathOperator{\tF}{\mathsf{F}}
\DeclareMathOperator{\tG}{\mathsf{G}}
\DeclareMathOperator{\tH}{\mathsf{H}}
\DeclareMathOperator{\tX}{\mathsf{X}}
\DeclareMathOperator{\tY}{\mathsf{Y}}
\DeclareMathOperator{\tS}{\mathsf{S}}
\DeclareMathOperator{\tU}{\mathsf{U}}
\DeclareMathOperator{\tN}{\mathsf{N}}
\DeclareMathOperator{\Ob}{\mathsf{O}}
\DeclareMathOperator{\Pe}{\mathsf{P}}
\DeclareMathOperator{\tA}{\mathsf{A}}
\DeclareMathOperator{\Always}{\mathsf{A}}
\DeclareMathOperator{\Sometimes}{\mathsf{S}}
\DeclareMathOperator{\Mostly}{\mathsf{M}}
% better smallcaps:
\newcommand{\smallcaps}[1]{{\text{\footnotesize \MakeUppercase{#1}}}}
\newcommand{\stit}{\mathop{\textsf{\smallcaps{stit}}}}
\newcommand{\OK}{\mathop{\textsf{\smallcaps{N}}}}

\newcommand{\Ax}[1]{\smallcaps{\text{A#1}}}
\newcommand{\prov}[1]{\vdash_{\!\scriptscriptstyle{#1}}}
\newcommand{\notprov}[1]{\scaleobj{1.3}{\not\vdash}_{\!\scriptscriptstyle{#1}}}
\newcommand{\CM}[1]{M_{\!\scriptscriptstyle{\Ax{#1}}}}


% shortcut for displaying named axiom schemas:
\newcommand{\principle}[2]{\begin{equation}\tag{#1}#2\end{equation}}
\newenvironment{principles}{\gather}{\endgather}
\newcommand{\pri}[2]{#2 \tag{#1}}

\newcommand{\pr}[1]{(#1)}

% custom list format in exercises:
\usepackage[shortlabels]{enumitem} % \begin{enumerate}[(a)]
\newenvironment{exlist}{\vspace{-2mm}\begin{enumerate}[(a)]\itemsep-1.5mm}{\end{enumerate}\vspace{-3mm}}
\newenvironment{sollist}{\vspace{-5mm}\begin{enumerate}[(a)]\itemsep-1.5mm}{\end{enumerate}\vspace{-3mm}}
\setlist[itemize]{leftmargin=*}
\setlist[enumerate]{leftmargin=*}

% BOXES:

\usepackage[most]{tcolorbox}

%\definecolor{teal}{HTML}{ba5536}
%\definecolor{terra}{HTML}{363237}
\definecolor{teal}{HTML}{34675c}
\definecolor{terra}{HTML}{324851}

% plain box:

\newtcolorbox{justabox}{
  enhanced,
  colback=terra!2!white,
  colframe=terra!80!white,
  boxrule=0.3mm,
  drop small lifted shadow,
  beforeafter skip=\baselineskip,
}

% Definition environment with counter:
\newtcbtheorem[number within=chapter]{definition}{Definition}{%
  enhanced,
  breakable,
  enhanced jigsaw,
  colback=terra!2!white,
  colframe=terra!80!white,
  colbacktitle=terra!2!white,
  boxrule=0.3mm,
  drop small lifted shadow,
  titlerule=-0.5mm,
  toptitle=2mm,
  beforeafter skip=\baselineskip,
  fonttitle=\upshape\bfseries\sffamily,
  coltitle=terra!90!black,
  fontupper=\normalfont,
  }{def}

% Observation environment without title bar:
\newtcbtheorem[number within=chapter]{obs}{Observation}{%
  theorem style=plain,
  enhanced,
  colback=terra!2!white,
  colframe=terra!80!white,
  boxrule=0.3mm,
  beforeafter skip=\baselineskip,
  fonttitle=\upshape\bfseries\sffamily,
  coltitle=terra!90!black,
  fontupper=\normalfont,
  drop small lifted shadow,
  }{obs}
\newenvironment{observation}[1]{\begin{obs}{}{#1}}{\end{obs}}

% Theorem environment without 'Theorem' label:
\newenvironment{theorem}[2]{%
  \begin{tcolorbox}[%
    enhanced,
    title={#1},
    colframe=terra!80,
    colback=terra!2!white,
    colframe=terra!80!white,
    colbacktitle=terra!2!white,
    coltitle=terra!90!black,
    boxrule=0.3mm,
    titlerule=-0.5mm,
    toptitle=2mm,
    fonttitle=\upshape\bfseries\sffamily,
    drop small lifted shadow,
    beforeafter skip=\baselineskip,
    ]\label{thm:#2}%
  }%
  {\end{tcolorbox}}

    
\renewtcolorbox{proof}{
  blanker,enforce breakable,left=3mm,
  before skip=10pt,after skip=10pt,
  parbox=false, % for paragraph indentation
  borderline west={0.2mm}{0pt}{terra!60!white}
}
%  \setlength{\parskip}{\normalparskip}%
%  \setlength{\parindent}{\normalparindent}}

% % Exercises:
% \NewTColorBox[auto counter,number within=chapter]{exercise}{+!O{}}{%
%   enhanced,
%   breakable,
%   enhanced jigsaw,
%   fonttitle=\sffamily\upshape\bfseries,
%   coltitle=teal!10!white,
%   colback=teal!02!white,
%   colbacktitle=teal,
%   colframe=teal!70!black,
%   boxrule=0.3mm,
%   attach boxed title to top left={yshift=-1mm, xshift=1mm},
%   beforeafter skip=.9\baselineskip,
%   title=Exercise~\thetcbcounter,
%   lowerbox=ignored,
%   savelowerto=solutions/exercise-\thetcbcounter.tex,
%   record={\string\solution{\thetcbcounter}{solutions/exercise-\thetcbcounter.tex}},
%   #1
% }
% \tcbset{no solution/.style={no recording,after upper=}}


\ifcompilesolutions

\usepackage[clear-aux]{xsim}
\xsimsetup{
  exercise/template=tcb,
  exercise/within=chapter,
  exercise/the-counter = \thechapter.\arabic{exercise},
  solution/template=tcbsol,
  solution/name=Exercise,
  print-solutions/headings=false,
}

\DeclareExerciseEnvironmentTemplate{tcb} {%
  \tcolorbox[
  breakable,
  enhanced jigsaw,
  fonttitle=\sffamily\upshape\bfseries,
  coltitle=teal!10!white,
  colback=teal!02!white,
  colbacktitle=teal,
  colframe=teal!70!black,
  boxrule=0.3mm,
  attach boxed title to top left={yshift=-1mm, xshift=1mm},
  beforeafter skip=.9\baselineskip,
  title = Exercise~\GetExerciseProperty{counter}%
        \GetExercisePropertyT{subtitle}{ \textit{\PropertyValue}}%
        \IfInsideSolutionF{%
            \GetExercisePropertyT{points}{ % notice the space
                (%
                \PropertyValue
                \IfExerciseGoalSingularTF{points}
                    {\XSIMtranslate{point}}
                    {\XSIMtranslate{points}}%
                )%
            }%
        }%
    ]%
    % \IfInsideSolutionT{%
    %   \tcbsubtitle{Exercise}
    %   \input{\jobname-\ExerciseType-\ExerciseID-exercise-body.tex}
    %   \tcbsubtitle{Solution}
    % }
}{\endtcolorbox}

% Solution headings:
\DeclareExerciseEnvironmentTemplate{tcbsol}{%
  \tcbox[
  fontupper=\sffamily\upshape\bfseries,
  coltext=teal!10!white,
  colback=teal!80!white,
  colframe=teal!50!white,
  boxrule=0.3mm,
  boxsep=-0.5mm,
  before skip=\baselineskip,
  after skip=0.5\baselineskip,
  ]{Exercise~\GetExerciseProperty{counter}}
  \noindent
}

\else

\newtcolorbox[auto counter,number within=chapter]{exercise}{
  breakable,
  enhanced jigsaw,
  fonttitle=\sffamily\upshape\bfseries,
  coltitle=teal!10!white,
  colback=teal!02!white,
  colbacktitle=teal,
  colframe=teal!70!black,
  boxrule=0.3mm,
  attach boxed title to top left={yshift=-1mm, xshift=1mm},
  beforeafter skip=.9\baselineskip,
  title=Exercise~\thetcbcounter,
}
\usepackage{comment}
\excludecomment{solution}
\newcommand{\printsolutions}{}

\fi

% \DeclareExerciseType{exercise}{
%     exercise-env=ltxexercise,
%     solution-env=ltxsolution,
%     exercise-name=\XSIMtranslate{exercise},
%     solution-name=\XSIMtranslate{solution},
%     exercise-template=tcb,
%     solution-template=tcb,
%     within=chapter,
% }

%\newcommand{\beginwithlist}{\leavevmode\vspace{-7mm}}
\newcommand{\beginwithlist}{\vspace{1mm}}

\usepackage{wrapfig}

\let\description=\eqlist
\let\enddescription=\endeqlist
\let\eqlistlabel\descriptionlabel
\newenvironment{semantics}{\begin{description}}{\end{description}}

% graphs for models, from http://www.actual.world/resources/tex/doc/TikZ.pdf
\usetikzlibrary{positioning,arrows,calc}
\tikzset{
modal/.style={>=stealth',shorten >=1pt,shorten <=1pt,auto,node distance=1.5cm,
semithick},
world/.style={circle,draw,minimum size=0.5cm,fill=gray!15,font=\small},
point/.style={circle,draw,inner sep=0.5mm,fill=black},
reflexive above/.style={->,loop,looseness=7,in=120,out=60},
reflexive below/.style={->,loop,looseness=7,in=240,out=300},
reflexive left/.style={->,loop,looseness=7,in=150,out=210},
reflexive right/.style={->,loop,looseness=7,in=30,out=330}
}
\definecolor{rgreen}{RGB}{0,150,0}
\definecolor{rblue}{RGB}{0,0,150}
\definecolor{rred}{RGB}{180,0,0}


% Trees:
\usepackage{xyling}
\usepackage{xfp}
\usepackage{ifthen}
\definecolor{annotcol}{rgb}{0.3, 0.4, 0.5}
\definecolor{red}{rgb}{1,0,0}
\def\checkmark{\color{annotcol}\tikz\fill[scale=0.4](0,.35) -- (.25,0) -- (1,.7) -- (.25,.15) -- cycle;}
\newcommand{\xmark}{\ding{55}}%
\newcommand{\tree}[2][0]{\Treek[#1]{0.7}{#2}}
%\newcommand{\dotline}{\GB{d}{.}}
\newcommand{\wlabel}[2]{\Kk[#1]{0}{\ifthenelse { \equal {#2} {} } {} {\ensuremath{(#2)}}}}
\newcommand{\ticklabel}[1]{\Kk[#1]{0}{\checkmark}}
\newcommand{\annotlabel}[2]{\Kk[#1]{0}{{\color{annotcol}#2}}}
\newcommand{\llabel}[2]{\Kk[-#1]{0}{{\color{annotcol}#2}}}
\newcommand{\branchbelow}{\B{ddl}\B{ddr}}
\newcommand{\tribranchbelow}{\B{ddl}\B{dd}\B{ddr}}
\newcommand{\closed}{\Below{\sffamily x}}
\newcommand{\treenode}[5]{\llabel{\fpeval{(#1)-1}}{#2} #3 \wlabel{#1}{#4} \annotlabel{\fpeval{(#1)+10}}{#5}}
\newcommand{\treenodeticked}[5]{\llabel{\fpeval{(#1)-1}}{#2} #3 \wlabel{#1}{#4} \annotlabel{\fpeval{(#1)+10}}{#5}  \ticklabel{\fpeval{(#1)+18}}  }
\newcommand{\nnode}[5]{\treenode{#1}{#2}{\K{#3}}{#4}{#5}}
\newcommand{\nnodeticked}[5]{\treenodeticked{#1}{#2}{\K{#3}}{#4}{#5}}
\newcommand{\nnodeclosed}[5]{\treenode{#1}{#2}{\K{#3}\closed}{#4}{#5}}
\newcommand{\dotbelowbnode}[5]{\treenode{#1}{#2}{\K{#3}\GB{dd}{.}\B[-8]{dddl}\B[-8]{dddr}}{#4}{#5}}
\newcommand{\dotbelownode}[5]{\treenode{#1}{#2}{\K{#3}\GB{dd}{.}}{#4}{#5}}
\newcommand{\bnode}[5]{\treenode{#1}{#2}{\K{#3}\branchbelow}{#4}{#5}}
\newcommand{\bnodeticked}[5]{\treenodeticked{#1}{#2}{\K{#3}\branchbelow}{#4}{#5}}
\newcommand{\tribnode}[5]{\treenode{#1}{#2}{\K{#3}\tribranchbelow}{#4}{#5}}
\newcommand{\barenode}[1]{\K{#1}}
\newcommand{\dotbelowbarenode}[1]{\K{#1}\GB{dd}{.}}
\newcommand{\dotbelowbarebnode}[1]{\K{#1}\GB{dd}{.}\B[-8]{dddl}\B[-8]{dddr}}
\newcommand{\dotbelowbaretribnode}[1]{\K{#1}\GB{dd}{.}\B[-8]{dddl}\B[-8]{ddd}\B[-8]{dddr}}

% Syntax:
%
% \tree[1]{ ... } => optional 1 adds width to branching
%
% Trees are defined like tabular tables:
%
% \tree{ & node1 & \\ node2 & & node3 }
%
% A tree node is defined by
%
% \nnode{10}{1.}{$A \land B$}{w}{Ass.}
%
% The first argument (10) controls the distance between the formula
% and the line number/world/annotation.
%
% Nodes under which the tree branches are defined by \bnode instead of
% \nnode. A bnode must be followed by an empty row:
%
%\tree[2]{
%    & \bnode{13}{1.}{$A \lor B$}{w}{Ass.} & \\
%    && \\
%    \nnode{8}{2.}{$A$}{w}{(1)} && \nnode{8}{3.}{$B$}{w}{(1)}
%}

\usepackage{tabularx}


\usepackage[type={CC}, modifier={by-nc-sa}, version={4.0}, imageposition=left, imagewidth=20mm, imagedistance=3mm]{doclicense}

\begin{document}

\tcbstartrecording

\begingroup
\thispagestyle{empty}

\begin{center}

%\renewcommand{\bfdefault}{sb}%
%\fontfamily{put}\bfseries

  % {\Large A Philosophical Introduction to}
  
  % \vspace{2mm}

  {\huge Logic 2: Modal Logic\par}

  \vspace{5mm}
  
  % {\Large A Philosophical Introduction}
  

  \vspace{10mm}

  {\normalsize Wolfgang Schwarz}

  \vspace{2mm}
  {\normalsize \today}

%\renewcommand{\bfdefault}{bx}%
\end{center}

\vfill
\endgroup
{

  \footnotesize

 \noindent \copyright\ \the\year\ Wolfgang Schwarz

 \smallskip
 \noindent \href{https://github.com/wo/logic2}{github.com/wo/logic2}

 \vspace{-1.5mm}
 \doclicenseThis

}

  

{
\newpage
\small
\tableofcontents 
}

\normalsize

%\cleardoublepage % Forces the first chapter to start on an odd page so it's on the right


%%% Local Variables: 
%%% mode: latex
%%% TeX-master: "logic2.tex"
%%% End:

\chapter*{Preface}\label{ch:preface}
\addcontentsline{toc}{chapter}{Preface}

These notes are aimed at philosophy students who have taken an introductory
course in formal logic. They provide an introduction to modal logic, with plenty
of philosophical applications. I also use the opportunity to introduce general
ideas that might be taught in an intermediate logic course: the concept of a
model, soundness and completeness, compactness, three-valued logics, free
logics, supervaluation, properties of relations and orders, etc.

Chapters 1--3 introduce the standard toolkit of modal propositional logic:
Kripke models, frame correspondence, some popular systems, the tableau method
and axiomatic calculi. Chapter 4 goes through soundness and completeness.
Chapters 5--8 turn to philosophical applications. Each of these chapters also
extends the toolkit from chapter 3. Chapter 5 introduces multi-modal logics,
chapter 6 ordering models and neighbourhood semantics, chapter 6 two-dimensional
semantics and supervaluationism, chapter 7 conditional logics and
Lewis-Stalnaker models. Chapters 9 and 10 then introduce some of the
complexities that arise in first-order modal logic.

Apart from chapter 9, which sets the stage for chapter 10, every chapter after
chapter 3 can be skipped or skimmed without affecting the accessibility of later
chapters.

The best way to learn logic is by solving problems. That's why the text is
frequently interrupted by exercises. As a student, you should try to do the
exercises as soon as you reach them, before continuing with the text.

\bigskip
--- Wolfgang Schwarz, \today.


% And remember: It is more effective to learn a little every day than to cram 

% ``If you rely on passively rereading your course materials, you’ll tend to end up using your memory to try to reproduce the author’s understanding of the subject rather than generating your own. So, what is the best catalyst for generating your own understanding of what you read? The answer is to question what you read as you’re reading it.''

% Don't cram. space out.

% - focus is on understanding, not on rote learning of rules


%%% Local Variables: 
%%% mode: latex
%%% TeX-master: "ml.tex"
%%% End:

\chapter{Modal Operators}\label{ch:operators}

\section{A new language}
\label{sec:intro}

% Modal Logic

Modal logic is an extension of propositional and predicate logic that is widely
used to reason about possibility and necessity, obligation and permission, the
flow of time, the processing of computer programs, and a range of other topics.
Each of these applications begins by adding new symbols to the formal language
of classical propositional or predicate logic. Before we explore such additions,
let's briefly review why we use formal languages in the first place.

% Validity

When reasoning about a given topic, we sometimes want to make sure that the
stated conclusions really follow from the stated premises. If they do, we say
that the reasoning is \emph{valid}. By this we mean that there is no conceivable
scenario in which the premises are true while the conclusions are false.

% Logical validity

Here is an example of a valid argument.
%
\begin{quote}
  All myriapods are oviparous.\\
  Some arthropods are myriapods.\\
  Therefore: Some arthropods are oviparous.
\end{quote}
%
You can tell that this argument is valid even if you don't understand the
zoological terms, because every argument of the same \emph{logical form} is
valid. The relevant logical form might be expressed as follows.
%
\begin{quote}
  All $F$ are $G$.\\
  Some $H$ are $F$.\\
  Therefore: Some $H$ are $G$.
\end{quote}
%
No matter what descriptive terms you plug in for $F$, $G$, and $H$, you get a
valid argument. The argument about myriapods is therefore not just valid, but
\emph{logically valid} -- valid in virtue of its logical form.

% The case for formal languages

In natural languages like English, the logical form of sentences is not always
transparent. `Every dog barked at a tree' can mean either that there is a single
tree at which every dog barked, or that for each dog there is a tree at which it
barked. The two readings have different logical consequences, so it would be
good to keep them apart. Worse, the meaning of logical expressions (`all',
`some', `and', etc.) in natural language is often unclear and complicated. `Paul
and Paula got married and had children' suggests that the marriage came before
the children. In `Paul went to the zoo and Paula stayed at home', the word `and'
does not seem to have this temporal meaning.

To get around these problems, we invent formal languages in which there are no
ambiguities of logical form and in which all logical expressions have
determinate, precise meanings. If we want to evaluate natural-language arguments
for logical validity, we first have to translate them into the formal language.
(Sometimes an argument will be valid on one translation and invalid on another.)
With some practice, one can also reason directly in a formal language.

% A modal argument

Now consider the following argument.
%
\begin{quote}
  It might be raining.\\
  It is certain that we will  get wet if it is raining.\\
  Therefore: We might get wet.
\end{quote}
%
The argument looks valid. Indeed, any argument of this form is plausibly valid:
\begin{quote}
  It might be that $A$.\\
  It is certain that $B$ if $A$.\\
  Therefore: It might be that $B$.
\end{quote}
But it's hard to bring out the validity of these arguments in classical
propositional or predicate logic. We need formal expressions corresponding to
`it might be that' and `it is certain that'. The languages of classical
logic do not have such expressions.

% The standard languages of modal logic

So let's add them. Let's invent a new formal language with two new logical
symbols. It doesn't matter what these look like; a popular choice is a
diamond $\Diamond$ and a box $\Box$. We use the diamond to formalize `it might be
that', and the box for `it is certain that'.

If we add these symbols to the language of propositional logic, we get the
standard language of modal propositional logic. If we add them to the language
of predicate logic, we get the standard language of modal predicate logic. We
will stick with propositional logics until chapter \ref{ch:qml}.

% Translating the modal argument

% We can translate the above argument into the language of modal propositional
% logic, using $r$ to mean that it is raining and $w$ that we will get wet.
% \begin{quote}
%   $\Diamond r$\\
%   $\Box(r \to w)$\\[-3mm]
%   \rule{2cm}{0.2mm}\\
%   $\Diamond w$
% \end{quote}

% A formal definition

Let's officially define the standard language of modal propositional logic.

\begin{definition}{The language $\L_{M}$}{LM}
  A \emph{sentence letter} of $\L_{M}$ is any lower-case letter of the Latin
  alphabet ($a,b,c,\ldots,z$), possibly followed by numerical subscripts
  ($a_{1}, p_{18}, \ldots$). 

  A \emph{sentence} of $\L_{M}$ is either a sentence letter of $\L_{M}$ or an
  expression of the form $\neg A$, $(A \land B)$, $(A \lor B)$, $(A \to B)$,
  $(A \leftrightarrow B)$, $\Box A$, or $\Diamond A$, where $A$ and $B$ are
  $\L_{M}$-sentences.
\end{definition}

% Some conventions

I use lower-case letters $a,b,c,\ldots$ as atomic $\L_{M}$-sentences and
upper-case letters $A,B,C,\ldots$ when I want to talk about arbitrary
$\L_{M}$-sentences. To reduce clutter, I generally omit outermost parentheses
and quotation marks when I mention $\L_{M}$-symbols or sentences: $p \land q$ is
treated as an abbreviation of `$(p \land q)$'.

\begin{exercise}
  Which of these are $\L_M$-sentences?
  \begin{exlist}
  \item $p$
  \item $\Diamond$
  \item $\Diamond p \lor (\Box p \to p)$
  \item $\Box \Box p$
  \item $\Box A \to A$
  \item $(\Diamond r \land \Diamond qr) \land \Diamond \Box\Diamond\Box p$
  \end{exlist}
\end{exercise}
\begin{solution}
  (a), (c), and (d) are $\L_{M}$-sentences, (b), (e), and (f) are not.
\end{solution}

% Outstanding tasks

Having new symbols is only the beginning. We also need to lay down 
rules for reasoning with these symbols. The rules should be motivated by
what the symbols are supposed to mean. So we shall also assign a more precise
meaning to the diamond and the box -- just as classical logic assigns a precise
meaning to the symbol $\land$ that may or may not exactly match the meaning of
`and' in English.

% Truth tables

The meaning of $\land$ can be given by a \emph{truth table}:
\begin{center}
  \begin{tabular}{cc|ccccc}
    A & B & $A \land B$ \\\hline
    T & T & T\\
    T & F & F\\
    F & T & F\\
    F & F & F
  \end{tabular}
\end{center}
This tells us how the truth-value of $A \land B$ depends on the truth-value of
$A$ and $B$: the compound sentence is true iff (if and only if) both of its
subsentences are true. If you know this, you know all there is to know about the
meaning of $\land$. (You can see, for example, that $A \land B$ does not imply
anything about the temporal order of $A$ and $B$.)

\begin{exercise}
  Draw the truth tables for $\neg, \lor, \to$, and $\leftrightarrow$.
\end{exercise}
\begin{solution}
  Here is a combined truth table for all the classical connectives: 
  \begin{center}
    \begin{tabular}{cc|ccccc}
      A & B & $\neg A$ & $A \land B$ & $A\lor B$ & $A\to B$ & $A\leftrightarrow B$\\\hline
      T & T & F & T & T & T & T\\
      T & F & F & F & T & F & F\\
      F & T & T & F & T & T & F\\
      F & F & T & F & F & T & T\\
    \end{tabular}
  \end{center}
\end{solution}

% Truth-functionality

The sentence operators (or connectives) of classical propositional logic
($\neg, \land, \lor, \to$, and $\leftrightarrow$) are all truth-functional. Recall
that an operator is \textbf{truth-functional} if the truth-value of a compound
sentence formed by applying the operator to other sentences is always determined
by the truth-value of these other sentences. The truth tables for the classical
operators spell out this dependence. They tell us how to compute the truth-value
of a compound sentence from the truth-values of its constituents.

% 'Might' is not truth-functional

The diamond operator can't be truth-functional if it is supposed to mean
anything like `it might be that' in English. To see why, note first that `it
might be that $P$' can be true if $P$ is true, but also if $P$ is false. `It
might be raining' doesn't entail that it is actually raining, nor that it isn't
raining. It merely says that our evidence is compatible with rain. Now, if the
diamond were truth-functional, then what would follow from the fact that
$\Diamond p$ is \emph{sometimes} true when $p$ is true? It would follow that
$\Diamond p$ is \emph{always} true when $p$ is true. (Make sure you understand
why.) Likewise, from the fact that $\Diamond p$ is sometimes true when $p$ is
false, it would follow that $\Diamond p$ is true whenever $p$ is false.
$\Diamond p$ would be a logical truth. But `it might be raining' is surely not a
logical truth.

% \begin{exercise}\label{ex:might-truth-func}
%   Explain why the $\Diamond$ operator that formalises `it might be that' is not truth-functional.
% \end{exercise}
% \begin{solution}
%   In most situations, there are false propositions $P$ for which `it might be
%   that $P$' is true (because we are not omniscient) and other false propositions
%   $Q$ for which `it might be that $Q$' is false (because there are at least some truths we actually know).
% \end{solution}

% Possible worlds

If an operator isn't truth-functional, its meaning can't be defined by a truth
table. The standard approach to defining the meaning of modal operators instead
involves the concept of possible worlds. Roughly, we'll interpret $\Diamond A$
as saying that $A$ is true at some possible world, and $\Box A$ as saying that
$A$ is true at all possible worlds. Much more on this later.

\begin{exercise}\label{ex:truth-func}
  Which of these English expressions are truth-functional?
  \begin{exlist}
  \item It used to be the case that \ldots
  \item It is widely known that \ldots
  \item It is false that \ldots
  \item It is necessary that \ldots
  \item I can see that \ldots
  \item God believes that \ldots
  \item Either 2+2=4 or it is practically feasible that \ldots
  \end{exlist}
\end{exercise}
\begin{solution}
  An operator $O$ is truth-functional if you can figure out the
  truth-value of $Op$ from the truth-value of $p$.

  (c) and (g) are truth-functional; (a), (b), (d), and (e) are not
  truth-functional.

  (f) is truth-functional if God is omniscient (and infallible); it is also
  truth-functional if God doesn't exist, or if God believes all and only false
  things; otherwise (f) is not truth-functional.
\end{solution}

\section{Flavours of modality}
\label{sec:flavours}

% Historically, modal logic grew out of the study of necessity and contingency,
% which medieval logicians regarded as ``modes of truth''. Hence the name `modal
% logic'. Today, the study of necessity and contingency is but one of many
% subfields within modal logic. To a first (and rough) approximation, any part of
% logic that involves \emph{non-truth-functional sentence operators} is part of
% modal logic.

% Epistemic modality

`It might be that' and `it is certain that' express an \emph{epistemic} kind of
possibility and necessity, related to evidence and knowledge. There are other
kinds -- or \emph{flavours} -- of possibility and necessity.

% Deontic modality

Consider `John must leave'. This expresses a kind of necessity, but it would
typically not be understood as a statement about the available evidence. On its
most natural interpretation, it says that some relevant norms require John to
leave. This flavour of necessity is called \emph{deontic} (from Greek
\emph{deontos}: `of that which is binding').

% Circumstancial modality

Other statements about possibility and necessity are neither deontic nor
epistemic. If I say that you can't travel from Auckland to Sydney by train, I
don't just mean that my information implies that you won't make that journey;
nor do I mean that you're not permitted to make it. Rather, I mean that relevant
circumstances in the world -- such as the presence of an ocean between Auckland
and Sydney -- preclude the journey. This flavour of modality is sometimes called
\emph{circumstantial}. It comes in many sub-flavours, depending on what kinds of
circumstances are in play.

% In natural language, modals often have a mixed flavour. E.g. "I have to go
% now", "I can pick you up from the station", "I can't stay", "I'll do it as
% soon as possible", "the food is edible".

% Flavoured logics

Each of these flavours of modality corresponds to a branch of modal logic.
\emph{Epistemic logic} formalizes reasoning about knowledge and information.
\emph{Deontic logic} deals with norms, permissions, and obligations. A third
branch of modal logic might be called \emph{circumstantial logic}, but nobody
uses that label. Some authors speak of \emph{alethic modal logic} (from
\emph{aletheia}: `truth'), but this label is also not used widely, and it is
used for different things by different authors.

% The division into epistemic, deontic, and alethic logic was made popular in \cite{wright51essay}

% Metaphysical modality

Confusingly, some philosophers use `modal logic' for the logic of a certain
sub-flavour of circumstantial modality, known as \emph{metaphysical} modality.
Metaphysical modality is concerned with what is or isn't compatible with the
nature of things. We will follow the more common practice of using `modal logic'
as an umbrella term that covers all the applications I have mentioned, as well
as many others.

% Preview

We will take a closer look at epistemic logic in chapter \ref{ch:epistemic} and
at deontic logic in chapter \ref{ch:deontic}. In chapter \ref{ch:time} we are
going to study a branch of modal logic called \emph{temporal logic} that is
concerned with reasoning about time. Chapter \ref{ch:conditionals} is on
\emph{conditional logic}. Here we will introduce (non-truth-functional)
two-place operators that are meant to formalise certain `if \ldots then \ldots'
constructions in English. In chapter \ref{ch:proofs}, we will briefly look at
\emph{provability logic}, which investigates formal properties of mathematical
provability. What unifies the different branches of modal logic is not a
particular subject matter, but a loosely defined collection of abstract ideas
and techniques that turn out to be useful in all these applications.

% Boxes and diamonds

When we study some flavour of possibility or necessity, the diamond $\Diamond$
is generally used for the relevant kind of possibility and the box $\Box$ for
the corresponding kind of necessity. In this context, you may pronounce the
diamond `it is possible that' and the box `it is necessary that'. In general,
however, I would recommend pronouncing the diamond `diamond' and the box `box'.

% A variety of logics

Different interpretations of the box and the diamond often motivate different
rules for reasoning with these expressions. Consider, for example, the inference
from $\Box p$ to $p$. If the box expresses a circumstantial kind of necessity,
then this inference is plausibly valid: if the circumstances ensure that
something is the case, then it really is the case. On a deontic reading of the
box, by contrast, the inference is invalid. We can easily imagine scenarios in
which, say, it is required that all library books are returned on time
($\Box p$) and yet it is not the case that all library books are returned on
time ($\neg p$).

So we can't say, once and for all, whether $\Box p$ entails $p$. We will develop
different ``logics'' or ``systems'' of modal logic. In some systems, the inference is
valid, in others it is invalid.

% English modals

The diamond and the box are sentence operators. English expressions for
necessity and possibility often don't have this form. We can talk about what's
necessary or possible using `must', `might', or `can', which are (auxiliary)
verbs. We can also use adjectives like `feasible', `certain', and 'obligatory',
or adverbs like `possibly', `certainly', and `inevitably'.

% Translating from English

When translating from English into $\L_{M}$, it is often helpful to first
paraphrase the English sentence with `it is necessary that' and `it is possible
that' (or other suitable sentence operators). For example,
\begin{quote}
  You can't go from Auckland to Sydney by train
\end{quote}
might be paraphrased as
\begin{quote}
  It is not possible [in light of relevant circumstances] that you go from
  Auckland to Sydney by train
\end{quote}
An adequate translation is $\neg\Diamond p$, where $p$ represents `you go from
Auckland to Sydney by train' and the diamond represents the relevant kind of
circumstantial possibility.

\begin{exercise}
  Translate the following sentences, as well as possible, into $\L_{M}$,
  assuming that the diamond expresses epistemic possibility (`it might be that')
  and the box epistemic necessity (`it must be that').
  \begin{exlist}
  \item I may have offended the principal.
  \item It can't be raining.
  \item Perhaps there is life on Mars.
  \item If the murderer escaped through the window, there must be
    traces on the ground.
    % Against narrow scope: e->[]t together with <>-t logically
    % entails -e. And in the context, <>-t is plausibly true. But it
    % would be wrong to infer -e.
  \item If the murderer escaped through the window, there might be
    traces on the ground.
  \end{exlist}
\end{exercise}
\begin{solution}
  \begin{sollist}
    \item $\Diamond p$ \quad  $p$: I offended the principal.
    \item $\neg \Diamond p$ \quad  $p$: It is raining.
    \item $\Diamond p$ \quad  $p$: There is life on Mars.
    \item $\Box(p \to q)$ \quad  $p$: The murderer escaped through the window; $q$: There are traces on the ground.
    \item $\Diamond(p \land q)$ \quad $p$: The murderer escaped through the window; $q$: There are traces on the ground.
  \end{sollist}
\end{solution}

\begin{exercise}
  Translate the following sentences, as well as possible, into $L_{M}$, assuming
  that the diamond expresses deontic possibility (`it is permitted that') and
  the box deontic necessity (`it is obligatory that').
  \begin{exlist}
  \item I must go home.
  \item You don't have to come.
  \item You can't have another beer.
  \item If you don't have a ticket, you must pay a fine.
  % This plausibly has two readings, one with an epistemic must, another with
  % the deontic must restricted by the conditional. How do they come apart?
  % Imagine everyone who doesn't have to pay a fine is handed a "ticket" for the
  % no-fine queue. There is no rule that those without a ticket have to pay a
  % fine. Then [](~t -> f) is plausibly false, but ~t -> []f is true. For the
  % converse direction, normally we'd assume that it's a matter or law that
  % whoever doesn't have a ticket must pay a fine. So [](~t -> f) would seem
  % correct. In that case ~t -> []f often will also seem fine, because we are
  % restricting the domain of relevant worlds by the facts about whether you
  % have a ticket: i.e., given that ~t is true, it is also true among the worlds
  % over which the box ranges, and so []f will be true. To make the narrow-scope
  % reading false, we'd have to imagine a scenario in which we don't hold fixed
  % whether you have a ticket when we evaluate the modal.
  % 
  % \item You need a special visa to enter Chukotka.
  \end{exlist}
\end{exercise}
\begin{solution}
  \begin{sollist}
    \item $\Box p$ \quad  $p$: I go home.
    \item $\neg \Box p$ \quad  $p$: You come.
    \item $\neg\Diamond p$ \quad  $p$: You have another beer.
    \item $\Box(\neg p \to q)$ \quad  $p$: You have a ticket; $q$: You pay a fine.
  \end{sollist}
\end{solution}

\begin{exercise}
  Translate the following sentences, as well as possible, into $L_{M}$, assuming
  that the diamond expresses (some relevant sub-flavour of) circumstantial
  possibility and the box circumstantial necessity.
  \begin{exlist}
  \item I could have studied architecture.
  % \item It's impossible for me to both cook and entertain the children.
  \item The bridge is fragile.
  \item I can't hear you if you're talking to me from the kitchen. 
  \item If you have a smartphone, you can use an electronic ticket.
  % Here `can' plausibly expresses some kind of circumstantial possibility.
  % Again, we have to choose between two formalizations: $\Diamond(p \to q)$ and
  % $p \to \Diamond q$, where $p$ represents you having a smartphone and $q$
  % using an electronic ticket. In this case, the second formalization is
  % arguably better. The statement plausibly does say that if you have a
  % smartphone, then the following is possible, in a relevant circumstantial
  % sense: you use an electronic ticket.
    
  \end{exlist} 
\end{exercise}
\begin{solution}
  \begin{sollist}
    \item $\Diamond p$ \quad  $p$: I study architecture.
    % Notice the fake past in the English sentence.
    \item $\Diamond p$ \quad  $p$: The bridge collapses.
    \item $\neg\Diamond(p \land q)$ \quad  $p$: You are talking to me from the kitchen; $q$: I hear you.
    \item $p \to \Diamond q$ \quad $p$: You have a smartphone; $q$: You use an electronic ticket.
  \end{sollist}
\end{solution}

% Conditionals

Special care is required when translating English sentences that contain both
modal expressions and an `if' clause. The surface form of English can be
misleading. A good strategy is to first rephrase the English sentence so that it
no longer contains any conditional expression, then translate that paraphrase.
The paraphrase, and therefore the translation, will often sound rather unlike
the original sentence, but that's OK. What's important is that it has the same
truth-conditions. There should be no conceivable scenario in which the original
sentence is true and the paraphrase (or translation) false, or the other way
round.

\section{The turnstile}
\label{sec:turnstile}

In section \ref{sec:intro}, I said that an argument is valid if there is no
conceivable scenario in which the premises are true and the conclusion is false.
An argument is logically valid, I said, if it is valid ``in virtue of its
logical form''. Can we make this more precise?

Consider this English argument.
%
\begin{quote}
  Some cats are black.\\
  Therefore: Some animals are black.
\end{quote}
%
The argument is valid, but not logically valid. Its validity turns on the
meaning of `cat', which we don't consider a logical expression.

To bring out how the argument's validity depends on the meaning of `cat', we can
imagine a language that is much like English except that `cat' means
\emph{chair}. In this language, the argument just displayed is invalid. It is
invalid because there are conceivable scenarios in which there are black chairs
but no black animals. In any such scenario, the argument's premise is true (in
our imaginary language) while the conclusion is false.

When we say that an argument is valid ``in virtue of its logical form'', we mean
that its validity does not depend on the meaning of the non-logical expressions.
In other words, there is no conceivable scenario in which the premises are true
and the conclusion is false, \emph{no matter what meaning we assign to the
  non-logical expressions}.

The concept of validity for arguments is closely related to that of entailment.
If an argument is valid, we say that the premises entail the conclusion. If an
argument is logically valid, we say that the premises logically entail the
conclusion. In logic, we're interested in logical entailment. We adopt the
following definition.

\begin{definition}{}{entailment-informal}
  Some sentences $\Gamma$ ('gamma') \textbf{(logically) entail} a sentence $A$
  iff there is no conceivable scenario in which all sentences in $\Gamma$ are
  true and $A$ is false, under any interpretation of the non-logical
  expressions.
\end{definition}

Instead of saying that the sentences $\Gamma$ logically entail $A$, we also say
that $A$ is a \emph{logical consequence of} $\Gamma$, or that $A$
\emph{logically follows from} $\Gamma$. Two sentences are \emph{(logically)
  equivalent} if either logically follows from the other.

Logicians often use the symbol `$\models$' (the ``double-barred turnstile'') for
entailment. The claim that $\Box (p \to q)$ and $\Box p$ together entail $q$,
for example, could be expressed as
\begin{equation*}
  \Box (p \to q), \Box p \models q.
\end{equation*}

This is not a sentence of $\L_{M}$. The comma and the turnstile belong to
the \textbf{meta-language} we use to talk about the \textbf{object language}
$\L_M$. (The rest of our meta-language is mostly English.) We use the turnstile
to express a certain relationship between $\L_M$-sentences, not to construct
further $\L_{M}$-sentences.

\begin{exercise}
  % If we are allowed to re-interpret the non-logical expressions, do we even need
  % to consider alternative scenarios? 
  What do you think of this simpler alternative to definition
  \ref{def:entailment-informal}? ``Sentences $\Gamma$ entail a sentence $A$ iff
  there is no interpretation of non-logical expressions that renders all
  sentences in $\Gamma$ true and $A$ false.''
\end{exercise}
\begin{solution}
  The proposed definition is equivalent to definition
  \ref{def:entailment-informal} for many languages, but not for all. Consider
  the sentence $\exists x \exists y \neg(x = y)$ in the language of predicate
  logic. If we treat the identity symbol as logical, this sentence contains no
  non-logical expressions at all. And the sentence is true, because there is in
  fact more than one object. So the sentence is true under any interpretation of
  its non-logical vocabulary. But it's not logically true; it doesn't logically
  follow from any premises whatsoever. The sentence is false in any scenario in
  which there is only one object.
  % For another example, if there is actually just one object, then the proposed
  % definition would imply that Fa entails Fb. 
\end{solution}

The following fact about logical consequence often proves useful.

\begin{observation}{semantic-deduction-theorem}
  If $A$ and $B$ are sentences and $\Gamma$ is a (possibly empty) list of sentences, then
  \vspace{-1mm}
  \[
    \Gamma,A \models B \text{ \;iff\; }\Gamma \models A \to B.
  \]
  \vspace{-5mm}
\end{observation}
%
\begin{proof}
  \emph{Proof}. Look at the statement on the right-hand side of the `iff'.
  `$\Gamma \models A \to B$' says that there is no conceivable scenario in which
  all sentences in $\Gamma$ are true while $A\to B$ is false, under any
  interpretation of the non-logical expressions. By the truth-table for `$\to$',
  $A\to B$ is false iff $A$ is true and $B$ is false. So we can rephrase the
  statement on the right-hand side as saying that there is no conceivable
  scenario and interpretation that makes all sentences in $\Gamma$ true and
  $A$ true and $B$ false. That's just what the statement on the left-hand
  side asserts. \qed
\end{proof}

Observation \ref{obs:semantic-deduction-theorem} tells us that if we start with
a claim of the form $A_{1},A_{2},A_{3}\ldots \models B$, we can always generate
an equivalent claim by moving the turnstile to the left of the sentence that
precedes it and putting an arrow in its original place. For example, instead of
\begin{equation*}
  \Box (p \to q), \Box p \models \Box q
\end{equation*}
we can equivalently say
\begin{equation*}
  \Box (p \to q) \models \Box p \to \Box q.
\end{equation*}
We can go further to
\begin{equation*}
  \models \Box (p \to q) \to (\Box p \to \Box q).
\end{equation*}
This says that $\Box (p \to q) \to (\Box p \to \Box q)$ logically follows from
no premises at all. A sentence that follows from no premises is called
\emph{logically true} or \emph{(logically) valid}.

(So an \emph{argument} is called valid if the conclusion follows from the
premises, while a \emph{sentence} is called valid if it follows from no
premises.)

Sentence validity is implicitly covered by definition
\ref{def:entailment-informal}, using an empty list of sentences for $\Gamma$.
But it's worth making the definition more explicit.
%
\begin{definition}{}{valid-informal}
  A sentence $A$ is \textbf{valid} (for short, $\models A$) iff there is no
  conceivable scenario in which $A$ is false, under any interpretation of the
  non-logical expressions.
\end{definition}

Make sure you don't confuse the arrow with the turnstile. It's not just that the
two symbols belong to different languages -- one to $\L_{M}$, the other to our
meta-language. They also have very different meanings. $p \to q$ is true iff
either $p$ is false or $q$ is true (or both). $p \models q$, on the other hand,
is true iff there is no conceivable scenario in which $p$ is true and $q$ is
false, under any interpretation of $p$ and $q$. Nonetheless, there is an
important connection between the arrow and the turnstile: $A \models B$ is
\emph{true} iff $A \to B$ is \emph{valid}.

The definitions of this section are still somewhat imprecise. Eventually we will
want to prove various claims about entailment and validity. To this end, we will
need to give rigorous meanings to `conceivable scenario' and `interpretation of
non-logical expressions'. Let's leave this task until the next chapter.

% \begin{exercise}
%   We can generalise our interpretation of the double-barred turnstile. Let's
%   read `$\ldots \models \ldots$' as saying that there is no scenario and
%   interpretation (of the non-logical expressions) that makes everything on the
%   left of the turnstile true while making everything on the right false. On this
%   interpretation, what do the following statements mean? (a)
%   `$p \lor \neg p \models$' (b) `$p \models p,q$', (c) `$\models$'? Which of
%   them are true?
% \end{exercise}

\section{Duality}%
\label{sec:duality}

% Two translations

`Neville can't be the murderer', says Watson. His claim could be paraphrased as
`it is not possible that Neville is the murderer'. This suggests that
$\neg\Diamond p$ is an adequate translation (where $p$ expresses that Neville is
the murderer). But Watson's claim might also be paraphrased as `it is certain
that Neville is not the murderer', which we might translate as $\Box\neg p$.

% Equivalence

The two paraphrases are plausibly equivalent. In general, `it is not
(epistemically) possible that $A$' seems to say the same as `it is certain that
not $A$'. Similarly, `it is not certain that $A$' arguably says the same as `it
is possible that not $A$'.

% Dual1 and Dual2

Whether or not the equivalence holds in English, we stipulate that it holds in
$\L_{M}$: for any $\L_{M}$-sentence $A$,
%
\begin{principles}
\pri{Dual1}{\neg \Diamond A \text{ is equivalent to } \Box \neg A};\\
\pri{Dual2}{\neg \Box A \text{ is equivalent to } \Diamond \neg A}.
\end{principles}

% Dual operators

Operators that stand in the relationship expressed by \pr{Dual1} and \pr{Dual2}
are called \textbf{duals} of each other. There is a convention in modal logic to
use the symbols $\Box$ and $\Diamond$ only for concepts that are duals of each
other.

\begin{exercise}
  Find all pairs of duals among the following English expressions.
  \begin{exlist}
  \item It is necessary that \ldots
  \item It is impossible that \ldots
  \item It is possible that \ldots
  \item It is possibly not the case that \ldots
  % \item It is neither necessary nor impossible that \ldots
  \item It was at some point the case that \ldots
  \item It will at some point be the case that \ldots
  \item It has always been the case that \ldots
  \item It will always be the case that \ldots
  \item The law requires that \ldots
  \item The law does not require that \ldots
  \item The law allows that \ldots
  \item It is true that \ldots
  \item It is false that \ldots
  \end{exlist}
\end{exercise}
\begin{solution}
  The following pairs are duals: (a) and (c), (b) and (d), (e) and
  (g), (f) and (h), (i) and (k), (l) and (l), (m) and (m).
\end{solution}

% An equivalence

\pr{Dual1} implies that $\neg\Diamond\neg p$ is equivalent to $\Box\neg\neg p$,
choosing $\neg p$ as the sentence $A$. In standard modal logic, logically
equivalent expressions are interchangeable.\label{claim:replacement} So we can
simplify $\Box \neg\neg p$ to $\Box p$, drawing on the equivalence between
$\neg\neg p$ and $p$. So $\neg\Diamond\neg p$ is equivalent to
$\Box p$.

% Generalised equivalences

The same reasoning could be applied to any other sentence $A$ in place of
$p$. \pr{Dual1} therefore implies that for any sentence $A$,
\[
   \Box A\text{ is equivalent to }\neg\Diamond\neg A.
\]
In the same way, \pr{Dual2} implies that (for any sentence $A$)
\[
   \Diamond A\text{ is equivalent to }\neg\Box\neg A.
\]

% Redundancy

This shows that the box and the diamond can be defined in terms of one another.
We could have used a language whose only primitive modal operator is the box,
and read $\Diamond A$ as an abbreviation of $\neg\Box\neg A$. Alternatively, we
could have used the diamond as the only primitive modal operator and read
$\Box A$ as an abbreviation of $\neg\Diamond\neg A$.

\begin{exercise}
  Which of these sentences are equivalent to $\Diamond\Diamond \neg p$? (a)
  $\Diamond \neg \Diamond p$, (b) $\Diamond\neg \Box p$, (c)
  $\neg\Box\Diamond p$, (d) $\neg\Diamond\Box p$, (e) $\neg\Box\Box p$
\end{exercise}
\begin{solution}
  (b) and (e) are equivalent to $\Diamond\Diamond \neg p$, (a), (c), and (d) are
  not.
  
  As a rule, you can always replace a modal operator by its dual, insert a
  negation on both sides, and remove any double negations to get an equivalent
  sentence.
\end{solution}

% Does 'possible' imply 'not necessary'?

A digression: you might think that there is another connection between `possible' and
`necessary'. When we say that something is possible (or that it might be the
case), we often convey that it is not necessary (or not certain). This suggests
that $\Diamond p$ entails $\neg \Box p$. We've just assumed, however, that
$\Diamond p$ is equivalent to $\neg \Box \neg p$. If $\Diamond p$ entails
$\neg \Box p$, we would have to conclude that $\neg\Box\neg p$ entails
$\neg\Box p$. By contraposition, we could infer that $\Box p$ entails
$\Box \neg p$. But `it is necessary that $P$' surely doesn't entail `it is
necessary that not-$P$'!

% A choice

We have to reject either the duality of `possible' and `necessary' or the
apparent entailment from `possible' to `not necessary'. On reflection, the case
for duality is stronger. There is a good explanation of why `possible' often
\emph{appears} to entail `not necessary' even if it actually doesn't.

% An example.

Take an example. Suppose Watson says `Neville might be the murderer'. Let's
assume that `might' is the dual of `certain', so that `it might be that $P$' is
equivalent to `it is not certain that not $P$'. On this interpretation, what
Watson said -- that Neville might be the murderer -- is merely that it isn't
certain that Neville is \emph{not} the murderer. It may well be certain that
Neville \emph{is} the murderer. Why, then, does his statement convey that
Neville's guilt is an open question?

% A pragmatic explanation

Well, suppose Watson had known that Neville is the murderer. In that case, he
shouldn't have said `Neville might be the murderer'. These words would still
have been true -- or so we assume -- but they would not have been helpful.
Watson would have been in a position to say something more informative: that
Neville is the murderer, or that he is known to be the murderer. We generally
assume that speakers are trying to be helpful, that they are not hiding relevant
information. Assuming that Watson is trying to be helpful, his \emph{statement}
that Neville might be the murderer implies that he considers Neville's guilt an
open question. This follows not from \emph{what he said}, but from the fact
\emph{that he said it}, together with the assumption that he is trying to be
helpful.

% Scalar implicature

This kind of effect is studied in the field of pragmatics, where it is known as
a \emph{scalar implicature}. Scalar implicatures arise when an utterance of a
logically weaker sentence conveys that a certain stronger sentence is false.
`Some students passed the test', for example, conveys that not all students
passed the test, although the statement would be true even if all students had
passed. In that case, however, it would not have been helpful: the speaker
should have used `all students passed'.
End of digression.

% Schemas

I want to say a little more about duality. To do so, I need to introduce the
concept of a schema.

% I already mentioned that I use upper-case letters $A,B,C,\ldots$ when I want to
% talk about arbitrary $\L_{M}$-sentences. Above, for example, I said that we
% could have read $\Diamond A$ as an abbreviation of $\neg\Box\neg A$. By this, I
% mean that we could have read $\Diamond p$ as an abbreviation of
% $\neg\Box\neg p$, $\Diamond (p \lor q)$ as an abbreviation of
% $\neg\Box\neg (p \lor q)$, and so on, for all $\L_{M}$-sentences. The
% expressions `$\Diamond A$' and `$\neg\Box\neg A$' are schemas.

Formally, a \textbf{schema} (for $\L_{M}$-sentences) is simply an
$\L_{M}$-sentence with upper-case schematic variables in place of sentence
letters. Every $\L_{M}$-sentence that results from a schema by (uniformly)
replacing the schematic variables with object-language sentences is called an
\textbf{instance} of the schema.

% An example

$\Box A \to A$, for example, is a schema. Three of its instances are
$\Box p \to p$ and $\Box (p \lor q) \to (p \lor q)$ and
$\Box \Box p \to \Box p$. The sentence $\Box p \to q$ is not an instance: the
same schematic variable must always be replaced by the same object-language
sentence. (That's what I meant by ``uniformly''.)

\begin{exercise}
  Which of the following expressions are instances of
  $\Box(A\to \Diamond (A \land B))$?
  \begin{exlist}
  \item $\Box(p \to \Diamond (q\land p))$
  \item $\Box(\Diamond p \to \Diamond (\Diamond p\land p))$
  \item $\Box\Box(p \to \Diamond (p \land q))$
  \item $\Box((p \to \Diamond (p \land q)) \to \Diamond((p \to \Diamond (p \land q)) \land \Diamond p))$
  \item $\Box((A\land C) \to \Diamond ((A\land C) \land (B\land C)))$
  \end{exlist}
\end{exercise}
\begin{solution}
  (b) and (d)
\end{solution}

% Use of schemas

Schemas are useful when we want to talk about all $\L_{M}$-sentences of a
certain form. In the next section, for example, we are going to define a system
of modal logic by giving a list of schemas all instances of which are considered
valid.

% Dual schemas

Now compare the schemas $\Box A \to A$ and $A \to \Diamond A$. Given the duality
of the box and the diamond, and the fact that logically equivalent expressions
can be freely exchanged for one another, we can show that \emph{every instance
  of one of them is equivalent to an instance of the other}. In this sense, the
two schemas are equivalent. And because their equivalence relies on the duality
of the box and the diamond, the two schemas are called duals of one another.

% Proof of duality

To see why every instance of $\Box A \to A$ is equivalent to an instance of
$A \to \Diamond A$, take a simple instance: $\Box p \to p$. By the truth-table
for the arrow, this is equivalent to $\neg p \to \neg \Box p$. By \pr{Dual2},
$\neg \Box p$ is equivalent to $\Diamond \neg p$. So $\neg p \to \neg \Box p$ is
equivalent to $\neg p \to \Diamond \neg p$. And this is an instance of
$A \to \Diamond A$. The same line of reasoning obviously works for any other
sentence in place of $p$, and a similar line of reasoning shows the converse,
that every instance of $A \to \Diamond A$ is equivalent to an instance of
$\Box A \to A$.

% Instances not equivalent

It's crucial that we're talking about schemas here. We have not shown that the
\emph{sentence} $\Box p \to p$ is equivalent to $p \to \Diamond p$. In fact, the
duality principles and the replacement of equivalents don't suffice to show that
these sentences are equivalent.

% Equivalence of axioms

The equivalence of the \emph{schemas}, however, is enough to show that it
doesn't matter which of them we use when we list schemas to define a logic. We
can say that all instances of $\Box A \to A$ are valid in a certain logic, or we
can say that all instances of $A \to \Diamond A$ are valid -- it amounts to the
same thing, because every instance of either schema is equivalent to an instance
of the other.

% Generalising 

The equivalence between $\Box A \to A$ and $A \to \Diamond A$ is an example of
a more general pattern. Any schema with an arrow ($\to$ or $\leftrightarrow$) as
the only truth-functional operator can be converted into an equivalent schema --
its \textbf{dual} -- by swapping antecedent and consequent and replacing every
box with a diamond and every diamond with a box. 

% More generally, the dual of a schema with -> as its main connective and no
% other -> is the schema in which antecedent and consequent are interchanged and
% & and v and [] and <> are swapped by one another. (To find the dual of e.g. GL
% first replace an embedded -> by not or.)

\begin{exercise}
 Find the duals of (a) $\Box A \to \Box\Box A$, (b) $\Diamond A \to \Box\Diamond A$, (c) $\Box A \to \Diamond A$.
\end{exercise}
\begin{solution}
  (a) $\Diamond\Diamond A \to \Diamond A$, (b) $\Diamond\Box A \to \Box A$, (c)
  $\Box A \to \Diamond A$.
\end{solution}

\begin{exercise}
  A proposition is \emph{contingent} if it neither necessary nor impossible. Let
  $\nabla$ be a sentence operator for `it is contingent that'. Reading the box as `it is necessary that' and the diamond as `it is possible that', try to find
  \begin{exlist}
    \item a sentence whose only modal operator is $\Box$ that is equivalent to
    $\nabla p$;
    \item a sentence whose only modal operator is $\Diamond$ that is equivalent to
    $\nabla p$;
    \item a sentence whose only modal operator is $\nabla$ that is equivalent to $\Box p$.
  \end{exlist}
\end{exercise}
\begin{solution}
  (a) $\neg \Box p \land \neg\Box\neg p$; (b)
  $\Diamond p \land \Diamond \neg p$; (c) $\neg\nabla p \land p$. The last
  answer assumes that every necessary proposition is true. Without that
  assumption there is no answer to (c). % as shown in \cite{cresswell1988necessity}
\end{solution}

\section{A system of modal logic}%
\label{sec:systems}

Whether a sentence is logically valid, or logically entailed by other sentences,
never depends on the meaning of the non-logical expressions. But it may well
depend on the meaning of the logical expressions. In modal logic, the box and
the diamond are treated as logical expressions, but their interpretation varies
from application to application. Sometimes the box means epistemic necessity,
sometimes it means deontic necessity, sometimes it means something else. As I
mentioned in section \ref{sec:flavours}, this has the consequence that we need
to distinguish different ``systems of modal logic''. In some applications, we
want $\Box p$ to entail $p$, in others we don't.

Suppose, now, that we want to fully spell out one of these ``systems''. We want
to completely specify which $\L_{M}$-sentences are valid, and which are entailed
by which others, on a particular understanding of the modal operators.

There are many ways of approaching this task. We could, for example, define
precise notions of conceivable scenarios and interpretations and apply the
definitions of the previous section. But let's choose a more direct route. When
we think about circumstantial necessity, we can intuitively see that $\Box p$
entails $p$, without going through sophisticated considerations about scenarios
and interpretations. Assume, then, that we simply start with direct judgements
about entailment and validity.

We still face a problem. There are infinitely many $\L_{M}$-sentences. We can't
look at every sentence and argument one by one. We need to find some
shortcuts.

We can begin by drawing on a consequence of observation
\ref{obs:semantic-deduction-theorem}. Above I said that in order to spell out a
system of modal logic, we need to specify (i) which $\L_{M}$-sentences are valid
and (ii) which $\L_{M}$-sentences are entailed by which others. Observation
\ref{obs:semantic-deduction-theorem} tells us that we can ignore part (ii) of
the task. Once we have settled which sentences are valid, we have implicitly
also settled which sentences entail which others. If, for example, we decide
that $\Box p \to p$ is valid, we have also decided that $\Box p$ entails $p$.

Our task of spelling out a system of modal logic therefore reduces to
the task of specifying which $\L_{M}$-sentences are valid. That's why a
\textbf{system of modal logic} is usually defined simply as a set of
$\L_{M}$-sentences.

To make this more concrete, let's look at a particular sub-flavour of
circumstantial necessity, sometimes called \emph{historical necessity}.
Something is historically necessary if it is ``settled'': it is true and there
is nothing anyone can do about it. Facts about the past are plausibly settled.
Nothing we can do is going to make a difference to what happened yesterday. By
contrast, some facts about the future are intuitively ``open''.

Let's use the box to formalise this (admittedly vague) concept of historical
necessity. So $\Box p$ says that $p$ is settled. Since the diamond is the dual
of the box, $\Diamond p$ expresses that it not settled that $p$ is false. In
other words, $p$ is either open or settled as true.

% Don't say: <>p means that p is open: otherwise []p -> <>p looks implausible.
% 'open' conveys 'could be made true and could be made false'.

Our task is to specify all $\L_{M}$-sentences that are valid on this
understanding of the box and the diamond. This will give us a system of modal
logic, a set of $\L_{M}$-sentences that are valid on a certain interpretation of
the box and the diamond. We want to know which sentences are in the system --
for short, which sentences are ``in'' -- and which are not.

If the box expresses historical necessity then $\Box p$ clearly entails $p$. So
$\Box p \to p$ is in. There is nothing special here about the sentence $p$.
Whatever is settled is true. Every instance of the schema $\Box A \to A$ is in.
(As mentioned in section \ref{sec:duality}, it follows that every instance of
$A \to \Diamond A$ is in as well.)

In the same vein, we may now look at other schemas. Arguably, all instances of
the following schemas -- listed here with their conventional names -- are valid,
and therefore in our target system:
%
\begin{principles}
  \pri{Dual}{\neg\Diamond A \leftrightarrow \Box\neg A}\\
  \pri{T}{\Box A \to A}\\
  \pri{K}{\Box(A\to B) \to (\Box A \to \Box B)}\\
  \pri{4}{\Box A \to \Box \Box A}\\
  \pri{5}{\Diamond A \to \Box \Diamond A}
\end{principles}

\pr{Dual} corresponds to the duality principle \pr{Dual1} from section
\ref{sec:duality}. Its instances are guaranteed to be valid by the fact that we
have introduced the diamond as the dual of the box.

We've already talked about \pr{T}.

\pr{K} is a little easier to understand as a claim about entailment:
\[
  \Box(A \to B), \Box A \models \Box B.
\]
On our present interpretation, this says that if a material conditional $A\to B$
is settled, and its antecedent $A$ is settled, then its consequent $B$ is
guaranteed to be settled as well. Why should we accept this? Let $A$ and
$B$ be arbitrary propositions, and assume that $A\to B$ and $A$ are both
settled. It follows that they are both true. Since $A\to B$ and $A$ entail $B$,
it follows that $B$ is true as well. Could it be that $B$ is true but open?
Arguably not: If we could bring about a situation in which $B$ is false then we
could also bring about a situation in which either $A\to B$ or $A$ is false,
since one of these is guaranteed to be false in any situation in which $B$ is
false. The assumption that $A\to B$ and $A$ are settled therefore implies that
$B$ is settled. So all instances of \pr{K} are in.

% If we read settled as 'unaffected by our beliefs, desires, and intentions',
% then Lycan claims that (K) is implausible. (Lycan 1994:186f., drawing on
% Slote 1982)

\pr{4} and \pr{5} assert that facts about what is settled are themselves
settled. \pr{4} says that if something is settled then it is settled that it is
settled. \pr{5} says that if something is not settled then it is settled that it
is not settled. Here it is important that we adopt a consistent point of view.
It is easy to think of situations in which something is open to us (say, we
could read a certain letter) and we can do something (say, burn the letter) that
would make it no longer open. This doesn't contradict \pr{5}, since \pr{5}
concerns what is open and settled \emph{now}. If something is now open, then
arguably there is nothing we can do that would change the fact that it is now
open. Likewise, if something is now settled, then arguably there is nothing we
can do that would change the fact that it is now settled.

I could have listed further schemas. For example, whenever a conjunction is
settled, then both its conjuncts are plausibly settled as well. So every
instance of $\Box(A\land B) \to (\Box A \land \Box B)$ should be in. There are,
in fact, infinitely many further schemas, not covered by the five above, whose
instances belong to our target system.

That's the bad news. The good news is that we don't need to list any of them. We
can replace the whole lot by specifying two rules for generating new sentences
from sentences we have already classified as ``in''.

The first of these rules captures the plausible thought that anything that
follows from a valid sentence by classical (non-modal) propositional logic is
itself valid. Since we've decided that $\Box p \to p$ is valid (in the logic of
historical necessity), we can, for example, infer that $(\Box p \to p) \lor q$
is also valid, because $A \lor B$ follows from $A$ in classical propositional
logic.
% By adopting this rule, we effectively assume that (i) the meaning of the
% truth-functional connectives is given by their standard truth tables, and that
% (ii) every sentence is either true or false and not both.
Our system of modal
logic thereby becomes an \textbf{extension} of classical propositional
logic. \label{claim:extension}

To state the rule concisely, let $\Gamma \models_{P} A$ mean that $A$ follows
from $\Gamma$ in classical propositional logic -- as can be determined, for
example, by the truth table method. Then our rule says that for any list of
sentences $\Gamma$ and any sentence $A$,
%
\begin{principles}
  \pri{CPL}{\text{If }\Gamma \models_{P} A\text{ and all members of }\Gamma\text{ are in, then }A\text{ is in}.}
\end{principles}

As a special case, \pr{CPL} implies that every propositional tautology is
``in'', since tautologies follow in classical propositional logic from any
premises whatsoever (and even from no premises).

Our second rule reflects the idea that all logical truths are settled: For any
sentence $A$,
%
\begin{principles}
  \pri{Nec}{\text{If $A$ is in, then }\Box A\text{ is in}.}
\end{principles}

% It is easy to see that these two rules generate infinitely many sentence schemas
% from any basis of axioms. Indeed (Nec) alone tells us that since (e.g.) all
% instances of []A->A are in S, so are all instances of []([]A->A), all instances
% of [][]([]A->A), and so on.

And now we're done. I claim -- and this may seem rather mysterious at the moment
-- that there is a natural understanding of historical necessity (of `settled')
on which the sentences that are valid in the logic of historical necessity are
precisely the sentences that can be generated from instances of \pr{T}, \pr{K},
\pr{4}, \pr{5} and \pr{Dual} by \pr{CPL} and \pr{Nec}. (In fact, \pr{4} is
redundant: any instance of \pr{4} can be derived from the remaining axioms and
rules.)

The system of modal logic defined by these schemas and  rules is perhaps the
best known of all systems of modal logic. Its conventional name is `S5' because
it was introduced as the fifth system in an influential list of systems
published by C.I.\ Lewis and C.H.\ Langford in 1932.

Other systems of modal logic can be defined by different schemas or rules.
Lewis and Langford's system S4, for example, is defined by \pr{T},
\pr{K}, \pr{4}, \pr{Dual}, \pr{CPL} and \pr{Nec}, without
\pr{5}. This system is adequate for other interpretations of the box and the
diamond, where we don't want to treat all instances of \pr{5} as valid.

\begin{exercise}
  Instead of reading the box as `it is settled that', we might give it one of these interpretations (with the diamond defined as the box's dual):
  \begin{exlist}
    \item it is true that
    \item it is false that
    \item it is either true or false that
    \item it is logically true that
  \end{exlist}
  \medskip\noindent%
  For each of these interpretations, evaluate whether the schemas \pr{T}, \pr{K}, \pr{4}, \pr{5}, and the rules \pr{CPL} and \pr{Nec} are plausible. 
\end{exercise}
\begin{solution}
  \begin{sollist}
  \item All of them.
  \item Only \pr{K} and \pr{CPL}.
  \item All except \pr{T}.
  \item All of them.
  % This is discussed in Burgess 1999, Which modal logic is the right one. The
  % idea goes back to Hallden 1963 and further to Carnap 1946.) See Schurz 2001
  % for arguments that the logic of validity is much stronger than S5: see
  % Kracht and Kutz 2007:964.

  % For (K): If a conditional $A \to B$ is true in virtue of its form, and so is
  % $A$, we can conclude that $B$ is true in virtue of its form? Arguably yes.
  % Intuitively, if $A$ is true in any conceivable scenario under any
  % interpretation of the sentence letters, and $A\to B$ is true in any
  % conceivable scenario under any interpretation of the sentence letters, then
  % so is $B$ -- for $B$ is bound to be true in any scenario in which $A$ and
  % $A \to B$ are both true.

  % What about \pr{4}? Take an example. $p\lor \neg p$ is logically true; so
  % $\Box (p \lor \neg p)$ is true. You don't need to know what $p$ means in
  % order to see that $\Box (p\lor \neg p)$ is true, nor do you need to know any
  % substantive facts about the world: the statement is true in virtue of its
  % logical form. So $\Box\Box (p \lor \neg p)$ is true as well. In general, if
  % $A$ is true in virtue of its form, then $\Box A$ is also true in virtue of
  % its form. So $\Box A$ does entail $\Box\Box A$.

  % Next, \pr{5} says that if $\neg A$ is not true in virtue of its form, then
  % it is true in virtue its form that $\neg A$ is not true in virtue of its
  % form. This isn't easy to understand, but the following line of thought shows
  % that it is plausible.

  % Suppose the antecedent of \pr{5} is true: $\neg A$ is not true in virtue of
  % its form. If a sentence is not true in virtue of its form, then evidently
  % any sentence of the same form also isn't true in virtue of its form. So,
  % given that $\neg A$ is not true in virtue of its form, one can tell merely
  % by the logical form of $\neg A$ that $\neg A$ is not true in virtue of its
  % form -- in other words, that $\Box \neg A$ is false. So one can tell merely
  % by the logical form of $\Box \neg A$ that it is false. And so one can tell
  % merely by the logical form of $\neg \Box \neg A$ -- equivalently,
  % $\Diamond A$ -- that it is true. So from the assumption $\Diamond A$ we can
  % logically infer $\Box \Diamond A$. So \pr{5} is valid.
\end{sollist}
\end{solution}

Remember that a system of modal logic is just a set of $\L_{M}$-sentences. I
have defined the system S5 in terms of \pr{T}, \pr{K}, \pr{4}, \pr{5}, or
\pr{Dual}, \pr{CPL} and \pr{Nec}, but the same system can be defined by many
other combinations of schemas and rules. (Lewis and Langford used a very
different definition.)

The schemas and rules that I have chosen are called an \textbf{axiomatisation}
of S5. The schemas -- or more precisely, their instances -- are called
\textbf{axioms} because they are the starting points if we want to show that a
sentence is in the system.

% This style of axiomatising modal logics goes back to G\"odel 1933: Eine
% interpretation des intuitionistischen aus- sagenkalkuls. In Ergebnisse eines
% mathematisches Kolloquiums, volume 4, pages 39-40.

To illustrate this point, think of how we could show that
$\Box(p \land q) \to \Box p$ is in S5 (that it is ``S5-valid''). The sentence is
not an instance of any of the schemas I have listed. Instead, we may start with
the non-modal sentence $(p \land q) \to p$. This is a propositional tautology,
so \pr{CPL} tells us that it is in S5. By \pr{Nec}, it follows that
$\Box((p \land q) \to p)$ is in S5 as well. Since all instance of
\pr{K} are in S5, the system contains
\[
  \Box((p \land q) \to p) \to (\Box(p \land q) \to \Box p).
\]
By Modus Ponens, $\Box((p \land q) \to p)$ and
$\Box((p \land q) \to p) \to (\Box(p \land q) \to \Box p)$ entail our target
sentence $\Box(p \land q) \to \Box p$. By \pr{CPL}, this means the target
sentence is also in S5.

Here is a more streamlined presentation of this line of reasoning.
%
\begin{alignat*}{2}
  1.\quad& (p \land q) \to p &\quad& \text{(CPL)}\\
  2.\quad& \Box((p \land q) \to p) &\quad& \text{(1, Nec)}\\
  3.\quad& \Box((p \land q) \to p) \to( \Box(p \land q) \to \Box p) &\quad& \text{(K)}\\
  4.\quad& \Box(p \land q) \to \Box p &\quad& \text{(2, 3, CPL)}
\end{alignat*}

We can use the same streamlined format to show that, say,
$\Box p \to \Diamond p$ is S5-valid.
%
\begin{alignat*}{2}
  1.\quad& \Box \neg p \to \neg p &\quad& \text{(T)}\\
  2.\quad& \neg\Diamond p \leftrightarrow \Box \neg p &\quad& \text{(Dual)}\\
  3.\quad& \neg\Diamond p \to \neg p &\quad& \text{(1, 2, CPL)}\\
  4.\quad& p \to \Diamond p &\quad& \text{(3, CPL)}\\
  5.\quad& \Box p \to p &\quad& \text{(T)}\\
  6.\quad& \Box p \to \Diamond p &\quad& \text{(4, 5, CPL)}
\end{alignat*}

These annotated lists look a lot like proofs. They \emph{are} proofs. Every
axiomatisation of a logical system defines a corresponding \textbf{axiomatic
  calculus}. A proof in an axiomatic calculus is simply a list of sentences each
of which is either an axiom or follows from earlier sentences in the list by one
of the rules. (The annotations on the right are not officially part of the proof.
They are added to help understand where the lines come from.)

\begin{exercise}
  Try to find axiomatic proofs showing that the following sentences are in S5.
  \begin{exlist}
    \item $\Box(\Box p \to p)$
    \item $(\Box p \land \Box q) \to \Box(p \land q)$
    \item $\Diamond \neg p \leftrightarrow \neg \Box p$
    % exercise: prove 4 from KT5
  \end{exlist}
\end{exercise}
\begin{solution}
  \begin{sollist}
    
    \item 
    \begin{alignat*}{2}
      1.\quad& \Box p \to p &\quad& \text{(T)}\\
      2.\quad& \Box(\Box p \to p) &\quad& \text{(1, Nec)}
    \end{alignat*}
    
    \item 
    \begin{alignat*}{2}
      1.\quad& p \to (q \to (p \land q)) &\quad& \text{(CPL)}\\
      2.\quad& \Box (p \to (q \to (p \land q))) &\quad& \text{(1, Nec)}\\
      3.\quad& \Box (p \to (q \to (p \land q))) \to (\Box p \to \Box (q \to (p \land q))) &\quad& \text{(K)}\\
      4.\quad& \Box p \to \Box (q \to (p \land q))) &\quad& \text{(2, 3, CPL)}\\
      5.\quad& \Box (q \to (p \land q))) \to (\Box q \to \Box (p\land q))  &\quad& \text{(K)}\\
      6.\quad& \Box p \to (\Box q \to \Box (p\land q))  &\quad& \text{(4, 5, CPL)}\\
      7.\quad& (\Box q \land \Box q) \to \Box (p\land q)  &\quad& \text{(6, CPL)}
    \end{alignat*}

    \item 
    \begin{alignat*}{2}
      1.\quad& \neg \Diamond \neg p \leftrightarrow \Box \neg\neg p &\quad& \text{(Dual)}\\
      2.\quad& \neg\neg \Diamond \neg p \leftrightarrow \neg \Box \neg\neg p &\quad& \text{(1, CPL)}\\
      3.\quad& \Diamond \neg p \leftrightarrow \neg \Box \neg\neg p &\quad& \text{(2, CPL)}\\
      4.\quad& \neg\neg p \to p &\quad& \text{(CPL)}\\
      5.\quad& \Box(\neg\neg p \to p) &\quad& \text{(4, Nec)}\\
      6.\quad& \Box(\neg\neg p \to p) \to (\Box \neg\neg p \to \Box p)&\quad& \text{(K)}\\
      7.\quad& \Box \neg\neg p \to \Box p&\quad& \text{(5, 6, CPL)}\\
      8.\quad& p \to \neg\neg p &\quad& \text{(CPL)}\\
      9.\quad& \Box(p \to \neg\neg p) &\quad& \text{(8, Nec)}\\
      10.\quad& \Box(p \to \neg\neg p) \to (\Box p \to \Box \neg\neg p)&\quad& \text{(K)}\\
      11.\quad& \Box p \to \Box \neg\neg p &\quad& \text{(9, 10, CPL)}\\
      12.\quad& \Box \neg\neg p \leftrightarrow \Box p &\quad& \text{(7, 11, CPL)}\\
      13.\quad& \neg \Box \neg\neg p \leftrightarrow \neg \Box p &\quad& \text{(12, CPL)}\\
      14.\quad& \Diamond \neg p \leftrightarrow \neg \Box p &\quad& \text{(3, 13, CPL)}
    \end{alignat*}
    
  \end{sollist}
\end{solution}

\begin{exercise}
  In the axiomatic calculus for S5, \pr{Nec} allows us to derive $\Box A$
  from $A$. Someone might object that this inference is obviously invalid, since
  a sentence might be true without being necessarily true. Can you explain why
  \pr{Nec} is an acceptable rule in the axiomatic calculus for S5?
\end{exercise}
\begin{solution}
  In an axiomatic calculus, every line in a proof is either an axiom or follows
  from an earlier line by one of the rules. \pr{Nec} therefore assumes that
  whenever a sentence $A$ is \emph{provable in the axiomatic calculus}, then it
  is necessarily true (reading the box as `it is necessary that'). 

  The rules of the axiomatic calculus cannot be used to directly derive
  assumptions from arbitrary premises. To show that $A$ entails $B$, you have to
  prove $A \to B$.
\end{solution}

The axiomatic method is the oldest formal method of proof. It has many virtues,
but user-friendliness is not among them. Even simple facts are often hard to
prove in an axiomatic calculus. In the next chapter, we will meet a different
method that is much easier to use.


% \begin{exercise}
%   Suppose the world is deterministic, so that everything that is going to happen
%   is entailed by the past state of the universe together with the laws of
%   nature. Intuitively, we can't affect the past, nor the laws of nature. Both
%   are settled. Explain why this entails that everything that is true is settled
%   ($A \to \Box A$), assuming that the logic of historical necessity is S5.
% \end{exercise}
% \begin{solution}
%   Let $p$ specify the past state of the universe and $q$ the laws. By
%   assumption, any true proposition $A$ about the past is settled, If $A$ is
%   about the future then it is entailed by $p$ and $q$, both of which are
%   settled. In S5, $\Box p$
% \end{solution}

  
%%% Local Variables: 
%%% mode: latex
%%% TeX-master: "logic2.tex"
%%% End:

\chapter{Possible Worlds}\label{ch:worlds}

\section{The possible-worlds analysis of possibility and necessity}

An important breakthrough in the history of modal logic was the development of
``possible-worlds semantics'' in the 1940s-60s.
%
% The discovery was made by a number of authors, including Carnap 1946, 1947,
% Kanger 1957, Hintikka 1957, 1961, Bayart 1958, 1959, Montague 1960, Kripke
% 1959, 1963a, 1963b.
%
The central idea of possible-worlds semantics is to analyze modal notions in
terms of truth at possible worlds. In its simplest form, the analysis goes like
this:
%
\begin{quote}
  A proposition is possible iff it is true at some possible world.\\
  A proposition is necessary iff it is true at all possible worlds.
\end{quote}

In philosophy jargon, a \textbf{possible world} is a maximally specific
possibility. An example of a possible world is the \textbf{actual world} -- the
totality of everything that is the case. In the actual world, light travels
faster than sound and the Conservatives won the 2019 UK election. In other
possible worlds, sound travels faster than light. In yet others, Labour won the
election.

The possible-worlds analysis translates modal statements into quantificational
statements about possible worlds. You may feel uneasy about this. Talking about
worlds other than the actual world may strike you as fanciful and unscientific.
Besides, you may wonder if anything is really gained by the translation, since
we now face the question of what sorts of worlds should be classified as
``possible''.

Remember that there are different flavours of modality. A proposition might be
epistemically possible, historically possible, metaphysically possible, and so
on. If we want to analyse all these kinds of possibility in terms of possible
worlds, we need different flavours of worlds. There must be epistemically
possible worlds, historically possible worlds, metaphysically possible worlds,
etc. And if we ask how these types of worlds are defined it looks like we have
to turn back to relevant features of the actual world. The ultimate reason why
you can't go from Auckland to Sydney by train is surely that there is no
suitable train line here in our world, not that you don't make the journey in
some non-actual worlds.

These objections cast doubt on the possible-worlds analysis as a piece of
reductive metaphysics. But the metaphysics of modality is not our topic. When we
use the possible-world analysis, we don't assume that the translation in terms
of possible worlds reveals the metaphysical grounds of the original modal
statements. We merely assume that the original statements can be paraphrased in
the fanciful language of possible worlds.

% The reductivist use of possible-worlds semantics is often associated with
% Lewis 1986, who does indeed give a non-modal analysis of `possible world'.

% If we take seriously the idea that a possible world is a complete way things
% might have been, then the hypothesis that something might have been the case
% iff it is the case at some possible world does not amount to much more than
% the hypothesis that any incomplete way things might have been (like, me having
% coffee with breakfast) can be extended to a complete way things might have
% been: If $p$ might have been the case, then there is a way the entire world
% might have been that would have included $p$.

% In practice, we don't even take the completeness of possible worlds all that
% seriously. We will often work with toy worlds that merely settle all questions
% in which we're currently interested, leaving lots of other questions open.

% When we go beyond modality in the application of modal logic, the ``possible
% worlds'' will look even less like genuine worlds. In temporal logic, the role of
% worlds is played by times; in the logic of relativistic spacetime, it is played
% by spacetime point, and in so-called dynamic logics by states of a computer
% program.

For a first glimpse of why this might be useful, consider the following
hypothesis.
%
\begin{equation*}
  \Box\Diamond\Box p \models \Box p
\end{equation*}
% 
Is this true? If something is necessarily possibly necessary, does it follow
that it is necessary? Hard to say. We know that $A$ logically entails $B$ iff
there is no conceivable scenario in which $A$ is true and $B$ false, under any
interpretation of the non-logical expressions. The problem is that it is not
obvious what a scenario would have to look like for $\Box \Diamond \Box p$ to be
true, under a given interpretation of $p$.

The possible-worlds analysis can help clear things up. By the possible-worlds
analysis, $\Box \Diamond \Box p$ says that $\Diamond \Box p$ is true at every
possible world. Since we want to know in which scenarios $\Box \Diamond\Box p$
is true, we will assume that every scenario contains a whole range of possible
worlds. The hypothesis that $\Box \Diamond \Box p$ is true in a scenario now
reduces to the hypothesis that $\Diamond \Box p$ is true at every world in the
scenario. $\Diamond\Box p$ says that $\Box p$ is true at some world. So if
$\Diamond\Box p$ is true at every world in a scenario then $\Box p$ is true at
some world in the scenario. And if $\Box p$ is true at some world in a scenario
then $p$ is true at every world in the scenario. This is just what $\Box p$
says. So whenever $\Box \Diamond\Box p$ is true in a scenario (under some
interpretation of $p$), then $\Box p$ is true in that scenario (under that
interpretation). That is, $\Box\Diamond\Box p$ entails $\Box p$.

\begin{exercise}
  Explain, in the same informal manner, why $\Diamond p$ does not entail
  $\Box p$, assuming the possible-worlds analysis of the box and the diamond.
\end{exercise}
\begin{solution}
  Consider a scenario in which (say) it is raining at some worlds and not
  raining at others. Let $p$ express that it is raining. In this scenario, under
  this interpretation, $\Diamond p$ is true, because $p$ is true at some world.
  But $\Box p$ is false, because $p$ is not true at all worlds. So there are conceivable scenarios and interpretations that render $\Diamond p$ true and $\Box p$ false.
\end{solution}

% \begin{exercise}
%   By the possible-worlds analysis, every modal statement about
%   necessity and possibility can be translated into a non-modal
%   statement that quantifies over possible worlds. The translation also
%   works backwards: many statements that quantify over possible worlds
%   can be translated into modal statements with boxes and diamonds. But
%   not all. Can you find a statement that quantifies over possible
%   worlds but cannot be translated into the language of boxes and
%   diamonds?
% \end{exercise}

\section{Models}
\label{sec:basicmodels}

In section \ref{sec:turnstile}, I defined validity and entailment in terms of
scenarios and interpretations. A sentence is valid, I said, iff it is true in
every conceivable scenario under every interpretation of the non-logical
expressions. This is a little vague. What, exactly, is a conceivable scenario,
and what counts as a relevant interpretation? Also, scenarios and
interpretations are unwieldy objects. It is difficult to give a full description
of a scenario and an interpretation. Fortunately, most of the details are
irrelevant if all we care about is which $\L_{M}$-sentences are true and which
are false in a scenario under a particular interpretation. This observation will
lead us to a more precise definition of validity and entailment.

Suppose I tell you the following about a scenario $S$ and an interpretation $I$
of the sentence letters.

\begin{quote}
  There are three worlds in $S$, $w_{1}, w_{2}$, and $w_{3}$. Under the
  interpretation $I$, the sentence $p$ expresses a proposition that is true at
  $w_{1}$, false at $w_{2}$, and true at $w_{3}$. All other sentence letters
  express propositions that are false at all three worlds.
\end{quote}

This tells you almost nothing about what the scenario looks like. You don't know
if $w_{1}$ is a world at which it is currently raining, or who won which
elections at $w_{2}$. You also don't know what the sentence letters mean under
my interpretation. Does $p$ mean that it is raining? That Labour won the
2019 election? I haven't told you. Yet the sparse information I have given is
enough to determine the truth-value of every $\L_{M}$-sentence at every world.

\begin{exercise}
  Which of the following sentences are true at $w_{1}$ in my scenario $S$ under
  my interpretation $I$?
  \begin{exlist}
  \item $\neg p$ % false
  \item $\neg p \to \Box p$ % true
  \item $\Box p$ % false
  \item $\Diamond\Box p$ % false
  \item $\Diamond \Diamond p \lor \Diamond \Box p$ % false
  \item $\Box (\Box p \to p)$ % true
  \end{exlist}
\end{exercise}
\begin{solution}
  (b), (e), and (f) are true at $w_{1}$, the others false.
\end{solution}

A joint representation of a scenario and an interpretation (of non-logical
expressions) that contains just enough information to determine the truth-value
of every sentence is called a \textbf{model}. Just as a model airplane often
leaves out important aspects of a real airplane -- the motor, the seats, etc. --
models in logic leave out many important aspects of the scenarios and
interpretations they represent.

Adopting the simple possible-worlds analysis of the box and the diamond, we can
define a model for $\L_{M}$ as consisting of two parts. First, there is a set of
things we call ``worlds''. They don't need to be genuine worlds. They can be
arbitrary (usually not further specified) objects whose job is to represent
genuine worlds. Second, there is an ``interpretation function'' that tells us
for each sentence letter at which of the worlds it is true.

\begin{definition}{}{basicmodel}
  A \textbf{basic model} of $\L_M$ is a pair $\t{W,V}$ of%
  \vspace{-3mm}
  \begin{itemize*}
  \item a non-empty set $W$, and
  \item a function $V$ that assigns to each sentence letter of $\L_M$
  a subset of $W$.
  \end{itemize*}
\end{definition}
\noindent
In the next chapter, we will replace this definition by a slightly more
complicated definition. That's why I've called models of the present kind
`basic'.

If you are not familiar with elementary concepts of set theory: A \emph{set} is
a collection of objects, called the \emph{members} or \emph{elements} of the
set. Sets can be defined by listing their members enclosed in curly braces:
`$\{ a, b, c \}$'. The \emph{empty set}, with no members, is denoted by
`$\emptyset$'. A \emph{subset} of a set $X$ is a set all whose members are
members of $X$. A \emph{function} is a mapping -- a kind of abstract machine
that takes objects of a certain kind as input and outputs objects of a possibly
different kind.

The interpretation function $V$ in a model maps each sentence letter to the set
of worlds at which the sentence is true. For example, if $W$ contains three
worlds $w_{1}, w_{2}$, and $w_{3}$, and $V(p) = \{ w_{1}, w_{3} \}$ -- meaning
that $V$ maps $p$ to the set $\{ w_{1}, w_{3} \}$ --, then $p$ is true at
$w_{1}$ and $w_{3}$ but not at $w_{2}$.

Notice that an interpretation function only specifies at which worlds the
\emph{sentence letters} are true. $V$ is defined for $p$, $q$, and $r$, but not
for $p \to q$ or $\Box p$ or $\Diamond \Box q$. This is the key idea behind the
possible-worlds analysis. Once we know at which worlds each sentence letter is
true, we have all we need to determine the truth-value of every sentence at
every world.

To formally define how the truth-value of complex sentences is determined, I
will use (meta-linguistic) statements of the form
\[
  M,w \models A
\]
%
as shorthand for
%
\[
  \text{$A$ is true at world $w$ in model $M$}.
\]
I use `$M,w \not\models A$' for the negation of `$M,w \models A$'.

Yes, it's the same turnstile that we use for entailment and validity. This
should cause no confusion because it is usually clear if the things to the left
of the turnstile are $\L_M$-sentences or meta-linguistic expressions for a model
and a world. (In its present use, the turnstile is often pronounced `makes true'
or `satisfies'.)

The relation $\models$ between a model, a world and an $\L_M$-sentence is
defined as follows.

\begin{definition}{Basic Possible-Worlds Semantics}{basicsemantics}
  If $M = \t{W,V}$ is a basic model, $w$ is a member of $W$, $P$ is
  any sentence letter, and $A,B$ are any $\L_M$-sentences, then

  \medskip
  \begin{tabular}{lll}
    (a) & $M,w \models P$ &iff $w$ is in $V(P)$.\\
    (b) & $M,w \models \neg A$ &iff $M,w \not\models A$.\\
    (c) & $M,w \models A \land B$ &iff $M,w \models A$ and $M,w \models B$.\\
    (d) & $M,w \models A \lor B$ &iff $M,w \models A$ or $M,w \models B$.\\
    (e) & $M,w \models A \to B$ &iff $M,w \not\models A$ or $M,w \models B$.\\
    (f) & $M,w \models A \leftrightarrow B$ &iff $M,w \models A\to B$ and $M,w \models B\to A$.\\
    (g) & $M,w \models \Box A$ &iff $M,v \models A$ for all $v$ in $W$.\\
    (h) & $M,w \models \Diamond A$ &iff $M,v \models A$ for some $v$ in $W$.
  \end{tabular}
\end{definition}

Let's go through the clauses in this definition.

Clause (a) says that a sentence letter is true at a world in a model iff the
world is an element of the set of worlds which the model's interpretation
function assigns to the sentence letter. This is just what I explained above.

Clause (b) says that the negation $\neg A$ of an $\L_M$-sentence $A$ is true at
a world in a model iff $A$ is not true at that world in that model. In other
words, the truth-table for negation applies locally at every world: at any
world, $\neg A$ is true iff $A$ is not true.
Clauses (c)--(f) similarly tell us that the truth-tables for the other
truth-functional connectives apply locally at each world.

Clauses (g) and (h) spell out the possible-worlds analysis of the box and the
diamond. According to (g), a sentence $\Box A$ is true at a world in a model iff
$A$ is true at all worlds in the model. According to (h), $\Diamond A$ is true
at a world in a model iff $A$ is true at some world in the same model.

The whole definition is called a \emph{semantics} because a semantics for a
language is an account of what the expressions in the language mean, and
definition \ref{def:basicsemantics} can be seen as giving the meaning of the
logical expressions in $\L_M$. (The non-logical expressions in $\L_{M}$ don't
have a fixed meaning.)

Since every $\L_M$-sentence is built up from sentence letters with the operators
covered in definition \ref{def:basicsemantics}, the definition settles the
truth-value of every sentence at every world in every model.

Consider, for example, the following model $M$:
%
\begin{gather*}
  W = \{ w_{1},w_{2} \}\\
  V(p) = \{ w_{1},w_{2} \}\\
  V(q) = \{ w_{1} \}\\
  V(P) = \emptyset \text{ for all other sentence letters $P$ }
\end{gather*}
This model contains only two worlds, $w_{1}$ and $w_{2}$. The interpretation
function $V$ indicates that $p$ is true at both worlds, $q$ is true at $w_{1}$,
and all other sentence letters are true nowhere. With the help of definition
\ref{def:basicsemantics}, we can figure out at which of the two worlds, say,
$\Box\Diamond(\Box q \to \Diamond\Box p)$ is true. We start with the smallest
parts of the sentence.

\begin{enumerate*}
  \item $p$ is true at $w_{1}$ and $w_{2}$ (by clause (a) of definition
  \ref{def:basicsemantics}).
  \item $q$ is true at $w_{1}$ and not true at $w_{2}$ (by clause (a) of definition
  \ref{def:basicsemantics}).
  \item $\Box p$ is true at $w_{1}$ and $w_{2}$ (by 1 and clause (g) of definition
  \ref{def:basicsemantics}).
  \item $\Box q$ is true at no world (by 2 and clause (g) of definition
  \ref{def:basicsemantics}).
  \item $\Diamond\Box p$ is true at $w_{1}$ and $w_{2}$ (by 3 and clause (h) of
  definition \ref{def:basicsemantics}).
  \item $(\Box q \to \Diamond\Box p)$ is true at $w_{1}$ and $w_{2}$ (by 4, 5,
  and clause (e) of definition \ref{def:basicsemantics}).
  \item $\Diamond(\Box q \to \Diamond\Box p)$ is true at $w_{1}$ and $w_{2}$ (by
  6 and clause (h) of definition \ref{def:basicsemantics}).
  \item $\Box\Diamond(\Box q \to \Diamond\Box q)$ is true at $w_{1}$ and $w_{2}$ (by 7
  and clause (g) of definition \ref{def:basicsemantics}).
\end{enumerate*}

\begin{exercise}
  At which worlds in the model just described is
  $\Diamond p \to (q \lor \Diamond\Box p)$ true?
\end{exercise}
\begin{solution}
  $\Diamond p \to (q \lor \Diamond\Box p)$ is true at both worlds.
\end{solution}

% I should perhaps clarify that models and worlds often represent the same kind
% of things: conceivable scenarios. But they represent different parts of them.
% E.g., a world represents only non-modal aspects of a scenario, whereas a model
% also represents what's accessible.

% In the early days of possible-worlds semantics, philosophers thought that a
% possible world is more or less the same thing as a conceivable scenario, and
% they often identified the space of possible worlds with the class of all
% models of predicate logic. However, it has proved useful to treat the space of
% possible worlds as a non-logical matter, so that different models may involve
% different possible worlds, just as different models of predicate logic may
% involve different sets of individuals.

\section{Basic entailment and validity}%
\label{sec:redefining}

Using the concept of a model, we can sharpen the hand-wavy definitions of
entailment and validity from section \ref{sec:turnstile}.

Imagine a list of all conceivable scenarios and all possible interpretation of
the sentence letters. By definition \ref{def:valid-informal}, a sentence is
valid iff it is true in all of these scenarios under each of these
interpretations. Every combination of a scenario $S$ and an interpretation $I$
is represented by a model. The model contains enough information to figure out
whether any given sentence is true or false in $S$ under $I$. Assuming that,
conversely, every model represents some combination of a scenario and an
interpretation, it follows that a sentence is valid iff it is true in every
model. In the same way, some sentences $\Gamma$ entail a sentence $A$ iff $A$ is
true in every model in which all members of $\Gamma$ are true.

That's the idea. There is, however, a small problem. Take a model with two
worlds, $W = \{ w_{1}, w_{2} \}$, and assume that $V(p) = \{ w_{1} \}$. Is $p$
true in this model? We can't say. Definition \ref{def:basicsemantics} only
specifies under what conditions a sentence is true \emph{at a world in a model}.
We have not defined what it means for a sentence to be true in a model. So we
can hardly say that a sentence is valid iff it is true in all models.

There are two ways to fix this. The conceptually cleaner response is to change
the definition of a model. Intuitively, the worlds in a scenario are not all on
a par. Think of a scenario in which it is raining although it might have been
snowing. This scenario has worlds at which it is raining and others at which it
is snowing. One of these worlds -- a rain world -- is special: it represents the
actual world in the scenario. `It is raining' is true in the scenario because it
is raining in the actual world of the scenario. Following this line of thought,
we could define a model to consist of \emph{three} elements: a set of worlds
$W$, an interpretation function $V$, and a ``designated element of $W$'' that
indicates which world in $W$ represents the actual world of the scenario. We
could then say that a sentence is \emph{true in a model} iff it is true at the
actual world of the model. Models of this type -- with a designated element of
$W$ -- are called \emph{pointed models}.

We will adopt the more popular second response. Here we change the definition of
entailment and validity. Instead of saying that a sentence is valid iff it is
true in every model, we say that a sentence is valid iff it is true \emph{at
  every world in every model}. Similarly, we say that some sentences $\Gamma$
entail a sentence $A$ iff $A$ is true at every world in every model at which all
members of $\Gamma$ are true.

The two responses amount to the same thing. Since every world in every basic
(un-pointed) model could be chosen as the designated world in a corresponding
pointed model, a sentence is true at all worlds in all basic models just in case
it is true in all pointed models. The response we adopt has the minor advantage
of keeping models slightly simpler, and logicians want their models to be as
simple as possible.

\begin{definition}{}{valid}
  A sentence $A$ is \textbf{valid} (for short: $\models A$) iff
  it is true at every world in every basic model.
\end{definition}
%
\begin{definition}{}{basicconsequence}
  Some sentences $\Gamma$ \textbf{(logically) entail} a sentence $A$ (for short:
  $\Gamma \models A$) iff there is no world in any basic model at which all sentences in $\Gamma$ are true while $A$ is false.
\end{definition}

\begin{exercise}
  Call a sentence true \emph{throughout} a model iff it is true at every world
  in the model. What do you think of the following definition?
  `$\Gamma \models A$ iff there is no model throughout which all sentences in 
  $\Gamma$ are true and throughout which $A$ is false.' Is this equivalent to
  definition \ref{def:basicconsequence}? (Hint: consider the hypothesis that
  $p \models \Box p$.)
\end{exercise}
\begin{solution}
  % The corresponding notion of validity is equivalent to ours, so the question
  % is whether Observation 1.1 holds. |= A->B says that there are no worlds in
  % any model at which A,~B are true. A |= B allows for such worlds, as long as
  % any model in which A is true at /all/ worlds is also a model in which B is
  % true at all worlds. 
  The two definitions are not equivalent, as can be seen from the fact that the
  definition proposed in the exercise would render $p \models \Box p$ true.
  Whenever $p$ is true at every world in a model then (by definition
  \ref{def:basicsemantics}) $\Box p$ is also true at every world in the model.
  Definition \ref{def:basicconsequence} renders $p \models \Box p$ false, since
  there are models in which $p$ is true at some worlds and not at others.
\end{solution}

Above I mentioned an assumption implicit in our new definitions: that every
model represents a pair of a conceivable scenario and interpretation. This isn't
obvious. For example, if our topic is metaphysical possibility and necessity, it
may be hard to conceive of a scenario with exactly two possible worlds. Is it
really conceivable that there are only two ways a world might have been,
compatible with the nature of things? We could stipulate that a model, at least
for this application, must contain at least (say) a million worlds, or
infinitely many. It turns out, however, that this would make no difference to
the logic. The very same sentences are valid whether we impose the restriction
or not. So we'll allow for models with very few worlds. Such models are often
useful as toy models to illustrate facts about entailment and validity.

\section{Explorations in S5}%
\label{sec:basiclogic}

By definition \ref{def:valid}, a sentence is valid iff it is true at all worlds
in all (basic) models. Definition \ref{def:basicmodel} explains what a (basic)
model is; definition \ref{def:basicsemantics} specifies the truth-value of any
sentence at any world in any model. Together, these definitions settle which
sentences are valid.

Take, for instance, $\Box p \to p$. This turns out to be valid. To see why, let
$w$ be an arbitrary world in an arbitrary model $M$. Either $p$ is true at $w$
or not. If $p$ is true at $w$, then by clause (e) of definition
\ref{def:basicsemantics}, $\Box p \to p$ is also true at $w$. If $p$ is not true
at $w$, then by clause (g) of definition \ref{def:basicsemantics}, $\Box p$ is
not true at $w$ in $M$ either, and then $\Box p \to p$ is true at $w$ by clause
(e). Either way, $\Box p \to p$ is true at $w$. Since $w$ and $M$ were chosen
arbitrarily, this shows that every instance of $\Box p \to p$ is true at every
world in every model.

(In the last section of the previous chapter, I mentioned that for some
applications of modal logic, we don't want $\Box p \to p$ to be valid. In the
next chapter, we will see how this can be achieved by a slight tweak to the
definitions of the present chapter.)

How about, say, $\Box p \to \Box \Box p$? If something is necessary, is it
necessarily necessary? Our semantics says yes. Let $w$ be an arbitrary world in
an arbitrary model. If $\Box p$ is false at $w$, then $\Box p \to \Box\Box p$ is
true at $w$, by clause (e) of definition \ref{def:basicsemantics}. Suppose then
that $\Box p$ is true at $w$. In that case, $p$ is true at all worlds, by clause
(g) of definition \ref{def:basicsemantics}. And then $\Box p$ is true at all
worlds, again by clause (g). And so $\Box\Box p$ is also true at all worlds, by
clause (g). So whenever $\Box p$ is true at a world in a model, then so is
$\Box\Box p$. By clause (e) of definition \ref{def:basicsemantics}, it follows
that $\Box p \to \Box\Box p$ is true at every world in every model.

\begin{exercise}
  Show that $\Box p \to \Diamond p$ is valid. 
\end{exercise}
\begin{solution}
  By definition \ref{def:valid}, a sentence is valid iff it is true at every
  world in every model. Suppose for reductio that $\Box p \to \Diamond p$ is
  false at some world $w$ in some model. By definition \ref{def:basicsemantics},
  $\Box p$ is then true at $w$ and $\Diamond p$ false. But if $\Diamond p$ is
  false at $w$ then (by definition \ref{def:basicsemantics}) $p$ is false at
  every world in the model. And then $\Box p$ isn't true at $w$ (by definition
  \ref{def:basicsemantics}). Contradiction.
\end{solution}

There is a shorter way to show that $\Box p \to \Box\Box p$ is valid. Definition
\ref{def:basicsemantics} entails that if a sentence starts with a modal
operator, then its truth-value never varies from world to world. For example, if
$\Diamond p$ is true at some world $w$ in some model, then $\Diamond p$ is true
at all worlds in the model. It follows that if a sentence starts with a modal
operator, then its truth-value doesn't change if you stack further modal
operators in front. If $\Diamond p$ is true at a worlds in a model, then so
are $\Box\Diamond p$ and $\Diamond \Diamond p$.

This means that any sentence that begins with a sequence of modal operators is
equivalent to the same sentence with all but the last operator removed.
$\Diamond\Box\Box\Diamond\Diamond p$ is equivalent to $\Diamond p$. $\Box\Box p$
is equivalent to $\Box p$. Since replacing logically equivalent sentences inside
a larger sentence never affects the larger sentence's truth-value at any world,
$\Box\Box p \to \Box p$ is equivalent to $\Box p \to \Box p$. And this is
obviously valid.

Do not conflate the concepts of necessity and validity. Necessity means truth at
all worlds (or so we currently assume). Validity means truth at all worlds
\emph{in all models}. Whether an $\L_M$ sentence is necessary generally varies
from model to model. In a model whose interpretation function makes $p$ true at
all worlds, $p$ is necessary insofar as $\Box p$ is true at all worlds. In a
model whose interpretation function makes $p$ false at some world, $\Box p$ is
false at all worlds. Validity, by contrast, is not relative to a model. The
sentence $p$ is definitely not valid. The sentence $\Box p \to p$ is.

\begin{exercise}
  Show that if a sentence $A$ is valid, then so is $\Box A$.
\end{exercise}
\begin{solution}
  Suppose $A$ is valid -- true at all worlds in all models (definition
  \ref{def:valid}). It follows that in any given model, $A$ is true at every
  world. By definition \ref{def:basicsemantics}, it follows that $\Box A$ is
  true at every world in any model.
\end{solution}

Here is an example of an invalid sentence:
\[
  \Box(p \lor q) \to (\Box p \lor \Box q)
\]
How could we show that this is invalid? By definition \ref{def:valid}, a
sentence is valid iff it is true at all worlds in all models. So we have to find
some model in which there is some world at which the sentence is false. Such a
model is called a \textbf{countermodel} for the sentence. The following model is
a countermodel for the sentence above, as you should verify with the help of
definition \ref{def:basicsemantics}.
%
\begin{gather*}
  W = \{ w,v \}\\
  V(p) = \{ w \}\\
  V(q) = \{ v \}
\end{gather*}
%
I haven't explained at which worlds sentence letters other than $p$ and $q$ are
true, because it doesn't matter.

% In general, to specify a countermodel for a sentence $A$, you have to specify
% two things: a set $W$ of worlds, and an interpretation function $V$ that assigns
% a subset of $W$ to all the sentence letters in $A$.

\begin{exercise}
  Show that $p \to \Box p$ is invalid (and thus $p \not\models \Box p$), by
  giving a countermodel. Explain why this doesn't contradict the previous
  exercise.
\end{exercise}
\begin{solution}
  $p\to \Box p$ is false at world $w$ in the model(s) given by
  $W = \{ w, v \}, V(p) = \{ w \}$.

  This shows that the \emph{truth} of $p$ (at a world in a model) does not
  entail the truth of $\Box p$ (at the world in the model), even though the
  \emph{validity} of $p$ entails the validity of $\Box p$, as per the previous
  exercise.
\end{solution}

% \begin{exercise}
%   The sentences that are true at a world $w$ in a model $M$ contain
%   information not just about $w$ but also about other worlds in
%   $M$. For example, if $\Box p$ is true at $w$, we know that $w$ is
%   true at all worlds, and if $\neg p$ and $\Diamond p$ are both true
%   at $w$, we know that $p$ is true at some other world. Question: do
%   the sentences true at $w$ completely determine the model $M$? If
%   not, give an example of two worlds $w_1$ and $w_2$ in two models
%   $M1$, $M2$ that verify the same sentences even though the models are
%   not isomorphic.
% \end{exercise}

\begin{exercise}
  Show that for any sentences $A$, $B$, if $\models A \to B$, then also
  $\models \Box A \to \Box B$.
\end{exercise}
\begin{solution}
  Assume $\models A \to B$. Then there is no world in any model at which $A$ is
  true and $B$ is false. So if $A$ is true at every world in a model, then $B$
  is also true at every world in the model. It follows that $\Box A \to \Box B$
  is true at every world in every model.
\end{solution}

Earlier in this section, I showed that $\Box p \to p$ and
$\Box p \to \Box\Box p$ are valid. The arguments I gave easily generalise to
other sentences in place of $p$. All instances of $\Box A \to A$ and
$\Box A \to \Box\Box A$ are valid.

You may remember these schemas from section \ref{sec:systems}. There I defined
the system S5 by stipulating that it contains all instances of the following
schemas:
%
\begin{principles}
  \pri{Dual}{\neg\Diamond A \leftrightarrow \Box\neg A}\\
  \pri{T}{\Box A \to A}\\
  \pri{K}{\Box(A\to B) \to (\Box A \to \Box B)}\\
  \pri{4}{\Box A \to \Box \Box A}\\
  \pri{5}{\Diamond A \to \Box \Diamond A}
\end{principles}
%
All instances of these schemas are valid by the definitions of the present
chapter.

% \begin{exercise}
%   Show that all instances of the \textbf{D}-schema $\Box A \to \Diamond A$ are
%   valid by the definitions from sections \ref{sec:basicmodels} and
%   \ref{sec:redefining}.
% \end{exercise}
% \begin{solution}
%   By definition \ref{def:valid}, a sentence is valid iff it is true at every
%   world in every model. Suppose for reductio that some instance of
%   $\Box A \to \Diamond A$ is false at some world $w$ in some model. By
%   definition \ref{def:basicsemantics}, $\Box A$ is then true at $w$ and
%   $\Diamond A$ false. But if $\Diamond A$ is false at $w$ then (by definition
%   \ref{def:basicsemantics}) $A$ is false at every world in the model, including
%   $w$. And then $\Box A$ isn't true at $w$ (by definition
%   \ref{def:basicsemantics}). Contradiction.
% \end{solution}

% Exercise: $\Box(A \lor B) \leftrightarrow (\Box A \lor \Box B)$ is
% invalid, but one direction is valid; which one. Dito for the
% dual. [MLOM 18]

I also specified two rules for S5. The first says that any truth-functional
consequence of any sentences in S5 is itself in S5. The second says that
whenever a sentence $A$ is in S5, then so is $\Box A$. As we will show chapter
\ref{ch:proofs}, these rules preserve validity (as defined in the previous
section). Indeed, you will learn how to show that the sentences that come out as
valid by our present definitions are precisely the sentences in S5.

You may pause a moment to ponder how this could be done. In the meantime, let's
prove a simpler fact to which I have appealed above (as well as on page
\pageref{claim:replacement} in the previous chapter): that replacing logically
equivalent sentences inside a larger sentence never affects the larger
sentence's truth-value at any world.

\begin{observation}{replacement-theorem}
  If $A$ is an $\L_{M}$-sentence and $A'$ results from $A$ by replacing a
  subsentence of $A$ with a logically equivalent sentence, then $A$ and $A'$ are
  logically equivalent.
\end{observation}
\begin{proof}
  \emph{Proof.} Remember that two sentences are logically equivalent if each
  entails the other. By definition \ref{def:basicconsequence}, this means that
  the two sentences are true at the same worlds in every model.

  Now let $A$ be an arbitrary $\L_{M}$-sentence and assume that $A'$ results
  from $A$ by replacing a subsentence of $A$ with a logically equivalent
  sentence. To show that $A$ and $A'$ are equivalent, we first show that this
  holds for the special case where $A$ is a sentence letter. Then we consider
  different ways in which $A$ might be built up from simpler sentences and show
  that \emph{if the observation holds for those simpler sentences}, then it also
  holds for $A$ itself.

  So assume that $A$ is a sentence letter. In that case, $A$ has no sentences as
  proper parts. The observation is vacuously true. (There is no way of turning
  $p$ into a non-equivalent sentence by replacing a subsentence within $p$.)

  Next we assume that $A$ is a complex sentence and that the observation holds
  for all simpler sentences.

  To begin, assume that $A$ is the negation of another sentence $B$. So $A$ is
  $\neg B$ and $A'$ is $\neg B'$ for some sentence $B'$ that is either
  equivalent to $B$ (if $B$ is the subsentence of $A$ that has been replaced to
  yield $A'$) or that results from $B$ by replacing a subsentence within $B$ by
  an equivalent sentence (if the subsentence of $A$ that has been replaced to
  yield $A'$ isn't $B$). In the latter case, our assumption that the observation
  holds for sentences simpler than $A$ implies that $B$ and $B'$ are equivalent.
  Either way, then, $B$ and $B'$ are logically equivalent: they are true at the
  same worlds in every model. By clause (b) of definition
  \ref{def:basicsemantics}, it follows that $A$ and $A'$ are also true at the
  same worlds in every model.

  Essentially the same reasoning applies in the case where $A$ is a conjunction
  $B \land C$, a disjunction $B \lor C$, a conditional $B \to C$, a
  biconditional $B \leftrightarrow C$, a box sentence $\Box B$, and a diamond
  sentence $\Diamond B$. I won't bore you by going through all of them. Here is
  the case for $\Box B$.
  
  Assume that $A$ has the form $\Box B$. So $A$ is $\Box B$ and $A'$ is
  $\Box B'$ for some sentence $B'$ that is equivalent to $B$ (by the same
  reasoning as before). By clause (g) of definition \ref{def:basicsemantics} it
  follows that $A$ and $A'$ are also equivalent. \qed

\end{proof}

The style of proof I have used here is called an \textbf{induction on
  complexity}. It is widely used when reasoning about formal languages. In
general, if you want to show that every sentence of a language has a certain
property, it suffices to show that (i) all atomic sentences in the language have
the property, and (ii) \emph{if} all sentences that are simpler than a sentence
have the property then so does the sentence itself. (In this context, the
assumption that all simpler sentences have the property is called the
\emph{induction hypothesis}.)


\section{Trees}\label{sec:trees}

I will now introduce a streamlined method for working through definition
\ref{def:basicsemantics} to check whether a sentence is valid: the method of
\textbf{analytic tableau} or \textbf{tree proofs}. (You may be familiar with
this method for non-modal logic. If so, good. If not, no problem.)

The tree method is a method not just for proving validity, but also for finding
countermodels. It is best introduced by example.

Let's try to find a countermodel for $\Diamond p \to \Box p$. That is, we want
to construct a model in which there is some world $w$ at which
$\Diamond p\to \Box p$ is false. We start our search by assuming that the
\emph{negation} of $\Diamond p \to \Box p$ is \emph{true} at $w$. We write this
down as follows.

\begin{center}
  \Tree{%
    \nnode{18}{1.}{$\neg(\Diamond p \to \Box p)$}{w}{(Ass.)}%
  }
\end{center}
%
`1.' and `(Ass.)' are for book-keeping; `Ass.'\ is short for `Assumption', since
we're \emph{assuming} that $\neg(\Diamond p \to \Box p)$ is true at $w$. Now we
unfold this assumption in accordance with definition \ref{def:basicsemantics}.
The definition tells us that a conditional $A\to B$ is false at a world $w$ iff
the antecedent $A$ is true at $w$ and the consequent $B$ is false at $w$. So the
assumption on line 1 implies that $\Diamond p$ is true at $w$ and that $\Box p$
is false at $w$. We expand our ``tree'' (or ``tableau'') by adding these consequences.

\begin{center}
  \tree{%
    \nnodeticked{18}{1.}{$\neg(\Diamond p \to \Box p)$}{w}{(Ass.)}\\
    \nnode{18}{2.}{$\Diamond p$}{w}{(1)}\\
    \nnode{18}{3.}{$\neg\Box p$}{w}{(1)}%
  }
\end{center}
%
I have ticked off line 1 (with `$\checkmark$') to mark that we won't need to
look at it again. All the information in line 1 is contained in lines 2 and 3.
The parenthetical `(1)' at lines 2 and 3 reminds us that these assumptions are
derived from line 1.

We continue drawing out further consequences. What does the truth of
$\Diamond p$ at $w$ imply for the subsentence $p$? By definition
\ref{def:basicsemantics}, there must be some world -- let's call it $v$ -- at
which $p$ is true.

\begin{center}
  \tree{%
    \nnodeticked{18}{1.}{$\neg(\Diamond p \to \Box p)$}{w}{(Ass.)}\\
    \nnodeticked{18}{2.}{$\Diamond p$}{w}{(1)}\\
    \nnode{18}{3.}{$\neg\Box p$}{w}{(1)}\\
    \nnode{18}{4.}{$p$}{v}{(2)}%
  }
\end{center}

Line 3 claims that $\Box p$ is false at $w$. By definition
\ref{def:basicsemantics}, $\Box p$ is true at $w$ iff $p$ is true at all worlds.
So if $\Box p$ is false at $w$, there must be some world at which $p$ is false.
Let's introduce such a world, naming it $u$. Our tree looks as follows.

\begin{center}
  \tree{%
    \nnodeticked{18}{1.}{$\neg(\Diamond p \to \Box p)$}{w}{(Ass.)}\\
    \nnodeticked{18}{2.}{$\Diamond p$}{w}{(1)}\\
    \nnodeticked{18}{3.}{$\neg\Box p$}{w}{(1)}\\
    \nnode{18}{4.}{$p$}{v}{(2)}\\
    \nnode{18}{5.}{$\neg p$}{u}{(3)}%
  }
\end{center}

Now the only unprocessed lines are assumptions about sentence letters and
negations of sentence letters. Sentence letters don't have (non-trivial)
subsentences, so we can't use definition \ref{def:basicsemantics} to further
break down 4 or 5. The tree is complete. We have found a countermodel for
$\Diamond p \to \Box p$.

Let's read off the countermodel. There are three worlds in our tree: $w$, $v$,
and $u$. So $W = \{ w, u, v \}$. By line 4, $p$ is true at $v$. By line 5, $p$
is false at $u$. We don't know whether $p$ is true or false at $w$, and it
doesn't matter -- otherwise the tree would say. Let's assume that
$V(p) = \{ v \}$. As you can verify, $\Diamond p\to \Box p$ is indeed false at
world $w$ in this model.

One more example, before I state the general rules. Let's try to find a
countermodel for $\Box(p\to q) \to (p \to \Box q)$. That's another conditional,
so we begin as before.
\begin{center}
  \tree[3]{%
    & \nnodeticked{32}{1.}{$\neg(\Box(p\to q) \to (p \to \Box q))$}{w}{(Ass.)} &&\\
    & \nnode{32}{2.}{$\Box(p\to q)$}{w}{(1)} &&\\
    & \nnode{32}{3.}{$\neg(p \to \Box q)$}{w}{(1)} && \\
  }
\end{center}
%
Line 1 assumes that the negation of the conditional is true at some world $w$.
Lines 2 and 3 break down this assumption, using the fact that $\neg (A \to B)$
is true (at a world) iff $A$ is true and $B$ false. We could deal with line 2
next, but it's better to ignore it for the moment and process 3 first, which is
yet another negated conditional.

\begin{center}
  \tree[3]{%
    & \nnode{32}{4.}{$p$}{w}{(3)} && \\
    & \nnode{32}{5.}{$\neg \Box q$}{w}{(3)} && \\
  }
\end{center}

Line 5 tells us that $\Box q$ is false at $w$. We can infer that there is a
world -- call it $v$ -- at which $q$ is false.

\begin{center}
  \tree[3]{%
    & \nnode{32}{6.}{$\neg q$}{v}{(5)} && \\
  }
\end{center}

Now we need to return to line 2. What can we infer from the hypothesis that
$\Box(p\to q)$ is true at $w$ about the subsentence $p \to q$? By definition
\ref{def:basicsemantics}, $p \to q$ must be true at \emph{every} world. So, in
particular, $p\to q$ must be true at $w$. Let's write that down. We'll add
another line for $v$ later, so we don't check off node 2.

\begin{center}
  \tree[3]{%
    & \nnode{32}{7.}{$p\to q$}{w}{(2)} && \\
  }
\end{center}

If you are used to proofs in the natural deduction style, you may now be tempted
to apply \emph{modus ponens} and infer that $q$ is true at $w$, from lines 4 and
7. In the tree method, however, we try not to draw inferences from multiple
premises. We simply look at any lines that can still be processed and check what
definition \ref{def:basicsemantics} tells us about the immediate subsentences of
the sentence on that line. So we process line 7 without looking at line 4.

What can we infer from the truth of $p\to q$ at $w$ about the subsentences $p$
and $q$? By definition \ref{def:basicsemantics}, $p \to q$ is true at $w$ if $p$
is false at $w$ \emph{or} $q$ is true at $w$. We have to keep track of both
possibilities. Our (upside down) tree will branch. Here is the full tree at its
present stage.

\begin{center}
  \tree[3]{%
    & \nnodeticked{32}{1.}{$\neg(\Box(p\to q) \to (p \to \Box q))$}{w}{(Ass.)} &&\\
    & \nnode{32}{2.}{$\Box(p\to q)$}{w}{(1)} &&\\
    & \nnodeticked{32}{3.}{$\neg(p \to \Box q)$}{w}{(1)} && \\
    & \nnode{32}{4.}{$p$}{w}{(3)} && \\
    & \nnodeticked{32}{5.}{$\neg \Box q$}{w}{(3)} && \\
    & \nnode{32}{6.}{$\neg q$}{v}{(5)} && \\
    & \bnodeticked{32}{7.}{$p\to q$}{w}{(2)} && \\
    &&& \\
    \nnodeclosed{8}{8.}{$\neg p$}{w}{(7)} && \nnode{8}{9.}{$q$}{w}{(7)} & \\
  }
\end{center}

So far, I have called the numbered items on a tree `lines'. The proper term is
\textbf{nodes}. Since nodes 8 and 9 are visually on the same line, it would be
confusing to call them lines. While we're at it, a \textbf{branch} of a tree is
series of nodes that extends from the top (or ``root'') node all the way down to
a node below which there is no other node. The present tree has two branches,
both of which contain 8 nodes.

What does this tree tell us? Remember that our aim is to construct a model in
which the sentence at node 1 is true at world $w$. At this stage, the tree tells
us that there are two worlds $w$ and $v$ in the model; nodes 4 and 5 tell us
something about the interpretation function in the model: $p$ is true at $w$,
$q$ is false at $v$. After node 7, the tree branches, meaning that there are two
ways of extending the model we have construed so far. On the left branch, we
assume that $p$ is false at $w$. On the right branch, we assume that $q$ is true
at $w$. But hold on. We already know that $p$ is true at $w$ (from node 4).
There's no model in which $p$ is both true and false at $w$. So the possibility
explored on the left branch is a dead-end. it doesn't lead to a countermodel.
That's why I've \emph{closed} the left branch by drawing a cross below node 8.

We continue on the right-hand branch. Here we expand node 2 again, this time for
world $v$, which leads to another branching.

\begin{center}
  \tree[3]{%
    & \nnodeticked{32}{1.}{$\neg(\Box(p\to q) \to (p \to \Box q))$}{w}{(Ass.)} &&\\
    & \nnode{32}{2.}{$\Box(p\to q)$}{w}{(1)} &&\\
    & \nnodeticked{32}{3.}{$\neg(p \to \Box q)$}{w}{(1)} && \\
    & \nnode{32}{4.}{$p$}{w}{(3)} && \\
    & \nnodeticked{32}{5.}{$\neg \Box q$}{w}{(3)} && \\
    & \nnode{32}{6.}{$\neg q$}{v}{(5)} && \\
    & \bnodeticked{32}{7.}{$p\to q$}{w}{(2)} && \\
    &&& \\
    \nnodeclosed{8}{8.}{$\neg p$}{w}{(7)} && \nnode{13}{9.}{$q$}{w}{(7)} & \\
    && \bnode{13}{10.}{$p\to q$}{v}{(2)} & \\
    &&& \\
    & \nnode{8}{11.}{$\neg p$}{v}{(10)} & & \nnodeclosed{8}{12.}{$q$}{v}{(10)} 
  }
\end{center}
%
On the right-most branch, $q$ is true at $v$ (by node 12) but also false at $v$
(by node 6), so that branch is closed. But the middle possibility is still open,
and there are no more assumptions to unfold. We have found a countermodel.

The countermodel is given by all the assumptions \emph{on the middle branch},
the one that remained open. (The other branches were dead-ends and can be
ignored.) We have two worlds, $W = \{ w,v \}$. The interpretation function $V$
makes $p$ true at $w$ (node 4) and false at $v$ (node 11); $q$ is also true at
$w$ (node 9) and false at $v$ (node ). Again, you may verify that the sentence
on node 1 is true at world $w$ in this model.

Now for the general rules.

In order to find a countermodel for a sentence $A$ with the help of the tree
method, you always begin by assuming that the \emph{negation} of $A$ is true at
world $w$:
%
\begin{center}
  \tree{%
    \nnode{10}{1.}{$\neg A$}{w}{(Ass.)}
    }
\end{center}
%
You then expand this node, and every new node that appears on the tree, until no
more nodes can be expanded.

To expand a node with a non-negated sentence, you consider what the truth of the
sentence at the node's world implies for the truth-value of the sentence's
immediate parts. The result may be added to the end of any open branch
containing the node.

(The immediate parts of a sentence of the form $A\land B$, $A \lor B$,
$A \to B$, or $A \leftrightarrow B$ are the corresponding sentences $A$ and $B$;
the only immediate part of $\Box A$, $\Diamond A$, and $\neg A$ is $A$.)

To expand a node with a negation $\neg A$, you consider what the falsity of the
relevant sentence $A$ at the node's world implies for the immediate parts of
$A$. The result may again be added to the end of any open branch containing the
node.

The following diagrams summarize how the different kinds of nodes are expanded.

\bigskip

% Perhaps I shouldn't use 'nu' and 'omega' in the schematic tree rules? students
% just thought they mean w and v, and asked if the box rule can be applied with
% an old world other than v. Also might need to clarify that v can be the same
% world as w.
\begin{minipage}{0.33\textwidth}\centering
\tree{
  \dotbelownode{12}{}{$A \land B$}{\omega}{}\\
  \\
  \nnode{12}{}{$A$}{\omega}{}\\
  \nnode{12}{}{$B$}{\omega}{}
}
\end{minipage}
\begin{minipage}{0.33\textwidth}\centering
\tree{
  & \dotbelowbnode{12}{}{$A \lor B$}{\omega}{} &\\
  && \\
  && \\
  \nnode{8}{}{$A$}{\omega}{} && \nnode{8}{}{$B$}{\omega}{}
}
\end{minipage}
\begin{minipage}{0.33\textwidth}\centering
\tree{
  & \dotbelowbnode{12}{}{$A \to B$}{\omega}{} &\\
  && \\
  && \\
  \nnode{8}{}{$\neg A$}{\omega}{} && \nnode{8}{}{$B$}{\omega}{}
}
\end{minipage}

\vspace{10mm}

\begin{minipage}{0.33\textwidth}\centering
\tree{
  & \dotbelowbnode{12}{}{$A \leftrightarrow B$}{\omega}{} & \\
  & \\
  & \\
  \nnode{8}{}{$A$}{\omega}{} & & \nnode{8}{}{$\neg A$}{\omega}{} & & \\
  \nnode{8}{}{$B$}{\omega}{} & & \nnode{8}{}{$\neg B$}{\omega}{} & & \\
}
\end{minipage}
\begin{minipage}{0.33\textwidth} \centering
\tree{
  \dotbelownode{8}{}{$\Box A$}{\omega}{}\\
  \\
  \nnode{8}{}{$A$}{\nu}{}\\
  \Kk[8]{0}{\color{red}$\uparrow$}\\
  \Kk[8]{0}{\color{red}\small old}
}
\end{minipage}
\begin{minipage}{0.33\textwidth}\centering
\tree{
  \dotbelownode{8}{}{$\Diamond A$}{\omega}{}\\
  \\
  \nnode{8}{}{$A$}{\nu}{}\\
  \Kk[8]{0}{\color{red}$\uparrow$}\\
  \Kk[8]{0}{\color{red}\small new}
}
\end{minipage}

\vspace{10mm}

\begin{minipage}{0.33\textwidth}\centering
\tree{
  & \dotbelowbnode{15}{}{$\neg(A \land B)$}{\omega}{} &\\
  && \\
  && \\
  \nnode{8}{}{$\neg A$}{\omega}{} && \nnode{8}{}{$\neg B$}{\omega}{}
}
\end{minipage}
\begin{minipage}{0.33\textwidth}\centering
\tree{
  \dotbelownode{15}{}{$\neg(A \lor B)$}{\omega}{}\\
  \\
  \nnode{15}{}{$\neg A$}{\omega}{}\\
  \nnode{15}{}{$\neg B$}{\omega}{}
}
\end{minipage}
\begin{minipage}{0.33\textwidth}\centering
\tree{
  \dotbelownode{15}{}{$\neg(A \to B)$}{\omega}{}\\
  \\
  \nnode{15}{}{$A$}{\omega}{}\\
  \nnode{15}{}{$\neg B$}{\omega}{}
}
\end{minipage}

\vspace{10mm}

\begin{minipage}{0.33\textwidth}\centering
\tree{
  & \dotbelowbnode{15}{}{$\neg(A \leftrightarrow B)$}{\omega}{} & \\
  & \\
  & \\
  \nnode{8}{}{$A$}{\omega}{} & & \nnode{8}{}{$\neg A$}{\omega}{} & & \\
  \nnode{8}{}{$\neg B$}{\omega}{} & & \nnode{8}{}{$B$}{\omega}{} & & \\
}
\end{minipage}
\begin{minipage}{0.33\textwidth}\centering
\tree{
  \dotbelownode{10}{}{$\neg \Box A$}{\omega}{}\\
  \\
  \nnode{10}{}{$\neg A$}{\nu}{}\\
  \Kk[10]{0}{\color{red}$\uparrow$}\\
  \Kk[10]{0}{\color{red}\small new}
}
\end{minipage}
\begin{minipage}{0.33\textwidth}\centering
\tree{
  \dotbelownode{10}{}{$\neg \Diamond A$}{\omega}{}\\
  \\
  \nnode{10}{}{$\neg A$}{\nu}{}\\
  \Kk[10]{0}{\color{red}$\uparrow$}\\
  \Kk[10]{0}{\color{red}\small old}
}
\end{minipage}

\vspace{10mm}

\begin{minipage}{0.33\textwidth}\centering
\tree{%
  \dotbelownode{12}{}{$\neg\neg A$}{\omega}{}\\
  \\
  \nnode{12}{}{$A$}{\omega}{}
}
\end{minipage}

\vspace{5mm}

If a branch of a tree contains a sentence $A$ as well as its negation $\neg A$,
for the same world $\omega$, then the branch is \emph{closed} with an {\sffamily
  x} at the bottom.

I've used `$\omega$' and `$\nu$' as placeholders for arbitrary world variables.

The rule for $\Box A$ says that from the assumption that $\Box A$ is true at a
world $\omega$ you may infer that $A$ is true at any world $\nu$ \emph{that
  already occurs on the branch to which the new node is added}. You're not
allowed to introduce a new world variable (`$v$', `$u$', etc.) when expanding
$\Box A$ nodes. The same is true for $\neg \Diamond A$ nodes (which by duality
means the same as $\Box \neg A$). When you expand a $\Diamond A$ node (or a
$\neg \Box A$ node), by contrast, you must introduce a new world variable.

Nodes of type $\Box A$ and $\neg \Diamond A$ can be expanded several times, once
for every world variable on any branch containing the node.

If you have expanded a node that is not of type $\Box A$ or $\neg \Diamond A$,
and you have added the new nodes to every open branch containing the node, then
you can tick off the node. You don't need to look at it again. Nodes of type
$\Box A$ and $\neg \Diamond A$ nodes are never ticked off.

If no more rules can be applied, the tree is complete. Any open branch on a
complete tree defines a countermodel for the target sentence.
%
\begin{exercise}
  Use the tree method to find countermodels for the following
  sentences. (Spell out the countermodel, in addition to drawing the tree.)
  \begin{exlist}
  \item $p\to q$ 
  \item $p \to \Box(p \lor q)$
  \item $\Box p \lor \Box \neg p$
  \item $\Diamond(p \to q) \to (\Diamond p \to \Diamond q)$
  % This is invalid, but one can only read off a counterexample once all rules
  % have been applied. Also emphasize that countermodels must come from /one/
  % branch only. 
  \item $\Box \Diamond p \to p$ % infinite
  \end{exlist}
\end{exercise}
\begin{solution}
  \medskip
    
  \begin{sollist}
  \item Target: $p\to q$
  
    \tree{
      &\nnode{15}{1.}{$\neg(p \to q)$}{w}{(Ass.)}&\\
      &\nnode{15}{2.}{$p$}{w}{(1)}&\\
      &\nnode{15}{3.}{$\neg q$}{w}{(1)}&
    }

    Countermodel: $W = \{ w \}, V(p) = \{ w \}, V(q) = \emptyset$.
    
    \medskip
  \item Target: $p \to \Box(p \lor q)$

    \tree{
      \nnode{25}{1.}{$\neg (p \to \Box(p \lor q))$}{w}{(Ass.)}\\
      \nnode{25}{2.}{$p$}{w}{(1)}\\
      \nnode{25}{3.}{$\neg \Box(p \lor q)$}{w}{(1)}\\
      \nnode{25}{4.}{$\neg(p \lor q)$}{v}{(3)}\\
      \nnode{25}{5.}{$\neg p$}{v}{(4)}\\
      \nnode{25}{5.}{$\neg q$}{v}{(4)}
    }

    Countermodel: $W = \{ w, v \}, V(p) = \{ w \}, V(q) = \emptyset$.

  \medskip
  \item Target: $\Box p \lor \Box \neg p$
    
    \tree{
      \nnode{25}{1.}{$\neg (\Box p \lor \Box \neg p)$}{w}{(Ass.)}\\
      \nnode{25}{2.}{$\neg \Box p$}{w}{(1)}\\
      \nnode{25}{3.}{$\neg \Box \neg p$}{w}{(1)}\\
      \nnode{25}{4.}{$\neg p$}{v}{(2)}\\
      \nnode{25}{5.}{$\neg \neg p$}{u}{(3)}\\
      \nnode{25}{6.}{$p$}{u}{(5)}
    }

    Countermodel: $W = \{ w,v,u \}, V(p) = \{ u \}$.
    
    \medskip
  \item Target: $\Diamond(p \to q) \to (\Diamond p \to \Diamond q)$

    \tree[3]{
      &\nnode{35}{1.}{$\neg(\Diamond(p \to q) \to (\Diamond p \to \Diamond q))$}{w}{(Ass.)}&\\
      &\nnode{35}{2.}{$\Diamond(p \to q)$}{w}{(1)}&\\
      &\nnode{35}{3.}{$\neg (\Diamond p \to \Diamond q)$}{w}{(1)}&\\
      &\nnode{35}{4.}{$\Diamond p$}{w}{(3)}&\\
      &\nnode{35}{5.}{$\neg \Diamond q$}{w}{(3)}&\\
      &\nnode{35}{6.}{$p\to q$}{v}{(2)}&\\
      &\nnode{35}{7.}{$p$}{u}{(4)}&\\
      &\nnode{35}{8.}{$\neg q$}{w}{(5)}&\\
      &\nnode{35}{9.}{$\neg q$}{v}{(5)}&\\
      &\bnode{35}{10.}{$\neg q$}{u}{(5)}&\\
      &&\\
      \nnode{10}{11.}{$\neg p$}{v}{(6)} && \nnodeclosed{10}{12.}{$q$}{v}{(6)}
    }

    Countermodel: $W = \{ w,v,u \}, V(p) = \{ u \}, V(q) = \emptyset$.
    
    \medskip
  \item $\Box \Diamond p \to p$
  
    \tree{
      &\nnode{20}{1.}{$\neg(\Box\Diamond p \to p))$}{w}{(Ass.)}&\\
      &\nnode{20}{2.}{$\Box \Diamond p$}{w}{(1)}&\\
      &\nnode{20}{3.}{$\neg p$}{w}{(1)}&\\
      &\nnode{20}{4.}{$\Diamond p$}{w}{(2)}&\\
      &\nnode{20}{5.}{$p$}{v}{(4)}&\\
      &\nnode{20}{6.}{$\Diamond p$}{v}{(2)}&\\
      &\nnode{20}{7.}{$p$}{u}{(6)}&\\
      &\nnode{20}{8.}{$\Diamond p$}{u}{(2)}&\\
      &\nnode{20}{9.}{$p$}{t}{(8)}&\\
      &\nnode{20}{}{\vdots}{}{}&
  }

  \medskip\noindent The tree grows forever. The target sentence isn't valid, but
  the tree method only gives us an infinite countermodel. In such a case, it may
  be useful to read off a model from an incomplete version of the tree and
  manually check whether it is a genuine countermodel. The model determined by
  the first five nodes of the present tree is $W = \{ w,v \}, V(p)=\{ v \}$,
  and you can confirm that it is a countermodel to the target sentence.

  If you read off a model from an \emph{incomplete} tree, you can't be sure that
  it is a countermodel for the target sentence. You must always double-check!
   
  \end{sollist}
  
\end{solution}

What if all branches on a tree close? Then there is no countermodel for the
target sentence. If there is no countermodel for a sentence, then the sentence
is valid. This is how the tree method is used to show that sentences are valid.

The following tree shows that
$\Diamond \neg p \leftrightarrow \neg \Box p$ is valid. Make sure you understand
each step. (I've omitted the check marks since these are only useful during the
construction phase.)

\begin{center}
  \tree[3]{
    & \bnode{22}{1.}{$\neg(\Diamond \neg p \leftrightarrow \neg \Box p)$}{w}{(Ass.)} & \\
  && \\
  \nnode{12}{2.}{$\Diamond \neg p$}{w}{(1)} && \nnode{12}{4.}{$\neg \Diamond\neg p$}{w}{(1)}  \\
  \nnode{12}{3.}{$\neg \neg \Box p$}{w}{(1)} && \nnode{12}{5.}{$\neg \Box p$}{w}{(1)}  \\
  \nnode{12}{6.}{$\Box p$}{w}{(3)} && \nnode{12}{9.}{$\neg p$}{v}{(5)}  \\
  \nnode{12}{7.}{$\neg p$}{v}{(2)} && \nnodeclosed{12}{10.}{$\neg\neg p$}{v}{(4)}  \\
  \nnodeclosed{12}{8.}{$p$}{v}{(6)} &&  \\
}
\end{center}

A similar tree could obviously be drawn for
$\Diamond \neg q \leftrightarrow \neg \Box q$, and for any other formula of the
form $\Diamond \neg A \leftrightarrow \neg \Box A$: we would simply replace each
occurrence of $p$ on the tree with $A$.

To show that all instances of a schema are valid, we can also directly draw
\textbf{schematic trees} in which we use schematic variables `$A$', `$B$', `$C$'
instead of sentence letters.

\begin{exercise}
  Use the tree method to show that all instances of the following schemas are valid.
   \begin{exlist}
   \item[\pr{K}] $\Box (A \to B) \to (\Box A \to \Box B)$
   \item[\pr{T}] $\Box A \to A$
   % \item[\pr{D}] $\Box A \to \Diamond A$ 
   \item[\pr{4}] $\Box A \to \Box \Box A$
   \item[\pr{5}] $\Diamond A \to \Box \Diamond A$
   % \item[\pr{G}] $\Diamond\Box A \to \Box \Diamond A$ 
   \end{exlist}
\end{exercise}
\begin{solution}
  You can enter the schemas at
  \href{https://www.umsu.de/trees/}{umsu.de/trees}.
  After entering a formula, tick the checkbox for `universal (S5)'.
  Alternatively, follow these links:
  \href{https://www.umsu.de/trees/#~8(A~5B)~5(~8A~5~8B)||universality}{\pr{K}}, \href{https://www.umsu.de/trees/#~8A~5A||universality}{\pr{T}},
  \href{https://www.umsu.de/trees/#~8A~5~9A||universality}{\pr{D}},
  \href{https://www.umsu.de/trees/#~8A~5~8~8A||universality}{\pr{4}},
  \href{https://www.umsu.de/trees/#~9A~5~8~9A||universality}{\pr{5}},
  \href{https://www.umsu.de/trees/#~9~8A~5~8~9A||universality}{\pr{G}}.
\end{solution}

\begin{exercise}
  For each of the following sentences, either show that it is valid or
  give a countermodel to show that it is invalid:
  \begin{exlist}
  \item $p \to \Box\Diamond p$
  \item $\Diamond\Diamond p \to \Diamond p$
  %\item $\Box\Box p \to \Box p$
  \item $\Diamond(p \land q) \to (\Diamond p \land \Diamond q)$
  \item $(\Diamond p \land \Diamond q) \to \Diamond(p \land q)$
  %\item $\Diamond p \lor \Box\neg p$
  \item $\Diamond(p \lor q) \leftrightarrow (\Diamond p \lor \Diamond q)$
  \item $\Box\Diamond p \to \Diamond\Box p$
  % \item $(\Diamond p \to \Box q) \to (\Box p \to \Box q)$
  \end{exlist}
\end{exercise}
\begin{solution}
  (a), (b), (c), (e), and (g) are valid. You can find the trees at
  \href{https://www.umsu.de/trees/}{umsu.de/trees} (Remember to tick the checkbox for `universal (S5)') or by following these links:
  \href{https://www.umsu.de/trees/#p~5~8~9p||universality}{(a)},
\href{https://www.umsu.de/trees/#~9~9p~5~9p||universality}{(b)},
\href{https://www.umsu.de/trees/#~9(p~1q)~5(~9p~1~9q)||universality}{(c)},
\href{https://www.umsu.de/trees/#~9(p~2q)~4(~9p~2~9q)||universality}{(e)}.
% \href{https://www.umsu.de/trees/#(~9p~5~8q)~5(~8p~5~8q)||universality}{(g)}.
  
  (d) and (f) are invalid. Here is a tree for (d):

  \medskip
  \tree[3]{
    &\nnode{35}{1.}{$\neg ((\Diamond p \land \Diamond q) \to \Diamond(p \land q))$}{w}{(Ass.)}&&\\
    &\nnode{35}{2.}{$\Diamond p \land \Diamond q$}{w}{(1)}&&\\
    &\nnode{35}{3.}{$\neg \Diamond(p \land q)$}{w}{(1)}&&\\
    &\nnode{35}{4.}{$\Diamond p$}{w}{(2)}&&\\
    &\nnode{35}{5.}{$\Diamond q$}{w}{(2)}&&\\
    &\nnode{35}{6.}{$p$}{v}{(4)}&&\\
    &\nnode{35}{7.}{$q$}{u}{(5)}&&\\
    &\bnode{35}{8.}{$\neg (p \land q)$}{v}{(3)}&&\\
    &&&\\
    \nnodeclosed{9}{9.}{$\neg p$}{v}{(8)}&&\nnode{15}{10.}{$\neg q$}{v}{(8)}&\\
    &&\bnode{15}{11.}{$\neg(p \land q)$}{u}{(3)}&\\
    &&&\\
    &\nnode{15}{12.}{$\neg p$}{u}{(11)}&&\nnodeclosed{9}{13.}{$\neg q$}{u}{(11)}&\\
    &\bnode{15}{14.}{$\neg(p \land q)$}{w}{(3)}&&\\
    &&&\\
    \nnode{9}{15.}{$\neg p$}{w}{(14)}&&\nnode{9}{16.}{$\neg q$}{w}{(14)}&
  }

  \medskip\noindent
  We can choose either of the open branches to read off a countermodel. In fact, here we get the same countermodel no matter which open branch we choose:
$W = \{ w,v,u \}, V(p)=\{ v\}, V(q)=\{u\}$.
  \medskip

  A tree for (e) might begin like this:

  \tree[3]{
    &\nnode{25}{1.}{$\neg (\Box\Diamond p \to \Diamond\Box p)$}{w}{(Ass.)}&&\\
    &\nnode{25}{2.}{$\Box\Diamond p$}{w}{(1)}&&\\
    &\nnode{25}{3.}{$\neg \Diamond\Box p$}{w}{(1)}&&\\
    &\nnode{25}{4.}{$\Diamond p$}{w}{(2)}&&\\
    &\nnode{25}{5.}{$p$}{v}{(4)}&&\\
    &\nnode{25}{6.}{$\neg \Box p$}{w}{(3)}&&\\
    &\nnode{25}{7.}{$\neg p$}{u}{(6)}&&\\
    &\nnode{25}{8.}{$\Diamond p$}{v}{(2)}&&\\
    &\nnode{25}{9.}{$p$}{s}{(8)}&&\\
    &\nnode{25}{10.}{$\neg \Box p$}{v}{(3)}&&\\
    &\nnode{25}{11.}{$\neg p$}{t}{(10)}&&\\
    &\nnode{25}{}{\vdots}{}{}&&
  }

  \medskip\noindent The tree grows forever. The model determined by
  the first seven nodes of the present tree is $W = \{ w,v,u \}, V(p)=\{ v \}$.
  It is a countermodel to the target sentence.
    
\end{solution}

When constructing a tree, you often have a choice of which node to expand next.
In that case, a good idea is to start with any $\Diamond A$ or $\neg \Box A$
nodes. If there are none, choose a node of type $A \land B$, $\neg (A \lor B)$
or $\neg(A \to B)$. Choose a node of another type only if none of the above are
available. This heuristic often helps to keep trees small, but it is not part of
the official tree rules.

\begin{exercise}\label{ex:trees-with-premises}
  Can we use the tree method to show that some premises $A_1,\ldots,A_n$ entail
  a conclusion $B$? Can we use it to show that two sentences $A$ and $B$ are
  equivalent?
\end{exercise}
\begin{solution}
  By observation \ref{obs:semantic-deduction-theorem}, $A_{1},\ldots,A_{n}$
  entail $B$ iff $(A_1\land\ldots\land A_n) \to B$ is valid. To show that
  $A_{1},\ldots,A_{n}$ entail $B$ we could therefore draw a tree for
  $(A_1\land\ldots\land A_n) \to B$. In practice, we can save a few steps by
  starting the tree with multiple assumptions: one for each of the premises
  $A_{1},\ldots, A_{n}$, and one for the negated conclusion $\neg B$. (All of
  these are assumed to be true at world $w$.) If the tree closes,
  $A_1,\ldots,A_n$ entail $B$.

  To show that $A$ and $B$ are equivalent, we can draw a tree for
  $A \leftrightarrow B$.
\end{solution}


%%% Local Variables: 
%%% mode: latex
%%% TeX-master: "logic2.tex"
%%% End:

\chapter{Accessibility}\label{ch:accessibility}

\section{Variable modality}

I have given two kinds of interpretation for the box and the diamond. First I
said that we use $\Box A$ to express that $A$ is certain, or historically
necessary, or obligatory, etc. Then I said that $\Box A$ means that $A$ is true
at all possible worlds. These were not meant to be competing interpretations.
Rather, I have assumed that the ordinary concepts of certainty, historical
necessity, obligation, and so on, can be analysed -- at least to some
approximation -- as universal quantifiers over possible worlds.

Take historical necessity. Informally, $A$ is historically necessary if there is
nothing we can do that might render $A$ false. Let's say that a world is
\emph{open} if there is something we can do that might bring about that world
(in the sense that if we were to perform the relevant action, then we might live
in that world). Now, if you think about it, $A$ is historically necessary iff
$A$ is true at all open worlds. So we can use $\Box A$ to mean both that $A$ is
historically necessary and that $A$ is true at all possible worlds, where
`possible' means `open'.

Other flavours of modality are associated with other domains of worlds. Suppose
the box formalizes `it is obligatory that'. To a first approximation, $A$ is
obligatory (relative to some norms) iff it is true at all worlds at which the
norms are respected. (Think about it!) These worlds are also called
\emph{ideal}. So we can use $\Box A$ to mean both that $A$ is obligatory and
that $A$ is true at all possible worlds, where `possible' now means `ideal'.

\begin{exercise}
  Try to analyse the following concepts in terms of universal quantification over possible worlds of a suitable kind.
  \begin{exlist}
    \item It is physically necessary that $A$.
    \item We know that $A$.
    \item It is true that $A$.
  \end{exlist}
\end{exercise}
\begin{solution}
  \begin{sollist}
    \item One answer: $A$ is physically necessary iff $A$ is true at all worlds
    that are compatible with the laws of physics. A possibly better answer: $A$
    is physically necessary iff $A$ is true at all worlds that are compatible
    with the laws of physics and the current state of the universe.
    
    \item A simple, if somewhat uninformative answer: We know that $A$ iff $A$
    is true at all worlds that are compatible which our knowledge.

    \item It is true that $A$ iff $A$ is true at all worlds that are identical
    to the actual world.
  \end{sollist}
\end{solution}

Unfortunately, there are good reasons to think that this approach won't always
work. Remember that the possible-worlds semantics from the previous chapter
determines a particular logic: S5. And this logic is not appropriate for every
application of modal logic. In deontic logic, for example, we don't want the
schema
%
\begin{principles}
  \pri{T}{\Box A \to A}
\end{principles}
%
to be valid. We can easily conceive of scenarios in which $\Box p$ is true (on
some interpretation of $p$) even though $p$ is false.

The semantics from the previous chapter renders the \pr{T}-schema valid.
Whenever a sentence $\Box A$ is true at a world $w$ in a model then $A$ is true
at $w$ as well, because the box quantifies over all worlds, including $w$. To
make room for deontic logic, we need a semantics in which not all worlds in $W$
are among the ``possible'' worlds over which the modal operators quantify. Not
all worlds are ideal.

We might also want to allow that the worlds over which the modal operators
quantify depend on the world at which the relevant sentence is evaluated.
Perhaps you are obligated to do the dishes in worlds where you have promised to
do the dishes, but not in worlds where you haven't made the promise. Worlds in
which you don't do the dishes are then ideal relative to the second kind of
world, but not relative to the first.

This type of variability is also needed for other flavours of modality. Suppose
the box quantifies over all worlds that are compatible with our knowledge. Which
worlds are compatible with our knowledge depends on what we know. But we don't
always know what we know. Sometimes we believe that we know something, but don't
actually know it because it is false. We don't know it, without knowing that we
don't know it. \label{par:notB}Among the worlds compatible with our knowledge
are then worlds in which we know more than we actually do. What's compatible
with our knowledge in \emph{these} worlds is different from what's compatible
with our knowledge in the actual world.

Let's assume, then, that for any world in any scenario there is a set of worlds
that are possible \emph{relative to $w$}. We assume that $\Box p$ as true at $w$
iff $p$ is true at all worlds that are possible relative to $w$. If a world $v$
is possible relative to $w$ we also say that $v$ is \textbf{accessible} from $w$,
or (informally) that $w$ \emph{can see} $v$.

Accessibility means different things in different applications. In epistemic
logic, a world $v$ is accessible from $w$ iff $v$ is compatible with what is
known at $w$. In the logic of historical necessity, $v$ is accessible from $w$
iff $v$ can be brought about at $w$. And so on.

Since facts about accessibility matter to the truth-value of modal sentences,
they must be represented by our models. From now on, a model for $\L_M$ will
therefore specify which worlds in $W$ are accessible from which others (and from
themselves). This marks the difference between a ``basic model'' and a ``Kripke
model'' -- named after Saul Kripke, who popularised models of this kind.
%
% see Burgess, ``Kripke Models'', 2011, for more on the history of Kripke models.
% 
\begin{definition}{}{kripkemodel}
  A \textbf{Kripke model} of $\L_M$ is a triple $\t{W,R,V}$ consisting of
  \vspace{-3mm}
  \begin{itemize*}
  \item a non-empty set $W$,
  \item a binary relation $R$ on $W$, and
  \item a function $V$ that assigns to each sentence letter of $\L_M$
  a subset of $W$.
  \end{itemize*}
\end{definition}
%
\noindent%
$R$ is the accessibility relation. It is called a relation ``on $W$'' because it
holds between members of $W$. We write `$wRv$' to express that $R$ holds between
$w$ and $v$.

We also need to update definition \ref{def:basicsemantics}, which settles under
what conditions an $\L_M$-sentence is true at a world in a model. The old
definition had the following clauses for the box and the diamond:

\bigskip
\begin{tabular}{lll}
  (g) & $M,w \models \Box A$ &iff $M,v \models A$ for all $v$ in $ W$.\\
  (h) & $M,w \models \Diamond A$ &iff $M,v \models A$ for some $v$ in $ W$.
\end{tabular}
\bigskip

\noindent%
In the new semantics, the box and the diamond only quantify over
accessible worlds:

\bigskip
\begin{tabular}{lll}
  (g) & $M,w \models \Box A$ &iff $M,v \models A$ for all $v$ in $ W$ such that $wRv$.\\
  (h) & $M,w \models \Diamond A$ &iff $M,v \models A$ for some $v$ in $ W$ such that $wRv$.
\end{tabular}
\bigskip

Here is the full definition, for completeness.

\begin{definition}{Kripke Semantics}{kripkesemantics}
  If $\Mfr = \t{W,R,V}$ is a Kripke model, $w$ is a member of $W$, $P$ is
  any sentence letter, and $A,B$ are any $\L_M$-sentences, then

  \medskip
  \begin{tabular}{lll}
    (a) & $M,w \models P$ &iff $w$ is in $V(P)$.\\
    (b) & $M,w \models \neg A$ &iff $M,w \not\models A$.\\
    (c) & $M,w \models A \land B$ &iff $M,w \models A$ and $M,w \models B$.\\
    (d) & $M,w \models A \lor B$ &iff $M,w \models A$ or $M,w \models B$.\\
    (e) & $M,w \models A \to B$ &iff $M,w \not\models A$ or $M,w \models B$.\\
    (f) & $M,w \models A \leftrightarrow B$ &iff $M,w \models A\to B$ and $M,w \models B\to A$.\\
    (g) & $M,w \models \Box A$ &iff $M,v \models A$ for all $v$ in $ W$ such that $wRv$.\\
    (h) & $M,w \models \Diamond A$ &iff $M,v \models A$ for some $v$ in $ W$ such that $wRv$.
  \end{tabular}
\end{definition}
%
When I speak of truth at a world in a Kripke model, this should always be
understood in accordance with definition \ref{def:kripkesemantics}. Definition
\ref{def:basicsemantics} defines truth at a world in a basic model.

To see definition \ref{def:kripkesemantics} in action, consider a simple model
with two worlds, $w$ and $v$. World $v$ is accessible from world $w$, but $w$ is
not accessible from $v$. Neither world can access itself. The interpretation
function assigns $\{ v \}$ to $p$ and the empty set $\emptyset$ to all other
sentence letters. The model can be pictured as follows, with an arrow
representing accessibility:

\begin{center}
  \begin{tikzpicture}[modal, world/.append style={minimum size=0.8cm}, node distance = 15mm]
    \node[world] (w1) [label=above:{$w$}] {};
    \node[world] (w2) [label=above:{$v$}, right=of w1] {$p$};
    \path[->] (w1) edge (w2);
  \end{tikzpicture}
\end{center}
%
Using definition \ref{def:kripkesemantics}, we can figure out which $\L_M$-sentences
are true at which worlds in the model. For example:

\begin{itemize*}
  \item By clause (a) of definition \ref{def:kripkesemantics}, $p$ is true at
  $v$ and false at $w$.
  \item By clause (h), $\Diamond p$ is true at $w$ because $p$ is true at $v$
  and $v$ is accessible from $w$. $\Diamond p$ is false at $v$ because there is
  no world accessible from $v$ at which $p$ is true.
  \item By clause (g), $\Box\Diamond p$ is false at $w$ because $\Diamond p$ is
  false at $v$ and $v$ is accessible from $w$. $\Box\Diamond p$ is true at $v$
  because there is no world accessible from $v$ at which $\Diamond p$ is false.
\end{itemize*}
%
Note that $\Diamond p$ and $\Box \Diamond p$ have different truth-values at $w$
(and at $v$). In the new semantics, we can no longer ignore all but the last in a
string of modal operators. Note also that $\Box p$ is true at $w$ even though
$p$ is false; $\Box p \to p$ is no longer valid.

\begin{exercise}
  Explain why every sentence of the form $\Box A$ is true at world $v$
  in the above model.
\end{exercise}
\begin{solution}
  $v$ has access to no world. So any sentence $A$ is true at
  \emph{all} (zero) worlds accessible from $v$.

  If this seems strange, remember that $\Box A$ is equivalent to
  $\neg \Diamond \neg A$. And $\Diamond \neg A$ means that there's an
  accessible world where $\neg A$ is true. If there are no accessible
  worlds, then this is false. So $\neg \Diamond \neg A$ is true.
\end{solution}

The next three exercises refer to the following model:
%
\begin{center}
  \begin{tikzpicture}[modal, world/.append style={minimum size=0.8cm}, node distance = 15mm]
    \node[world] (w1) [label=above:{$w_1$}] {$p$};
    \node[world] (w2) [label=above:{$w_2$}, right=of w1] {};
    \node[world] (w3) [label=below:{$w_3$}, below=of w1] {$p$};
    \node[world] (w4) [label=below:{$w_4$}, right=of w3] {$q$};
    \path[->] (w1) edge (w4);
    \path[<-] (w1) edge (w2);
    \path[<->] (w1) edge (w3);
    \path[->] (w2) edge (w4);
    \path[->] (w4) edge [reflexive right] (w4);
  \end{tikzpicture}
\end{center}

\begin{exercise}
  At which worlds in the  model are the following sentences true?
  \begin{exlist}
  \item $p \lor \neg q$
  \item $\Box(p \lor \neg q)$
  \item $\Diamond(\neg p \land \neg q)$
  \item $\Diamond\Box q$
  \item $\Diamond\Diamond\Box q$
  \end{exlist}
\end{exercise}
\begin{solution}
  (a) $w_{1}, w_{2}$, and $w_{3}$; (b) $w_{3}$; (c) --; (d) $w_{1}, w_{2}$ and $w_{4}$; (e) all.
\end{solution}

\begin{exercise}
  For each world in the model, find an $\L_M$-sentence that is true only at
  that world.
\end{exercise}
\begin{solution}
  There are infinitely many correct answers for each world. For
  example: $w_1: \Diamond\Box p$, $w_2: \neg p \land \neg q$,
  $w_3: \Box p$, $w_4: \Box q$.
\end{solution}

\begin{exercise}
  Can you draw a diagram of a smaller model (with fewer worlds) in
  which the exact same $\L_M$-sentences are true at $w_1$?
\end{exercise}
\begin{solution}
  \begin{tikzpicture}[modal, world/.append style={minimum size=0.8cm}, node distance = 15mm]
    \node[world] (w1) [label=above:{$w_1$}] {$p$};
    \node[world] (w3) [label=below:{$w_3$}, below=of w1] {$p$};
    \node[world] (w4) [label=below:{$w_4$}, right=of w3] {$q$};
    \path[->] (w1) edge (w4);
    \path[<->] (w1) edge (w3);
    \path[->] (w4) edge [reflexive right] (w4);
  \end{tikzpicture}
\end{solution}

% \begin{exercise}
%   This exercise is due to Johan van Benthem. Consider the following structure.

%   \begin{center}
%     \begin{tikzpicture}[modal, world/.append style={minimum size=0.8cm}, node distance = 10mm]
%       \node[world] (w1) [label=above:{$w_1$}] {};
%       \node[world] (w2) [label=above:{$w_2$}, right=of w1] {};
%       \node[world] (w3) [label=above:{$w_3$}, right=of w2] {$p$};
%       \node[world] (w4) [label=left:{$w_4$}, below=of w1] {};
%       \node[world] (w5) [label={[label distance=-1mm]above left:{$w_5$}}, below=of w2] {};
%       \node[world] (w6) [label=right:{$w_6$}, below=of w3] {$p$};
%       \node[world] (w7) [label=below:{$w_7$}, below=of w4] {$p$};
%       \node[world] (w8) [label=below:{$w_8$}, below=of w5] {};
%       \node[world] (w9) [label=below:{$w_9$}, below=of w6] {$t$};
%       \path[->] (w1) edge (w2);
%       \path[->] (w2) edge (w3);
%       \path[->] (w1) edge (w4);
%       \path[->] (w2) edge (w5);
%       \path[->] (w3) edge (w6);
%       \path[->] (w4) edge (w5);
%       \path[->] (w5) edge (w6);
%       \path[->] (w4) edge (w7);
%       \path[->] (w5) edge (w8);
%       \path[->] (w6) edge (w9);
%       \path[->] (w7) edge (w8);
%       \path[->] (w8) edge (w9);
%     \end{tikzpicture}
%   \end{center}
%   % 
%   There's a treasure at $w_9$, marked by the sentence letter $t$;
%   pirates are standing at $w_3, w_6,$ and $w_7$. At which worlds are
%   the following sentences true?
%   \begin{exlist}
%   \item $\Diamond t$
%   \item $\Diamond \Box t$
%   \item $\Diamond p$
%   \item $\Box \Diamond p$
%   \end{exlist}

% \end{exercise}

% \begin{exercise}
%   For each world in the previous exercise, find a sentence that is
%   true only there.
% \end{exercise}


\section{The systems K and S5}\label{sec:systems-k-s5}

As in the previous chapter, we call a sentence \emph{valid} if it is true at all
worlds in all models. But we now use a different conception of models, and a
different definition of truth at a world in a model. To avoid confusion, it is
best to use different expressions for different kinds of validity. Let's call
the new kind of validity \emph{K-validity}. (`K' for Kripke.) The old kind will
henceforth be called \emph{S5-validity}, because the sentences that are valid by
the definition from the previous chapter are precisely the sentences in C.I.\
Lewis's system S5.

\begin{definition}{}{kvalid}
  A sentence $A$ is \textbf{K-valid} (for short, $\models_K A$) iff $A$
  is true at every world in every Kripke model.
\end{definition}

% Many properties of S5-consequence and S5-validity carry over to
% K-consequence and K-validity. In particular, observation \ref{obs:semantic-deduction-theorem} (p.\ \pageref{obs:semantic-deduction-theorem}), the
% propositional extension theorem (p.\
% \pageref{thm:propositional-extension-theorem}), and the replacement
% theorem (p.\ \pageref{thm:replacement-theorem}) still hold, and for
% the same reasons as before. I won't go through the arguments again.

The same distinction applies to the concept of entailment. Entailment in the old
sense (definition \ref{def:basicconsequence}) will henceforth be called
\emph{S5-entailment}. Our new definition of models and truth lead to the
concept of \emph{K-entailment}.

\begin{definition}{}{kconsequence}
  Some sentences $\Gamma$ \textbf{K-entail} a sentence $A$ (for short:
  $\Gamma \models_{K} A$) iff there is no world in any Kripke model at which all
  sentences in $\Gamma$ are true while $A$ is false.
\end{definition}

The set of K-valid sentences is a system of modal logic. This system did
not figure in C.I.\ Lewis's list of systems. It is known as \textbf{system K}.

K is \textbf{weaker} than S5, by which we mean that not all S5-valid
sentences are K-valid. $\Box p \to p$, for example, is S5-valid but not K-valid.
Conversely, however, every K-valid sentence is S5-valid. Let's prove this.

\begin{observation}{kins5}
  Every K-valid sentence is S5-valid.
\end{observation}
%
\begin{proof}
  \emph{Proof:} In essence, observation \ref{obs:kins5} holds because the basic
  models from the previous chapter can be simulated by Kripke models in which
  all worlds have access to all worlds. If a sentence $A$ is K-valid, meaning
  that $A$ is true throughout every Kripke model, then $A$ is true throughout
  every Kripke model of this kind, and so $A$ is also true in every basic model.

  It is worth going through this more carefully. For any basic model
  $M = \langle W,V \rangle$, let $M^*$ be the Kripke model
  $\langle W,R,V \rangle$ with the same worlds $W$ and the same interpretation
  function $V$, and with an accessibility relation $R$ that holds between all
  worlds in $W$. That is, every world in $M^*$ can see every other world as well
  as itself. If every world can see every world, then it makes no difference
  whether we use definition \ref{def:basicsemantics} or definition
  \ref{def:kripkesemantics} to evaluate the truth of sentences at a world.
  That's because the two definitions only differ for the case of the modal
  operators, which definition \ref{def:basicsemantics} interprets as quantifiers
  over all worlds, while definition \ref{def:kripkesemantics} interprets them as
  quantifiers over the accessible worlds. So we have:
  \begin{quote}
  \begin{itemize}
  \item[(*)] A sentence is true at a world $w$ in a basic model $M$ iff
    it is true at $w$ in the corresponding Kripke model $M^*$.
  \end{itemize}
  \end{quote}
  (A full proof of (*) would proceed by induction on complexity of the
  sentence.)

  Now suppose a sentence $A$ is \emph{not} S5-valid, meaning that it is false at
  some world $w$ in some basic model $M$. By (*), it follows that $A$ is also
  false at some world in some Kripke model -- namely, at the same world $w$ in
  $M^*$. And if $A$ is false at some world in some Kripke model, then $A$ is not
  K-valid. By contraposition, it follows that if $A$ is K-valid, then $A$ is
  S5-valid. \qed
\end{proof}

You may remember from section \ref{sec:systems} that S5 can be axiomatized by
five axiom schemas and two rules:
%
\begin{principles}
  \pri{Dual}{\neg\Diamond A \leftrightarrow \Box\neg A}\\
  \pri{K}{\Box(A\to B) \to (\Box A \to \Box B)}\\
  \pri{T}{\Box A \to A}\\
  \pri{4}{\Box A \to \Box \Box A}\\
  \pri{5}{\Diamond A \to \Box \Diamond A}\\
  \pri{Nec}{\text{If $A$ is in the system, then so is }\Box A.}\\
  \pri{CPL}{\text{If }\Gamma \models_{0} A\text{ and all members of }\Gamma\text{ are in the system, then so is }A.}
\end{principles}
%
All instances of \pr{Dual}, \pr{K}, \pr{T}, \pr{4}, and \pr{5} are S5-valid, and
all and only the S5-valid sentences can be derived from instances of these axioms
by \pr{Nec} and \pr{CPL}.

The system K can be axiomatized by dropping three of the axiom schemas: \pr{T},
\pr{4}, and \pr{5}, leaving only \pr{Dual} and \pr{K}. All and only the K-valid
sentences can be derived from instances of \pr{Dual} and \pr{K} by \pr{Nec}
and \pr{CPL}.

(Many authors define $\Box$ as $\neg\Diamond\neg$ or $\Diamond$ as
$\neg\Box\neg$, in which case \pr{Dual} is true by definition. The only
remaining axiom schema is then \pr{K}. Don't confuse the schema \pr{K} with the
system K!)

\begin{exercise}
  \begin{exlist}
    \item Describe a Kripke model in which some instance of \pr{4} is false at
    some world.
    \item Describe a Kripke model in which some instance of \pr{5} is false at
    some world.
  \end{exlist}
\end{exercise}
\begin{solution}
  \begin{sollist}
    \item For example: $W = \{ w,v \}$, $R = \{ (w,v), (v,w) \}$,
    $V(p) = \{ v \}$. $\Box p \to \Box\Box p$ is false at $w$.
    (`$R = \{ (w,v), (v,w) \}$' means that $R$ relates $w$ to $v$ and $v$ to $w$
    and nothing else to anything else.)
    \item For example: $W = \{ w,v \}$, $R = \{ (w,w), (w,v) \}$,
    $V(p) = \{ w \}$. $\Diamond p \to \Box\Diamond p$ is false at $w$.
  \end{sollist}
\end{solution}

\begin{exercise}
  Can you find an instance of the \pr{T}-schema $\Box A \to A$ that is K-valid?
\end{exercise}
\begin{solution}
  For example: $\Box(p \lor \neg p) \to (p \lor \neg p)$.
\end{solution}

\begin{exercise}
  Show that $\Box(p \lor \neg p)$ is K-valid, using definition
  \ref{def:kripkesemantics}.
\end{exercise}
\begin{solution}
  By clause (g) of definition \ref{def:kripkesemantics}, $\Box(p \lor \neg p)$
  is false at a world $w$ in a Kripke model only if $p \lor \neg p$ is false at
  some world accessible from $w$. By clause (d) of definition
  \ref{def:kripkesemantics}, $p \lor \neg p$ is false at a world only if both
  $p$ and $\neg p$ are false at the world, which by clause (a) means that $p$ is
  both true and false at the world. This is impossible. So $\Box(p \lor \neg p)$
  is not false at any world in any Kripke model.
\end{solution}

% Other valid principles: $\Box$ distributes over $\land$ and vice versa.
% $\Box (A \to B) \to (\Diamond A \to \Diamond B)$.
% $(\Box A \land \Diamond B) \to \Diamond (A \land B)$.
  
% Also: If $A \to B$ is valid, then so are $\Box A \to \Box A$ and
% $\Diamond A \to \Diamond B$.

% \begin{exercise}
%   The sentences that are true at a world $w$ in a model $M$ contain
%   information not just about $w$ but also about other worlds in $M$. For
%   example, if $\Box p$ is true at $w$, we know that $w$ is true at all worlds,
%   and if $\neg p$ and $\Diamond p$ are both true at $w$, we know that $p$ is
%   true at some other world. Question: do the sentences true at $w$ completely
%   determine the model $M$? If not, give an example of two worlds $w_1$ and
%   $w_2$ in two models $M1$, $M2$ that verify the same sentences even though
%   the models are not isomorphic.
% \end{exercise}

\section{Some other normal systems}\label{sec:normalsystems}

For many applications of modal logic, we need a concept of validity that lies in
between K-validity and S5-validity. Suppose, for example, we read the box as
physical necessity and the diamond as physical possibility, understood as
compatibility with the laws of nature. On a popular conception of what it means
to be a law of nature, nothing that happens is ever incompatible with the laws
of nature. Equivalently, anything that is physically necessary is actually the
case. We therefore want $\Box A$ to entail $A$. On the other hand, it is not
clear if $\Box A$ should entail $\Box\Box A$: if $A$ is physically necessary,
can we infer that it is physically necessary that $A$ is physically necessary?
Below I will argue that we can't. If that is right, then the logic of physical
necessity is neither K nor S5. We want a logic with \pr{T} ($\Box A \to A$)
but without \pr{4} ($\Box A \to \Box\Box A$). S5 gives us both, K gives us
neither.

Our current semantics makes it easy to define systems in between K and S5 by
putting restrictions on the accessibility relation in Kripke models.

Let's say that an $\L_M$-sentence is \textbf{valid in a class of Kripke models}
iff the sentence is true at every world in every model that belongs to the
class. K-validity is validity in the class of all Kripke models. S5-validity is
validity in the class of Kripke models in which every world has access to every
world (as mentioned earlier, in the proof of observation \ref{obs:kins5}).

% Humberstone says valid "over" a class of frames or models

If you inspect countermodels to the K-validity of $\Box p \to p$, you may notice
that all of them involve worlds that don't have access to themselves. If we
require that every world can see itself then all instances of the \pr{T}-schema
become valid.

\begin{observation}{treflexive}
  All instances of \pr{T} are valid in the class of Kripke models in which every
  world is accessible from itself.
\end{observation}
%
\begin{proof}
  \emph{Proof:} According to clause (e) of definition \ref{def:kripkesemantics},
  an instance of $\Box A \to A$ is false at a world $w$ only if $\Box A$ is true
  at $w$ and $A$ is false; but if $\Box A$ is true at $w$ and $w$ has access to
  itself, then by clause (g) of definition \ref{def:kripkesemantics}, $A$ is
  true at $w$. So if $\Box A \to A$ is false at $w$, and $w$ is accessible from
  itself, then $A$ is both true and false at $w$, which is impossible. Hence
  $\Box A \to A$ is true at every world in every model in which every world is
  accessible from itself. \qed
\end{proof}

A relation $R$ on a set $W$ is called \textbf{reflexive} if each member of $W$
is $R$-related to itself. If the accessibility relation in a Kripke model is
reflexive, we also call the model itself reflexive. Observation
\ref{obs:treflexive} therefore states that all instances of \pr{T}
are valid in the class of reflexive Kripke models.

The set of all sentences that are valid in the class of reflexive Kripke models
is known as \textbf{system T}. Accordingly, any sentence that is valid in this
class of Kripke models (every member of system T) is called \textbf{T-valid}.

% The system T was first discussed in Feys 1937, who dropped one of the axioms
% in Godel 1933.

System T is stronger than K, but weaker than S5. The system can be axiomatized
by adding the axiom schema \pr{T} to the axioms and rules of K. We don't have
\pr{4} or \pr{5}. $\Box p \to \Box \Box p$ is S5-valid but not T-valid.

Systems of modal logic sometimes share their name with a schema. For
disambiguation, I always put schema names in parentheses. \pr{T} is a schema, T
is a system. \pr{K} is a schema, K is a system. All instances of \pr{T} are in
T, but many sentences in T -- for example, all instances of \pr{K} -- are not
instances of \pr{T}.

\begin{exercise}
  Show that $\Box p \to \Diamond p$ is T-valid.
\end{exercise}
\begin{solution}
  By definition \ref{def:kripkesemantics}, $\Box p \to \Diamond p$ is false at a
  world $w$ in a Kripke model only if $\Box p$ is true at $w$ and $\Diamond p$
  is false at $w$. But if $w$ has access to itself then the truth of $\Box p$ at $w$ implies that $p$ is true at $w$, and then $\Diamond p$ is false at $w$. So $\Box p \to \Diamond p$ can't be false at any world in any Kripke model in which each world has access to itself.
\end{solution}

In chapter \ref{ch:time}, we will study a temporal application of modal
logic in which the box is read as `it is always going to be the case that'.
The ``worlds'' in a Kripke model here represent times. $\Box p$ is
understood to be true at a time $t$ iff $p$ is true at all times after $t$. The
accessibility relation is the earlier-later relation: $t_1Rt_2$ iff $t_1$ is
earlier than $t_2$. In this application, we don't want to assume that $R$ is
reflexive, which would mean that every point in time is earlier than itself.
% On the contrary, we may want to assume that $R$ is
% \emph{irreflexive}, meaning that no member of $W$ is $R$-related to
% itself.
But we'll want something else. Suppose $t_1$ is earlier than $t_2$, and $t_2$ is
earlier than $t_3$. Then surely $t_1$ is earlier than $t_3$. 

A relation $R$ is called \textbf{transitive} if whenever $xRy$ and $yRz$ then
$xRz$. As before, we call a Kripke model transitive if its accessibility
relation is transitive. When we do temporal logic, we will restrict the relevant
models to transitive models.

The set of sentences that are valid in the class of transitive Kripke models is
known as \textbf{system K4}. The name alludes to the fact that this system can
be axiomatized by adding schema \pr{4} to the axioms and rules of K.

\begin{observation}{4trans}
  All instances of \pr{4} are valid in the class of transitive Kripke models.
\end{observation}
%
\begin{proof}
  \emph{Proof:} Suppose for reductio that there is some transitive Kripke model
  in which some instance of $\Box A \to \Box \Box A$ is false at some world $w$.
  By clause (e) of definition \ref{def:kripkesemantics}, it follows that (i)
  $\Box A$ is true at $w$ and (ii) $\Box\Box A$ is false at $w$. By clause (g)
  of definition \ref{def:kripkesemantics}, (ii) implies that there is some world
  $v$ accessible from $w$ where $\Box A$ is false. And that, in turn implies
  that there is some world $u$ accessible from $v$ at which $A$ is false. Since
  $R$ is transitive, $u$ is accessible from $w$. By (i), $A$ is true at $u$. So
  $A$ is both true and false at $u$. Contradiction. \qed
\end{proof}

We can combine the systems T and K4 by requiring both reflexivity and
transitivity. The set of sentences valid in the class of reflexive and
transitive Kripke models is C.I.\ Lewis's \textbf{system S4}. It is stronger
than K, T, and K4, but weaker than S5.

There are many other conditions we could impose on the accessibility relation,
and many combinations of these conditions. Each of them defines a system of
modal logic. The following table lists some well-known model classes with the
conventional names for the corresponding systems, repeating (for future
reference) the ones we already know. We will have a closer look at some of these
systems in later chapters, when we turn to applications of modal logic.

\bigskip
\begin{tabular}{ll}
  \toprule
  \emph{System} & \emph{Constraint on $R$}\\
  \midrule
  K & --\\
  T & $R$ is \textbf{reflexive}: every world in $W$ can access itself\\
  D & $R$ is \textbf{serial}: every world in $W$ can access some world\\
  K4 & $R$ is \textbf{transitive}: whenever $wRv$ and $vRu$, then $wRu$\\
  K5 & $R$ is \textbf{euclidean}: whenever $wRv$ and $wRu$, then $vRu$\\
  KD45 & $R$ is serial, transitive, and euclidean\\
  B & $R$ is reflexive and \textbf{symmetric}: whenever $wRv$ then $vRw$\\
  S4 & $R$ is reflexive and transitive\\ 
  S4.2 & $R$ is reflexive, transitive, and \textbf{convergent}: whenever $wRv$ and $wRu$,\\[-0.5mm]
      & then there is some $t$ such that $vRt$ and $uRt$ \\ 
  S5 & $R$ is reflexive, transitive, and symmetric\\ 
  S5 & $R$ is \textbf{universal}: every world has access to every world\\
  \bottomrule
\end{tabular}\label{table:systems}

\bigskip

% Any system that can be defined by putting constraints on the accessibility
% relation in Kripke models is called \textbf{normal}. So K, T, D, K4, K5, B,
% S4, and S5 are examples of normal systems, or normal logics. There are also
% non-normal systems/logics. These require a different kind of semantics. We
% will look at one alternative in section~\ref{sec:neighbourhood}, but mostly we
% will stay within the realm of the normal.

S5 occurs twice in the list. We already know S5 as the system for
universal models, in which the box and the diamond quantify unrestrictedly over
the whole space $W$. But we also get S5 if we merely require the accessibility
relation to be reflexive, transitive, and symmetric.

Relations that are reflexive, transitive, and symmetric are called
\textbf{equivalence relations}. An equivalence relation on a set divides the
members of the set into classes within which everything stands in the relation
to everything. (These classes are called \textbf{equivalence classes}.)

For example, let $S$ be the relation that holds between two people iff they have
the same birthday. This is an equivalence relation. It is reflexive: everyone
has the same birthday as themselves. It is transitive: if $aSb$ and $bSc$ then
$aSc$. And it is symmetric: if $aSb$ then $bSa$. For any person $a$, consider
the class $[a]_S$ of everyone who has the same birthday as $a$. (A ``class'' is
essentially the same thing as a set.) Everyone in $[a]_S$ has the same birthday
as everyone else in $[a]_S$. So within $[a]_S$, the same-birthday relation $S$
is universal.

Now let me explain why the above two characterisations of S5 are equivalent.

\begin{observation}{equivalence-universal}
  A sentence is valid in the class of Kripke models whose accessibility relation
  is universal iff it is valid in the class of Kripke models whose accessibility
  relation is an equivalence relation.
\end{observation}
%
\begin{proof}
  \emph{Proof sketch:} The right-to-left direction is easy. If $R$ is the universal
  relation on $W$, then $R$ is reflexive, transitive, and symmetric. So the
  universal relation on $W$ is a special kind of equivalence relation on $W$. If
  a sentence is valid in every model in which $R$ is an equivalence relation, it
  must therefore be valid in every model in which $R$ is universal.

  The other direction is more interesting. We argue by contraposition, showing
  that if a sentence $A$ is not valid in the class of models in which $R$ is an
  equivalence relation, then $R$ is also not valid in the class of universal
  models. So assume $A$ is not valid in the class of models in which $R$ is an
  equivalence relation. Then there is some world $w$ in some such model
  $M = \t{W,R,V}$ such that $M,w \not\models A$. Define a new model
  $M' = \t{W',R',V'}$ as follows:
  \begin{quote}
    $W'$ is the class of worlds accessible in $M$ from $w$ (i.e., the
    equivalence class $[w]_R$).

    $R'$ is the universal relation on $W'$.

    $V'$ is the restriction of $V$ to $W'$, so that for any sentence letter $B$,
    $V'(B) = V'(B) \cap W'$.
  \end{quote}
  (If $X$ and $Y$ are sets, then $X \cap Y$ -- the \emph{intersection} of $X$ and
  $Y$ -- is the set of all things that are both in $X$ and in $Y$.)

  $M'$ has a universal accessibility relation. But from the perspective of $w$,
  $M$ and $M'$ are indistinguishable. \emph{Any sentence is true at $w$ in $M$ iff
    it is true at $w$ in $M'$.} This could be shown by induction, but I hope you
  see intuitively why it is the case.

  Granting the italicized sentence, the assumption that $A$ is false at some
  world in some model whose accessibility relation is an equivalence relation
  entails that $A$ is false at some world in some model whose accessibility
  relation is universal. \qed
\end{proof}

\begin{exercise}
  Let $R$ be the relation on the set of all people such that $aRb$ iff $b$ is a
  sibling of $a$. Is $R$ reflexive? serial? transitive? euclidean? symmetric?
  universal?
\end{exercise}
\begin{solution}
  Reflexive no, serial no, transitive no, euclidean no, symmetric yes,
  universal no.
\end{solution}

\begin{exercise}\label{ex:relations}
  Explain these facts:
  \begin{exlist}
  \item If $R$ is symmetric and transitive, then $R$ is euclidean.
  \item If $R$ is symmetric and euclidean, then $R$ is transitive.
  \item If $R$ is reflexive and euclidean, then $R$ is symmetric.
  \end{exlist}
\end{exercise}
\begin{solution}
  \begin{sollist}
    \item Suppose $R$ is symmetric and transitive, and that $xRy$ and $xRz$. By
    symmetry, $yRx$. By transitivity, $yRz$.
    \item Suppose $R$ is symmetric and euclidean, and that $xRy$ and $yRz$. By
    symmetry, $yRx$. By euclidity, $xRz$.
    \item Suppose $R$ is reflexive and euclidean, and that $xRy$. By
    reflexivity, $xRx$. By euclidity, $yRx$.
  \end{sollist}
\end{solution}

\begin{exercise}
  What is wrong with the following argument? ``If $R$ is symmetric,
  then $wRv$ implies $vRw$; if $R$ is transitive, it follows that
  $wRw$. So symmetry and transitivity together imply reflexivity.''
\end{exercise}
\begin{solution}
  It's true that if $R$ is symmetric and transitive then $wRv$ implies $vRw$
  which implies $wRw$. But this only shows that every world $w$ \emph{that can
    see some world $v$} can see itself. Symmetry, transitivity, \emph{and
    seriality} together imply reflexivity. Symmetry and transitivity alone do
  not.
\end{solution}

%\begin{exercise}
%  Show that seriality and transitivity and symmetry implies reflexivity.
%\end{exercise}


\section{Frames}\label{sec:frames}

There is a close connection between conditions on the accessibility relation in
Kripke models and modal schemas -- between reflexivity and the \pr{T}-schema,
between transitivity and the \pr{4}-schema, and so on. What exactly is the
connection?

You might think the connection between \pr{T} and reflexivity is this:
\begin{quote}
  \begin{itemize}
    \item[(?)] All instances of \pr{T} are valid in a model iff the model is reflexive.
  \end{itemize}
\end{quote}
%
But that's false. We know (observation \ref{obs:treflexive}) that all \pr{T}
instances are valid in the class of reflexive models. It follows that all \pr{T}
instances are valid in every reflexive model. But the other direction fails.
There are non-reflexive models in which all \pr{T} instances are valid. The
following model is an example.
\begin{center}
  \begin{tikzpicture}[modal, world/.append style={minimum size=0.8cm}, node distance = 15mm]
    \node[world] (w1) [label=above:{$w$}] {$p$};
    \node[world] (w2) [label=above:{$v$}, right=of w1] {$p$};
    \path[<->] (w1) edge (w2);
  \end{tikzpicture}
\end{center}
There are two worlds, both of which can see each other; neither can see itself.
$p$ is true at both worlds, all other sentence letters are false at both worlds.
This model is not reflexive, but no instance of the \pr{T}-schema $\Box A \to A$
is false at any world in the model. (Try to find a false instance!) The fact
that the \pr{T}-schema is valid in a class of models therefore does not entail
that all models in the class are reflexive. The class might contain models like
the one just described.

To understand the connection between modal schemas and conditions on the
accessibility relation, we need to talk about \emph{frames}. A frame is what you
get if take a model and remove the interpretation function.

\begin{definition}{}{Kripke frame}
  A \textbf{Kripke frame} is a pair of a non-empty set $W$ and a relation $R$ on $W$.
\end{definition}

Roughly speaking, if we think of a model as representing a scenario and an
interpretation, then a frame is the part of the model that represents the
scenario.

% In modal logic, we only care about the abstract structure of a
% scenario: how many worlds there are, and how they are accessible from one
% another.

Frames can be pictured just like Kripke models, but without any sentence letters
in the nodes. The frame of the model displayed above looks like this:
\begin{center}
  \begin{tikzpicture}[modal, world/.append style={minimum size=0.8cm}, node distance = 15mm]
    \node[world] (w1) [label=above:{$w$}] {$$};
    \node[world] (w2) [label=above:{$v$}, right=of w1] {$$};
    \path[<->] (w1) edge (w2);
  \end{tikzpicture}
\end{center}

Now remember that validity is truth in virtue of the meaning of the logical
expressions. Whether a sentence is valid should not depend on the
meaning of the non-logical expressions. So if we define a particular kind of
validity by reference to a class of Kripke models, the constraints we impose on
the models in the class should be constraints on the frame of the models, not on
the interpretation function.

To see the point, suppose I suggested that a sentence is ``$X$-valid'' iff it is true
at all worlds in all Kripke model whose interpretation function assigns the
empty set to the sentence letter $p$. So $\Box \neg p$ is $X$-valid, while
$\Box \neg q$ is $X$-invalid. But $\Box \neg p$ and $\Box \neg q$ have the same
logical form. If $\Box \neg p$ is true in virtue of its logical form, then
$\Box \neg q$ should also be true in virtue of its logical form. $X$-validity is
not a sensible concept of logical validity. The systems from the previous
section were all defined sensibly, by putting constraints on the frame of a
Kripke model, not on the interpretation function.

Let's say that a sentence is \textbf{valid on a frame} if it is true at all
worlds in all models with that frame. A sentence is \textbf{valid in a class of
  frames} if it valid on all frames in the class.

If a sentence is valid in the class of all models whose accessibility relation
satisfies a certain condition, then it is also valid in the class of all frames
whose accessibility relation satisfies that condition, and vice versa. We could
have defined the systems from the previous section in terms of frame classes
rather than model classes: K is the set of sentences valid in the class of all
frames, T is the set of sentences valid in the class of reflexive frames, and so
on. (A reflexive/transitive/etc.\ frame is a frame with a
reflexive/transitive/etc.\ accessibility relation.)

Now here is the connection between \pr{T} and reflexivity: All \pr{T} instances
are valid in a class of frames iff every frame in the class is reflexive. More
simply:

\begin{observation}{tcorrespondence}
  All instances of \pr{T} are valid on a frame iff the frame is reflexive.
\end{observation}
%
\begin{proof}
  \emph{Proof:} The right-to-left direction follows from observation
  \ref{obs:treflexive}, according to which all \pr{T} instances are valid in the
  class of reflexive models, and therefore in the class of reflexive frames, and
  therefore on any frame in that class. For the other direction, we have to show
  that if all instances of \pr{T} are valid on a frame $\t{W,R}$, then $R$ is
  reflexive. We do this by showing that if $R$ is not reflexive, then we can
  find an interpretation function $V$ that makes $\Box p \to p$ false at some
  world $w$. $w$ will be an arbitrary world in $W$ that can't see itself. (There
  must be some such world if $R$ is not reflexive.) Let $V(p)$ comprise all
  worlds in $W$ except $w$. Then $\Box p$ is true at $w$ and $p$ false. So
  $\Box p \to p$ is false at $w$. \qed
\end{proof}

If all instances of a schema are valid on all and only the frames whose
accessibility relation satisfies a certain property, the schema is said to
\textbf{correspond} to that property (and to \emph{define} the relevant class of
frames). Observation \ref{obs:tcorrespondence} says that the \pr{T} schema
corresponds to reflexivity.

% More generally, a class of frames is /modally defined/ by a set of formulas
% iff it is the class of frames for the set of formulas, i.e. the frames on
% which all the formulas are valid.
 
Instead of proving more facts about the correspondence between modal schemas and
frame conditions, I will simply give you a list of some important results.

\bigskip
\begin{tabular}{rll}
  \toprule
  \multicolumn{2}{l}{\emph{Schema}} & \emph{Corresponding Frame Condition}\\\midrule
  \pr{T} & $\Box A \to A$ & $R$ is reflexive: every world in $W$ is accessible from itself\\
  \pr{D} & $\Box A \to \Diamond A$ & $R$ is serial: every world in $W$ can access some world in $W$\\
  \pr{B} & $A \to \Box\Diamond A$ & $R$ is symmetric: whenever $wRv$ then $vRw$\\
  \pr{4} & $\Box A \to \Box\Box A$ & $R$ is transitive: whenever $wRv$ and $vRu$, then $wRu$\\
  \pr{5} & $\Diamond A \to \Box\Diamond A$ & $R$ is euclidean: whenever $wRv$ and $wRu$, then $vRu$\\
  \pr{G} & $\Diamond \Box A \to \Box\Diamond A$ & $R$ is convergent:
                                                      whenever $wRv$ and $wRu$, then there is\\[-0.5mm]
      && some $t$ such that $vRt$ and $uRt$ \\ 
  \bottomrule
\end{tabular}
\bigskip

Correspondence facts are often useful when trying to figure out which schemas
should be valid on a given interpretation of the modal operators. Return to the
case of physical possibility and necessity from the start of section
\ref{sec:normalsystems}. I claimed that on this interpretation of the box and
the diamond, we should not regard all instances of the \pr{4}-schema
$\Box A \to \Box\Box A$ as valid. My claim is not based on a direct intuition
that something could be physically necessary without it being physically
necessary that it is physically necessary. My claim is rather based on a
judgement about the non-transitivity of physical accessibility. My reasoning
goes like this. I assume that a world $v$ is physically possible relative to a
world $w$ if nothing that happens at $v$ contradicts the laws of nature at $w$.
This does not imply that $v$ has the same laws as $w$. For example, suppose the
only law at $w$ is that ravens are black; at $v$, there is no such law but there
happen to be no non-black ravens. Then what happens at $v$ does not contradict
the laws at $w$, even though $v$ has different laws. Relative to the laws of
$v$, worlds with white ravens are physically possible. So a world accessible
from a world that is accessible from $w$ need not itself be accessible from $w$.
Since \pr{4} corresponds to transitivity, I can infer that the logic of physical
necessity does not render all instances of that schema valid.

\begin{exercise}
  Can you find frame conditions that correspond to these schemas?
  \begin{exlist}
  \item $\Box A \leftrightarrow A$
  \item $\Box A$
  \end{exlist}
\end{exercise}
\begin{solution}
  \begin{sollist}
  \item Every world has access only to itself.
  \item No world has access to any world.
  \end{sollist}
\end{solution}

\section{More trees}
\label{sec:more-trees}

In section \ref{sec:trees}, I described the tree method for checking whether a
sentence is valid, and for constructing countermodels. These were the rules for
the box and the diamond:

\bigskip

\begin{minipage}{0.24\textwidth} \centering
\tree{
  \dotbelownode{8}{}{$\Box A$}{\omega}{}\\
  \\
  \nnode{8}{}{$A$}{\nu}{}\\
  \Kk[8]{0}{\color{red}$\uparrow$}\\
  \Kk[8]{0}{\color{red}\small old}
}
\end{minipage}
\begin{minipage}{0.24\textwidth}\centering
\tree{
  \dotbelownode{8}{}{$\Diamond A$}{\omega}{$\!\!\!\!\!\!\checkmark$}\\
  \\
  \nnode{8}{}{$A$}{\nu}{}\\
  \Kk[8]{0}{\color{red}$\uparrow$}\\
  \Kk[8]{0}{\color{red}\small new}
}
\end{minipage}
\begin{minipage}{0.24\textwidth}\centering
\tree{
  \dotbelownode{10}{}{$\neg \Box A$}{\omega}{$\!\!\!\!\!\!\checkmark$}\\
  \\
  \nnode{10}{}{$\neg A$}{\nu}{}\\
  \Kk[10]{0}{\color{red}$\uparrow$}\\
  \Kk[10]{0}{\color{red}\small new}
}
\end{minipage}
\begin{minipage}{0.24\textwidth}\centering
\tree{
  \dotbelownode{10}{}{$\neg \Diamond A$}{\omega}{}\\
  \\
  \nnode{10}{}{$\neg A$}{\nu}{}\\
  \Kk[10]{0}{\color{red}$\uparrow$}\\
  \Kk[10]{0}{\color{red}\small old}
}
\end{minipage}
%

\bigskip
\noindent
The rule for $\Box A$ allows us to infer, from the hypothesis that $\Box A$ is
true at some world, that $A$ is true at any world that occurs on a tree branch.
This made sense given the semantics of the previous chapter, where the box
quantified unrestrictedly over all worlds. With the new semantics of the present
chapter, we need to change the tree rules.

If $\Box A$ is true at a world $w$, and there's some other world $v$ on the
branch, we can only infer that $A$ is true at $v$ if $v$ is accessible from $w$.
So we need to keep track of which worlds are accessible from any world on a
tree. We do this by adding meta-linguistic statements about accessibility to the
tree.

For example, suppose we want to expand the following node.
%
\begin{center}
  \tree{%
    \nnode{18}{n.}{$\Diamond p$}{w}{} 
  }
\end{center}
%
The node represents the hypothesis that $\Diamond p$ is true at $w$.
It follows that $p$ is true at some world $v$. Moreover, that world
$v$ must be accessible from $w$. So we add two new nodes:

\begin{center}
  \tree{%
    \nnode{18}{m.}{$wRv$}{}{} \\
    \nnode{18}{m+1.}{$p$}{v}{} 
  }
\end{center}
%
Node $m+1$ is what we would have added by the old rules. Node $m$ is a
meta-linguistic statement reminding us that $v$ is accessible from
$w$. `$wRv$' is not a sentence of $\L_M$; it isn't true or false
relative to a world, which is why node $m$ has no world label.

What if we want to expand a box node?
%
\begin{center}
  \tree{%
    \nnode{18}{n.}{$\Box p$}{$w$}{} 
  }
\end{center}
%
By itself, this doesn't tell us anything about the truth-value of $p$
at any world. We can't infer that $p$ is true at $w$, because $w$
might not be accessible from itself. Indeed, if no world is accessible
from $w$, then $\Box p$ can be true even if $p$ is false at every
world. So we can't even infer that there is some world or other at
which $p$ is true.

However, suppose a branch that contains node $n$ also contains the
following node.
\begin{center}
  \tree{%
    \nnode{18}{m.}{$w R v$}{}{} 
  }
\end{center}
Now we can infer that $p$ is true at $v$. So to expand a box node on a
branch, there must be another node on the branch telling us that
the world $w$ at which the box sentence is true has access to some
world $v$.

Here are diagrams of the new rules for the box and the diamond.

\bigskip\noindent%
\begin{minipage}{0.24\textwidth} \centering
\tree{
  \nnode{12}{}{$\Box A$}{\omega}{}\\
  \dotbelownode{12}{}{$\omega R\nu$}{}{}\\
  \\
  \nnode{12}{}{$A$}{\nu}{}\\
  \Kk[12]{0}{}\\
  \Kk[12]{0}{}
}
\end{minipage}
\begin{minipage}{0.26\textwidth}\centering
\tree{
  \dotbelownode{12}{}{$\Diamond A$}{\omega}{$\!\!\!\!\!\!\checkmark$}\\
  \\
  \nnode{12}{}{$\omega R \nu$}{}{}\\
  \nnode{12}{}{$A$}{\nu}{}\\
  \Kk[12]{0}{\color{red}$\uparrow$}\\
  \Kk[12]{0}{\color{red}\small new}
}
\end{minipage}
\begin{minipage}{0.26\textwidth}\centering
\tree{
  \dotbelownode{12}{}{$\neg \Box  A$}{\omega}{$\!\!\!\!\!\!\checkmark$}\\
  \\
  \nnode{12}{}{$\omega R \nu$}{}{}\\
  \nnode{12}{}{$\neg A$}{\nu}{}\\
  \Kk[12]{0}{\color{red}$\uparrow$}\\
  \Kk[12]{0}{\color{red}\small new}
}
\end{minipage}
\begin{minipage}{0.24\textwidth} \centering
\tree{
  \nnode{12}{}{$\neg \Diamond  A$}{\omega}{}\\
  \dotbelownode{12}{}{$\omega R \nu$}{}{}\\
  \\
  \nnode{12}{}{$\neg A$}{\nu}{}\\
  \Kk[12]{0}{}\\
  \Kk[12]{0}{}
}
\end{minipage}

\bigskip\noindent%
If two nodes occur above the dotted line in a rule, as in the rule for $\Box A$,
this means that the rule can only be applied if both nodes already occur on the
relevant branch (in any order, and not necessarily adjacent to each other).

As before, the checkmark next to the rules for $\Diamond A$ and $\neg \Box A$
indicates that these nodes can only be expanded once on each branch.

The rules for negated boxes and diamonds are what you would expect from
the duality of the box and the diamond. Note that only nodes of type
$\Diamond A$ and $\neg \Box A$ allow us to introduce hypotheses about
accessibility into a tree.

The rule for the classical connectives all stay the same. Together,
all these rules are known as the \textbf{K-rules}; the tree rules from
section \ref{sec:trees} are the \textbf{S5-rules}.

Here is a schematic tree proof to show that
$\models_K \Box (A \land B) \to (\Box A \land \Box B)$.

\begin{center}
  \tree[4]{%
    & \nnode{32}{1.}{$\neg(\Box (A \land B) \to (\Box A \land \Box B))$}{w}{(Ass.)} &\\
    & \nnode{32}{2.}{$\Box(A \land B)$}{w}{(1)} &\\
    & \bnode{32}{3.}{$\neg(\Box A \land \Box B)$}{w}{(1)} & \\
    &&\\
    \nnode{12}{4.}{$\neg\Box A$}{w}{(3)} && \nnode{12}{5.}{$\neg\Box B$}{w}{(3)} \\
    \nnode{12}{6.}{$wRv$}{}{(4)} && \nnode{12}{11.}{$wRu$}{}{(5)} \\
    \nnode{12}{7.}{$\neg A$}{v}{(4)} && \nnode{12}{12.}{$\neg B$}{u}{(5)} \\
    \nnode{12}{8.}{$A\land B$}{v}{(2,6)} && \nnode{12}{13.}{$A \land B$}{u}{(2,11)} \\
    \nnode{12}{9.}{$A$}{v}{(8)} && \nnode{12}{14.}{$A$}{u}{(13)} \\
    \nnodeclosed{12}{10.}{$B$}{v}{(8)} && \nnodeclosed{12}{15.}{$B$}{u}{(13)} \\
  }
\end{center}
%
The annotation `(2,6)' for node 8 indicates that this node is based on two
assumptions from earlier in the branch: the assumption on node 2 that
$\Box (A \land B)$ is true at $w$, and the assumption on node 6 that $wRv$. Only
these two assumptions together allow us to infer that $A \land B$ is true at
$v$.

What happens if we try to prove $\Box p \to p$?

\begin{center}
  \tree{%
    & \nnode{15}{1.}{$\neg(\Box p \to p)$}{w}{(Ass.)} &\\
    & \nnode{15}{2.}{$\Box p$}{w}{(1)} &\\
    & \nnode{15}{3.}{$\neg p$}{w}{(1)} &
  }
\end{center}
%
At this point, no more rules can be applied. We can read off a countermodel from
the open branch:
\begin{center}
  $W = \{ w \}$\\
  $R = \emptyset$\\
  $V(p) = \emptyset$
\end{center}
This is the smallest possible Kripke model. It consists of a single world that
can't see itself. `$R = \emptyset$' is a way of saying that no world can see any
world. If you want to say that $R$ holds between $w$ and $v$ and between $v$ and
$u$, you might write `$R = \{ (w,v), (v,u) \}$' or simply `$wRv, vRu$'.

\begin{exercise}
  Use the K-rules to check which of the following sentences are K-valid. If a sentence is invalid, describe a countermodel.
  \begin{exlist}
  \item $(\Box p \land \Box q) \to \Box (p \land q)$
  \item $\Diamond (p \land q) \to (\Diamond p \land \Diamond q)$
  \item $(\Diamond p \land \Diamond q) \to \Diamond (p \land q)$
  \item $\Diamond(p \lor q) \leftrightarrow (\Diamond p \lor \Diamond q)$
  \item $\Box(p \lor q) \leftrightarrow (\Box p \lor \Box q)$
  \item $\Box (p \to q) \to (\Diamond p \to \Diamond q)$. % yes
  \item $(\Box p \land \Diamond q) \to \Diamond (p \land q)$. % yes
  % <>A->[]B entails []A->[]B.
  % $\Diamond A \to (\Box B \to \Diamond B)$
  % $\Diamond (A \to B) \leftrightarrow (\Box A \to \Diamond B)$
  \end{exlist}
\end{exercise}
\begin{solution}
  You can enter the sentences at
  \href{https://www.umsu.de/trees/}{umsu.de/trees}. To check for K-validity, leave all the checkboxes (for `universal' etc.) empty.
\end{solution}

For systems in between K and S5 that are characterised by certain constraints on
the accessibility relation, we add new rules for manipulating accessibility
nodes. For example, if we want to check whether a sentence is T-valid, we use a
\emph{reflexivity rule} in addition to the K-rules. The reflexivity rule says
that if a world variable $\omega$ occurs on a branch, then we may always add
$\omega R\omega$ to the branch.

Here is a proof of $\Box p \to p$, using the reflexivity rule.

\begin{center}
  \tree[4]{%
     \nnode{19}{1.}{$\neg(\Box p \to p)$}{w}{(Ass.)} \\
     \nnode{19}{2.}{$\Box p$}{w}{(1)} \\
     \nnode{19}{3.}{$\neg p$}{w}{(1)}  \\
     \nnode{19}{4.}{$wRw$}{}{(Ref.)}  \\
     \nnodeclosed{19}{5.}{$p$}{w}{(2,4)}  \\
  }
\end{center}

To test for validity in the class of transitive frames (or models), we need a
\emph{transitivity rule}, which allows us to infer $\omega R\upsilon$ from
$\omega R\nu$ and $\nu R\upsilon$. Here is a proof of $\Box p \to \Box\Box p$
that uses this rule.

\begin{center}
  \tree[4]{%
     \nnode{25}{1.}{$\neg(\Box p \to \Box\Box p)$}{w}{(Ass.)} \\
     \nnode{25}{2.}{$\Box p$}{w}{(1)} \\
     \nnode{25}{3.}{$\neg \Box\Box p$}{w}{(1)}  \\
     \nnode{25}{4.}{$wRv$}{}{(3)}  \\
     \nnode{25}{5.}{$\neg\Box p$}{v}{(3)}  \\
     \nnode{25}{6.}{$vRu$}{}{(5)}  \\
     \nnode{25}{7.}{$\neg p$}{u}{(5)}  \\
     \nnode{25}{8.}{$wRu$}{}{\quad(4,6,Tr.)}  \\
     \nnodeclosed{25}{9.}{$p$}{u}{(2,8)}  \\
  }
\end{center}

The following diagrams summarize the tree rules for the frame conditions we have
so far considered.
%
\begin{center}

  \begin{minipage}[t]{0.3\textwidth} \centering

    Reflexivity
    
    \vspace{-3mm}

    \tree{
      \dotbelowbarenode{}\\
      \\
      \barenode{$\omega R \omega$}\\
      \Kk[0]{0}{\color{red}$\uparrow$\hspace{6mm}}\\
      \Kk[0]{0}{\color{red}\small old\hspace{6mm}}
    }
  \end{minipage}
  \begin{minipage}[t]{0.3\textwidth} \centering
    Seriality

    \vspace{-3mm}

    \tree{
      \dotbelowbarenode{}\\
      \\
      \barenode{$\omega R \nu$}\\
      \Kk[0]{0}{\color{red}$\uparrow$\hspace{3mm}$\uparrow$}\\
      \Kk[0]{0}{\color{red}\small old \; new}
    }
  \end{minipage}
  \begin{minipage}[t]{0.3\textwidth} \centering
    Transitivity

    \medskip
    
    \tree{
      \barenode{$\omega R \nu$}\\
      \dotbelowbarenode{$\nu R \upsilon$}\\
      \\
      \barenode{$\omega R \upsilon$}\\
    }
  \end{minipage}
  \bigskip
  
  \begin{minipage}[t]{0.3\textwidth} \centering
    Symmetry

    \medskip
    
    \tree{
      \dotbelowbarenode{$\omega R \nu$}\\
      \\
      \barenode{$\nu R \omega$}\\
    }
  \end{minipage}
  \begin{minipage}[t]{0.3\textwidth} \centering
    Euclidity
    
    \medskip

    \tree{
      \barenode{$\omega R \nu$}\\
      \dotbelowbarenode{$\omega R \upsilon$}\\
      \\
      \barenode{$\nu R \upsilon$}\\
    }
  \end{minipage}
  \begin{minipage}[t]{0.3\textwidth} \centering
    Convergence
    
    \medskip

    \tree{
      \barenode{$\omega R \nu$}\\
      \dotbelowbarenode{$\omega R \upsilon$}\\
      \\
      \barenode{$\nu R \tau$}\\
      \barenode{$\upsilon R \tau$}\\
      \Kk[0]{0}{\hspace{6mm}\color{red}$\uparrow$}\\
      \Kk[0]{0}{\hspace{6mm}\color{red}\small new}
    }
  \end{minipage}

\end{center}

By selectively adding some of these rules to the K-rules, we get tree rules for
a variety of modal logics.  (Compare the table on p.\
\pageref{table:systems}.)

\bigskip
\begin{tabular}{ll}
  \toprule
  \emph{System} & \emph{Tree Rules}\\
  \midrule
  K & K-rules\\
  T & K-rules and reflexivity rule\\
  D & K-rules and seriality rule\\
  K4 & K-rules and transitivity rule\\
  K5 & K-rules and euclidity rule\\
  KD45 & K-rules, seriality rule, transitivity rule, and euclidity rule\\
  B & K-rules, reflexivity rule, and symmetry rule\\
  S4 & K-rules, reflexivity rule, and transitivity rule\\ 
  S4.2 & K-rules, reflexivity rule, transitivity rule, and convergence rule\\
  \bottomrule
\end{tabular}
\bigskip

% Seriality tree rule should only be allowed for w if there's not already a node
% wRv on the tree? (Priest.)

% Seriality messes things up anyway: serial trees grow forever, and you can
% never read off a countermodel from a truncated initial part of an open branch,
% because that part won't be serial. Mention in deontic chapter?

\begin{exercise}
  Use the tree method to check the following claims.
  \begin{exlist}
    \item $\models_{K4} \Diamond p \to \Diamond\Diamond p$.
    \item $\models_{D} (\Box p \land \Box q) \to \Diamond (p \lor q)$.
    \item $\models_{B} \Diamond p \to \Box\Diamond p$.
    \item $\models_{T} (\Diamond\Box(p \to q) \land \Box p) \to \Diamond q$.
    \item $\models_{T} \Diamond (p \to \Box \Diamond p)$.
    \item $\models_{S4} \Diamond\Box(\Diamond p \to \Box\Diamond p)$. 
  \end{exlist}
\end{exercise}
\begin{solution}
  You can enter the sentences at
  \href{https://www.umsu.de/trees/}{umsu.de/trees}. To
  test for K4-validity, check the `transitive' box. To test for D-validity, check `serial'. To test for B-validity, check `symmetric'. To test for T-validity, check `reflexive'.
\end{solution}


%%% Local Variables: 
%%% mode: latex
%%% TeX-master: "logic2.tex"
%%% End:

\chapter{Models and Proofs}\label{ch:proofs}

\section{Soundness and completeness}

You may find that this chapter is harder and more abstract than the previous
chapters. Feel free to skip or skim it if you're mostly interested in
philosophical applications.

We have introduced several kinds of validity: S5-validity, K-validity,
T-validity, and so on. All of these are defined in terms of models. K-validity
means truth at all worlds in all Kripke models. T-validity means truth at all
worlds in all reflexive Kripke models. S5-validity means truth at all worlds in
all universal Kripke models (equivalently, at all worlds in all ``basic''
models). And so on.

If you want to show that a sentence is, say, K-valid, you could directly work
through the clauses of definition \ref{def:kripkesemantics}, showing that there
is no world in any Kripke model in which the sentence is false. The tree method
regiments and simplifies this process. If you construct a tree for your sentence
in accordance with the K-rules and all branches close, then the sentence is
K-valid. If some branch remains open, the sentence isn't K-valid.

Or so I claimed. But these claims aren't obvious. The tree rule for the diamond,
for example, appears to assume that if $\Diamond A$ is true at a world then $A$
is true at some accessible world \emph{that does not yet occur on the branch}.
Couldn't $\Diamond A$ be true because $A$ is true at an accessible ``old'' world
instead? Also, why do we expand $\Diamond A$ nodes only once? Couldn't $A$ be
true at multiple accessible worlds?

In the next two sections, we are going to lay any such worries to rest. We are
going to prove that (1) if all branches on a K-tree close then the target
sentence is K-valid; conversely, (2) if some branch on a fully developed K-tree
remains open, then the target sentence is not K-valid. (1) establishes the
\emph{soundness} of the tree rules for K, (2) establishes their
\emph{completeness}.

When you use the tree method, you don't have to think of what you are doing as
exploring Kripke models. I could have introduced the method as a purely
syntactic game. You start the game by writing down the negation of the target
sentence, followed by `(w)' (and possibly `1.' to the left and `(Ass.)' to the
right, although in this chapter we will mostly ignore these book-keeping
annotations.) Then you repeatedly apply the tree rules until either all branches
are closed or no rule can be applied any more. At no point in the game do you
need to think about what any of the symbols you are writing might mean.

Soundness and completeness link this syntactic game with the ``model-theoretic''
concept of validity. Soundness says that if the game leads to a closed tree (a
tree in which all branches are closed) then the target sentence is true at all
worlds in all models. Completeness says that if the game doesn't lead to a
closed tree then the target sentence is not true at all worlds in all models.
This is called completeness because it implies that every valid sentence can be
shown to be valid with the tree method.

In general, a proof method is called \textbf{sound} if everything that is
provable with the method is valid. A method is \textbf{complete} if everything
that is valid is provable. Strictly speaking, we should say that a method is
sound or complete \emph{for a given concept of validity}. The tree rules for K
are sound and complete for \emph{K-validity}, but not for T-validity or
S5-validity.

The tree method is not the only method for showing that a sentence is K-valid
(or T-valid, or S5-valid). Instead of constructing a K-tree, you could construct
an axiomatic proof, trying to derive the target sentence from some instances of
\pr{Dual} and \pr{K} by \pr{Nec} and \pr{CPL}. This, too, can be done as a
purely syntactic exercise, without thinking about models and worlds. In section
\ref{sec:scaxiomatic}, we will show that the axiomatic calculus for K is indeed
sound and complete for K-validity: all and only the K-valid sentences can be
derived from \pr{Dual} and \pr{K} by \pr{Nec} and \pr{CPL}. The `all' part is
completeness, the `only' part soundness. Having shown soundness and completeness
for both the tree method and the axiomatic method, we will have shown that the
two methods are equivalent. Anything that can be shown with one method can also
be shown with the other.

There are other styles of proof besides the axiomatic and the tree format. Two
famous styles that we won't cover are ``natural deduction'' methods and
``sequence calculi''. Logicians are liberal about what qualifies as a proof
method. The only non-negotiable condition is that there must be a mechanical way
of checking whether something (usually, some configuration of symbols) is or is
not a proof of a given target sentence.

\begin{exercise}\label{ex:proofmethods}
  What do you think of the following proposals for new proof methods?
  \begin{exlist}
    \item In \emph{method A}, every $\L_{M}$-sentence is a proof of itself: To
    prove an $\L_{M}$-sentence with this method, you simply write down the
    sentence.
    \item In \emph{method B}, every $\L_{M}$-sentence that is an instance of
    $\Box(A \lor \neg A)$ is a proof of itself. Nothing else is a proof in
    method B.
    \item In \emph{method C}, a proof of a sentence $A$ is a list of
    $\L_{M}$-sentences terminating with $A$ and in which every sentence occurs
    in some logic textbook.
  \end{exlist}
  \medskip%
  
  Which of these qualify as genuine proof methods by the criterion I have
  described?
\end{exercise}
\begin{solution}
  Methods A and B are genuine proof methods. Method C is not because there is no
  simple mechanical check of whether a sentence occurs in some logic textbook.
\end{solution}

\begin{exercise}\label{ex:sillyproofmethod}
  Which, if any, of the methods from the previous exercise are sound for
  K-validity? Which, if any, are complete?
\end{exercise}
\begin{solution}
  Method A is complete, but not sound. Everything that's K-valid is provable
  with the method, but so is everything that's not K-valid.
  
  Method B is sound, but not complete. Since every instance of
  $\Box (A \lor \neg A)$ is K-valid, everything that is provable with method B
  is K-valid. But many K-valid sentences (e.g., $p \to p$) aren't provable with
  method B.

  Method C is neither sound nor complete. It is not sound because many K-invalid
  sentences figure in logic textbooks. It is not complete because there
  are infinitely many K-valid sentences almost all of which don't occur in any
  textbooks.
\end{solution}

% A set of sentences (system) is sometimes called complete iff there
% is a sound and complete proof method for it.

\section{Soundness for trees}%
\label{sec:soundnesstrees}

We are now going to show that the tree method for K is sound -- that every sentence
that can be proved with the method is K-valid. A proof in the tree method is a
tree in which all branches are closed. So this is what we have to show:
%
\begin{quote}
  Whenever all branches on a K-tree close then the target sentence is K-valid.
\end{quote}
%
By a \emph{K-tree} I mean a tree that conforms to the K-rules from the previous
chapter.

I'll first explain the proof idea, then I'll fill in the details. We will assume
that there is a K-tree for some target sentence $A$ on which all branches close.
We need to show that $A$ is K-valid. To this end, we suppose for reductio that
$A$ is \emph{not} K-valid. By definition \ref{def:kvalid}, a sentence is K-valid
iff it is true at all worlds in all Kripke models. Our supposition that $A$ is
not K-valid therefore means that $A$ is false at some world in some Kripke
model. Let's call that world `$w$' and the model `$M$'. Note that the closed
tree begins with

\begin{center}
  \tree{%
    \nnode{12}{1.}{$\neg A$}{w}{}%
  }
\end{center}

\noindent%
If we take the world variable `$w$' on the tree to pick out world $w$ in $M$,
then node 1 is a correct statement about $M$, insofar as $\neg A$ is indeed true
at $w$ in $M$. Now we can show the following:

\begin{quote}
    \emph{If all nodes on some branch of a tree are correct statements about
      $M$, and the branch is extended by the K-rules, then all nodes on at least
      one of the resulting branches are still correct statements about $M$.}
\end{quote}
%
Since our closed tree is constructed from node 1 by applying the K-rules, it
follows that all nodes on some branch of the tree are correct statements about
$M$. But every branch of a closed tree contains a pair of contradictory
statements, which can't both be correct statements about $M$. This completes the
reductio.

Let's fill in the details. We first define precisely what it means for the
nodes on a tree branch to be correct statements about a model.

\begin{definition}{}{correctstatement}
  A tree node is a \textbf{correct statement about} a Kripke model
  $M = \t{M,R,V}$ \textbf{under} a function $f$ that maps world variables to
  members of $W$ iff either the node has the form $\omega R \upsilon$ and
  $f(\omega)Rf(\upsilon)$, or the node has the form $A\; (\omega)$ and $A$ is
  true at $f(\omega)$ in $M$.

  A tree branch \textbf{correctly describes} a model $M$ iff there is a function
  $f$ under which all nodes on the branch are correct statements about $M$.
\end{definition}

% I don't like these labels, but can't think of better once.

We now prove the italicised statement above:

\begin{theorem}{Soundness Lemma}{soundnesslemma}
  If some branch $\beta$ on a tree correctly describes a Kripke model $M$, and
  the branch is extended by applying a K-rule, then at least one of the
  resulting branches correctly describes $M$.
\end{theorem}

\begin{proof}
  \emph{Proof:} We have to go through all the K-rules. In each case we assume
  that the rule is applied to some node(s) on a branch $\beta$ that correctly
  describes $M$, so that there is a function $f$ under which all nodes on
  the branch are correct statements about $M$. We show that once the rule has
  been applied, at least one of the resulting branches still correctly describes
  $M$.
  
  \medskip
  \begin{itemize}
  \itemsep1mm
  
    \item Suppose $\beta$ contains a node of the form $A \land B \;(\omega)$ and
          the branch is extended by two new nodes $A \;(\omega)$ and
          $B \;(\omega)$. Since $A \land B \; (\omega)$ is a correct statement
          about $M$ under $f$, we have $M,f(\omega) \models A \land B$. By
          clause (c) of definition \ref{def:kripkesemantics}, it follows that
          $M,f(\omega) \models A$ and $M,f(\omega) \models B$. So the extended
          branch still correctly describes $M$.

    \item Suppose $\beta$ contains a node of the form $A \lor B \;(\omega)$ and
          the branch is split into two, with $A \;(\omega)$ appended to one end
          and $B \;(\omega)$ to the other. Since the expanded node is a correct
          statement about $M$ under $f$, we have $M,f(\omega) \models A \lor B$.
          By clause (d) of definition \ref{def:kripkesemantics}, it follows that
          either $M,f(\omega) \models A$ or $M,f(\omega) \models B$. So at least
          one of the resulting branches also correctly describes $M$.
          
  \end{itemize}
  %
  The proof for the other non-modal rules is similar. Let's look at the rules
  for the modal operators.

  \begin{itemize}
    \itemsep1mm
    
    \item Suppose $\beta$ contains nodes of the form $\Box A \;(\omega)$ and
          $\omega R \upsilon$, and the branch is extended by adding
          $A \;(\upsilon)$. Since $\Box A \;(\omega)$ and $\omega R \upsilon$
          are correct statement about $M$ under $f$, we have
          $M,f(\omega) \models \Box A$ and $f(\omega)Rf(\upsilon)$. By clause
          (g) of definition \ref{def:kripkesemantics}, it follows that
          $M,f(\upsilon) \models A$. So the extended branch correctly describes
          $M$.

    \item Suppose $\beta$ contains a node of the form $\Diamond A \;(\omega)$
          and the branch is extended by adding nodes $\omega R \upsilon$ and
          $A \;(\upsilon)$, where $\upsilon$ is new on the branch. Since
          $\Diamond A \;(\omega)$ is a correct statement about $M$ under $f$, we
          have $M,f(\omega) \models \Diamond A$. By clause (h) of definition
          \ref{def:kripkesemantics}, it follows that $M,v \models A$ for some
          $v$ in $W$ such that $f(\omega)Rv$. Let $f'$ be the same as $f$ except
          that $f'(\upsilon) = v$. The newly added nodes are correct statements
          about $M$ under $f'$. Since $\upsilon$ is new on the branch, all
          earlier nodes on the branch are also correct statements about $M$
          under $f'$. So the expanded branch correctly describes $M$.

          % Note that f(w) may equal f(v). A reflexive model can be correctly
          % described by a tree branch that contains no "reflexive" statements
          % of the form wRw. As far as soundness is concerned, the rule for <>A
          % doesn't assume that A is true at a "new world". The rule merely
          % assumes that A is true at some world, for which we introduce a new
          % name. That the world is genuinely new is only assumed when we
          % read off countermodels.
           
  \end{itemize}
  %
  The cases for $\neg \Box$ and $\neg \Diamond$ are similar to the previous two
  cases. \qed

\end{proof}

\medskip

With the help of this lemma, we can prove that the method of K-trees is sound.

\begin{theorem}{Theorem: Soundness of K-trees}{soundness-tree-K}
  If a K-tree for a target sentence closes, then the target sentence is K-valid.
\end{theorem}

\begin{proof}
  \emph{Proof:} Suppose for reductio that some K-tree for some target sentence
  $A$ closes even though $A$ is not K-valid. Then $\neg A$ is true at some world
  $w$ in some Kripke model $M$. The first node on the tree, $\neg A \; (w)$, is
  a correct statement about $M$ under the function that maps the world variable
  `$w$' to $w$. Since the tree is created from the first node by applying the
  K-rules, the Soundness Lemma implies that some branch $\beta$ on the tree
  correctly describes $M$: all nodes on the tree are correct statements about
  $M$ under some function $f$. But the tree is closed. This means that $\beta$
  contains contradictory nodes of the form
  
  \begin{center}
    \tree{%
      \nnode{12}{n.}{$B$}{\upsilon}{}\\
      \nnode{12}{m.}{$\neg B$}{\upsilon}{}
    }
  \end{center}
  If both of these are correct statements about $M$ under $f$, then
  $M,f(\upsilon) \models B$ and also $M,f(\upsilon) \models \neg B$. This is
  impossible by definition \ref{def:kripkesemantics}. \qed
\end{proof}

\begin{exercise}
  Spell out the cases for $A \to B$ and $\neg\Diamond A$ in the proof of the
  Soundness Lemma.
\end{exercise}
\begin{solution}
  For $A\to B$: Suppose $\beta$ contains a node of the form $A \to B \;(\omega)$
  and the branch is split into two, with $\neg A \;(\omega)$ appended to one end
  and $B \;(\omega)$ to the other. Since the expanded node is a correct
  statement about $M$ under $f$, we have $M,f(\omega) \models A \to B$. By
  clause (e) of definition \ref{def:kripkesemantics}, it follows that either
  $M,f(\omega) \not\models A$ or $M,f(\omega) \models B$. By clause (b), this
  means that either $M,f(\omega) \models \neg A$ or $M,f(\omega) \models B$. So
  at least one of the resulting branches also correctly describes $M$.

  For $\neg\Diamond A$: Suppose $\beta$ contains nodes of the form
  $\neg\Diamond A \;(\omega)$ and $\omega R \upsilon$, and the branch is
  extended by adding $\neg A \;(\upsilon)$. Since $\neg\Diamond A \;(\omega)$
  and $\omega R \upsilon$ are correct statement about $M$ under $f$, we have
  $M,f(\omega) \models \neg\Diamond A$ and $f(\omega)Rf(\upsilon)$. By clause
  (b) of definition \ref{def:kripkesemantics},
  $M,f(\omega) \models \neg\Diamond A$ implies
  $M,f(\omega) \not\models \Diamond A$. By clause (h), it follows that
  $M,f(\upsilon) \models \neg A$. So the extended branch correctly describes
  $M$.
\end{solution}

\begin{exercise}
  Draw the K-tree for target sentence $\Box p$. The tree has a single open
  branch. Does this branch correctly describe the Kripke model in which there is
  just one world $w$, $w$ has access to itself, and all sentence letters are
  false at $w$?
\end{exercise}
\begin{solution}
  Yes. The function $f$ can map both `$w$' and `$v$' to $w$.
\end{solution}

The soundness proof for K-trees is easily adapted to other types of trees. The
tree rules for system T, for example, are all the K-rules plus the Reflexivity
rule, which allows adding $\omega R \omega$ for every world $\omega$ on the
branch. Suppose we want to show that everything that is provable with the
T-rules is T-valid -- true at every world in every reflexive Kripke model. All
the clauses in the Soundness Lemma still hold if we assume that the model $M$ is
reflexive. We only need to add a further clause for the Reflexivity rule, to
confirm that if a branch correctly describes a reflexive model $M$, and the
branch is extended by adding $\omega R \omega$, then the resulting branch also
correctly describes $M$. This is evidently the case.

\begin{exercise}
  How would we need to adjust the soundness proof to show that the tree rules
  for K4 are sound with respect to K4-validity?
\end{exercise}
\begin{solution}
  A sentence is K4-valid iff it is true at all worlds in all transitive Kripke
  models. We only need to check that the Transitivity rule is sound, in the
  sense that if a branch correctly describes a transitive model $M$, and the
  branch is extended by the Transitivity rule, then the resulting branch also
  correctly describes $M$. (The Transitivity rule allows adding a node
  $\omega R \upsilon$ to a branch that already contains nodes $\omega R \nu$ and
  $\nu R \upsilon$. If these nodes correctly describe a transitive model then so does $\omega R \upsilon$.)
\end{solution}

\section{Completeness for trees}%
\label{sec:completenesstrees}

Let's now show that the tree rules for K are complete -- that whenever a sentence is
K-valid then there is a closed K-tree for that sentence. In fact, we will show
something stronger:
\begin{quote}
  If a sentence is K-valid, then every fully developed K-tree for the sentence is closed.
\end{quote}
By a \emph{fully developed} tree, I mean a tree on which every node on any open
branch that can be expanded (in any way) has been expanded (in this way). A
fully developed tree may be infinite.

We will prove the displayed sentence by proving its contraposition:
\begin{quote}
  If a fully developed K-tree for a sentence does not close, then the sentence is
  not K-valid.
\end{quote}
Assume, then, that some fully developed K-tree for some target sentence has at
least one open branch. We want to show that the target sentence is false at some
world in some Kripke model.

We already know how to read off a countermodel from an open branch. All we need
to do is show that this method for generating countermodels really works. Let's
first define the method more precisely.

\begin{definition}{}{inducedmodel}
  The model \textbf{induced by} a tree branch is the Kripke model $(W,R,V)$
  where
  \begin{itemize}[leftmargin=12mm]
    \itemsep-1mm
    \item[(a)] $W$ is the set of world variables on the branch,
    \item[(b)] $\omega R \upsilon$ holds in the model iff a node
          $\omega R \upsilon$ occurs on the branch,
    \item[(c)] for any sentence letter $P$, $V(P)$ is the set of world variables
          $\omega$ for which a node $P \; (\omega)$ occurs on the branch.
  \end{itemize}
\end{definition}

Next we show that all nodes on any open branch on a fully developed tree are
correct statements about the Kripke model induced by the branch.

\begin{theorem}{Completeness Lemma}{completenesslemma}
  Let $\beta$ be an open branch on a fully developed K-tree, and let
  $M = \t{W,R,V}$ be the model induced by $\beta$. Then $M,\omega \models A$ for
  all sentences $A$ and world variables $\omega$ for which $A\; (\omega)$ is on
  $\beta$.
\end{theorem}

\begin{proof}
  We have to show that whenever $A\; (\omega)$ occurs on $\beta$ then
  $M,\omega \models A$. The proof is by induction on the length of $A$. We first
  show that the claim holds for sentence letters and negated sentence letters.
  Then we show that \emph{if} the claim holds for all sentences shorter than $A$
  (this is our induction hypothesis), \emph{then} it also holds for $A$ itself.
  
  \begin{itemize}
    
    \item If $A$ is a sentence letter then the claim is true by clause (c) of
          definition \ref{def:inducedmodel} and clause (a) of definition
          \ref{def:kripkesemantics}.

    \item If $A$ is the negation of a sentence letter $\rho$, then $rho\; (\omega)$
          does not occur on $\beta$, otherwise $\beta$ would be closed. By
          clause (c) of definition \ref{def:inducedmodel}, it follows that
          $\omega$ is not in $V(\rho)$, and so $M, \omega \models A$ by clauses (a)
          and (b) of definition \ref{def:kripkesemantics}.
          
    \item If $A$ is a doubly negated sentence $\neg\neg B$, then $\beta$
          contains a node $B \;(\omega)$, because the tree is fully developed.
          By induction hypothesis, $M,\omega \models B$. By clause (b) of
          definition \ref{def:kripkesemantics}, it follows that
          $M,\omega \models A$.
    
    \item If $A$ is a conjunction $B\land C$, then $\beta$ contains nodes
          $B \;(\omega)$ and $C \;(\omega)$. By induction hypothesis,
          $M,\omega \models B$ and $M,\omega \models C$. By clause (c) of
          definition \ref{def:kripkesemantics}, it follows that
          $M,\omega \models A$.

    \item If $A$ is a negated conjunction $\neg(B\land C)$, then $\beta$ contains
          either $\neg B \;(\omega)$ or $\neg C \;(\omega)$. By induction
          hypothesis, $M,\omega \models \neg B$ or $M,\omega \models \neg C$.
          Either way, clauses (b) and (c) of definition
          \ref{def:kripkesemantics} imply that $M,\omega \models A$.
    
          % \item If $A$ is a disjunction $B\lor C$, then branch also contains
          % either $B \;(\omega)$ or $C \;(\omega)$, because the tree is fully
          % developed. By induction hypothesis, $M,\omega \models B$ or
          % $M,\omega \models C$. So $M,\omega \models B\lor C$.
    
  \end{itemize} 
  
  I skip the cases where $A$ is a disjunction, a conditional, a
  biconditional, or a negated disjunction, conditional, or biconditional. The
  proofs are similar to one (or both) of the previous two cases.

  \begin{itemize}
    
    \item If $A$ is a box sentence $\Box B$, then $\beta$ contains a node
          $B \;(\upsilon)$ for each world variable $\upsilon$ for which
          $\omega R \upsilon$ is on $\beta$ (because the tree is fully
          developed). By induction hypothesis, $M, \upsilon \models B$, for each
          such $\upsilon$. By definition \ref{def:inducedmodel}, it follows that
          $M,\upsilon \models B$ for all worlds $\upsilon$ such that
          $\omega R \upsilon$. By clause (g) of definition
          \ref{def:kripkesemantics}, it follows that $M, \omega \models \Box B$.
    
    \item If $A$ is a diamond sentence $\Diamond B$, then there is a world
          variable $\upsilon$ for which $\omega R \upsilon$ and $B \;(\upsilon)$
          are on $\beta$. By induction hypothesis, $M, \upsilon \models B$. And
          by definition \ref{def:inducedmodel}, $\omega R\upsilon$. By clause
          (h) of definition \ref{def:kripkesemantics}, it follows that
          $M, \omega \models \Diamond B$.

  \end{itemize}

  For the case where $A$ has the form $\neg \Box B$ or $\neg \Diamond B$, the
  proof is similar to one of the previous two cases. \qed
  
\end{proof}
\medskip

To establish completeness, we need to verify one more point: that one can always
construct a fully developed tree for any invalid target sentence. Let's call a
K-tree \emph{regular} if it is constructed by (i) first applying all rules for
the truth-functional connectives until no more of them can be applied (without
adding only nodes to a branch that are already on the branch), then (ii)
applying the rules for $\Diamond$ and $\neg \Box$ until no more of them can be
applied, then (iii) applying the rules for $\Box$ and $\neg \Diamond$ until no
more of them can be applied, then starting over with (i), and so on.

\begin{observation}{regulartrees}
  Every regular open K-tree is fully developed.
\end{observation}
\begin{proof}
  \emph{Proof:} When constructing a regular tree, every iteration of (i), (ii),
  and (iii) only allows expanding finitely many nodes. So every node on every
  open branch that can be expanded in any way is eventually expanded in this way
  by some iteration of (i), (ii), and (iii). \qed
\end{proof}

Now we have all the ingredients to prove completeness.

\begin{theorem}{Theorem: Completeness of K-trees}{completeness-tree-k}
  If a sentence is K-valid, then there is a closed K-tree for that sentence.
\end{theorem}

\begin{proof}
  \emph{Proof:} Let $A$ be any K-valid sentence, and suppose for reductio that
  there is no closed K-tree for $A$. In particular, then, every regular K-tree
  for $A$ remains open. Take any such tree. By observation
  \ref{obs:regulartrees}, the tree is fully expanded. Choose any open branch on
  the tree. By the Completeness Lemma, $A$ is false at $w$ in the model induced
  by that branch. So $A$ is not true at all worlds in all Kripke models.
  Contradiction. \qed
\end{proof}

\begin{exercise}
  Fill in the cases for $B \to C$ and $\neg \Diamond B$ in the proof of the
  Completeness Lemma.
\end{exercise}
\begin{solution}
  For $B\to C$: If $A$ is a conditional $B\to C$, then $\beta$ contains either
  $\neg B \;(\omega)$ or $C \;(\omega)$. By induction hypothesis,
  $M,\omega \models \neg B$ or $M,\omega \models \neg C$. Either way, clauses
  (b) and (e) of definition \ref{def:kripkesemantics} imply that
  $M,\omega \models A$.

  For $\neg\Diamond B$: If $A$ is a negated diamond sentence $\neg\Diamond B$,
  then $\beta$ contains a node $\neg B \;(\upsilon)$ for each world variable
  $\upsilon$ for which $\omega R \upsilon$ is on $\beta$ (because the tree is
  fully developed). By induction hypothesis, $M, \upsilon \models \neg B$, for
  each such $\upsilon$. By definition \ref{def:inducedmodel}, it follows that
  $M,\upsilon \models \neg B$ for all worlds $\upsilon$ such that
  $\omega R \upsilon$. By clauses (b) and (g) of definition
  \ref{def:kripkesemantics}, it follows that $M, \omega \models A$.
\end{solution}

Like the soundness proof, the completeness proof for K is easily adapted to
other logics. To show that the T-rules are complete with respect to T-validity,
for example, we merely need check that the model induced by any open branch on a
fully developed T-tree is reflexive. It must be, because an open branch on a
fully developed T-tree contains $\omega R \omega$ for each world variable
$\omega$ on the branch.

% Strictly speaking, we also need to adjust the definition of regular trees.

\begin{exercise}
  What do we need to check to show that the K4-rules are complete with respect
  to K4-validity?
\end{exercise}
\begin{solution}
  We need to check that the model induced by an open branch on a fully developed
  K4-tree is transitive. (Suppose the model contains worlds $w$, $v$, $u$ for
  which $wRv$ and $vRu$. Then the Transitivity rule has been applied to the
  corresponding nodes on the branch, generating a node $wRu$. By definition
  \ref{def:inducedmodel}, $wRu$ holds in the induced model.)
\end{solution}

\begin{exercise}\label{ex:acyclical}
  A Kripke model is \emph{acyclical} if you can never return to the same world
  by following the accessibility relation. Show that if a sentence is true at
  some world in some Kripke model, then it is also true at some world in some
  acyclical Kripke model.

  (Hint: If $A$ is true at some world in some Kripke model then $\neg A$ is
  K-invalid. By the soundness theorem, there is a fully developed K-tree for
  $\neg A$ with an open branch. Now consider the model induced by this branch.)
\end{exercise}
\begin{solution}
  Suppose $A$ is true at some world in some Kripke model. Then $\neg A$ is
  K-invalid. Take any regular K-tree for $\neg A$. By observation
  \ref{obs:regulartrees}, that tree is fully developed. By the soundness theorem
  for K-trees, the tree has an open branch. Let $M$ be the model induced by some
  such branch $\beta$. Then $M$ is acyclical. This is because the only rules
  that allow adding a node $\omega R \upsilon$ to a branch of a K-tree are the
  rules for expanding $\Diamond A$ and $\neg\Box A$ nodes. In both cases, the
  rule requires that the relevant world variable $\upsilon$ is new on the
  branch. (Call this the \emph{novelty requirement}.) Now suppose the
  accessibility relation in $M$ has a cycle $\omega_{1} R \omega_{2}$,
  $\omega_{2} R \omega_{3}$, \ldots, $\omega_{n-1} R \omega_{n}$,
  $\omega_{n} R \omega_{1}$. Each of these facts about $R$ must correspond to a
  node on $\beta$. Of these nodes, the one that was added last (to $\beta$)
  violates the novelty requirement. So $M$ is acyclical.

  By the Completeness Lemma, the target sentence $\neg\neg A$ is true at world
  $w$ in $M$. So $A$ is true at $w$ in $M$. So $A$ is true at some world in some acyclical model.
\end{solution}

% This explains why it's OK to always introduce a new world when expanding <>A.
% Technically, we only introduce a new world /name/, and the completeness lemma
% exploits this technicality. But when we read off countermodels, we assume that
% different world names denote different worlds. If -- somehow -- there could be
% sentences C that are only true in cyclical models, our method might still be
% sound: we could not prove ~C. All ~C trees would remain open. But the models
% induced by these open branches would not be countermodels to C. The open
% branches might still "accurately describe" a countermodel, under a function
% that maps different world names to the same world. And then the method might
% even be complete.

% Why is it OK to only ever expand <>A once? What if the only countermodels to
% the target sentence have /two/ witnesses of a diamond sentence? If a tree
% closes with only one witness then we've already reached a contradiction
% assuming there is at least one witness. No point expanding <>A again. If a
% tree remains open, completeness says that we have a countermodel with one
% witness. This would fail if -- somehow -- there could be sentences that are
% true only in models with multiple witnesses. That is, the model induced by an
% open branch wouldn't be a countermodel.

% \begin{exercise}
%   When constructing a tree proof, one often has a choice of which rules to
%   apply in which order. Show that the choice doesn't matter, in the following
%   sense: If there is a K-tree for the target sentence in which all branches
%   close, then there is no fully developed K-tree for the target sentence in
%   which some branch remains open. xxx Well, if you develop []<>p & (q & ~q),
%   and you keep expanding []<>p without every dealing with (q & ~q) then the
%   tree never closes. so the choice /does/ matter.
% \end{exercise}
% \begin{solution}
%   We have to show that if there some K-tree for a given target sentence closes,
%   then there is no fully developed K-tree for the sentence that remains open.
%   So assume some K-tree for the target sentence closes. By the Completeness
%   Theorem, the target sentence is K-valid. If there were a fully developed but
%   open K-tree for the same target sentence, then by the Completeness Lemma the
%   negation of that sentence would be true at world $w$ in the model induced by
%   some open branch on that tree. This contradicts the fact that the target
%   sentence is K-valid.
% \end{solution}

\begin{exercise}
  The S5 tree rules from chapter \ref{ch:worlds} are sound and complete for
  S5-validity: all and only the S5-valid sentences can be proven. Are the rules
  sound for K-validity? Are they complete for K-validity?
\end{exercise}
\begin{solution}
  The S5 rules are not sound with respect to K-validity. For example,
  $\Box p \to p$ is provable with the S5 rules, but it isn't K-valid. The rules
  are, however, complete with respect to K-validity. This follows from the
  completeness of the S5 rules and the fact that every K-valid sentence is
  S5-valid (observation \ref{obs:kins5}).
\end{solution}

\iffalse
\section{Strong completeness and compactness}%
\label{sec:compactness}

% Greg Restall: trees are upside-down cut-free sequent calculus, rephrased for
% easy teaching. That’s it. If you rephrase them just a *tiny* bit (to make them
% +/- signed trees, instead of littering every second rule with negation) then
% you have an easy and direct proof of the subformula theorem: that any valid
% argument has a purely analytic proof—a proof using only subformulas of the
% premises and the conclusion. To show that for classical predicate logic in
% natural deduction by normalisation? No fun at all!

We have shown that every K-valid sentence is provable with the tree rules for K.
This kind of completeness -- linking validity with provability -- is sometimes
called \emph{weak} completeness. Strong completeness is concerned not with
validity, but with entailment: A proof method is \textbf{strongly complete} with
respect to a logic if, whenever a sentence is entailed (in the logic) by some
sentences $\Gamma$, then there is a proof (in the method) showing that the
sentence is entailed by $\Gamma$.

You will remember from observation \ref{obs:semantic-deduction-theorem} in
chapter 1 that claims about entailment can be converted into claims about
validity. $A$ entails $B$ iff $A \to B$ is valid; $A_{1}$ and $A_{2}$ together
entail $B$ iff $A_{1} \to (A_{2} \to B)$ is valid; and so on. We can therefore
show that, say, $A_{1}$ and $A_{2}$ entail $B$ by constructing a closed tree for
$A_{1} \to (A_{2} \to B)$. After two applications of the rule for negated
conditionals, this tree has three unexpanded nodes, corresponding to the
original premises and the negated conclusion: $A_{1}\; (w)$, $A_{2}\; (w)$, and
$\neg B\; (w)$. In general, to show that some premises $\Gamma$ and entail a
conclusion $B$, we can save a few steps by directly starting the tree with one
node $A_{i}\; (w)$ for each premise $A_{i}$ in $\Gamma$, and another node
$\neg B\; (w)$ for the negated conclusion. Together with observation
\ref{obs:semantic-deduction-theorem}, the soundness and completeness results
from the previous section immediately entail the following.

\begin{observation}{sckentailment}
  Some sentences $A_{1},\ldots,A_{n}$ K-entail a sentence $B$ iff there is a
  closed K-tree with starting nodes $A_{1}\; (w)$, \ldots, $A_{n}\; (w)$,
  $\neg B\; (w)$.
\end{observation}

But this isn't enough for strong completeness. It only shows that if a
conclusion is entailed by \emph{a finite collection} of premises, then the
entailment can be verified by the tree method. One may, however, reasonably ask
whether a sentence is entailed by some infinite collection of sentences. If
$\Gamma$ is infinite then the claim that $\Gamma$ entails $B$ can't be converted
into a claim about validity, because $L_{M}$-sentences are finite. Nor can we
start a tree with separate nodes for each premise followed by a node for the
negated conclusion: With infinitely many premises we would never get to the
negated conclusion, nor would we ever get to a stage where any of these nodes is
expanded.

Here is how we can nonetheless use the tree method do deal with infinitely many
premises. Instead of starting with all the premises
$A_{1}, A_{2}, A_{3}, \ldots$ at once, we proceed in stages. In the first stage,
we try to derive the conclusion $B$ from only the first premise $A_{1}$. If
that fails, or if the tree is getting too large, we move on to the second stage,
where we try to derive $B$ from the first two premises, $A_{1}$ and $A_{2}$,
allowing for somewhat larger trees. And so on. Even though we consider only
finitely many premises at each stage, every premise will eventually be
considered, unless the tree already closes without taking it into account.

Let me describe the method in a little more detail. Suppose we want to prove
that $B$ is K-entailed by infinitely many sentences
$A_{1}, A_{2}, A_{3},\ldots$. In the first stage, we develop a K-tree whose
starting assumptions are $A_{1}\; (w)$ and $\neg B\; (w)$. We don't develop that
tree fully, because this could take forever. Instead, we develop the tree by (i)
applying all rules for the truth-functional connectives until no more of them
can be applied (without repetition), then (ii) applying the rules for $\Diamond$
and $\neg \Box$ until no more of them can be applied, and finally (iii) applying
the rules for $\Box$ and $\neg \Diamond$ until no more of them can be applied.
If at that point the tree is closed, we're done: The proof is complete. If not,
we move on to stage 2. Here we construct another tree with starting assumptions
$A_{1}\; (w)$, $A_{2}\; (w)$ and $\neg B\; (w)$. To develop this tree, we first
repeat everything we did on the first tree, expanding $A_{1}\; (w)$ and
$\neg B\; (w)$. Then we go through another iteration of (i)--(iii). If the tree
is still open, we move on to stage 3, where we add $A_{3}\; (w)$ to the starting
assumptions, copy over all nodes from stage 2 and go through another iteration
of (i)--(iii). And so on.

This method is sound. For suppose at some stage $n$ the method yields a closed
tree. This tree is an ordinary K-tree with starting assumptions $A_{1}\; (w)$,
\ldots, $A_{n}\; (w)$, and $\neg B\; (w)$. By observation
\ref{obs:sckentailment}, it follows that $A_{1},\ldots,A_{n}$ K-entail $B$. And
then all of $A_{1},A_{2},A_{3},\ldots$ K-entail $B$.

More interestingly, the described method is complete. For suppose it never
yields a closed tree. At each stage, the tree has an open branch. Note that any
open branch at a stage after the first extends an open branch from the previous
stage. So there is a sequence of branches $b_{1}, b_{2}, b_{3}, \ldots$, one
from each stage, such that $b_{2}$ extends $b_{1}$, $b_{3}$ extends $b_{2}$, and
so on. Pick any such sequence. We may regard the set of nodes that occur
somewhere in the sequence as a single infinite branch. This infinite branch
contains a node $A_{i}\; (w)$ for every premise, as well as $\neg B\; (w)$. Just
as an ordinary open branch induces a Kripke model (as per definition
\ref{def:inducedmodel}), so does the infinite branch. And because every node on
the branch that can be expanded has been expanded, the Completeness Lemma goes
through just as in the previous section, showing that all nodes on the branch
are correct statements about the induced model. So there is a world in some
Kripke model at which all of $A_{1},A_{2},A_{3},\ldots$ are true and $B$ is
false.

\begin{theorem}{Theorem: Strong completeness of K-trees}{strongcompleteness}
  If a sentence $B$ is K-entailed by some sentences $\Gamma$ then there is a
  closed K-tree starting with $A_{1}\; (w)$, \ldots, $A_{n}\; (w)$ and
  $\neg B\; (w)$, where all of $A_{1},\ldots,A_{n}$ are in $\Gamma$.
\end{theorem}
\begin{proof}
  \emph{Proof:} Consider a sequence of K-trees, the first beginning with
  $A_{1}\; (w)$ and $\neg B\; (w)$, the second with $A_{1}\; (w)$,
  $A_{2}\; (w)$, and $\neg B\; (w)$, and so on -- each extending the previous
  one by an added premise from $\Gamma$ and one iteration of (i)--(iii). As
  we've just seen, if no tree in this sequence closes, then $B$ is not K-entailed
  by $\Gamma$. \qed
\end{proof}
%
Our proof easily carries over to other systems of modal logic. The only part
that needs adapting is the specification of (i)--(iii), where we must add the
extra tree rules in some way that ensures that all nodes are eventually
expanded.

Strong completeness has a noteworthy corollary:
%
\begin{theorem}{Theorem: Compactness of K}{compactness}
  If a sentence $B$ is K-entailed by some sentences $\Gamma$ then $B$ is
  K-entailed by a finite subset of $\Gamma$.
\end{theorem}
\begin{proof}
  \emph{Proof:} By strong completeness, if some sentences $\Gamma$ K-entail $B$,
  then there is a closed K-tree starting with $A_{1}\; (w)$, \ldots,
  $A_{n}\; (w)$ and $\neg B\; (w)$, where $A_{1},\ldots,A_{n}$ are finitely many
  sentences from $\Gamma$. By observation \ref{obs:sckentailment}, it follows
  that $B$ is entailed by $A_{1},\ldots,A_{n}$. \qed
\end{proof}

In general, a logic $S$ is called \textbf{compact} if any sentence that is
$S$-entailed by infinitely many sentences is also $S$-entailed by a
finite subset of these sentences.

Compactness is surprising. It is easy to think of cases in which a conclusion is
entailed by infinitely many premises, but not by any finite subset of these
premises. For example, suppose I like the number 0, I like the number 1, I like
the number 2, and so on, for all natural numbers 0,1,2,3,\ldots. Together, these
assumptions entail that I like all natural numbers. But no finite subset of the
assumptions entails that I like all natural numbers.

\begin{exercise}
  A set of sentences $\Gamma$ is called \emph{K-satisfiable} if there is a world
  in some Kripke model at which all members of $\Gamma$ are true. Show that an
  infinite set of sentences $\Gamma$ is K-satisfiable iff every finite subset of
  $\Gamma$ is K-satisfiable.
\end{exercise}
\begin{solution}
  Let $\Gamma$ is an infinite set of $\L_{M}$-sentences. If $\Gamma$ is
  K-satisfiable then obviously every finite subset of $\Gamma$ is satisfiable as
  well. For the converse direction, assume $\Gamma$ is not K-satisfiable: There
  is no world in any Kripke model at which all members of $\Gamma$ are true.
  Then there is no world in any Kripke model at which all members of $\Gamma$
  are true while $p \land \neg p$ is false. So $\Gamma \models p\land \neg p$.
  By the compactness theorem, it follows that there is a finite subset
  $\Gamma^{-}$ for which $\Gamma^{-} \models p \land \neg p$. If
  $\Gamma^{-} \models p \land \neg p$ then there is no world in any Kripke model
  at which all members of $\Gamma^{-}$ are true while $p \land \neg p$ is false.
  Since $p\land \neg p$ is false at every world in every Kripke model, it
  follows that there is no world in any Kripke model at which all members of
  $\Gamma^{-}$ are true. This shows that if $\Gamma$ is not K-satisfiable then
  there is a finite subset ($\Gamma^{-}$) of $\Gamma$ that is not K-satisfiable.
  Conversely, if every finite subset of $\Gamma$ is K-satisfiable then $\Gamma$
  is K-satisfiable.
\end{solution}

\fi 

\section{Soundness and completeness for axiomatic calculi}
\label{sec:scaxiomatic}

Next, we are going to show that the axiomatic calculus for system K is sound and
complete for K-validity. In the axiomatic calculus, a proof is a list of
sentences each of which is either an instance of \pr{Dual} or \pr{K} or can be
derived from earlier sentences on the list by application of \pr{CPL} or
\pr{Nec}. Expressed as a construction rule, \pr{Nec} says that whenever a list
contains a sentence $A$ then one may append $\Box A$. \pr{CPL} says that one may
append any truth-functional consequence of sentences that are already on the
list. (This is an acceptable rule because there is a simple mechanical test --
the truth-table method -- for checking whether a sentence is a truth-functional
consequence of finitely many other sentences.)

Soundness is easy. We want to show that everything that is derivable from some
instances of \pr{Dual} and \pr{K} by applications of \pr{CPL} and \pr{Nec} is
K-valid. We show this by showing that (1) every instance of \pr{Dual} and \pr{K}
is K-valid, and (2) every sentence that is derived from K-valid sentences by
\pr{CPL} or \pr{Nec} is itself K-valid.

\begin{theorem}{Theorem: Soundness of the axiomatic calculus for K}{soundnessK}
  Any sentence that is provable in the axiomatic calculus for K is K-valid.
\end{theorem}

\begin{proof}
  \emph{Proof:} We first show that every instance of \pr{Dual} and \pr{K} is
  K-valid.
  \begin{enumerate}[leftmargin=7mm]
  \itemsep0mm
  
    \item \pr{Dual} is the schema $\neg\Diamond A \leftrightarrow \Box\neg A$.
          By clauses (b), (g), and (h) of definition \ref{def:kripkesemantics},
          a sentence $\neg\Diamond A$ is true at a world $w$ in a Kripke model
          $M$ iff $\Box\neg A$ is true at $w$ in $M$. It follows by clauses (f)
          and (e) that all instances of \pr{Dual} are true at all worlds in all
          Kripke models.
          
    \item \pr{K} is the schema $\Box(A \to B) \to (\Box A \to \Box B)$. By
          clause (e) of definition \ref{def:kripkesemantics}, a sentence
          $\Box(A\to B) \to (\Box A \to \Box B)$ is false at a world $w$ in a
          Kripke model only if $\Box(A \to B)$ and $\Box A$ are both true at
          $w$ while $B$ is false. By clause (g) of definition
          \ref{def:kripkesemantics}, $\Box B$ is false at $w$ only if $B$ is
          false at some world $v$ accessible from $w$. But if $\Box(A \to B)$
          and $\Box A$ are both true at $w$, then $A\to B$ and $A$ are true at
          every world accessible from $w$, again by clause (g). And there can be
          no world at which $A\to B$ and $A$ are true while $B$ is false, by
          clause (e) of definition \ref{def:kripkesemantics}.
  \end{enumerate}
  %
  Next we show that \pr{CPL} and \pr{Nec} preserve K-validity.
  %
  \begin{enumerate}[leftmargin=9mm]
    
    \item By definition \ref{def:kripkesemantics}, the truth-functional
          operators have their standard truth-table meaning at every world in
          every Kripke model. It follows that all truth-functional consequences
          of sentences that are true at a world are themselves true at that
          world. In particular, if some sentences are true at every world in
          every Kripke model, then any truth-functional consequence of these
          sentences is also true at every world every Kripke model.

    \item Let $w$ be an arbitrary world in an arbitrary Kripke model. If $A$ is
          true at every world in every Kripke model, then $A$ is true at every
          world accessible from $w$, in which case $\Box A$ is true at $w$ by
          clause (g) of definition \ref{def:kripkesemantics}. So if $A$ is K-valid, then $\Box A$ is also K-valid. \qed
          
  \end{enumerate}
\end{proof}

The soundness proof for K is easily extended to other modal systems. Since all
instances of \pr{Dual} and \pr{K} are true at all worlds in all Kripke models,
they are also true at all worlds in any more restricted class of Kripke models.
The arguments for \pr{CPL} and \pr{Nec} also go through if we replace `every
Kripke model' by `every Kripke model of such-and-such type'. So if we want to
show that, say, the axiomatic calculus for T is sound with respect to the
concept of T-validity -- that is, if we want to show that anything that is
derivable from \pr{Dual}, \pr{K}, and \pr{T} by \pr{CPL} and \pr{Nec} is true at
all worlds in all reflexive Kripke models -- all that is left to do is to show
that every instance of the \pr{T}-schema is true at all worlds in all reflexive
Kripke model. (We've already shown this: see observation \ref{obs:treflexive}.)

% In general, if $\Gamma$ is any set of sentences that are valid in a given
% class C of frames, anything that's provable from $K+\Gamma$ is valid in that
% class. (HC.39)

\begin{exercise}
  Outline the soundness proof for the axiomatic calculus for S4, whose axiom
  schemas are \pr{Dual}, \pr{K}, \pr{T}, and \pr{4}.
\end{exercise}
\begin{solution}
  We need to show that everything that's derivable in the axiomatic calculus for
  S4 is true at every world in every transitive and reflexive Kripke model. From
  the soundness proof for K, we know that all instances of \pr{Dual} and \pr{K}
  are true at every world in every Kripke model. From observation
  \ref{obs:treflexive}, we know that all instances of \pr{T} are true at every
  world in every reflexive Kripke model. From observation \ref{obs:4trans}, we
  know that all instances of \pr{4} are true at every world in every transitive
  Kripke model. So all axioms in the S4-calculus are valid in the class of
  transitive and reflexive Kripke frames. Since \pr{CPL} and \pr{Nec} preserve
  validity in any class of Kripke frames, it follows that everything that's
  derivable in the S4-calculus is valid in the class of transitive and reflexive
  frames.
\end{solution}

Let's turn to completeness. We are going to show that every K-valid sentence is
derivable from some instances of \pr{Dual} and \pr{K} by \pr{CPL} and \pr{Nec}.
% You might want to grab a cup of tea before you continue.
As in section \ref{sec:completenesstrees}, we argue by contraposition. We will
show that any sentence that cannot be derived from \pr{Dual} and \pr{K} by
\pr{CPL} and \pr{Nec} is not K-valid. To show that a sentence is not K-valid, we
will give a countermodel -- a Kripke model in which the sentence is false at
some world. In fact, we will give the \emph{same} countermodel for every
sentence that isn't derivable in the calculus. You might think we need different
countermodels for different sentences, but it turns out that there is a
particular model in which every K-invalid sentence is false at some world. This
model is called the \emph{canonical model} for K.

In order to define the canonical model, let's introduce some shorthand
terminology. We'll say that an $\L_M$-sentence is \emph{K-provable} if it can be
proved in the axiomatic calculus for K. A set of $\L_M$-sentences is
\emph{K-inconsistent} if it contains a finite number of sentences
$A_1,\ldots,A_n$ such that $\neg (A_1 \land \ldots \land A_n)$ is K-provable. A
set is \emph{K-consistent} if it is not K-inconsistent.

(For example, the set $\{ \Box(p \land q), q \to p, \neg \Box q \}$ is
K-inconsistent, because it contains two sentences, $\Box(p \land q)$ and
$\neg \Box q$ whose conjunction is refutable in K, in the sense that the
negation $\neg(\Box(p \land q) \land \neg \Box q)$ of their conjunction is
derivable from some instances of \pr{Dual} and \pr{K} by \pr{CPL} and \pr{Nec}.)

A set of $\L_M$-sentences is called \emph{maximal} if it contains either $A$ or
$\neg A$ for every $\L_M$-sentence $A$. A set is \emph{maximal K-consistent} if
it is both maximal and K-consistent.

% Exercise (HC 114): Show that if $\Gamma$ is maximal consistent, then for any
% sentences $A,B$, (i) $A\in \Gamma$ iff $\neg A \not\in \Gamma$, (ii)
% $A \lor B \in \Gamma$ iff either $A \in \Gamma$ or $B \in \Gamma$, (iii) $A \land B$ ...

\begin{exercise}
  Which, if any, of these sets are K-consistent? (a) $\{ p \}$, (b)
  $\{ \neg p \}$, (c) the set of all sentence letters, (d) the set of all
  $\L_{M}$-sentences.
\end{exercise}
\begin{solution}
  (a), (b), and (c) are K-consistent, (d) is not.
\end{solution}

% The universe of the canonical model is always uncountable since there are
% uncountably many (2^\omega) sets of negated/unnegated sentence letters.

Now here's the canonical model for K.

\begin{definition}{}{cmk}
  %\leavevmode\vspace{-2em}
  
  The \textbf{canonical model} $M_K$ for K is the Kripke model $\t{W,R,V}$, where%
  \vspace{-2mm}
  \begin{itemize}[leftmargin=7mm]
    \itemsep-1mm
    \item $W$ is the set of all maximal K-consistent sets of $\L_M$-sentences,
    \item $wRv$ iff $v$ contains every sentence $A$ for which $w$ contains
          $\Box A$,
    \item for every sentence letter $P$, $V(P)$ is the set of all members of $W$
          that contain $P$.
  \end{itemize}
\end{definition}

% Exercise: show that for any consistent normal modal logic, the canonical
% accessibility relation R is such that wRv iff for all B\in v we have <>B\in w.

The ``worlds'' in the canonical model are sets of $\L_M$-sentences. The
interpretation function makes a sentence letter true at a world iff the letter
is a member of the world. As we are going to see, this generalizes to arbitrary
sentences:
\begin{enumerate}[leftmargin=10mm]
  \item[(1)] A world $w$ in $M_K$ contains all and only the sentences that are
        true at $w$ in $M_K$.
\end{enumerate}
We will also prove the following:
\begin{enumerate}[leftmargin=10mm]
  \item[(2)] If some sentence cannot be proved in the axiomatic calculus for K,
        then its negation is a member of some world in $M_K$.
\end{enumerate}

Together, these two lemmas will establish completeness for the axiomatic
calculus. Fact (2) tells us that if a sentence $A$ isn't K-provable, then
$\neg A$ is a member of some world $w$ in the canonical model $M_K$. By fact
(1), we can infer that $\neg A$ is true at $w$ in $M_K$, which means that $A$ is
false at $w$ in $M_K$. So any sentence that isn't K-provable isn't K-valid.

We are going to prove (2) first. We'll need the following observation. 

\begin{observation}{extensionconsistency}
  If a set $\Gamma$ is K-consistent, then for any sentence $A$, either
  $\Gamma \cup \{ A \}$ or $\Gamma \cup \{ \neg A \}$ is K-consistent.
\end{observation}
%
\noindent
($\Gamma \cup \{ A \}$, called the \emph{union} of $\Gamma$ and $\{ A \}$, is
the smallest set that contains all members of $\Gamma$ as well as $A$.)
%
\begin{proof}
  \emph{Proof}: Let $\Gamma$ be any K-consistent set and $A$ any sentence.
  Suppose for reductio that $\Gamma \cup \{ A \}$ and
  $\Gamma \cup \{ \neg A \}$ are both K-inconsistent.

  That $\Gamma \cup \{ A \}$ is K-inconsistent means there are sentences
  $A_1,\ldots,A_n$ in $\Gamma \cup \{ A \}$ such that
  $\neg (A_1\land\ldots\land A_n)$ is K-provable. Since $\Gamma$ itself is
  K-consistent, one of the sentences $A_1,\ldots,A_n$ must be $A$. Let $B$ be
  the conjunction of the other sentences in $A_1,\ldots,A_n$, all of which are
  in $\Gamma$. So $\neg(B \land A)$ is K-provable.

  That $\Gamma \cup \{ \neg A \}$ is K-inconsistent means that there are sentences $A_1,\ldots,A_n$ in $\Gamma \cup \{ \neg A \}$ such that $\neg (A_1\land\ldots\land A_n)$ is K-provable. As before, one of these sentences must be $\neg A$. Let $C$ be the conjunction of the others, all of which are in $\Gamma$. So $\neg(C \land \neg A)$ is K-provable.

  If $\neg(B \land A)$ and $\neg(C \land \neg A)$ are both K-provable, then so
  is $\neg(B \land C)$, because it is a truth-functional consequence of
  $\neg(B \land A)$ and $\neg(C \land \neg A)$. But $B \land C$ is a conjunction
  of sentences from $\Gamma$. So $\Gamma$ itself is K-inconsistent, contradicting our assumption. \qed
\end{proof}

Now we can prove fact (2).

\begin{theorem}{Lindenbaum's Lemma}{extensibility}
  Every K-consistent set is a subset of some maximal K-consistent set.
\end{theorem}
%
\begin{proof}
  \emph{Proof}: Let $S_0$ be some K-consistent set of sentences. Let
  $A_1,A_2,\ldots$ be a list of all $\L_M$-sentences in some arbitrary
  order. For every number $i\geq 0$, define
  \begin{gather*}
    S_{i+1} = \begin{cases} S_i \cup \{ A_i \} & \text{if $S_i \cup \{ A_i \}$ is K-consistent}\\
      S_i \cup \{ \neg A_i \} & \text{otherwise}.
    \end{cases}
  \end{gather*}
%
  This gives us an infinite list of sets $S_0,S_1,S_2,\ldots$. Each set in the
  list is K-consistent: $S_0$ is K-consistent by assumption. And if some set
  $S_i$ in the list is K-consistent, then either $S_i \cup \{ A_i \}$ is
  K-consistent, in which case $S_{i+1} = S_i \cup \{ A_i \}$ is K-consistent, or
  $S_i \cup \{ A_i \}$ is not K-consistent, in which case $S_{i+1}$ is
  $S_i \cup \{ \neg A_i \}$, which is K-consistent by observation
  \ref{obs:extensionconsistency}. So if any set in the list is K-consistent, then
  the next set in the list is also K-consistent. It follows that
  $S_0,S_1,S_2,\ldots$ are all K-consistent.

  Now let $S$ be the set of sentences that occur in at least one of the sets
  $S_{0},S_1, S_2,S_3\ldots$. (That is, let $S$ be the union of
  $S_{0},S_1,S_2,S_3,\ldots$.) Evidently, $S_0$ a subset of $S$. And $S$ is
  maximal. Moreover, $S$ is K-consistent. For if $S$ were not K-consistent, then
  it would contain some sentences $B_1,\ldots,B_n$ such that
  $\neg (B_1\land \ldots\land B_n)$ is K-provable. All of these sentences would
  have to occur somewhere on the list $A_1,A_2,\ldots$. Let $A_j$ be a sentence
  from $A_1,A_2,\ldots$ that occurs after all the $B_1,\ldots,B_n$. If
  $B_1,\ldots,B_n$ are in $S$, they would have to be in $S_j$ already, so $S_j$
  would be K-inconsistent. But we've seen that all of $S_0,S_1,S_2,\ldots$ are
  K-consistent. \qed
\end{proof}

Notice that the proof of Lindenbaum's Lemma does not turn on any assumptions
about the axiomatic calculus for K except that \pr{CPL} is one of its rules. The
lemma holds for every calculus with \pr{CPL} as a (possibly derived) rule.

To prove fact (1), we need another observation, which relies on the presence of
\pr{K} and \pr{Nec}, besides \pr{CPL}.

\begin{observation}{hcp117}
  If $\Gamma$ is a maximal K-consistent set of sentences that does not
  contain $\Box A$, and $\Gamma^{-}$ is the set of all sentences $B$ for which
  $\Box B$ is in $\Gamma$, then $\Gamma^{-} \cup \{ \neg A \}$ is
  K-consistent.
\end{observation}
\begin{proof}
  \emph{Proof:} We show that if $\Gamma^{-} \cup \{ \neg A \}$ is not
  K-consistent, then neither is $\Gamma$. If $\Gamma^- \cup \{ \neg A \}$ is not
  K-consistent, then there are sentences $B_1,\ldots,B_n$ in $\Gamma^{-}$ such
  that $\neg(B_1\land\ldots\land B_n \land \neg A)$ is K-provable. And then
  $(B_1\land\ldots\land B_n) \to A$ is K-provable, because it is a
  truth-functional consequence of $\neg(B_1\land\ldots\land B_n \land \neg A)$.
  By repeated application of \pr{Nec}, \pr{K}, and \pr{CPL}, one can derive
  $(\Box B_1\land\ldots\land \Box B_n) \to \Box A$ from
  $(B_1\land\ldots\land B_n) \to A$. Another application of \pr{CPL} yields
  $\neg (\Box B_1\land\ldots\land \Box B_n \land \neg\Box A)$. So
  $\{\Box B_1,\ldots,\Box B_n, \neg \Box A\}$ is K-inconsistent. But
  $\Box B_1,\ldots,\Box B_n$ are in $\Gamma$. And since $\Box A$ is not in
  $\Gamma$ and $\Gamma$ is maximal, $\neg \Box A$ is in $\Gamma$. So
  $\{\Box B_1,\ldots,\Box B_n, \neg \Box A\}$ is a subset of $\Gamma$. And so
  $\Gamma$ is K-inconsistent. \qed
\end{proof}

Here, then, is fact (1):

\begin{theorem}{Canonical Model Lemma}{lemma:cml}
  For any world $w$ in $M_K$ and any sentence $A$, $A$ is in $w$ iff
  $M_K,w \models A$.
\end{theorem}

\begin{proof}
  \emph{Proof:} The proof is by induction on complexity of $A$. We first show
  that the claim (that $A$ is in $w$ iff $M_{K},w\models A$) holds for sentence
  letters. Then we show that if the claim holds for the immediate parts of a
  complex sentence (this is our induction hypothesis), then the claim also
  holds for the sentence itself.%
  \vspace{-1mm}
  \begin{itemize}
    \itemsep0mm
    \item Suppose $A$ is a sentence letter. By definition \ref{def:cmk},
          $w\in V(A)$ iff $A\in w$. So by clause (a) of definition
          \ref{def:kripkesemantics}, $M_K,w \models A$ iff $A \in w$. (`$\in$'
          means `is a member of the set'.)
  
    \item Suppose $A$ is a negation $\neg B$. By clause (b) of definition
          \ref{def:kripkesemantics}, $M_K,w \models \neg B$ iff
          $M_K,w \not\models B$. By induction hypothesis, $M_K,w\not\models B$
          iff $B \not\in w$. Since $w$ is maximal K-consistent, $B \not\in w$
          iff $\neg B \in w$. So $M_K,w \models \neg B$ iff $\neg B \in w$.
          % I haven't actually shown that maximal K-consistency implies $B
          % \not\in w iff \neg B \in w$.
        
    \item Suppose $A$ is a conjunction $B\land C$. By clause (c) of definition
          \ref{def:kripkesemantics}, $M_K,w \models B\land C$ iff
          $M_K,w \models B$ and $M_K,w \models C$. By induction hypothesis,
          $M_K,w \models B$ iff $B\in w$, and $M_K,w \models C$ iff $C\in w$.
          Since $w$ is maximal K-consistent, $B$ and $C$ are in $w$ iff
          $B\land C$ is in $w$.
          % I haven't actually shown this.
          So $M_K,w \models B \land C$ iff $B \land C \in w$.
  \end{itemize}
  %
  The cases for the other truth-functional connectives are similar.
  \begin{itemize}
    \itemsep0mm
    \item Suppose $A$ is a box sentence $\Box B$, and that $\Box B \in w$. By
          definition \ref{def:cmk}, it follows that $B\in v$ for all $v$ with
          $wRv$. By induction hypothesis, this means that $M_K,v \models B$ for
          all $v$ with $wRv$. And then $M_K,w \models \Box B$, by clause (g) of
          definition \ref{def:kripkesemantics}.

          For the converse direction, suppose $\Box B \not\in w$. Let $\Gamma^-$
          be the set of all sentences $C$ for which $\Box C \in w$. By
          observation \ref{obs:hcp117}, $\Gamma^- \cup \{ \neg B \}$ is
          K-consistent. By definition \ref{def:cmk} and Lindenbaum's Lemma, it
          follows that there is some $v\in W$ such that $wRv$ and
          $\neg B \in v$. Since $v$ is K-consistent, $B \not\in v$. By induction
          hypothesis, it follows that $M_K,v \not\models B$. And so
          $M_K,w \not\models \Box B$, by clause (g) of definition
          \ref{def:kripkesemantics}.
          
    \item Suppose $A$ is a diamond sentence $\Diamond B$, and that
          $\Diamond B \in w$. By \pr{Dual} and \pr{CPL}, any set that contains
          both $\Diamond B$ and $\Box \neg B$ is K-inconsistent. So
          $\Box \neg B \not\in w$. By observation \ref{obs:hcp117} and
          Lindenbaum's Lemma (as in the previous case), it follows that there is
          some $v\in W$ such that $wRv$ and $B \in v$. By induction hypothesis,
          $M,v\models B$. So $M_{K},w\models \Diamond B$, by clause (h) of
          definition \ref{def:kripkesemantics}.

          For the converse direction, suppose $\Diamond B \not\in w$. Then
          $\Box \neg B \in w$, by \pr{Dual}, \pr{CPL}, and the fact that $w$ is
          maximal K-consistent. By definition \ref{def:cmk}, it follows that
          $\neg B\in v$ for all $v$ with $wRv$. Since all such $v$ are maximal
          K-consistent, none of them contain $B$. By induction hypothesis, $B$
          is not true at any of them. By clause (h) of definition
          \ref{def:kripkesemantics}, it follows that
          $M_{K}, w \not\models \Diamond B$. \qed
  \end{itemize}
\end{proof}

The completeness of the axiomatic calculus for K follows immediately from the
previous two lemmas, as foreshadowed above:

\begin{theorem}{Theorem: Completeness of the axiomatic calculus for K}{completenessK}
  If $A$ is K-valid, then $A$ is provable in the axiomatic calculus for K.
\end{theorem}
%
\begin{proof}
  \emph{Proof}: We show that if a sentence is not K-provable then it is not
  K-valid. Suppose $A$ is not K-provable. Then $\{ \neg A \}$ is K-consistent.
  It follows by Lindenbaum's Lemma that $\{ \neg A \}$ is included in some
  maximal K-consistent set $S$. By definition \ref{def:cmk}, that set is a world
  in $M_K$. Since $\neg A$ is in $S$, it follows from the Canonical Model Lemma
  that $M_K,S \models \neg A$. So $M_K,S \not\models A$. So $A$ is not true at
  all worlds in all Kripke models. \qed
\end{proof}

Done!

% We have shown that the axiomatic proof system for K is rightly
% called a proof system \emph{for K} because the sentences that are
% provable in this system are precisely the sentences that are K-valid.

Once again, the proof is easily adjusted to many axiomatic calculi for logics
stronger than K. All we have assumed about the K-calculus is that it contains
\pr{Dual}, \pr{K}, \pr{Nec}, and \pr{CPL}. So if we're interested in, say,
whether the axiomatic calculus for T is complete, we can simply replace
`K-consistent' by `T-consistent' throughout the proof, and almost everything
goes through as before. We only have to add a small step at the end.

% Did we also need the assumption that the calculus is consistent? Perhaps not.
% If the calculus is inconsistent and contains CPL then everything we said
% should hold vacuously.

By adapting the argument for K, we can show that if a sentence $A$ is not
T-provable then $A$ is false at some world in the canonical model for T. This
shows that $A$ is not K-valid. But we want to show that $A$ is not T-valid --
meaning that $A$ is not true at all worlds in all reflexive Kripke models. To
complete the proof, we need to show that the canonical model $M_{T}$ for T is
reflexive.

This isn't hard. Given how accessibility in canonical models is defined, a world
$w$ in a canonical model is accessible from itself iff whenever $\Box A \in w$
then $A \in w$. Since the worlds in $M_T$ are maximal T-consistent sets of
sentences, and every such set contains every instance of the \pr{T} schema
$\Box A \to A$, there is no world in $M_T$ that contains $\Box A$ but not $A$.
So every world in $M_T$ has access to itself.

In general, to show that a calculus that extends the K-calculus by further axiom
schemas is complete, we only need to show that the canonical model for the
calculus satisfies the frame conditions that correspond to the added axiom
schemas. This is usually the case. But not always. Sometimes, an axiomatic
calculus is sound and complete with respect to some class of Kripke models, but
the canonical model of the calculus is not a member of that class. (An example is
the calculus for the system GL, which I will describe at the very end of this
chapter.) Completeness must then be established by some other means.

% The present technique for proving completeness does not always work. In the next
% section, we will meet an axiomatic calculus \Ax{GL} that is
% characterised by a certain class of frames, but its canonical model doesn't
% belong to that class.  Worse, the axioms of \Ax{GL} are not valid on the frame
% of its canonical model. So the above technique can't be used to prove
% completeness. Other (normal) calculi are not even characterised by any class of
% frames. An example is the calculus \Ax{KH}, which results from \Ax{K} by adding
% the axiom schema
% %
% \principle{H}{\Box(\Box A \leftrightarrow A) \to \Box A}

\begin{exercise}
  Outline the completeness proof for the axiomatic calculus for S5.
\end{exercise}
\begin{solution}
  We have to show that all S5-valid sentences are provable in the axiomatic
  calculus for S5, which extends the calculus for T by the axiom schemas
  $\Box A \to \Box\Box A$ and $\Diamond A \to \Box\Diamond A$. (The second
  schema alone would be sufficient, as I mentioned in chapter 1, but it doesn't
  hurt to have the first.) The argument is by contraposition: We suppose that
  some sentence is not S5-provable and show that it is not S5-valid.

  Suppose $A$ is not S5-provable. Then $\{ \neg A \}$ is S5-consistent. It
  follows by Lindenbaum's Lemma that $\{ \neg A \}$ is included in some maximal
  S5-consistent set $\Gamma$. By definition of canonical models, this set is a
  world in the canonical model $M_{S5}$ for S5. Since $\neg A$ is in $\Gamma$,
  it follows from the Canonical Model Lemma that $M_{S5},\Gamma \models \neg A$.
  So $M_{S5},S \not\models A$.

  It remains to show that the accessibility relation in $M_{S5}$ has the right
  formal properties. We know that a sentence is S5-validity iff it is valid in
  the class of Kripke models whose accessibility relation is an equivalence
  relation. So we will show that the accessibility relation in $M_{S5}$ is reflexive, transitive, and symmetric.

  By definition, a world $v$ in a canonical model is accessible from $w$ iff
  whenever $\Box A \in w$ then $A \in v$. Since the worlds in $M_{S5}$ are
  maximal S5-consistent sets of sentences, and every such set contains every
  instance of the \pr{T}-schema $\Box A \to A$, there is no world in $M_{S5}$
  that contains $\Box A$ but not $A$. So every world in $M_{S5}$ has access to
  itself.

  For transitivity, suppose for some worlds $w,v,u$ in $M_{S5}$ we have $wRv$
  and $vRu$. We need to show that $wRu$. Given how $R$ is defined in $M_{S5}$,
  we have to show that $u$ contains all sentences $A$ for which $w$ contains
  $\Box A$. So let $A$ be an arbitrary sentence for which $w$ contains $\Box
  A$. Since every world in $M_{S5}$ contains every instance of
  $\Box A \to \Box\Box A$, we know that $w$ also contains $\Box\Box A$. From
  $wRv$, we can infer that $v$ contains $\Box A$. And from $vRu$, we can infer
  that $u$ contains $A$. 

  For symmetry, suppose for some worlds $w,v$ in $M_{S5}$ we have $wRv$ and not
  $vRw$. Given how $R$ is defined, this means that there is some sentence $A$
  for which $\Box A$ is in $v$ but $\neg A$ is in $w$. Since $w$ contains the
  T-provable sentence $\neg A \to \Diamond \neg A$ and the \pr{5}-instance
  $\Diamond \neg A \to \Box \Diamond \neg A$, it also contains
  $\Box \Diamond \neg A$. So $v$ contains $\Diamond \neg A$. This contradicts
  the assumption that $v$ is S5-consistent, given that $v$ contains $\Box A$. 
\end{solution}

% in ch.2 I say we will prove here that S5 is the logic of "basic" models. Do we?

% We can show that S5 is complete wrt the simple semantics as per Humberstone
% 72: if A is not in S5 then by completeness for equivalence frames there is a
% world in an equivalence model where A is false. Take the submodel generated by
% that world...

% Exercise (H65): show that if all instances of T are provable in a consistent
% normal logic, then the canonical acc rel is reflexive. (Similarly for
% transitive with 4 and serial with D)

\begin{exercise}
  The set of all $\L_{M}$-sentences is a system of modal logic. Let's call this
  system $X$ (for ``explosion''). (a) Describe a sound and complete proof method
  for $X$. (b) Explain why $X$ does not have a canonical model.
\end{exercise}
\begin{solution}
  (a) Method A from exercise \ref{ex:proofmethods} is sound and complete for
  $X$. (b) No set of $\L_{M}$-sentences is $X$-consistent, but every Kripke model
  must have at least one world.
\end{solution}

% \begin{exercise}
%   Suppose we add the (``McKinsey'') schema $\Box\Diamond A \to \Diamond \Box A$
%   to the axiomatic calculus for S5. Explain why we can then prove all instances
%   of the (``Triv'') schema $\Box A \leftrightarrow A$.
%    % 1. A -> <>A (T, PL)
%    % 2. <>A -> []<>A (p5)
%    % 3. []<>A -> <>[]A (M)
%    % 4. A -> <>[]A (1,2,3)
%    % 5. <>[]A -> []A (p5, PL)
%    % 6. A -> []A (1,5)
% \end{exercise}
% \begin{solution}
%   \begin{alignat*}{2}
%     1.\quad& A \to \Diamond A &\quad& \text{(\pr{T}, Prop.\ Logic)}\\
%     2.\quad& \Diamond A \to \Box \Diamond A &\quad& \text{(\pr{5})}\\
%     3.\quad& \Box\Diamond A \to \Diamond\Box A &\quad& \text{(\pr{M})}\\
%     4.\quad& \Diamond\Box A \to \Box A &\quad& \text{(\pr{5}, Prop.\ Logic)}\\
%     5.\quad& A \to \Box A &\quad& \text{(1, 2, 3, 4, Prop.\ Logic)}
%   \end{alignat*}
% \end{solution}


\section{Loose ends}\label{sec:looseends}

You will remember from observation \ref{obs:semantic-deduction-theorem} in
chapter 1 that claims about entailment can be converted into claims about
validity. $A$ entails $B$ iff $A \to B$ is valid; $A_{1}$ and $A_{2}$ together
entail $B$ iff $A_{1} \to (A_{2} \to B)$ -- equivalently,
$(A_{1}\land A_{2}) \to B$ -- is valid; and so on. But what if there are
infinitely many premises $A_{1},A_{2},A_{3},\ldots$? Sentences of $\L_{M}$ are
always finite, so we can't convert the claim that $A_{1},A_{2},A_{3},\ldots$
entail $B$ into a claim that some $\L_{M}$-sentence is valid.

We also can't use the tree method or the axiomatic method to directly show that
a conclusion follows from infinitely many premises. A proof in either method is
a finite object that can only invoke finitely many sentences.

As it turns out, this is not a serious limitation. In many logics -- including
classical propositional and predicate logic and all the modal logics we have so
far encountered -- a sentence is entailed by infinitely many premises only if it
is entailed by a finite subset of these premises. Logics with this property are
called \textbf{compact}.

Let's show that K is compact. To this end, I'll say that a sentence $B$ is
\emph{K-derivable} from a (possibly infinite) set of sentences $\Gamma$ if there
are finitely many members $A_{1},\ldots,A_{n}$ of $\Gamma$ for which
$(A_1 \land \ldots \land A_n) \to B$ is provable in the axiomatic calculus for
K. Now we first show that whenever $\Gamma \models_{K} B$ then $B$ is
K-derivable from $\Gamma$. This is called \emph{strong completeness} because it
is stronger than the (``weak'') kind of completeness that we have established in
the previous section.

\begin{theorem}{Theorem: Strong completeness of the axiomatic calculus for K}{strongcompletenessK}
  Whenever $\Gamma \models_{K} B$ then $B$ is K-derivable from $\Gamma$.
\end{theorem}
\begin{proof}
  \emph{Proof}: Suppose $B$ is not K-derivable from $\Gamma$. Then there are no
  $A_1, \ldots, A_n$ in $\Gamma$ such that $(A_1 \land \ldots \land A_n) \to B$
  is K-provable. This means that $\Gamma \cup \{ \neg B \}$ is K-consistent. By
  Lindenbaum's Lemma, it follows that $\Gamma \cup \{ \neg B \}$ is included in
  some maximal K-consistent set and thereby in some world in the canonical model
  $M_{K}$ for K. (Lindenbaum's lemma says that every K-consistent set of
  $\L_{M}$-sentences, even if it is infinite, is included in a maximal
  K-consistent set.) By the Canonical Model Lemma, $M_{K}, w \models_K A$ for
  all $A$ in $\Gamma$, and $M_{K}, w \not\models_K B$. Thus
  $\Gamma \not\models_K B$. \qed
\end{proof}

\begin{theorem}{Theorem: Compactness of K}{compactnessK}
  If a sentence $B$ is K-entailed by some sentences $\Gamma$, then $B$ is
  K-entailed by a finite subset of $\Gamma$.
\end{theorem}
\begin{proof}
  \emph{Proof:} Suppose $\Gamma \models_{K} B$. By strong completeness, it
  follows that there are finitely many sentences $A_{1},\ldots,A_{n}$ in
  $\Gamma$ for which $(A_{1}\land\ldots\land A_{n}) \to B$ is K-provable. By the
  soundness of the K-calculus, $(A_{1}\land\ldots\land A_{n}) \to B$ is valid.
  So $A_{1},\ldots,A_{n} \models_{K} B$, by observation
  \ref{obs:semantic-deduction-theorem}. \qed
\end{proof}

Compactness is surprising. It is easy to think of cases in which a conclusion is
entailed by infinitely many premises, but not by any finite subset of these
premises. For example, suppose I like the number 0, I like the number 1, I like
the number 2, and so on, for all natural numbers 0,1,2,3,\ldots. Together, these
assumptions entail that I like every natural number. But no finite subset of the
assumptions has this consequence.

\begin{exercise}
  A set of sentences $\Gamma$ is called \emph{K-satisfiable} if there is a world
  in some Kripke model at which all members of $\Gamma$ are true. Show that an
  infinite set of sentences $\Gamma$ is K-satisfiable iff every finite subset of
  $\Gamma$ is K-satisfiable.
\end{exercise}
\begin{solution}
  Let $\Gamma$ be an infinite set of $\L_{M}$-sentences. If $\Gamma$ is
  K-satisfiable then obviously every finite subset of $\Gamma$ is satisfiable as
  well. For the converse direction, assume $\Gamma$ is not K-satisfiable: There
  is no world in any Kripke model at which all members of $\Gamma$ are true.
  Then there is no world in any Kripke model at which all members of $\Gamma$
  are true while $p \land \neg p$ is false. So $\Gamma \models p\land \neg p$.
  By the compactness theorem, it follows that there is a finite subset
  $\Gamma^{-}$ for which $\Gamma^{-} \models p \land \neg p$. If
  $\Gamma^{-} \models p \land \neg p$ then there is no world in any Kripke model
  at which all members of $\Gamma^{-}$ are true while $p \land \neg p$ is false.
  Since $p\land \neg p$ is false at every world in every Kripke model, it
  follows that there is no world in any Kripke model at which all members of
  $\Gamma^{-}$ are true. This shows that if $\Gamma$ is not K-satisfiable then
  there is a finite subset ($\Gamma^{-}$) of $\Gamma$ that is not K-satisfiable.
  Conversely, if every finite subset of $\Gamma$ is K-satisfiable then $\Gamma$
  is K-satisfiable.
\end{solution}

To conclude this chapter, I want to take a quick look at the logic of
mathematical provability.

Our proofs of soundness, completeness, compactness, etc.\ were informal. We have
not translated the relevant claims into a formal language, nor have we used a
formal method of proof. In principle, however, this can be done. All our
proofs could be formalized in an axiomatic calculus for predicate logic with a
few additional axioms about sets. A well-known calculus of that kind is ZFC
(named after Ernst Zermelo, Abraham Fraenkel, and the Axiom of Choice). ZFC is
strong enough to prove not just soundness and completeness in modal logic, but
practically everything that can be proved in any branch of maths.

An interesting feature of ZFC is that it can not only prove facts about what's
provable in simpler axiomatic calculi; it can also prove facts about what's
provable in ZFC itself. For example, one can prove in ZFC that one can prove in
ZFC that 2+2=4.

This gives us  an interesting application of modal logic. Let's read the box as
`it is mathematically provable that', which we understand as provability in ZFC.
One can easily show (in ZFC) that this operator has all the properties of the
box in the basic logic K. For example, all instances of the \pr{K}-schema are
provable in ZFC. (The language of ZFC doesn't have a box symbol. But one can
encode the \pr{K}-schema into a schema of ZFC, given the present reading of the
box, and all instances of that schema are ZFC-provable.)

So the logic of mathematical provability is at least as strong as K. In fact, it
is stronger. One can prove in ZFC that whenever a sentence is ZFC-provable then
it is ZFC-provable that the sentence is ZFC-provable. This gives us the
\pr{4}-schema $\Box A \to \Box\Box A$.

You might expect that we also have the \pr{T}-schema $\Box A \to A$ or the
\pr{D}-schema $\Box A \to \Diamond A$. The latter says that if something is
provable then its negation isn't provable (since $\Diamond A$ means
$\neg\Box\neg A$). And surely ZFC can't prove both a sentence and its negation
-- which would make ZFC inconsistent. I say `surely', but can we
prove (in ZFC) that ZFC is consistent? The answer is no. More precisely, one can
prove that if one can prove that ZFC is consistent then ZFC is
\emph{in}consistent. This bizarre fact is a consequence of \emph{G\"odel's
  second incompleteness theorem}, established by Kurt G\"odel in 1931. It is
reflected by the following schema (named after G\"odel and Martin L\"ob), all
whose instances are provable in ZFC:
%
\principle{GL}{\Box(\Box A \to A) \to \Box A}
%
The system GL, which is axiomatized by \pr{K}, \pr{GL}, \pr{Nec}, and \pr{CPL},
completely captures what ZFC can prove about provability in ZFC. (Schema \pr{4}
isn't needed as a separate axiom schema because it can be derived.)%
% To show []p->[][]p you need the GL instance with A = □p∧p

\begin{exercise}
  Suppose ZFC can prove its own consistency, so that there is a proof of
  $\neg \Box (p \land \neg p)$. Explain how this proof could be extended to a
  proof of $\Box (p \land \neg p)$. You need
  each of \pr{GL}, \pr{Nec}, and \pr{CPL}.
\end{exercise}
\begin{solution}
  Suppose there is a proof of $\neg\Box (p \land \neg p)$. By \pr{CPL}, we can
  infer $\Box (p \land \neg p) \to (p \land \neg p)$, because $A \to B$ is a
  truth-functional consequence of $\neg A$. By \pr{Nec}, we get
  $\Box (\Box (p \land \neg p) \to (p \land \neg p))$. By \pr{GL} and
  \emph{modus ponens} (an instance of \pr{CPL}), we can derive
  $\Box (p \land \neg p)$.
\end{solution}

% \begin{exercise}
%   Let $p$ be the statement that 2+2=5. Properly encoded into the language of
%   ZFC, one can prove in ZFC that $p$ is false. And this fact itself is provable
%   in ZFC: One can prove in ZFC that $\Box\neg p$. Explain (informally) why it
%   follows that $\neg\Box p$ is not provable in ZFC, assuming that ZFC is
%   consistent and drawing on the fact that the logic of ZFC-provability is
%   axiomatized by \pr{K}, \pr{GL}, \pr{Nec}, and \pr{CPL}.
% \end{exercise}
% \begin{solution}
%   Suppose $\neg\Box p$ is ZFC-provable. By \pr{CPL}, any proof of $\neg\Box p$
%   could be extended to a proof of $\Box p \to p$, because $\Box p \to p$ is a
%   truth-functional consequence of $\neg p$. By \pr{Nec}, we would get
%   $\Box(\Box p \to p)$. By \pr{GL} and Modus Ponens (an instance of \pr{CPL}),
%   we would get $\Box p$. So if $\neg\Box p$ is ZFC-provable, then $\Box p$ is
%   ZFC-provable. And if both $\Box p$ and $\neg \Box p$ are ZFC-provable, then
%   ZFC is inconsistent.
% \end{solution}

% Surprisingly, Kripke models play an important role in the study of mathematical
% provability: the standard technique for proving that all instance of \pr{GL}
% are provable in ZFC draws on the fact that GL is sound and complete with respect
% to the class of finite, transitive, and irreflexive Kripke models.
% (As
% I said earlier, the completeness proof is non-trivial, because the canonical
% model of \Ax{GL} has infinitely many worlds and therefore does not itself belong
% to the relevant model class.)
%
% why irreflexive? can't we drop that? no; if we do so, we allow for
% infinite chains.
%
% This is surprising because intuitively, mathematical truths are true at all
% possible worlds, so it is hard to see how mathematical provability could be
% usefully analysed in terms of truth at accessible worlds.

% [Boolos p.xx (yes, xx in the introduction) says that this is the
% decisive scientific justification for modal logic and Kripke
% models.]

% So we should not think of logical necessity as provability. Otherwise
% the logic of logical necessity certainly isn't S5. Whatever logical
% possibility means, it can't be defined in terms of provability.


%%% Local Variables: 
%%% mode: latex
%%% TeX-master: "logic2.tex"
%%% End:

\chapter{Epistemic Logic}\label{ch:epistemic}

% ``I know you think you understand what you thought I said but I'm not sure you
% realize that what you heard is not what I meant.'' -- Alan Greenspan%

\section{Epistemic accessibility}

When we say that something is possible, we often mean that it is compatible with
our information. This ``epistemic'' flavour of possibility -- along with related
concepts such as knowledge, belief, information, and communication -- is studied
in epistemic logic.

Standard epistemic logic relies heavily on the possible-worlds semantics
introduced in chapters \ref{ch:worlds} and \ref{ch:accessibility}. The guiding
idea is that information rules out possibilities. Suppose, for example, we are
investigating a crime. There are three suspects: the gardener, the butler, and
the cook. A credible eye-witness tells us that the gardener was out of town at
the time of the crime. This allows us to rule out the previously open
possibility that the gardener is the culprit. When we gain information, the
space of open possibilities shrinks.

% This model of attitudes goes back to Hintikka [1962]. The representation as a
% binary relation is formally interchangeable with the “possibility
% correspondences” as introduced by Aumann [1976] (see also Aumann [1999]) and
% used throughout economic theory. For the interchangeability, see, e.g. Fagin
% et al. [1995].

Let's say that a world is \emph{epistemically accessible} for an agent if it is
compatible with the agent's knowledge. Recall that a world is a maximally
specific possibility. For any such possibility, we may ask whether it might be
the actual world. If your information allows you to give a negative answer then
the world is not epistemically possible for you -- it is epistemically
inaccessible. Before we learned that the gardener was out of town, our
epistemically accessible worlds included worlds at which the gardener committed
the crime. When we received the eye-witness report, these worlds became
inaccessible.

\begin{exercise}
  Which worlds are epistemically accessible for an agent who knows all truths?
  Which worlds are epistemically accessible for an agent who knows nothing?
\end{exercise}
\begin{solution}
  For an agent who knows all truths only the actual world is epistemically accessible. For an agent who knows nothing all worlds are epistemically accessible.
\end{solution}

We will interpret the box and the diamond in terms of epistemic accessibility.
In this context, the box is usually written `$\Kn$', which for once doesn't
stand for Kripke but for knowledge. I will use `$\Mi$' (`might') for the
diamond. So $\Kn A$ means that $A$ is true at all epistemically accessible
worlds, while $\Mi A$ means that $A$ is true at some epistemically accessible
world. If we want to clarify which agent we have in mind, we can add a
subscript: $\Mi_{\text{b}} A$ might say that $A$ is epistemically possible for
Bob.

We often informally read $\Kn$ as `the agent knows'. In at least one respect,
however, our $\Kn$ operator does not match the knowledge operator of ordinary
English.

To see why, note that if some propositions are true at a world, then anything
that logically follows from these propositions is also true at that world. For
example, if $p\to q$ and $p$ are both true at $w$, then so is $q$ (by definition
\ref{def:kripkesemantics}). As a consequence, if $p \to q$ and $p$ are true at
all epistemically accessible worlds (for some agent), then $q$ is also true at
all these worlds. $\Kn (p\to q)$ and $\Kn p$ together entail $\Kn q$. More
generally, the $\Kn$ operator is \textbf{closed under logical consequence},
meaning that if $B$ logically follows from $A_1,\ldots,A_n$, and
$\Kn A_1, \ldots,\Kn A_n$, then $\Kn B$.

Our ordinary conception of knowledge does not seem to be closed under logical
consequence. If you know the axioms of a mathematical theory, you don't
automatically know everything that logically follows from the axioms. Our $\Kn$
operator might be taken to formalise the concept of \emph{implicit knowledge},
where an agent implicitly knows a proposition if the proposition follows from
things the agent knows. An agent's implicit knowledge represents the information
the agent has about the world. If what you know entails $p$, then the
information you have settles that $p$, even though you may not realise that it
does.

% \begin{exercise}
%   One might think that ordinary knowledge is closed under \emph{obvious
%   logical consequence}: If we know some propositions, and these propositions
%   obviously entail another proposition (say, by modus ponens, or by a similar
%   elementary rule), then we also know that other proposition. Explain why any
%   operator that is closed under obvious logical consequence is closed under
%   logical
%   consequence. % this needs to be made more precise. What if logical consequence is 2nd-order?
% \end{exercise}
% \begin{solution}
%   If some proposition $B$ is logically entailed by $A_{1}, \ldots, A_{n}$, then
%   there is a derivation of $B$ from $A_{1}, \ldots, A_{n}$ in which each
%   individual step is obvious.
% \end{solution}

\begin{exercise}
  Translate the following sentences into the language of epistemic logic,
  ignoring my warnings about the mismatch between $\Kn$ and the ordinary concept
  of knowledge.
  \begin{exlist}
  \item Alice knows that it is either raining or snowing.
  \item Either Alice knows that it is raining or that it is snowing.
  \item Alice knows whether it is raining.
  \item You know that you're guilty if you don't know that you're innocent.
  \end{exlist}
\end{exercise}
\begin{solution}
  \begin{sollist}
  \item $\Kn(r \lor s)$\\
    $r$: It is raining; $s$: It is snowing\\[-2mm]
  \item $\Kn r \lor \Kn s$\\
    $r$: It is raining; $s$: It is snowing\\[-2mm]
  \item $\Kn r \lor \Kn \neg r$\\
    $r$: It is raining\\[-2mm]
  \item This sentence is ambiguous. On one reading, it could be translated as $\Mi g\to Kn g$, on the other as $\Kn(\Mi g \to g)$\\
    $g$: You are guilty
  \end{sollist}
\end{solution}

% \begin{exercise}
%   The ordinary concept of knowledge logically ill-behaved in more than one
%   respect. Let $\Kn^*$ be an operator that applies to a sentence $A$ iff we
%   would intuitively say that some fixed agent knows $A$. Assume the agent in
%   question knows the axioms of ZFC set theory. Define $\Kn^+$ as the logical
%   closure of $\Kn^*$; that is,

%   \bigskip
%   $\Kn^+ A \;\Leftrightarrow_{\text{def}}\; \text{$A$ is entailed by sentences $A_1,\ldots,A_n$ such that $\Kn^* A_1,\ldots,\Kn^* A_n$}.$
%   \smallskip%
  
%   Note the similarity between $\Kn^+$ and the mathematical provability operator
%   from section \ref{sec:provability}.  Indeed, with minimal further assumptions
%   one can prove that $\Kn^+$ validates the \textbf{GL} schema:
%   \[
%     \Kn^+(\Kn^+ A \to A) \to \Kn^+ A
%   \]
%   From the definition of $\Kn^+$, we can infer that the following is also valid:
%   \[
%     \Kn^*(\Kn^+ A \to A) \to \Kn^+ A.
%   \]
%   Explain why this is an intuitively unacceptable principle about
%   knowledge.
% \end{exercise}
% \begin{solution}
%   One possible answer: There are many propositions that I don't know, and that
%   don't logically follow from things I know. And for many of these propositions
%   I \emph{know} that that they don't follow from things I know. (For example, I
%   know that it doesn't follow from anything I know that Edinburgh is in Italy.)
%   Let $p$ be some such proposition. Since I know that $\neg \Kn^+ p$ is true, I
%   can easily figure out that $\Kn^+ p \to p$ is true as well (by the truth-table
%   for the arrow). So $\Kn^*(\Kn^+p \to p)$. If
%   $\Kn^*(\Kn^+ A \to A) \to \Kn^+ A$ were valid, it would follow that $p$ does
%   after all follow from what I know!
% \end{solution}

% \begin{exercise}
%   We could add ``impossible'' worlds to avoid closure under logical entailment
%   (Rantala 1982). Explain why definition \ref{def:possibleworldssemantics}
%   needs to be changed for impossible worlds $w$.
% \end{exercise}

% Mention centring?


\section{The logic of knowledge}

What is the logic of (implicit) knowledge? Which sentences in the language of
epistemic logic are valid? Which are logical consequences of which others?

The basic system K is arguably too weak. There are Kripke models in which
$\Box p$ is true at some world while $p$ is false. But knowledge entails truth.
If $p$ is genuinely known (or entailed by what is known) then $p$ is true. In
the logic of knowledge, all instance of the \pr{T}-schema are valid.
%
\principle{T}{\Kn A \to A}

We know from section \ref{sec:frames} that the \pr{T}-schema corresponds to
reflexivity, in the sense that all instances of the schema are valid on a frame
iff the frame is reflexive. To ensure that all \pr{T} instances are valid, we
will therefore assume that Kripke models for epistemic logic are always reflexive.
Every world is accessible from itself.

This makes sense if you remember what accessibility means in epistemic logic. We
said that a world $v$ is (epistemically) accessible from a world $w$ if $v$ is
compatible with what the agent knows at $w$. Whatever the agent knows at $w$
must be true at $w$. So any world in any conceivable scenario must be
accessible from itself.

% Reflexivity implies seriality, which corresponds to the schema
% \begin{equation}\tag{\pr{D}}
%   \Kn A \to \Mi A
% \end{equation}
% Intuitively, this means that the information available to an agent is never
% contradictory. If the information entails $A$ (as $\Kn A$ asserts), then it
% does not entail $\neg A$ (that is, then $\neg \Kn\neg A$).

Let's look at other properties of the epistemic accessibility relation. Is the
relation symmetric? If $v$ is compatible with what is known at $w$, is $w$
compatible with what is known at $v$? I will give two arguments for a negative
answer.

My first argument assumes that we have non-trivial knowledge about the external
world. Let's say we know that we have hands. Now consider a possible world in
which we are brains in a vat, falsely believing that we have hands. In that
world, we know very little. We don't know that we have hands, nor that we are
handless brains in a vat. Perhaps we know that we are conscious, and what kinds
of experiences we have. But since our experiences are the same in the vat world
and in the actual world (let's assume), the actual world is compatible with what
little we know in the vat world. So the actual world is accessible from the vat
world. But the vat world is not accessible from the actual world -- otherwise we
wouldn't know that we have hands. If the actual world is accessible from the vat
world and the vat world is inaccessible from the actual world then the
accessibility relation isn't symmetric.

My second argument starts with a scenario in which someone has misleading
evidence that some proposition $p$ is false. This is easily conceivable. In that
scenario, $p$ is true but the agent believes $\neg p$. Often, when we believe
something, we also believe that we know it. Let's assume that our agent believes
that they know $\neg p$. Let's also assume that their beliefs are consistent, so
they don't believe that they \emph{don't} know $\neg p$. Since they don't
believe this proposition (that they don't know $\neg p$) they don't know it
either: they don't know that they don't know $\neg p$. So we have a scenario in
which $p$ is true but $\Kn \neg \Kn \neg p$ false.

Can you see what this has to do with symmetry? In section \ref{sec:frames} I
mentioned that symmetry corresponds to the schema
%
\principle{B}{A \to \Kn \Mi A.}
%
This means that all instances of \pr{B} are valid on a frame iff the frame is
symmetric. If the epistemic accessibility relation were symmetric, then all
instances of \pr{B} would be valid. But I've just described a scenario in which
an instance of \pr{B} is false. So the epistemic accessibility relation isn't
symmetric.

What about transitivity, which corresponds to schema \pr{4}?
%
\principle{4}{\Kn A \to \Kn\Kn A}
%
In epistemic logic, \pr{4} is known as the \textbf{KK principle}, or
(misleadingly) as \textbf{positive introspection}. There is an ongoing debate
over whether the principle should be considered valid. I will review one
argument for either side.

A well-known argument against the KK principle draws on the idea that knowledge
requires ``safety'': you know $p$ only if you couldn't easily have been wrong
about $p$. To motivate this idea, consider a Gettier case. Suppose you are
looking at the only real barn in a valley which, unbeknownst to you, is full of
fake barns. Your belief that you're looking at a barn is true, and it seems to
be justified. But intuitively, it isn't knowledge. You don't know that what
you're looking at is a real barn. Why not? Advocates of the safety condition
suggest that you don't have knowledge because you could easily have been wrong.
You genuinely know $p$ only if there is no ``nearby'' possibility at which $p$
is false, where ``nearness'' is a matter of similarity in certain respects.

On the safety account, you know \emph{that you know $p$} only if there is no
nearby world at which you don't know $p$. That is, you know at world $w$ that
you know $p$ only if you know $p$ at all worlds $v$ that are relevantly similar
to $w$. And you know $p$ at $v$ only if $p$ is true at all worlds $u$ that are
relevantly similar to $v$. But similarity isn't transitive: the fact that $u$ is
similar to $v$ and $v$ is similar to $w$ does not entail that $u$ is similar to
$w$. So it can happen that $p$ holds at all nearby worlds, but not at all worlds
that are nearby a nearby world. In that case, you may know $p$ without knowing
that you know $p$.

Not everyone accepts the safety condition. Other accounts of knowledge vindicate
the KK principle. For example, some have argued that an agent knows $p$
(roughly) iff the agent's belief state \emph{indicates} $p$, in the sense that
%
\begin{quote}
\begin{enumerate*}
\item[(1)] under normal conditions, being in that state implies $p$, and
\item[(2)] conditions are normal.
\end{enumerate*}
\end{quote}
% 
We can formalize this concept in modal logic. Let $N$ mean that conditions are
normal (whatever exactly that means), and let $\Box$ be a non-epistemic operator
that formalizes `at all worlds'. $\Box(N \to A)$ then means that $A$ is true at
all world at which conditions are normal. According to the definition I just
gave, a belief state $s$ indicates $p$ iff
\begin{equation}\tag{*}
  \Box(N \to (s\to p)) \land N.
\end{equation}
The state $s$ indicates that $s$ indicates $p$ iff
\begin{equation}\tag{**}
  \Box(N \to (s \to (\Box(N \to (s \to p)) \land N))) \land N.
\end{equation}
A quick tree proof reveals that (*) entails (**). That is, whenever a state
indicates $p$ then it also indicates that it indicates $p$. On the indication
account of knowledge, a belief state that constitutes knowledge therefore
automatically constitutes knowledge of knowledge: the \pr{4} schema is valid.

\begin{exercise}
  Give an S5 tree proof to show that (*) entails (**). Why can we
  assume S5 here?
\end{exercise}
\begin{solution}
  You can use
  \href{https://www.umsu.de/trees/}{umsu.de/trees/} to
  see the tree proof. We can assume S5 for the box because quantifies
  unrestrictedly over all worlds (as in chapter \ref{ch:worlds}).
\end{solution}

The \pr{4}-schema says that people have knowledge of their knowledge. The
\pr{5}-schema says that people have knowledge of their ignorance: if you don't
know something, then you know that you don't know it. This hypothesis is
(misleadingly) known as \textbf{negative introspection}.
%
\principle{5}{\Mi A \to \Kn \Mi A.}
%
We know that the \pr{5}-schema corresponds to euclidity. This gives us a quick
argument against the schema. As you showed in exercise \ref{ex:relations},
reflexivity and euclidity together entail symmetry. The epistemic accessibility
relation is reflexive. If it were euclidean, it would be symmetric. But I've
argued that it isn't symmetric. So the logic of knowledge doesn't validate
\pr{5}.

We can also give a more direct argument against negative introspection. Consider
again a scenario in which someone has misleading evidence that some proposition
$p$ is false. Since $p$ is actually true, the agent doesn't know $\neg p$. But
the agent might not know that they don't know $\neg p$. (On the contrary, they
might believe that they do know $\neg p$.) In that scenario, $\neg\Kn \neg p$ is
true but $\Kn \neg \Kn \neg p$ is false.

Here it is important to not be misled by a curiosity of ordinary language. When
we say that someone doesn't know $p$, this seems to imply that $p$ is true. If I
told you that my neighbour doesn't know that I have a pet aardvark, you could
reasonably infer that I have a pet aardvark. You might therefore be tempted to
regard all instances of the following schema as valid:
%
\principle{NT}{\neg \Kn A \to A}
%
On reflection, however, \pr{NT} is unacceptable. If $\neg \Kn A$ entails $A$,
then by contraposition $\neg A$ entails $\Kn A$: everything that is false would
be known! Indeed, if I \emph{don't} have a pet aardvark then surely my neighbour
does not know that I have one. We shall therefore not regard the inference from
$\neg \Kn A$ to $A$ as valid.

\begin{exercise}
  Can you find a Kripke frame on which \pr{NT} is valid?
\end{exercise}
\begin{solution}
  \pr{NT} is valid on all and only the frames in which no world can see any world. 
\end{solution}

\begin{exercise}
  Let's say that an agent is \emph{ignorant of} a proposition if they don't know
  the proposition and the proposition is true. (In English, saying that someone
  doesn't know a proposition normally conveys that they are ignorant of the
  proposition, in this sense.) Show that if the logic of knowledge is at least as
  strong as K, then ignorance of $A$ entails ignorance of ignorance of $A$.
    % I(A) = A . -KA.
    % II(A) = A . -KA . -K(A . -KA).
\end{exercise}
\begin{solution}
  We assume that ignorance of $A$ can be formalized as $A \land \neg \Kn A$. Ignorance of ignorance of $A$ is therefore formalized as $(A \land \neg \Kn A) \land \neg \Kn(A \land \neg \Kn A)$. A tree proof shows that the former K-entails the latter. 
  % See Fine 2018:
  % \begin{enumerate}
  % \item $\vdash_{S4} \neg Ip \to K\neg Ip$
  % \item $\not\vdash_{S4} Ip \to K Ip$
  % \item $\vdash_{S4} IIp \to Ip$
  % \item $\not\vdash_{S4} Ip \to IIp$
  % \item $\vdash_{S4} IIp \to \neg KIIp$
  % \item $\vdash_{S4} IIp \leftrightarrow IIIp$
  % \item $\vdash_{S5} \neg IIp$
  % \item $\vdash_{S4M} Ip \to IIp$
  % \end{enumerate}
  % \end{exercise}
\end{solution}
  
% Rumsfeld suggested there are things of which we don't know that we don't know
% them. We might say that someone is Rumsfeld ignorant of $p$ iff
% $\neg K\neg K p$. But that's not quite what Rumsfeld had in mind. For note that
% if $Kp$ then $\neg K \neg Kp$, because knowledge is factive. But $Kp$ is not a
% case of Rumsfeld ignorance. (I.e., one reason why we don't know that we don't
% know $p$ is that we actually know $p$, but that's not the interesting case.) So
% let's say someone is Rumsfeld ignorant whether $p$ iff
% $\neg K p \land \neg K \neg Kp$. (Fine says it's $Ip \land \neg K Ip$. Suppose
% $K\neg p$. Then $\neg K p$ and $K \neg Kp$. So that doesn't look problematic.)

% Notice that this is a Fitchean truth. So we can't know that we're Rumsfeld
% ignorant. Which is why it's hard to give an \emph{example} of something of which
% we're Rumsfeld ignorant.


% Oddly, when computer scientists reason about knowledge and information, they
% often assume both positive and negative introspection, along with the \pr{T}
% schema. Since every transitive, euclidean, and reflexive relation is an
% equivalence relation, the logic of knowledge then becomes S5. The comparative
% simplicity of S5 -- think of the simple tree rules from chapter \ref{ch:worlds}
% -- may be one reason to make the philosophically dubious posit of negative
% introspection. However, the posit can also be justified by assumptions about the kinds of system we want to model.

% Imagine an artificial agent whose database can store the truth-value for a
% finite number of propositions $p_1,\ldots,p_n$. The agent receives information
% through a reliable channel, so that the database is guaranteed to never
% contain false information. Since the agent knows, say, $p_{1}$ iff their
% database says that $p_{1}$ is true, the agent can easily find out whether or
% not they know $p_{1}$ by scanning their own database.

We have looked at six schemas: \pr{T}, \pr{B}, \pr{4}, \pr{5}, and \pr{NT}.
Philosophers working in epistemic logic generally reject \pr{B}, \pr{5}, and
\pr{NT}, accept \pr{T}, and are divided over \pr{4}. Theorists in other
disciplines often assume that the logic of knowledge is S5, which would render
all instances of \pr{T}, \pr{4}, \pr{B}, and \pr{5} valid. If we drop \pr{B} and
\pr{5} but keep \pr{T} and \pr{4}, we get S4. If we also drop \pr{4}, we get
system T.

We might, however, look at further schemas, corresponding to further conditions
on the accessibility relation. For example, some have argued that we should
adopt a weakened form of negative introspection. The above counterexample to
negative introspection -- schema \pr{5} -- involved an agent who doesn't know
that they don't know a certain proposition because they don't know that the
proposition is false. This kind of counterexample can't arise if the relevant
proposition is true. One might therefore suggest that if an agent doesn't know a
proposition $p$ \emph{and $p$ is true}, then the agent always knows that they
don't know $p$. This would give us a schema known as 0.4:
%
\principle{0.4}{(\neg \Kn A\land A) \to \Kn \neg \Kn A}
%
All instances of \pr{0.4} are S5-valid, but not all of them are S4-valid. Adding
the schema to S4 leads to a system known as S4.4.

% .4 corresponds to xRy & xRz -> (yRz v x=z)

\begin{exercise}
  Explain why Gettier cases cast doubt on \pr{0.4}.
\end{exercise}
\begin{solution}
  In a Gettier case, the relevant proposition $p$ (say, that you're looking at a
  barn) is true but unknown. By \pr{0.4}, it would follow that the agent knows
  that they don't know $p$. But in a typically Gettier case the agent does not
  know that they don't know $p$.
\end{solution}
  
A more modest extension of S4 adds the schema \pr{G}, which corresponds to
convergence of the accessibility relation:
%
\principle{G}{\Mi\Kn A \to \Kn\Mi A}
%
The resulting logic is called S4.2; it is weaker than S4.4 but stronger than S4.
We will meet an argument in favour of \pr{G} in section \ref{sec:kb}.

% \begin{exercise}
%   Give an S4 tree proof to show that
%   $(A \land \neg \Kn A) \to \Kn \neg \Kn A$ and
% %  $(\neg A \land \neg \Kn \neg A) \to \Kn \neg \Kn \neg A$ (both of which
%   $(\neg A \land \Mi A) \to \Kn \Mi A$ (both of which
%   are covered by 0.4) together entail \pr{G}.
% \end{exercise}

\begin{exercise}
  Use the tree method to check the following claims. (See the table at the end
  of chapter \ref{ch:accessibility} for the tree rules that go with B, S4, and
  S4.2.)
  \begin{exlist}
  \item $\models_{T} \Mi\Kn p \to \Kn\Mi p$.
  \item $\models_{B} \Mi\Kn p \to \Kn\Mi p$.
  \item $\models_{S4} \Mi\Kn\Mi p \to \Mi p$.
  \item $\models_{S4} \Mi\Kn p \leftrightarrow \Kn\Kn p$.
  \item $\models_{S4} \Mi\Kn(p \to \Kn\Mi p)$.
  \item $\models_{S4.2} (\Mi\Kn p \land \Mi\Kn q) \to \Mi \Kn(p \land q)$.
  \end{exlist}
\end{exercise}
\begin{solution}
  All except (a) and (d) are correct. You can find trees or counterexamples for (a)-(e) on
  \href{https://www.umsu.de/trees/}{umsu.de/trees/} if
  you write $\Kn$ as a box and $\Mi$ as a diamond. Here is a tree for (f):
  \begin{center}
  \tree{
   & \nnode{35}{1.}{$\neg((\Mi\Kn p \land \Mi\Kn q) \to \Mi \Kn(p \land q))$}{w}{(Ass.)} & \\
    & \nnode{35}{2.}{$\Mi\Kn p \land \Mi\Kn q$}{w}{(1)} & \\
    & \nnode{35}{3.}{$\neg\Mi \Kn(p \land q)$}{w}{(1)} & \\
    & \nnode{35}{4.}{$\Mi\Kn p$}{w}{(2)} & \\
    & \nnode{35}{5.}{$\Mi\Kn q$}{w}{(2)} & \\
    & \nnode{35}{6.}{$wRv$}{}{(4)} & \\
    & \nnode{35}{7.}{$\Kn p$}{v}{(4)} & \\
    & \nnode{35}{8.}{$wRu$}{}{(5)} & \\
    & \nnode{35}{9.}{$\Kn q$}{u}{(5)} & \\
    & \nnode{35}{10.}{$vRt$}{}{(6,8,Con)} & \\
    & \nnode{35}{11.}{$uRt$}{}{(6.8,Con)} & \\
    & \nnode{35}{12.}{$wRt$}{}{(6.10,Tr)} & \\
    & \nnode{35}{13.}{$\neg\Kn(p \land q)$}{t}{(3,12)} & \\
    & \nnode{35}{14.}{$tRs$}{}{(13)} & \\
    & \bnode{35}{15.}{$\neg(p \land q)$}{s}{(13)} & \\
    &&\\
    \nnode{10}{16.}{$\neg p$}{s}{(15)} && \nnode{10}{17.}{$\neg q$}{s}{(15)}\\
    \nnode{10}{18.}{$vRs$}{}{(10.14,Tr)} && \nnode{10}{19.}{$uRs$}{}{(11.14,Tr)}\\
    \nnodeclosed{10}{20.}{$p$}{s}{(7,18)} && \nnodeclosed{10}{21.}{$q$}{s}{(9,19)}\\
  }
  \end{center}
\end{solution}


\section{Multiple Agents}
\label{sec:multi}

A world that is epistemically accessible for one agent may not be accessible for
another. If we want to reason about the information available to different
agents, we need separate $\Kn$ operators and accessibility relations for each
agent.

We can easily expand the language $\L_M$ to a \textbf{multi-modal language} by
introducing a whole series of box operators $\Kn_1, \Kn_2, \Kn_3, \ldots$ with
their duals $\Mi_1, \Mi_2, \Mi_3, \ldots$. This multi-modal language is
interpreted in multi-modal Kripke models.

\begin{definition}{}{multikripkemodel}
  A \textbf{multi-modal Kripke model} consists of
  \vspace{-3mm}
  \begin{itemize*}
  \item a non-empty set $W$,
  \item a set of binary relation $R_1,R_2,R_{3},\ldots$ on $W$, and
  \item a function $V$ that assigns to each sentence letter a subset of $W$.
  \end{itemize*}
\end{definition}
%
In our present application, every accessibility relation $R_i$ represents the
information available to a particular agent. A world $v$ is $R_i$-accessible
from $w$ iff $v$ is compatible with the information agent $i$ has at world $w$.

The definition of truth at a world in a Kripke model (definition
\ref{def:kripkesemantics}) is easily extended to multi-modal Kripke models.
Instead of clauses (g) and (h), we have the following conditions, for each pair
of a modal operator $\Kn_i$/$\Mi_i$ and the corresponding accessibility
relation $R_i$:

\bigskip
\begin{tabular}{lll}
  & $M,w \models \Kn_i A$ &iff $M,v \models A$ for all $v$ in $W$ such that $wR_iv$.\\
  & $M,w \models \Mi_i A$ &iff $M,v \models A$ for some $v$ in $W$ such that $wR_iv$.
\end{tabular}
\bigskip

As an application of this machinery, let's look at the \emph{Muddy Children}
puzzle.

\begin{quote}
  Three (intelligent) children have been playing outside. They can't see or feel
  if their own face is muddy, but they can see who of the others has mud on
  their face. Coming inside, mother tells them: `At least one of you has mud on
  their face'. She then asks, `Do you know if you have mud on your face?''.
  All three children say that they don't know. Mother asks again, `Do you know
  if you have mud on your face?'. This time, two children say that they know.
  How many children have mud on their face? And what happens when the mother
  asks her question a third time?
\end{quote}

We begin by drawing a model. I'll call the three children Alice, Bob, and Carol,
and I'll use $m_a, m_b, m_c$ as sentence letters expressing, respectively, that
Alice/Bob/Carol is muddy; $c_a, c_b, c_c$ mean that Alice/Bob/Carol is clean.
Before the mother's first announcement, there are eight relevant possibilities.

\begin{center}
  \resizebox{ 8.5cm}{!}{%
  \begin{tikzpicture}[modal, world/.append style={minimum size=1.7cm}]
    \node[world] (w1) {$c_am_b\newline m_c$};
    \node[world] (w2) [above right=of w1] {$c_am_bc_c$};
    \node[world] (w3) [below=3cm of w1] {$c_ac_bm_c$};
    \node[world] (w4) [above right=of w3] {$c_ac_bc_c$};
    \draw[<->, rblue] (w1) -- (w2) node[midway,above left]{$c$};
    \draw[<->, rblue] (w3) -- (w4) node[midway,above left]{$c$};
    \draw[<->, rgreen] (w1) -- (w3) node[midway,left]{$b$};
    \draw[<->, rgreen] (w2) -- (w4) node[near end,left]{$b$};
    \node[world] (w5) [right=3cm of w1]{$m_am_bm_c$};
    \node[world] (w6) [above right=of w5] {$m_am_bc_c$};
    \node[world] (w7) [below=3cm of w5] {$m_ac_bm_c$};
    \node[world] (w8) [above right=of w7] {$m_ac_bc_c$};
    \draw[<->, rblue] (w5) -- (w6) node[midway,above left]{$c$};
    \draw[<->, rblue] (w7) -- (w8) node[midway,above left]{$c$};
    \draw[<->, rgreen] (w5) -- (w7) node[near end,left]{$b$};
    \draw[<->, rgreen] (w6) -- (w8) node[midway,left]{$b$};
    \draw[<->, rred] (w1) -- (w5) node[near end,above]{$a$};
    \draw[<->, rred] (w2) -- (w6) node[midway,above]{$a$};
    \draw[<->, rred] (w3) -- (w7) node[midway,above]{$a$};
    \draw[<->, rred] (w4) -- (w8) node[near start,above]{$a$};
  \end{tikzpicture}
  }%
\end{center}
%
Since we have three epistemic agents, we have three accessibility relations, one
for Alice (drawn in red), one for Bob (green), and one for Carol (blue). To
remove clutter, I have left out the three times eight arrows leading from each
world to itself, but we should keep in mind that every world is also accessible
from itself, for each agent.

Don't confuse an arrow in the diagram of a model with an accessibility relation.
We have three accessiblity relations, but more than three arrows. All the red
arrows in the picture represent one and the same accessibility relation. The
accessibility relation for Alice holds between a world and another whenever a
red arrow leads from the first world to the second.

Notice how the fact that every child can see the others is reflected in the
diagram. For example, at the top left world ($c_am_bc_c$), Alice sees that Bob
is muddy and that Carol is clean; the only epistemic possibilities for Alice at
that world are the two worlds at the top: $c_am_bc_c$ itself and $m_am_bc_c$. In
general, the only accessible worlds for a given child at a given world $w$ are
worlds at which the other children's state of muddiness is the same as at $w$.

What changes through the mother's first announcement, `At least one child has
mud on their face'? The announcement tells \emph{us} that we're not in the
$c_ac_bc_c$ world. More importantly, it allows \emph{each child} to rule out the
$c_ac_bc_c$ world (since they all hear and accept the announcement).

\begin{center}
  \resizebox{ 8.5cm}{!}{%
  \begin{tikzpicture}[modal, world/.append style={minimum size=1.7cm}]
    \node[world] (w1) {$c_am_bm_c$};
    \node[world] (w2) [above right=of w1] {$c_am_bc_c$};
    \node[world] (w3) [below=3cm of w1] {$c_ac_bm_c$};
    \node[world] (w4) [above right=of w3, opacity=0.4] {$c_ac_bc_c$};
    \draw[<->, rblue] (w1) -- (w2) node[midway,above left]{$c$};
    %\draw[<->, rblue] (w3) -- (w4) node[midway,above left]{$c$};
    \draw[<->, rgreen] (w1) -- (w3) node[midway,left]{$b$};
    %\draw[<->, rgreen] (w2) -- (w4) node[near end,left]{$b$};
    \node[world] (w5) [right=3cm of w1]{$m_am_bm_c$};
    \node[world] (w6) [above right=of w5] {$m_am_bc_c$};
    \node[world] (w7) [below=3cm of w5] {$m_ac_bm_c$};
    \node[world] (w8) [above right=of w7] {$m_ac_bc_c$};
    \draw[<->, rblue] (w5) -- (w6) node[midway,above left]{$c$};
    \draw[<->, rblue] (w7) -- (w8) node[midway,above left]{$c$};
    \draw[<->, rgreen] (w5) -- (w7) node[near end,left]{$b$};
    \draw[<->, rgreen] (w6) -- (w8) node[midway,left]{$b$};
    \draw[<->, rred] (w1) -- (w5) node[near end,above]{$a$};
    \draw[<->, rred] (w2) -- (w6) node[midway,above]{$a$};
    \draw[<->, rred] (w3) -- (w7) node[midway,above]{$a$};
    %\draw[<->, rred] (w4) -- (w8) node[near start,above]{$a$};
  \end{tikzpicture}
  }%
\end{center}

Next, the mother asks if anyone knows whether they are muddy. No child says yes.
So no-one knows whether they are muddy. And everyone now knows that no-one knows
whether they are muddy. We can go through the above seven possibilities to see
if at any of them, anyone knows whether they are muddy. At the top left world
($c_am_bc_c$) Alice doesn't know whether she is muddy, because the $m_am_bc_c$
world (top right) is $a$-accessible; nor does Carol know whether she is muddy,
because $c_am_bm_c$ is $c$-accessible. But Bob knows that he is muddy: no other
world is $b$-accessible. Intuitively, at the $c_am_bc_c$ world, Bob sees two
clean children (Alice and Carol), and he has just been told that not all
children are clean. So he can infer that he is muddy. But we know that Bob
didn't say that he knows whether he is muddy. So we (and all the children) can
rule out the top left world as an open possibility.

By the same reasoning, every world connected with only two arrows to other
worlds can be eliminated at this stage.

\begin{center}
  \resizebox{ 8.5cm}{!}{%
  \begin{tikzpicture}[modal, world/.append style={minimum size=1.7cm}]
    \node[world] (w1) {$c_am_bm_c$};
    \node[world] (w2) [above right=of w1, opacity=0.4] {$c_am_bc_c$};
    \node[world] (w3) [below=3cm of w1, opacity=0.4] {$c_ac_bm_c$};
    \node[world] (w4) [above right=of w3, opacity=0.4] {$c_ac_bc_c$};
    %\draw[<->, rblue] (w1) -- (w2) node[midway,above left]{$c$};
    %\draw[<->, rblue] (w3) -- (w4) node[midway,above left]{$c$};
    %\draw[<->, rgreen] (w1) -- (w3) node[midway,left]{$b$};
    %\draw[<->, rgreen] (w2) -- (w4) node[near end,left]{$b$};
    \node[world] (w5) [right=3cm of w1]{$m_am_bm_c$};
    \node[world] (w6) [above right=of w5] {$m_am_bc_c$};
    \node[world] (w7) [below=3cm of w5] {$m_ac_bm_c$};
    \node[world] (w8) [above right=of w7, opacity=0.4] {$m_ac_bc_c$};
    \draw[<->, rblue] (w5) -- (w6) node[midway,above left]{$c$};
    %\draw[<->, rblue] (w7) -- (w8) node[midway,above left]{$c$};
    \draw[<->, rgreen] (w5) -- (w7) node[midway,left]{$b$};
    %\draw[<->, rgreen] (w6) -- (w8) node[midway,left]{$b$};
    \draw[<->, rred] (w1) -- (w5) node[midway,above]{$a$};
    %\draw[<->, rred] (w2) -- (w6) node[midway,above]{$a$};
    %\draw[<->, rred] (w3) -- (w7) node[midway,above]{$a$};
    %\draw[<->, rred] (w4) -- (w8) node[near start,above]{$a$};
  \end{tikzpicture}
  }%
\end{center}

When the mother asks again if anyone knows whether they are muddy, two children
say `yes'. So everyone comes to know that two children know whether they are
muddy. In the middle world of the above model ($m_am_bm_c$), however, no child
knows whether they are muddy. That world is not actual, and it is no longer
accessible for anyone. The remaining open possibilities are $c_am_bm_c$,
$m_ac_bm_c$, and $m_am_bc_c$, each of which is only accessible from itself.

Now we can answer the questions. In the three remaining worlds, every child
knows who is muddy and who is clean. If the mother asks her question for the
third time, everyone says yes. Also, exactly two children have mud on their
face.

\begin{exercise}
  Albert and Bernard just met Cheryl. `When is your birthday?', Albert asks.
  Cheryl answers, `I'll give you some clues'. She writes down a list of 10
  dates:
  % 
  \begin{quote}
    5 May, 6 May, 9 May\\
    7 June, 8 June\\
    4 July, 6 July\\
    4 August, 5 August, 7 August
  \end{quote}
  %
  `My birthday is one of these', she says. Then she announces that she will
  whisper the month of her birthday in Albert's ear and the day in Bernard's.
  After the whispering, she asks Albert if he knows her birthday. Albert says,
  `no, but I know that Bernard doesn't know either'. To which Bernard responds:
  `Right. I didn't know until now, but now I know'. Albert: `Now I know too!'
  Draw a multi-modal Kripke model for each stage of the conversation. When is
  Cheryl's birthday?
\end{exercise}
\begin{solution}
  see
  \href{https://plato.stanford.edu/entries/dynamic-epistemic/appendix-B-solutions.html}{https://plato.stanford.edu/entries/dynamic-epistemic/appendix-B-solutions.html} (where all the dates are 10 days later than they are in my version).
\end{solution}

What logic do we have for our multi-modal language? Each pair of a $\Kn_{i}$ and
$\Mi_{i}$ operator should obey whatever conditions we want to impose on the
logic of knowledge. Are there also new principles governing the interaction
between operators for different agents?

We plausibly want all instances of the following to come out valid:
\[
  \Kn_1 \Kn_2 A \to \Kn_1 A.
\]
If I know that you know that it's raining, then I (implicitly) also know that
it's raining. Principles like this, containing multiple modal operators that are
not definable in terms of each other, are called \textbf{interaction
  principles}.

A common assumption in epistemic logic is that there are no genuinely new
interaction principles for the knowledge of multiple agents -- no principles
that don't already follow from the logic of individual knowledge. The above
principle, for example, is entailed by the assumption that the \pr{T}-schema
holds for $\Kn_2$. Think of the relevant Kripke models. Suppose, as
$\Kn_1 \Kn_2 A$ asserts, that $A$ holds at each world that is $R_2$-accessible
from any $R_1$-accessible world. If the \pr{T}-schema holds for $\Kn_2$, then
every world is $R_{2}$-accessible from itself. In particular, then, any
$R_1$-accessible world is $R_2$-accessible from itself. It follows that $A$
holds at every $R_1$-accessible world. So $\Kn_1 A$ is true.

% Girle says Hintikka has a rule to the effect that $K_aK_b A \to K_a A$.
% ``Transmissability of Knowledge''. But that seems redundant.

We can use the tree rules to streamline arguments like this. When multiple
agents are in play, we need to keep track of which world is accessible for which
agent. When expanding a node of type $\Mi_{i} A\; (w)$, for example, we add a
node $wR_{i}v$, with subscript $i$, and another node $A\; (v)$.

Here is a tree proof of the schema $\Kn_{1}\Kn_{2} A \to \Kn_{1} A$, assuming
that $R_{2}$ is reflexive. \bigskip
  \begin{center}
  \tree{
    \nnode{25}{1.}{$\neg(\Kn_1 \Kn_2 A \to \Kn_1 A)$}{w}{(Ass.)}\\
    \nnode{25}{2.}{$\Kn_1 \Kn_2 A$}{w}{(1)}\\
    \nnode{25}{3.}{$\neg \Kn_1 A$}{w}{(1)}\\
    \nnode{25}{4.}{$wR_1 v$}{}{(3)}\\
    \nnode{25}{5.}{$\neg A$}{v}{(3)}\\
    \nnode{25}{6.}{$\Kn_{2} A$}{v}{(2,4)}\\
    \nnode{25}{7.}{$vR_{2}v$}{}{(Refl.)}\\
    \nnodeclosed{25}{8.}{$A$}{v}{(6,7)}
  }
  \end{center}

\begin{exercise}
  Use the tree method to check which of the following interaction principles are
  valid if the logic of individual knowledge is S4. If a principle is invalid,
  give a counterexample.
  \begin{exlist}
  \item $\Mi_1 \Kn_2 p \to \Mi_1 p$ 
  \item $\Mi_1 \Kn_2 p \to \Mi_2\Mi_1 p$
  \item $\Mi_1 \Kn_2 p \to \Mi_2\Kn_1 p$
  \item $\Kn_1\Kn_2 p \to \Kn_2\Kn_1 p$
  \end{exlist}
\end{exercise}
\begin{solution}
  (a) and (b) are valid, (c) and (d) are invalid. Here is a tree proof for (a).
  
  \bigskip
  \begin{center}
  \tree{
    \nnode{25}{1.}{$\neg(\Mi_1 \Kn_2 p \to \Mi_1 p)$}{w}{(Ass.)}\\
    \nnode{25}{2.}{$\Mi_1 \Kn_2 p$}{w}{(1)}\\
    \nnode{25}{3.}{$\neg \Mi_1 p$}{w}{(1)}\\
    \nnode{25}{4.}{$wR_1 v$}{}{(2)}\\
    \nnode{25}{5.}{$\Kn_2 p$}{v}{(2)}\\
    \nnode{25}{6.}{$\neg p$}{v}{(3,4)}\\
    \nnode{25}{7.}{$vR_2 v$}{}{\;\;(Refl.)}\\
    \nnodeclosed{25}{8.}{$p$}{v}{(5,7)}
  }
  \end{center}

  The tree for (c) doesn't close:
  
  \bigskip
  \begin{center}
  \tree{
    \nnode{25}{1.}{$\neg(\Mi_1 \Kn_2 p \to \Mi_2\Kn_1 p)$}{w}{(Ass.)}\\
    \nnode{25}{2.}{$\Mi_1 \Kn_2 p$}{w}{(1)}\\
    \nnode{25}{3.}{$\neg \Mi_2\Kn_1 p$}{w}{(1)}\\
    \nnode{25}{4.}{$wR_1 v$}{}{(2)}\\
    \nnode{25}{5.}{$\Kn_2 p$}{v}{(2)}\\
    \nnode{25}{6.}{$vR_2v$}{}{\;\;(Refl.)}\\
    \nnode{25}{7.}{$p$}{v}{(5,6)}\\
    \nnode{25}{8.}{$wR_2 w$}{}{\;\;(Refl.)}\\
    \nnode{25}{9.}{$\neg\Kn_1 p$}{w}{(3,8)}\\
    \nnode{25}{10.}{$wR_1u$}{}{(9)}\\
    \nnode{25}{11.}{$\neg p$}{u}{(9)}
  }
  \end{center}
  % 
  We could add a few more applications of Reflexivity, but the tree would remain
  open. It also gives us a countermodel: let $W$ = $\{ w,v,u \}$; $w$ has
  1-access to $v$ and $u$; each world has 1- and 2-access to itself;
  $V(p,v) = 1$ and $V(p,u) = 0$. In this model, at world $w$, $\Mi_1\Kn_2 p$ is
  true while $\Mi_2\Kn_1 p$ is false.

  Cases (b) and (d) are similar.
  
\end{solution}

We can also define new modal operators for groups of agents. A proposition is
said to be \textbf{mutually known} in a group $G$ if it is known by every member
of the group. Let $\EKn_G$ be an operator for mutual knowledge. Clearly,
$\EKn_G A$ can be defined as $\Kn_1 A \land \Kn_2 A \land \ldots \land \Kn_n A$,
where $\Kn_1, \Kn_2, \ldots, \Kn_n$ are the knowledge operators for the members
of the group. So we can't say anything new with the help of $\EKn_G$ (at least
for finite groups). But it can be instructive to see how $\EKn_G$ behaves
depending on the behaviour of the underlying operators $\Kn_1,\Kn_2,$ etc. For
example, if each individual knowledge operator validates the \pr{T}-schema, then
so does $\EKn_G$; but if each $\Kn_i$ validates \pr{4} (positive introspection),
it does not follow that $\EKn_G$ validates \pr{4}. As a counterexample, consider
a group of two agents; both know $p$, and both know of themselves that they know
$p$, but agent 1 does not know that agent 2 knows $p$. Then $\EKn_G p$ but
$\neg \EKn_G \EKn_G p$.

\begin{exercise}
  Give an example to show that if each $\Kn_i$ validates \pr{5},
  it does not follow that $\EKn_G$ validates \pr{5}.
\end{exercise}
\begin{solution}
  The \pr{5}-schema for $\EKn_G$ states that
  $\neg \EKn_G \neg A \to \EKn_G \neg \EKn_G \neg A$. To show that some instance
  of this is invalid, we need to find a case where some instance of
  $\neg \EKn_G \neg A$ is true while $\EKn_G \neg \EKn_G \neg A$ is false. We
  can take the simplest instance, with $A=p$. Assume the relevant group has two
  agents, and consider a world $w$ at which $\Kn_1 \neg p$ and
  $\neg \Kn_2 \neg p$ are true. By the assumption that \pr{5} is valid for
  $\Kn_i$, $\Kn_2\neg\Kn_2\neg p$ is also true at $w$. But
  $\Kn_1\neg \Kn_2\neg p$ can be false (at $w$). If it is, then
  $\neg \EKn_G \neg p$ is true at $w$ while $\EKn_G \neg \EKn_G \neg p$ is
  false.
\end{solution}

% \begin{exercise}
%   Define the accessibility relation $R_G$ for $\EKn_G$ in
%   terms of the accessibility relations for the members of $G$, so that
%   \[
%     M,w \models \EKn_G A \text{ iff } M,w' \models A \text{ for all $w'$ such that } wRw'.
%   \]
%   \cmnt{%
%     Suppose we define $R_E = \bigcup_i R_i$. Let $M,w \models E^*(A)$
%     iff $M,w \models A$ for all $v$ such that $wR_E v$. Show that
%     $\models E^*(A) \leftrightarrow E(A)$.
%   } %
% \end{exercise}

A more interesting concept that has proved useful in many areas is that of
common knowledge. A proposition is \textbf{commonly known} in a group if
everyone knows it, everyone knows that everyone knows it, everyone knows that
everyone knows that everyone knows it, and so on forever. Let's use $\CKn_{G}$
as an operator for common knowledge. $\CKn_G$ is not definable in terms of
$\Kn_1, \ldots,\Kn_n$. Still, we can define it semantically in terms of the
accessibility relations for the individual agents: $\CKn_{G} A$ is true at a
world $w$ iff $A$ is true at all worlds that are reachable from $w$ by some
finite sequence of steps following the agents' accessibility relations.

%  So the accessibility relation is the transitive closure of $R_E$.

It is easy to see that common knowledge validates (all instances of) \pr{4}. It
validates \pr{T} whenever individual knowledge validates \pr{T}. So the logic of
common knowledge is at least S4. The complete logic of common knowledge also
contain some non-trivial interaction principles, which are easiest to state in
terms of $\EKn_G$:
%
\begin{principles}
  \pri{CK1} \CKn_{G}A \leftrightarrow (A \land \EKn_{G}\CKn_{G} A)\\
  \pri{CK2} (A \land \CKn_{G}(A \to \EKn_{G} A)) \to \CKn_{G} A
\end{principles}
%
You may want to confirm that these are valid. (They also provide a complete
axiomatization of common knowledge when added to an axiomatic calculus for
individual knowledge, but that is much harder to see.)

% \begin{exercise}
%   Show that if the logic of knowledge is S4, then the logic of common
%   knowledge is also S4. And if the logic of K is S4.2, the logic of
%   common knowledge is still S4. (Stalnaker 175, 195f.)
% \end{exercise}

\section{Knowledge, belief, and other modalities}
\label{sec:kb}

Issues in the logic of knowledge can sometimes be clarified by looking at the
connections between knowledge and belief. To formalise these connections, let's
introduce a new operator $\Bel$ for belief -- or rather, for \emph{implicit
  belief}, since $\Bel$, like $\Kn$, will be closed under logical
consequence.

An agent's belief state represents the world as being a certain way. For every
possible world, we can ask whether it matches what the agent believes. If, for
example, your only non-trivial belief is that there are seventeen types of
parrot, then every world in which there are seventeen types of parrot matches
your beliefs. Every such world is \emph{doxastically accessible} for you. As you
acquire further beliefs, the space of doxastically accessible worlds becomes
smaller and smaller.

We interpret $\Bel p$ as saying that $p$ is true at all doxastically accessible
worlds (for whatever agent we have in mind). Since we won't spend a lot of time
with this operator, we will simply write its dual as $\neg\Bel\neg$.

The logic of $\Bel$ is different from the logic of $\Kn$, if only because
beliefs can be false. So we will not regard all instances of 
%
\principle{T}{\Bel A \to A}
%
as valid. We may, however, accept the weaker schema
%
\principle{D}{\Bel A \to \neg \Bel \neg A.}
%
This reflects the assumption that a belief state that represents the world as
being a certain way $A$ can't also represent the world as being the opposite way
$\neg A$.

In the previous section, I argued that (implicit) knowledge does not validate
the negative introspection principle \pr{5}, and I reviewed an argument against
the positive introspection principle \pr{4}. Neither argument carries over to
belief. Many epistemic logicians accept positive and negative introspection for
(implicit) belief:
%
\begin{principles}
  \pri{4}{\Bel A \to \Bel \Bel A}\\
  \pri{5}{\neg \Bel A \to \Bel \neg \Bel A}
\end{principles}

The logic that results by adding the schemas \pr{D}, \pr{4}, and \pr{5} to the
axiomatic basis for K is known as KD45.

\begin{exercise}
  Is a transitive, serial, and euclidean relation always symmetric? If yes,
  explain why. If no, give a counterexample. What does your result mean for
  schema \pr{B} in KD45?
\end{exercise}
\begin{solution}
  No, a transitive, serial, and euclidean relation is not always symmetric.
  Counterexample: wRv, vRv. This means that not all instances of \pr{B} (which
  corresponds to symmetry) are valid in KD45.
\end{solution}

\begin{exercise}
  Show (in any way you like) that $\Bel(\Bel A \to A)$ is valid if the logic of
  belief is KD45.
\end{exercise}
\begin{solution}
  You can e.g. do a tree proof, using $\Bel$ as the box.
\end{solution}

If we want to model the connection between knowledge and belief, we need a
multi-modal language with both the $\Kn$ operator and the $\Bel$ operator.
Models for this language will have two accessibility relations $R_{e}$ and
$R_{d}$. The first represents epistemic accessibility and is used for the
interpretation of $\Kn$, the second represents doxastic accessibility and is
used to interpret $\Bel$.

The power of combined logics for (implicit) knowledge and belief lies in the
interaction principles that might link the two concepts. Here is a list of
popular principles that don't follow from the individual logics of knowledge and
belief.
\begin{principles}
  \pri{KB} \Kn A \to \Bel A\\
  \pri{PI} \Bel A \to \Kn\Bel A\\
  \pri{NI} \neg \Bel A \to \Kn\neg \Bel A\\
  \pri{SB} \Bel A \to \Bel \Kn A
\end{principles}

\pr{KB} assumes that knowledge implies belief. \pr{PI} and \pr{NI} strengthen
the introspection principles for belief. They assume that a state of belief or
disbelief is always known to the agent. \pr{SB} assumes that if an agent
believes something then they also believe that they know it. This is sometimes
said to reflect a conception of ``strong belief'', on which belief is
incompatible with doubt. If you believe $p$ in the sense that you have no doubt
that $p$, then you plausibly believe that you know $p$.

These interaction principles, together with the \pr{D}-schema for belief, imply
that an agent believes a proposition just in case they don't know that they
don't know it:
%
\principle{BMK}{\Bel A \leftrightarrow \Mi\Kn A}
%
Somewhat surprisingly, then, we could define belief in terms of knowledge.

Here is how we can get from $\Bel A$ to $\Mi\Kn A$.
%
\begin{enumerate}[leftmargin=10mm]
  \itemsep-1mm
\item Suppose $\Bel A$.
\item By \pr{SB}, it follows that $\Bel \Kn A$.
\item By \pr{D}, it follows that $\neg \Bel \neg \Kn A$.
\item By \pr{KB}, it follows that $\neg \Kn \neg \Kn A$, and so that $\Mi\Kn A$.
\end{enumerate}

To show that $\Mi\Kn A$ entails $\Bel A$, I'll show that $\neg \Bel A$
entails $\neg \Mi\Kn A$.
%
\begin{enumerate}[leftmargin=10mm]
  \itemsep-1mm
\item By \pr{KB}, $\neg \Bel A \to \neg \Kn A$ is a logical truth.
\item Since logical truths are true at every world, we have $\Kn(\neg \Bel A \to \neg \Kn A)$.
\item By the \pr{K}-schema, it follows that $\Kn\neg \Bel A \to \Kn \neg \Kn A$.
\item Now suppose $\neg \Bel A$.
\item By \pr{NI}, it follows that $\Kn \neg \Bel A$.
\item By 3 above, it follows that $\Kn \neg \Kn A$, which is equivalent to $\neg \Mi\Kn A$.
\end{enumerate}

Given the equivalence between $\Bel A$ and $\Mi\Kn A$, the \pr{D}-schema for
belief
\[
  \Bel A \to \neg \Bel\neg A
\]
is equivalent to
\[
  \Mi \Kn A \to \neg\Mi\Kn \neg A
\]
which in turn is equivalent to
\[
  \Mi \Kn A \to \Kn\Mi A.
\]
This is the \pr{G}-schema for knowledge. So if we accept the above interaction
principles, and principle \pr{D} for belief, then the logic of knowledge must
validate \pr{G}.

(In fact, we don't need to assume that the interaction principles and \pr{D}
hold for our ordinary concept of belief. As long as one can coherently define a
concept $\Bel$ that validates these principles we can derive the \pr{G}-schema
for $\Kn$.)

\begin{exercise}
  Show that the interaction principles entail principles \pr{4} and \pr{5} for
  belief: $\Bel A \to \Bel\Bel A$ and $\neg\Bel\neg A \to \Bel\neg\Bel\neg A$.
\end{exercise}
\begin{solution}
  Let $A$ be an arbitrary proposition.

  By \pr{PI}, $\Bel A \to \Kn\Bel A$ is valid. By \pr{KB}, so is
  $\Kn\Bel A \to \Bel\Bel A$. By propositional logic, these entail
  $\Bel A \to \Bel\Bel A$.

  By \pr{NI}, $\neg \Bel \neg A \to \Kn\neg \Bel \neg A$ is valid. By \pr{KB},
  so is $\Kn\neg \Bel \neg A \to \Bel\neg \Bel \neg A$. By propositional logic,
  these entail $\neg \Bel \neg A \to \Bel\neg \Bel \neg A$.
\end{solution}

\begin{exercise}
  Suppose the logic of knowledge validates \pr{5}, the logic of belief validates
  \pr{D}, and we have the interaction principles \pr{KB} and \pr{SB}. Show that
  knowledge is then equivalent to belief: $KA \leftrightarrow BA$ comes out as
  valid. (Another reason to think that \pr{5} is not valid in the logic of
  knowledge.)
\end{exercise}
\begin{solution}
  The left-to-right direction is \pr{KB}. For the right-to-left direction, let
  $A$ be an arbitrary proposition. By \pr{SB}, $\Bel A \to \Bel\Kn A$ is valid.
  By \pr{D} for belief, $\Bel \Kn A \to \neg\Bel \neg \Kn A$ is valid. \pr{KB}
  gives us $\neg \Bel \neg \Kn A \to \neg\Kn\neg\Kn A$. Finally, \pr{5} for
  knowledge yields $\neg \Kn\neg A \to \Kn A$. The target proposition
  $\Bel A \to \Kn A$ is a truth-functional consequence of these four
  propositions.
  % BA -> BKA (SB)
  % BKA -> -B-KA (D)
  % -B-KA -> -K-KA (KB)
  % -K-KA -> KA (5)
\end{solution}

% In KD45 R is an equivalence relation ``once removed'': any accessible world sees
% itself, and there's symmetry among all accessible worlds.

\begin{exercise}
  There seems to be no natural expression in English for the dual of belief. A
  common way to express that someone does not believe not $p$ is to say that
  they believe that it might be that $p$, which has the surface form
  $\Box\Diamond p$. Can you explain why this might be an adequate way of expressing $\Diamond p$?
\end{exercise}
\begin{solution}
  If the logic of belief is KD45 then $\Box\Diamond p$ is equivalent to
  $\Diamond p$ (as you can show, for example, with a tree proof).
\end{solution}

It can also be instructive to combine epistemic with non-epistemic operators.
Philosophers have often been interested not just in what we \emph{do} know, but
also in what we \emph{can} know. Various skeptical arguments, for example,
suggest that we \emph{cannot know} that we have hands. For another example, the
``verificationist'' movement in the early 20th century assumed that a sentence
is meaningful only if its truth-value can in principle be settled by
mathematical proof or empirical investigation. This would imply that a sentence
is meaningful only if \emph{it is possible to know} that it is true.

We can formalize claims like these in a multi-modal language with a knowledge
operator $\Kn$ and a diamond $\Diamond$ for the relevant kind of circumstantial
possibility. The verificationist hypothesis that every truth is in principle
knowable is then expressed by the following interaction principle:
%
\begin{equation}\tag{Knowability}
  A \to \Diamond \Kn A
\end{equation}

The principle is refuted by the following argument, due to Alonzo Church.

\begin{enumerate}[leftmargin=10mm]
  \itemsep-1mm
  \item Let $p$ be any unknown truth. (Nobody thinks all truths are actually
        known.)
  \item So we have $p \land \neg \Kn p$.
  \item In any logic that extends the minimal system K,
        $\Kn(p \land \neg \Kn p)$ entails $\Kn p \land \Kn \neg \Kn p$.
  \item By the \pr{T}-schema for knowledge, $\Kn \neg \Kn p$ entails
        $\neg \Kn p$.
  \item So $\Kn(p \land \neg \Kn p)$ entails both $\Kn p$ and $\neg \Kn p$.
  \item So the hypothesis $\Kn(p \land \neg \Kn p)$ is inconsistent.
  \item So $\neg \Diamond \Kn(p \land \neg \Kn p)$.
  \item Lines 2 and 7 together provide a counterexample to the Knowability
        principle.
\end{enumerate}

\begin{exercise}
  Show that if the logic of belief is at least KD4, then there are
  \emph{unbelievable truths}: truths of which it is impossible that anyone
  believes them. (You can assume that there are truths which no-one in fact
  believes.)
\end{exercise}
\begin{solution}
  Suppose $\Bel(p \land \neg \Bel p)$. In any logic that extends K, it follows
  that $\Bel p$ and $\Bel \neg \Bel p$. By \pr{4}, $\Bel p$ entails
  $\Bel\Bel p$. Now we have $\Bel \neg \Bel p$ and $\Bel\Bel p$, which violates \pr{D}.
\end{solution}








%%% Local Variables: 
%%% mode: latex
%%% TeX-master: "logic2.tex"
%%% End:

\chapter{Deontic Logic}\label{ch:deontic}

\section{Permission and obligation}

Deontic logic studies formal properties of obligation, permission, prohibition,
and related normative concepts. The box in deontic logic is usually written
`$\Ob$' (for `obligation' or `ought'), the diamond `$\Pe$' (for `permission').
If we read $q$ as stating that you cook dinner, we might use $\Ob q$ to express
that you are obligated to cook dinner.

We assume that obligation and permission are duals. You are not obligated to
cook dinner iff you are permitted to not cook dinner; you are not permitted to
cook dinner iff you are obligated to not cook dinner.

There are many kinds of norms: legal norms, moral norms, prudential norms,
social norms, and so on. There may also be overarching norms that combine some
or all of the others. Deontic logic is applicable to norms of all kinds. We do
not have to settle whether $\Ob$ expresses legal obligation or moral obligation
or some other kind of obligation. It is important, however, that we don't
equivocate. If the law requires $q$ and morality $\neg q$, we should not
formalize this as $\Ob q \land \Ob\! \neg q$. It would be better to use a
multi-modal language with different operators for legal and moral obligation.

% Humberstone 245 discusses the idea of using multi-relational frames in which
% the different relations correspond to different sources of norms or values.
% One may then read OA as true iff /some/ norm says that A is ideal. In this
% semantics, O(A) and O(B) do not entail O(A & B). Compare Lewis on
% compartmentalised beliefs.

Obligations and permissions often vary from agent to agent. If it is your turn
to cook dinner then you are obligated to cook dinner, but I am not. To capture
this agent-relativity, we could add agent subscripts to the operators, as we did
in epistemic logic. We could then express our different obligations as
$\Ob_1 \!q \land \neg \!\Ob_2 \!q$. But what does the sentence letter $q$ stand for?
When I say that you are obligated to cook dinner, the object of the obligation
appears to be a type of act: cooking dinner. In the language of modal
propositional logic, $\Ob$ and $\Pe$ are sentence operators. Unless we want to
say that verb phrases in English (like `cook dinner') should be translated into
sentences of $\L_M$ -- which is possible, but non-standard -- we have to
transform the acts that appear to be the true objects of obligation and
permission into propositions.

Consider sentence (1), which is arguably equivalent to (2).
\begin{itemize}[leftmargin=10mm]
\itemsep-1mm
\item[(1)] You ought to cook dinner.
\item[(2)] You ought to see to it that you cook dinner.
\end{itemize}
In (2), the operator `you ought to see to it that' attaches to a sentence, `you
cook dinner'. So we can translate (1) via (2) as $\Ob_1 \!q$, where $q$ translates
`you cook dinner', and $\Ob_1$ corresponds to `you ought to see to it that'.

The subject (you) is mentioned twice in (2). A common assumption in deontic
logic is that we can drop the agent subscripts from deontic operators, since the
embedded proposition will tell us upon whom the obligation or permission falls.
Informally, the idea is that (2) is equivalent to (3), with an impersonal
`ought'.
\begin{itemize}[leftmargin=10mm]
\itemsep1mm
\item[(3)] It ought to be the case that you cook dinner.
\end{itemize}

The impersonal `ought' also figures in statements like (4).
\begin{itemize}[leftmargin=10mm]
\itemsep1mm
\item[(4)] Nobody ought to die of hunger.
\end{itemize}
When I say (4), I don't mean that nobody is obligated to die of hunger. Nor do I
mean that everybody is obligated to not die of hunger. Rather, I mean that a
certain state of affairs -- that nobody dies of hunger -- ought to be the case.
Without further assumptions, this does not impose any obligations on anyone.

% The reducibility of personal to impersonal ought was suggested by Meinong and
% Chisholm, see Handbook of Modal Logic p 1204.

There are reasons to question the equivalence between agent-relative `ought'
statements like (2) and impersonal `ought' statements like (3).
Suppose Amy has promised to play with Betty. Then Amy is obligated to play with
Betty. But Betty is not thereby obligated to play with Amy. Betty may even have
promised not to play with Amy. It is hard to express these facts in terms of
impersonal oughts. If we say that it ought to be the case that Amy plays with
Betty, we're missing the fact that the obligation falls on Amy, not on Betty
(who might be under a contrary obligation). So perhaps it would be better to
keep the agent subscripts after all.

It can also be useful to make the `see to it that' component in statements like
(2) explicit. That Amy ought to play with Betty could then be translated as
$\Ob_a \stit p$, where $\stit$ formalizes `sees to it that'. This allows us to
distinguish between the following three claims.

\bigskip
\begin{tabular}{ll}
  $\Ob_a \stit \neg p$ & Amy ought to see to it that she doesn't play with Betty.\\
  $\Ob_a \neg \stit p$ & Amy ought to not see to it that she plays with Betty.\\
  $\neg \Ob_a \stit p$ & It is not the case that Amy ought to see to it that she plays with\\[-0.5mm]
                       & Betty.
\end{tabular}
\bigskip

% Others have suggested that an adequate deontic logic should include special
% terms for actions in addition to terms for propositions. We could then write
% $\Ob_a \alpha$, where $\alpha$ stands for the (possible) action of Amy playing
% with Betty. This leads to the field of \emph{dynamic deontic logic}. Here,
% action terms are themselves treated as modal operators, associated with their
% own accessibility relation: a world $w'$ is accessible from $w$ through action
% $\alpha$ if performing $\alpha$ in $w$ could bring about $w$. Thus if $p$ is
% the proposition that Betty is happy, one could use $[\alpha]p$ to express that
% Amy's dancing with Betty would definitely lead to Betty being happy, while
% $\langle \alpha \rangle p$ would expresses that the action $\alpha$
% \emph{could} lead to $p$.

% Hilpinen says stit accounts are more sophisticated versions of these.

The $\stit$ operator has proved useful to represent different concepts of rights
and duties. In what follows, we will nonetheless stick to the simplest (and
oldest) approach, without a $\stit$ operator and without agent subscripts. This
approach is sufficient for many applications, but its limitations should be kept
in mind.

% \begin{exercise}
%   Let $\mathsf{F}A$ mean that $A$ is forbidden. Can you define
%   $\mathsf{F}$ in terms of $\Ob$ or $\Pe$ (or both)?
% \end{exercise}
% \begin{solution}
%   For example: $\Ob \neg$, or $\neg \Pe$.
% \end{solution}

\begin{exercise}\label{ex:translate-sdl}
  Translate the following sentences into the standard language of deontic logic (without $\stit$ or agent subscripts).
  \begin{exlist}
  \item You must not go into the garden.
  \item You may not go into the garden.
  \item Jones ought to help his neighbours.
  \item If Jones is going to help his neighbours, then he ought to tell them
    he's coming.
  \item If Jones isn't going to help his neighbours, then he ought to not tell
    them he's coming.
  \end{exlist}
\end{exercise}
\begin{solution}
  \begin{sollist}
    \item $\Ob \neg p$; \quad $p$: You go into the garden.
    \item $\Ob \neg p$; \quad $p$: You go into the garden.
    \item $\Ob p$; \quad $p$: Jones helps his neighbours.
    \item $\Ob (p \to q)$; \quad $p$: Jones helps his
    neighbours, $q$: Jones tells his neighbours that he's coming.
    \item You might try $\Ob (\neg p \to \neg q)$ or $\neg p \to \Ob \neg q$ \quad $p$: Jones helps his neighbours, $q$: Jones tells his neighbours that he's coming.
  \end{sollist}
  See section \ref{sec:oblig-circ}, especially exercise \ref{ex:chisholmsparadox},
  for why neither translation of (e) is fully satisfactory.
\end{solution}


\section{Standard deontic logic}

Think of a possible world as a history of events. For any such history, and any
system of norms, we can ask whether the history conforms to the norms. Let's
call a world \emph{acceptable} relative to some norms if everything that happens
at the world conforms to the norms. That is, a world is acceptable if it
contains no violation of any relevant norm.

By definition, whatever happens at an acceptable world is permitted, in the
sense that it does not violate any (relevant) norms. The converse is plausible
as well: whenever something is permitted then it is the case at some acceptable
world. For example, if it is permitted that Amy plays with Betty, then there
should be a complete history of events in which Amy plays with Betty and no
norms are violated. If there were no such history, then Amy's playing with Betty
would logically entail the violation of some norms; but if an act entails the
violation of some norms, then it is hard to see how the act could be permitted
relative to these norms.

So we have the following connection between permission and acceptable worlds,
which amounts to a possible-worlds analysis of permission:
\begin{quote}
  $A$ is permitted (relative to some norms) iff $A$ is the case at
  some possible world that is acceptable (relative to these norms).
\end{quote}
%
Given the duality of permission and obligation, we also get a possible-worlds
analysis of obligation:
%
\begin{quote}
  $A$ is obligatory (relative to some norms) iff $A$ is the case at
  all worlds that are acceptable (relative to these norms).
\end{quote}

In logic, we are not interested in who is in fact obligated to do what, but in
whether a given deontic statement is logically valid, or whether it logically
follows from other statements.

Validity means truth in every conceivable scenario under every interpretation of
the non-logical vocabulary. A scenario for deontic logic has to specify the
relevant norms. This can be done by specifying which worlds are acceptable
relative to which other worlds.

A Kripke model represents a scenario of this type, together with an
interpretation of the sentence letters. In this application, a world $v$ in the
model is accessible from a world $w$ if $v$ is acceptable relative to the norms
at $w$ -- equivalently, if everything that ought to be the case at $w$ is the
case at $v$. Worlds that are accessible from $w$ in this sense are called
\textbf{ideal} relative to $w$.

Our possible-worlds analysis of obligation and permission is reflected in
definition \ref{def:kripkesemantics}, which settles under what conditions a
sentence is true at a world in a model. Writing the box as `$\Ob$' and the
diamond as $\Pe$', clause (g) of the definition states that $\Ob A$ is true at a
world $w$ in a model $M$ iff $A$ is true at all worlds of $M$ that are ideal
relative to $w$. Clause (h) states that $\Pe A$ is true at $w$ in $M$ iff $A$ is
true at some world that is ideal relative to $w$.

A sentence is valid iff it is true at every world in every suitable model. If we
count all Kripke models as suitable, the logic of obligation and permission will
be the minimal normal modal logic K. We can get stronger logics by imposing
constraints on the accessibility relation. Let's have a look at a few options.

We might stipulate that the deontic accessibility relation is reflexive, so that
every world can see itself. This would make all instances of the \pr{T}-schema
valid:
%
\principle{T}{\Ob A \to A}
%
In deontic logic, the \pr{T}-schema is highly implausible. The fact that
something ought to be the case does not entail that it is the case. Semantically
speaking, many worlds are not ideal relative to themselves. We will not assume
reflexivity.

We might, however, impose the weaker condition of seriality -- that each world
can see some world. This would validate principle \pr{D}:
%
\principle{D}{\Ob A \to \Pe A}
%
Intuitively, \pr{D} says that the norms are consistent: if you're obligated to
do $A$, then you are not obligated to do not-$A$. (Remember that $\Pe A$
is equivalent to $\neg\! \Ob\! \neg A$.) Semantically, \pr{D} corresponds to the
assumption that there is always at least one world at which all the norms are
satisfied.

Without seriality, we have to allow for worlds from which no world is
accessible. At such a world, all sentences of the form $\Ob A$ are true, and all
sentences of the form $\Pe A$ are false. Everything is obligatory, but nothing
is allowed. It is hard to make sense of such a situation. If we use Kripke
semantics for deontic logic, we should rule out inconsistent norms and accept
\pr{D} as valid.

Here it may be important to distinguish \emph{prima facie} obligations from
\emph{actual}, or \emph{all-things-considered} obligations. If you've promised
to cook dinner, you are under a \emph{prima facie} obligation to cook dinner.
But the obligation can be overridden by intervening circumstances or contrary
obligations. If your child has an accident and needs urgent medical care, the
right thing to do may well be to not cook dinner and instead bring your child to
the hospital. In a sense, you are under conflicting obligations: you ought to
cook dinner, and you ought to look after your child (and not cook dinner). There
is no world at which you meet both of these obligations. But that is not a
counterexample to \pr{D}, if we understand $\Ob$ as all-things-considered
obligation. You are \emph{prima facie} obligated to cook dinner, but all things
considered, you should not cook dinner.

Let's return to the non-reflexivity of the deontic accessibility relation. Many
things that are not the case nonetheless ought to be the case. Some have argued
that this is only true in non-ideal worlds. In an ideal world, everything that
ought to be the case is the case. By this line of thought, if a world $v$ is
accessible from some world $w$ -- meaning that $v$ is ideal relative to $w$ --
then $v$ should be accessible from itself. This condition is sometimes called
``shift reflexivity'' and corresponds to the following schema \pr{U} (for
``utopia'')
%
\principle{U}{\Ob(\Ob A \to A)}
%
In words: it ought to be the case that whatever ought to be the case is the
case.

The \pr{U} principle is entailed by an alternative way of formalizing obligation
and permission that goes back to Leibniz. Let `$\OK$' be a propositional
constant whose intended meaning is that all norms are satisfied, no obligations
violated. Suppose we add this expression to $\L_M$, and we interpret the box of
$\L_M$ as a suitable kind of circumstantial necessity. Leibniz's idea was that
$\Ob A$ is definable as $\Box (\OK \to A)$: it ought to be that $A$ iff,
necessarily, $A$ is the case whenever all obligations are met. It is not hard to
show that if the \pr{T}-schema is valid for the circumstantial box, and $\Ob A$
is defined as $\Box(\OK \to A)$, then the \pr{U}-schema is valid for $\Ob$.

% Kanger uses 'Q' instead of '$\OK$'. Note that the same approach could be used
% in other contexts, e.g. having 'Q' specify that the laws of nature are
% satisfied, or someone's beliefs. See Humberston 257.

% Humberstone 259 shows that if the box satisfies T and we have <>Q, then the
% Q-logic is precisely KDU.

\begin{exercise}
  \beginwithlist
  \begin{exlist}
    \item Translate the \pr{U}-schema into the Leibnizian language just
    proposed.
  \item Give a tree proof for the translated \pr{U}-schema, using the T-rules
    for the box. 
  \end{exlist}
\end{exercise}
\begin{solution}
  (a): $\Box(\OK \to (\Box(\OK \to A) \to A))$. (b): use 
  \href{https://www.umsu.de/trees/}{umsu.de/trees/}.
\end{solution}

% \begin{exercise}
%   Show that if $\Box$ satisfies \pr{4}, then so does $\Ob$.
% \end{exercise}

\begin{exercise}
  How could we define $\Pe$ in terms of $\Box$ and $\OK$, so that $\Pe$ is the
  dual of $\Ob$? 
\end{exercise}
\begin{solution}
  $\Pe A$ could be defined as $\neg\Box(\OK \to \neg A)$, or more simply (and
  equivalently) as $\Diamond (\OK \land A)$.
\end{solution}



% A weakening of $U$ is 4C or density:
% \[
%   \Ob\Ob A \to \Ob A.
% \]
% This means that every deontic alternative is a deontic alternative to some
% deontic alternative.

Turning to more familiar schemas and frame conditions, what shall we say about
transitivity and euclidity, and the corresponding schemas \pr{4} and \pr{5}?
%
\begin{principles}
\pri{4}{\Ob A \to \Ob\Ob A}\\
\pri{5}{\Pe A \to \Ob\Pe A}
\end{principles}
%
If something ought to be the case, ought it to be the case that it ought to be
the case? If something is permitted, is it obligatory that it is permitted?
Iterations of deontic operators sound strange in ordinary language. But they
have a well-defined meaning in our Kripke semantics. The validity of \pr{4}
would mean that whenever something is obligatory at a world, then it is also
obligatory at all ideal alternatives to that world. \pr{5} would mean that if
something is permissible at a world, then it's also permissible at all ideal
alternatives to that world. On the background of \pr{D}, these two assumptions
together imply that for each world there is a class of ideal worlds all of which
are ideal relative to one another.

To get a clearer grip on whether that is plausible, we need to clarify how
obligations and permissions can vary from world to world.

One obvious sense in which norms can vary across worlds is that people subscribe
to different norms at different worlds. In our world, UK traffic law requires
driving on the left, and most people think it is morally wrong to torture animals
for fun. At other worlds, the laws and attitudes are different.

Let $v$ be a world at which the traffic laws require driving on the right, and
at which everyone thinks it is fine to torture animals. Suppose Norman at $v$ is
torturing kittens, while driving on the right (in the UK). Is Norman doing
something that's morally wrong? Is he doing something that violates the traffic
laws? The answer depends on whether we evaluate Norman's acts relative to our
norms -- the norms at our world -- or relative to the norms at Norman's world.
Both perspectives are intelligible. They lead to different deontic logics.

On an \textbf{absolutist} conception, the basic norms do not vary from world to world.
Whichever world we look at, we always assess it relative to the same set of
norms. On this conception, it is natural to assume that the very same worlds are
ideal relative to any world: a world will be accessible from any world just in
case it contains no violation of the (fixed) norms. The resulting logic of
obligation and permission is KD45.

% Strictly speaking, I am here also assuming that not all worlds are ideal, and
% that there is more than one ideal world.

\begin{exercise}
  Explain why the deontic accessibility relation is transitive and euclidean if the same worlds are ideal relative to any world.
\end{exercise}
\begin{solution}
  Transitivity (if $wRv$ and $vRu$ then $wRu$) and euclidity (if $wRv$ and $wRu$
  then $vRu$) both state that if $v$ is ideal and $u$ is ideal then $u$ is
  ideal.
\end{solution}
\begin{exercise}
  Show that euclidity implies shift reflexivity.
  % This is needed later in applying the euclidity rule for trees.
\end{exercise}
\begin{solution}
  $R$ is euclidean if $\forall x \forall y \forall z((xRy \land xRz) \to yRz)$.
  Suppose $wRv$. Instantiating the universal formula with $w$ for $x$ and with
  $v$ for $y$ and $z$, we have $(wRv \land wRv) \to vRv$. So $vRv$.
\end{solution}

% when I draw a diagram for euclidity I shouldn't use a symmetrical arrow from v
% to u: if I draw the reverse arrow, I should also draw the loop at v and u.
% Maybe that's a good point to highlight in the text, by way of explaining that
% the variables don't need to pick out distinct worlds.

On a \textbf{relativist} conception of norms, we evaluate the events at other
worlds relative to the norms at these worlds. Transitivity and euclidity now
become implausible, as does shift reflexivity. To see why, add another world $u$
to the Norman scenario. The laws at $u$ say that one must drive on the right.
But the inhabitants of $u$ are rebellious: everyone at $u$ drives on the left.
Nothing that happens at $u$, we may assume, violates the traffic laws of our
world. So $u$ is deontically accessible from the actual world. But if we
evaluate the events at $u$ relative to the laws at $u$, then much of what
happens at $u$ violates the norms, so $u$ is not deontically accessible from
itself. Shift reflexivity fails.

\begin{exercise}
  Explain why deontic accessibility is neither transitive nor euclidean, on
  the relativist conception.
\end{exercise}
\begin{solution}
  Consider the example from the text, where $w$ is the actual world (in the UK)
  and $u$ is a $w$-accessible world at which everyone drives on the left
  although the law says that one must drive on the right. A typical world
  accessible from $u$ will be a world at which people drive on the right. This
  world will not be accessible from $w$. So we have a counterexample to
  transitivity. We also have a counterexample to euclidity because we have $wRu$
  and $wRu$ but not $uRu$. (Euclidity entails shift reflexivity.)
\end{solution}

The relativist conception is more common in deontic logic. So-called
\textbf{standard deontic logic} assumes only that the accessibility relation is
serial, making the system D the complete logic of obligation and permission.

The proposed logics of absolutism and relativism only disagree about sentences
in which a deontic operator occurs in the scope of another deontic operator. Any
sentence that does not contain an $\Ob$ or $\Pe$ operator embedded under another
$\Ob$ or $\Pe$ operator is D-valid iff it is KD45-valid.

% Proof: Clearly everything that's D-valid is KD45-valid, so we need to show
% that if a degree-1 sentence A is KD45-valid, then A is also D-valid. By
% contraposition, suppose A is not D-valid, meaning that there is a world w in a
% serial model M at which A is false. We construct a world w' in a
% serial+euclidean+transitive model M' at which A is false. The worlds of M' are
% w and every world that can be seen from w in M. Every world that can be seen
% from w can see every other such world in M'. The interpretation function in M'
% is like that in M, restricted to the relevant worlds. Trivially, all degree-0
% sentences have the same truth-values at all worlds in M' as they have in M.
% For degree-1 sentences A, we show by induction on complexity that M,w |= A iff
% M',w |= A. Case (1): A is atomic. Trivial. Case (2): A is -B. Trivial by i.h.
% Case (3): A is BvC. Trivial by i.h. Case (4): A is []B. Then B is degree-0. We
% know that B has the same truth-value at any w-accessible world in M and M'. So
% B is true at all w-accessible worlds in M iff it is true at all w-accessible
% worlds in M'. So []B is true at w in M iff []B is true at w in M'. QED.

\begin{exercise}
  Use the tree method to check which of the following sentences are D-valid and
  which are KD45-valid.
  \begin{exlist}
    % \item $\Ob p \to \Ob (p \lor q)$
    \item $\Pe (p \lor q) \to (\Pe p \land \Pe q)$
    % \item $(\Ob p \land \Ob q) \to \Ob(p \land q)$
    % \item $\Pe p \to \Pe(p \lor q)$
    \item $\Ob\Pe p \to \Pe p$
    % \item $\neg (\Ob p \land \Ob \neg p)$
    \item $\neg\Pe(p \lor q) \to (\Pe \neg p \lor \Pe\neg q)$
    \item $\Ob\Pe p \lor \Pe\Ob p$
  \end{exlist}
\end{exercise}
\begin{solution}
  Use
  \href{https://www.umsu.de/trees/}{https://www.umsu.de/trees/}.
  (Write $\Ob$ as a box and $\Pe$ as a diamond. For D, make the accessibility
  relation serial; for KD45, make it serial, transitive, and euclidean.)
\end{solution}

\begin{exercise}
  Consider a world in which there are no sentient beings, and nothing else that
  could introduce norms or laws. Since there are no norms at this world, one
  might hold that nothing is obligatory relative to the world's norms, and
  nothing is permitted. Explain why this casts doubt on the validity of
  \pr{Dual1} and \pr{Dual2} in the logic of relativist obligation and
  permission.
\end{exercise}
\begin{solution}
  \pr{Dual1} says that $\neg \Diamond A$ is equivalent to $\Box \neg A$. If
  nothing is permitted then $\neg \Diamond A$ is true for all $A$. But if
  nothing is forbidden then $\Box \neg A$ is false for all $A$.

  \pr{Dual2} says that $\neg \Box A$ is equivalent to $\Diamond \neg A$. If
  nothing is forbidden then $\neg \Box A$ is true for all $A$. But if
  nothing is permitted then $\Diamond \neg A$ is false for all $A$.
\end{solution}

\begin{exercise}\label{ex:amybettycarla}
  Amy ought to have either promised to help Betty or to help Carla. She hasn't
  made either promise. If she had promised to help Betty, she would be obligated
  to help Betty. If she had promised to help Carla, she would be obligated to
  help Carla. So it ought to be the case that Amy is either obligated to help
  Betty or obligated to help Carla. In fact, since Amy made neither promise, she
  is neither obligated to help Betty nor to help Carla. Explain why this casts
  doubt on the assumption that deontic accessibility is euclidean.
\end{exercise}
\begin{solution}
  In the described situation, it ought to be the case that Amy is either
  obligated to help Betty or obligated to help Carla, but Amy is neither
  obligated to help Betty nor to help Carla. So if $p$ translates `Amy helps
  Betty' and $q$ `Amy helps Carla', we seem to have $\Ob(\Ob p \lor \Ob q)$ and
  $\neg \Ob p$ and $\neg \Ob q$. But these assumptions are inconsistent in K5.
  You can draw a K5-tree (using the K-rules and the Euclidity rule) starting
  with $\Ob(\Ob p \lor \Ob q)$ and $\neg \Ob p$ and $\neg \Ob q$ on which all
  branches close. This shows that there is no world in any euclidean model at
  which the three assumptions are true.
  % The deeper point here is that obligations often depend on the circumstances
  % -- for example on what Amy has promised to do. Even if we hold fixed the
  % underlying norms ("keep your promises"), we shouldn't hold fixed the
  % specific obligations and permissions that arise from these norms together
  % with the circumstances. See section \sec{sec:oblig-circ} for more on this
  % theme. (This is a bank robber paradox type case. If we don't hold fixed what
  % was promised then having promised to p does not entail Op:
  % not-promise-and-not-p worlds are OK.)
\end{solution}

% \begin{exercise}
%   The \pr{C4}-schema $\Ob\Ob A \to \Ob A$ is entailed by the \pr{U}-schema
%   $\Ob(\Ob A \to A)$ in the sense that whenever an instance of \pr{U} is true at
%   a world in a model then so is the corresponding instance of \pr{C4}. What about the other direction? Does the \pr{C4}-schema entail the \pr{U}-schema? 
%   \begin{exlist}
%   \item if all instances of \pr{U} are valid on a frame, then so are all instances of \pr{C4};
%   \item it is not the case that if all instances of \pr{C4} are valid on a frame,
%     then so are all instances of \pr{U}.
%     % Need to give a counterexample frame where C4 is valid but not U.
%     % U invalid means not shift reflexive. But still dense. That's
%     % hard. Obvious example: W = Reals, R = <. Then wherever we can
%     % get in one step we can get in two steps (C4), but we don't have
%     % shift reflexivity. Does a tree help?
%   \end{exlist}
% \end{exercise}
% \begin{solution}
%   \begin{sollist}
%     \item I argue by contraposition. Suppose some sentence $\Ob\Ob A \to \Ob A$
%     is invalid on a frame. This means that at some world $w$ in some model $M$
%     based on the frame, $\Ob\Ob A$ is true while $\Ob A$ is false. It follows
%     that there is a world accessible from $w$ at which $A$ is false and $\Ob A$
%     true. So $\Ob A \to A$ is false at $v$. So $\Ob (\Ob A \to A)$ is false at
%     $w$. (You could also give a tree proof with the K-rules showing that \pr{U}
%     entails \pr{C4}.)
    
%     \item It is not enough to give a model in which some instance of \pr{C4} is
%     true at some world while the corresponding instance of \pr{U} is false. For
%     a counterexample, you need to give a \emph{frame} on which every instance of
%     \pr{C4} is valid but not every instance of \pr{U}. Here is one such frame:
%     $W = \{w,v\}$, $wRw$, $wRv$, and $vRw$.
%   \end{sollist}
% \end{solution}

\section{Norms and circumstances}\label{sec:oblig-circ}

The possible-worlds analysis from the previous section assumes that something
ought to be the case iff it is the case at all ideal worlds, where no norms are
violated. Many ordinary statements about oughts and obligations do not fit this
analysis.

Suppose you are walking past a drowning baby. You ought to save the baby. But
are you saving the baby at every world at which no norms are violated? Clearly
not. There are worlds at which the baby never fell into the pond, and others at
which you are overseas and have no means to rescue the baby. These worlds need
not involve any violations of norms.

Whether something ought to be the case depends not just on the norms but also on
the circumstances. Under circumstances in which you have the opportunity to
save a drowning baby, you ought to save it. Under other circumstances you do
not.

% The point also applies to impersonal oughts. In worlds where an increase of
% greenhouse gases threatens to destabilise the climate, greenhouses gases ought
% to be reduced; in worlds where greenhouse gases never increased, there is no
% imperative for reduction.

We can account for the dependence of obligations on circumstances by changing
our interpretation of the accessibility relation. Previously, we assumed that a
world $v$ is accessible from $w$ iff all the norms at $w$ are respected at $v$.
On the new interpretation, we also require that the relevant circumstances at
$w$ are preserved at $v$. If $w$ is a world at which you come across a drowning
baby then any accessible world will also be a world at which you come across a
drowning baby.

As a first stab, we might redefine deontic accessibility as follows:

\begin{quote}
  A world $v$ is deontically accessible from a world $w$ iff (a) the relevant
  circumstances at $w$ also obtain at $v$, and (b) no norms from $w$ are
  violated at $v$.
\end{quote}
%
I use `relevant circumstances' as a placeholder for the circumstances we hold
fixed when we consider what ought to be the case. Often we hold fixed everything
that is \emph{settled} in the sense we studied in section \ref{sec:systems} --
everything that can no longer be changed. If the baby has fallen into the pond
at $w$, then there is nothing anyone can do to undo the falling; the falling is
a ``relevant circumstance'' that takes place at every world accessible from $w$.

% Humberstone 238 mentions that an example due to Aqvist reveals that the
% circumstances that are held fixed can't be defined temporally. This point is
% also made in von Fintel & Heim: "this fence should be white".

% The worlds of deontic logic are plausibly centred (or world-time pairs) since
% the truth-value of Op at w depends on what is still open at w.

% There's also the issue of actualism. We may or may not hold fixed that Prof
% Procrastinate won't write the review. Or suppose you ought to go to work, but
% don't. If you had gone, you'd have walked past a drowning baby, and you'd have
% been obligated to save it. It seems that at all accessible worlds, you rescue
% the baby. But it's odd to say that you should save the baby. Why? Are we
% holding fixed that you're not going to work? Then you're not even obligated to
% go to work. Apparently what we hold fixed depends on the prejacent.

Clause (b) in the above definition assumes that no norms are violated at any
accessible world. But if accessibility is restricted by circumstances, then this
is implausible because the relevant circumstances will often involve violations of
norms.

The problem is brought ought by Arthur Prior's ``Samaritan Paradox''.  Suppose
someone has been injured in a robbery, and Jones has the opportunity to help. We
want to say that Jones ought to help the victim. On the possible-worlds analysis
of `ought', this means that Jones helps the victim at all worlds accessible from
the actual world. It follows that the robbery took place at all these
worlds. (In a world without a robbery, there is no victim to help.)  But then
all the accessible worlds contain a violation of norms. In a truly ideal world,
nobody would have been robbed and injured.

In the Samaritan Paradox, the robbery is settled; it has happened at all worlds
that are compatible with the ``relevant circumstances''. None of these worlds
are ideal. Among these worlds, however, worlds at which Jones doesn't help the
victim are even \emph{worse}, in terms of norm violations, than worlds at which
he helps the victim. Both kinds of worlds are bad, because the victim got
robbed. But our norms don't just divide the possible worlds into good and bad;
they allow for finer distinctions between bad worlds and even worse worlds.
Jones ought to help the victim because that's what he does in the \emph{best}
worlds among those he can bring about, even though none of these worlds are
ideal.

% So we need a "better-than" ordering over the worlds, where a world is "better"
% if the agent comes closer to satisfying her obligations. Naively, this might
% be a matter of counting obligation violations, but we may want to classify two
% shopliftings as better than one murder.

% Lewis 1973:98f. discusses whether, in a relativist approach, one needs to
% account for "abnormal" worlds that induce no betterness order, for example
% because they have no god. If there are no spheres around such a world at all,
% nothing is obligatory and everything is permissible. He suggests that one
% might want W to be the trivial unique sphere, so that tautologies are
% obligatory. /Normality/ means no abnormal worlds. Relatedly, he wonders
% whether each world v is in the order induced by any given world w. (He calls
% this /universality/.) /Absoluteness/ is the idea that betterness is not
% world-relative.
%
% If we interpret the counterfactual in terms of betterness, we get conditional
% obligation; the might counterfactual is conditional permission. [100]
%
% Lewis mentions [102] that several treatments of conditional obligation are
% problematic because they validate inferences from 'if A ought C' to 'if A and
% B ought C' or conversely. He suggests that there are Sobel-type
% counterexamples where 'ought C' alternates with further conjuncts in the
% antecedent: `given that Jesse robbed the bank, he ought to confess, but given
% in addition that his confession would send his mother to an early grave, he
% ought not to; etc.' [Note that like Sobel's sequence, these are hard
% to reverse!]
%
% Lewis also mentions in a footnote [102] the circumstance-relativity of
% ordinary oughts: There is a natural way to construe 'It ought to be that psi'
% so that it does become true when Jesse robs the bank. It can be taken as
% tacitly conditional, meaning something like 'Given those actual circumstances
% that now cannot be helped, it ought to be that psi '. But this tacitly
% conditional and time-dependent construal is not the appropriate one when 'It
% ought to be that psi' is used as a reading for the unconditional obligation
% operator of standard tenseless deontic logic.

So here is a second pass at the revised definition of deontic accessibility.

\begin{quote}
  A world $v$ is deontically accessible from a world $w$ iff (a) the relevant
  circumstances at $w$ are also the case at $v$, and (b) $v$ is one of the best
  worlds, by the norms at $w$, among worlds at which the relevant circumstances
  from $w$ are the case.
\end{quote}

The revised accessibility relation combines circumstantial and purely deontic
conditions. It can be useful to separate these two components. To this end,
let's first add a circumstantial accessibility relation to our models. In
addition, a model needs to specify which worlds are better than others, relative
to the norms at any given world (which may be the norms at every world, on an
absolutist approach).

Let `$u \prec_w v$' mean that world $u$ is better than world $v$ relative to the
norms at $w$. The symbol `$\prec$' hints at the idea that $u$ contains
\emph{fewer} violations of norms than $v$. We assume that for any world $w$,
the relation $\prec_w$ is transitive. We also assume that it is asymmetric,
meaning that if $u \prec_w v$ then it is not the case that $v \prec_w u$.
Asymmetric and transitive relations are known as \textbf{strict partial orders}.

\begin{definition}{}{orderingmodel}
  A \textbf{deontic ordering model} consists of
  \vspace{-3mm}
  \begin{itemize*}
    \item a non-empty set $W$ (the worlds),
    \item a binary relation $R$ on $W$ (the circumstantial accessibility
    relation),
    \item for each world $w\in W$, a strict partial order $\prec_w$ on $W$ (the
    world-relative ranking of worlds as better or worse), and
    \item a function $V$ that assigns to each sentence letter of $\L_M$ a subset
    of $W$.
  \end{itemize*}
\end{definition}

Now we need to say under what conditions a sentence of the form $\Ob A$ is true
at a world in an ordering model. Informally, $\Ob A$ will be true at $w$ iff
$A$ is true at the best worlds among those that are circumstantially accessible.
Let's introduce one more piece of notation. For any set of worlds $S$ and any
partial order $\prec$, let $\emph{Min}^{\prec}(S)$ be the set of $\prec$-minimal
members of $S$:
\[
  \emph{Min}^{\prec}(S) =_\text{def} \{ v: v \in S \land \neg\exists u(u \in S \land u \prec v) \}.
\]
An expression of the form `$\{ x: \ldots x \ldots \}$' denotes the set of all
things $x$ that satisfy the condition $\ldots x \ldots$. So $Min^{<}(S)$ is the
set of all things $v$ that are members of $S$ and for which there are no members
$u$ of $S$ for which $u \prec v$.

Here, then, are the truth-conditions for $\Ob A$ and $\Pe A$ in deontic ordering
models:


\begin{definition}{Ordering semantics}{orderingsemantics}
  If $M$ is a ordering model and $w$ a world in $M$, then\\[1mm]
  $M,w \models \Ob A$ \text{ \;iff\; $M,v \models A$ for all $v \in \emph{Min}^{\prec_w}(\{ u: wRu\})$}\\
  $M,w \models \Pe A$ \text{ \;iff\; $M,v \models A$ for some $v \in \emph{Min}^{\prec_w}(\{ u: wRu\})$}
\end{definition}

This is just a formal way of saying that $\Ob A$ is true at $w$ iff $A$ is true
at the best worlds (by the norms at $w$) among the worlds that are circumstantially accessible at $w$.

If we want the \pr{D}-schema to be valid, we have to assume that there is always
at least one best world among the circumstantially accessible worlds, so that
$\emph{Min}^{\prec_w}(\{ u: wRu\})$ is never empty. Let's make this assumption.

The logic of obligation and permission now depends on formal properties of the
circumstantial accessibility relation $R$ and the deontic orderings $\prec_w$.
In section \ref{sec:systems}, I argued that the logic of historical necessity
(of what is settled and open) is S5. This suggests that in normal contexts, $R$
is an equivalence relation. If we adopt an absolutist approach, on which the
orderings $\prec_{w}$ are the same for every world $w$, we then still get KD45.
If we allow the orderings to vary from world to world, we still get D, unless we
impose further restrictions on the orderings.

% Why? If R is an equivalence relation then we can ignore the restriction to
% R-accessible worlds: A is true at w in M iff A is true at w in M*, where W* =
% [w]_R and R* is the universal relation. So []A is true at w iff A is true at
% all the w-best worlds. On the absolutist conception, the w-best worlds are
% best relative to every world. So []A is true at w iff A is true at all worlds
% within some fixed set. This yields KD45. On the relativist conception, the
% w-best worlds may be different from the v-best worlds. Every world has a set
% of best worlds, and the box quantifies over that set. This is just as before.

\begin{exercise}
  Suppose fatalism is true and the only world that is open (circumstantially
  accessible) relative to any world $w$ is $w$ itself. Can you describe the
  resulting deontic logic (on either an absolutist or a relativist approach)?
\end{exercise}
\begin{solution}
  Since we assume that there is always at least one best world among the
  accessible worlds, and the accessible worlds comprise just one world, it
  follows that $\Ob A$ is true at $w$ iff $A$ is true at $w$. The logic we get
  is the ``Triv'' logic that is axiomatized by adding the \pr{Triv}-schema
  $\Box A \leftrightarrow A$ to the standard axioms and rules for K. This logic
  is stronger than S5: all S5-valid sentences are Triv-valid. (We also have,
  among other things, all instances of $\Box A \leftrightarrow \Diamond A$.) The
  choice between absolutism and relativism makes no difference.
\end{solution}

% \begin{exercise}
%   Can you describe a deontic ordering model for the scenario from exercise
%   \ref{ex:amybettycarla}? 
% \end{exercise}
% \begin{solution}
%   One possible answer: Let $w$ be the world in which the story takes place. At
%   $w$, Amy doesn't make any promises and isn't helping anyone. Let $v$ be a
%   world at which Amy promises to help Betty and keeps her promise. Let $u$ be a
%   world at which Amy promises to help Carla and keeps her promise. $v$ and $u$
%   are better than $w$, and neither is better than the other. $v$ and $u$ are
%   circumstantially accessible from $w$, but not from each other: we can easily
%   make promises, but if we've made a promise we can't easily dispose of the
%   commitment.

  % This is tricky. Strictly, v and u are open only if the promising at these
  % worlds takes place in the future. Then u is accessible from v: at a world
  % where Amy is about to promise to help Betty she can still promise to help
  % Carla. Likewise, v is accessible from u. Both v and u contain zero norm
  % violations. So we can't explain why Ob is true at v. Indeed, in some sense
  % it isn't true that Amy ought (right now) to be obligated to help either
  % Betty or Carla, if it's OK for her to make the promise in the future. (If
  % she will make the promise to help Carla, she /will/ be obligated to help
  % Carla, but she isn't already under that obligation.)
  %
  % Does the puzzle arise more clearly if we assume that Amy ought to have made
  % the promise yesterday? The problem is that if Amy has promised yesterday to help Betty, and we leave open what promise she made (if any) then it isn't true that at the best worlds she helps Betty: she might instead have promised to help Carla and do that. 
% \end{solution}

% \begin{exercise}
%   It is safe to assume that the relevant circumstances at $w$ are the
%   case at $w$ itself. What does this imply about the ``absolutist''
%   logic of obligation and permission on the revised definition of
%   deontic accessibility, where we hold fixed the norms of the actual
%   world?
% \end{exercise}

Ordering models prove useful when we want to formalize statements with
modal operators and if-clauses, like (1)--(3).

\begin{itemize}[leftmargin=10mm]
\itemsep-1mm
\item[(1)] If you smoke then you must smoke outside.
\item[(2)] If you miss the deadline for tax returns then you must pay a fine.
\item[(3)] If you have promised to call your parents then you must call them.
\end{itemize}
%
How would you translate these into our language $\L_M$? We seem to face a choice
between (W) and (N).
\begin{itemize}[leftmargin=10mm]
\itemsep-1mm
\item[(W)] $\Ob(p \to q)$
\item[(N)] $p \to \Ob q$
\end{itemize}
In (W), the operator $\Ob$ is said to have \textbf{wide scope} because it
applies to the entire conditional $p \to q$. In (N), the operator has
\textbf{narrow scope} because it only applies to the consequent $q$.

On reflection, neither translation is satisfactory. Starting with (N), note that
$p \to \Ob q$ and $\neg \Ob q$ together entail $\neg p$. But from (1), together
with the assumption that you are not required to smoke ($\neg \Ob q$), we surely
can't infer that you do not in fact smoke.

% Also, consider (3*) "If you have promised to call your parents then you must
% kill the Prime Minister". If this is translated as (N) then anyone who is
% unsure about $p$ can't be sure that (3*) is false, for $p \to \Ob k$ is true
% whenever $p$ is false.
%
% Note also that (N) is true whenever $\Ob q$ is true. So the narrow-scope
% approach implies that whenever you ought to do something, then you have a
% conditional obligation to do it under any condition whatsoever. But
% intuitively, the fact that you ought to cook dinner does not imply that if
% your child needs urgent medical care then you ought to cook dinner.

(W) is not much better. For one, in our Kripke-style semantics, $\Ob(p \to q)$
is entailed by $\Ob(\neg p)$. But it is easy to imagine a scenario in which you
must not smoke, or you must submit your tax return before the deadline, but in
which (1) and (2) are false.

% Suppose you should not have promised to call your parents: $\Ob \neg p$. On
% the wide-scope approach, we could infer that if you promised to call your
% parents, then you must kill the Prime Minister.

Another problem with both (N) and (W) is that they would license a problematic
form of ``strengthening the antecedent''. For example, they both suggest that
(3)  entails (4).
\begin{itemize}[leftmargin=10mm]
\itemsep-1mm
\item[(4)] If you have promised to call your parents and you know that someone
  has attached a bomb to your parents' phone that will go off if you call, then
  you must call them.
\end{itemize}

\begin{exercise}
  Give tree proofs with the K-rules to show that $p \to \Ob r$ entails
  $(p \land q) \to \Ob r$, and that $\Ob (p \to r)$ entails
  $\Ob((p \land q) \to r)$.
\end{exercise}
\begin{solution}
  Use \href{https://www.umsu.de/trees/}{umsu.de/trees/}.
\end{solution}

Let's think about what is expressed by statements like (1)--(4). Intuitively,
when we ask what must be done if $p$ is the case, we are limiting our attention
to situations in which $p$ is the case, and consider which of \emph{these}
situations best conform to the relevant norms. It is irrelevant whether $p$ is
in fact the case or whether it ought to be the case. (1) says -- roughly -- that
among worlds where you smoke, the ``best'' worlds are worlds where you smoke
outside. Worlds where you smoke inside are worse than worlds where you smoke
outside. Similarly for (2). A world at which you miss the deadline for tax
returns and pay the fine contains only one violation of the tax rules. Worlds at
which you miss the deadline and don't pay the fine contain two. The ``best''
worlds among those at which you miss the deadline are worlds at which you pay
the fine. Likewise for (3). Among worlds at which you have promised to call your
parents, the ``best'' are worlds at which you keep the promise and call them.

The if-clause in sentences like (1)--(3) therefore seems to \emph{restrict} the
worlds over which the modal operator quantifies. Whereas `ought $q$' alone says
that $q$ is true at the best of the open worlds, `if $p$ then ought $q$' says
that $q$ is true at the best of the open worlds \emph{at which $p$ is true}.

There is no way to express these truth-conditions with the resources of $\L_M$.
But we can introduce a new, binary operator for \textbf{conditional obligation}.
The operator is often written `$\Ob(\cdot/\cdot)$', with a slash separating the
two argument places. Intuitively, $\Ob(B/A)$ means that $B$ ought to be the
case if $A$ is the case.

% The precise semantics of $\Ob(\cdot/\cdot)$ is a matter of debate; I will
% sketch one attractive approach, drawing on ideas from Bengt Hansson and
% Angelika Kratzer.

% Hansson 1981:143 suggests something like Kratzer's account, according to
% Hilpinen 170. In the best worlds among those where -h, we have -t. Express this
% by O(-t/-h). We now assume that for any consistent proposition p there is a
% nonempty set of optimal p-worlds, generalising D.

% Chellas 276 suggests that O(/) can be defined as $A \Rightarrow O(B)$, for a
% Lewis-Stalnaker-selection-type conditional $\Rightarrow$. Others suggest to use
% defeasible conditionals.

% Neighbourhood semantics doesn't help much. In CTD situations, we get conflicting
% obligations. But we want to know more about how the obligations relate.

The formal truth-conditions for $\Ob(B/A)$ are much like those for $\Ob B$, except
that we add the assumption $A$ to the circumstances that are held fixed:

\begin{definition}{Ordering semantics for conditional obligation}{conditionalobligationsemantics}
  If $M$ is a ordering model and $w$ a world in $M$, then\\[1mm]
$M,w \models \Ob (B/A) \text{ iff $M,v \models B$ for all
  $v \in \emph{Min}^{\prec_w}(\{ u: wRu \text{ and } M,u\models A \})$}$.
\end{definition}
%
\noindent%
Here, $\{ u: wRu \text{ and } M,u\models A \}$ is the set of worlds $u$ that are
circumstantially accessible from $w$ and at which $A$ is true.
$\emph{Min}^{\prec_{w}}(\{ u: wRu \text{ and } M,u\models A \})$ is the set that
comprises the best of these worlds. So $\Ob(B/A)$ is true at $w$ iff $B$ is true
at all of the best $A$-worlds that are accessible at $w$.

% Hilpinen: Suppose p entails r, and p is true at some of the r-optimal worlds.
% I.e., some of the best r-worlds are (more strongly) p-worlds. Then the best
% p-worlds are the r-optimal worlds where p is true. I.e., optimality is subject
% to the condition:

% If $[[p]] \subseteq [[r]]$ and $[[p]] \cap Opt(r,w)$ is non-empty, then
% $Opt(p,w) = [[p]] \cap Opt(r,w)$.

% It looks like this is entailed by my ordering semantics.

% Syntactically, this means that $\Ob p$ and $\Ob (q/p)$ entail $\Ob(q)$. I.e., we
% have ``deontic detachment''. But we don't have ``factual detachment'': $p$ and
% $\Ob(q/p)$ does not entail $\Ob q$.
%
\begin{exercise}
  ``Deontic detachment'' is the inference from $\Ob A $ and $\Ob(B/A)$ to
  $\Ob B$. ``Factual detachment'' is the inference from $A$ and $\Ob(B/A)$ to
  $\Ob B$. Which of these are valid on the present semantics?
\end{exercise}
\begin{solution}
  Deontic detachment is valid. Suppose $A$ is true at the best of the
  (circumstantially) accessible worlds, and $B$ is true at the best of the
  accessible worlds at which $A$ is true. Then $B$ is true at the best of the
  accessible worlds.

  Factual detachment is invalid. A counterexample is the ``gentle murder
  puzzle''. Suppose John is determined to kill his grandmother. \emph{If he will
    go ahead and kill her, he ought to do so gently}. Can we conclude that John
  ought to gently kill his grandmother? Arguably not. He shouldn't kill her at
  all! We have $k$ and $\Ob(g/k)$, but not $\Ob(g)$. Formally, $g$ is true at
  the best of the accessible $k$-worlds, but since all the $k$-worlds are quite
  bad, $g$ is not true at the best of the accessible worlds.
\end{solution}

% \begin{exercise}
%   Shelly in standing in front of a burning building. Trapped inside
%   are two small babies. Shelly could enter the building and rescue
%   them, but at the cost of suffering crippling and probably fatal
%   burns. Consider the following three possibilities:
%   \begin{enumerate*}
%   \item[(A)] Shelly stays out and rescues neither baby.
%   \item[(B)] Shelly enters the building and rescues only one of the
%     babies, although he could easily and without any further costs
%     have rescued both.
%   \item[(C)] Shelly enters the building and rescues both the babies.
%   \end{enumerate*}
%   Intuitively, (A) and (C) are permissible but (B) is not.
% \end{exercise}

\begin{exercise}\label{ex:chisholmsparadox}
  In exercise \ref{ex:translate-sdl}, you were asked to translate the following
  statements.
  \begin{exlist}
    \item[(c)] Jones ought to help his neighbours.
    \item[(d)] If Jones is going to help his neighbours, then he ought to tell them
    he's coming.
    \item[(e)] If Jones isn't going to help his neighbours, then he ought to not  tell them he's coming.
  \end{exlist}
  \medskip\noindent%
  Let's add a fourth statement:
  \begin{exlist}
    \item[(f)] Jones is not going to help his neighbours.
  \end{exlist}
  \medskip\noindent%
  Intuitively, none of these four statements is entailed by one of the others.
  Moreover, they don't impose contradictory requirements on Jones: it is easy to
  think of a scenario in which they are all true and Jones is not obligated to
  perform some act and also obligated to not perform the act. This shows that
  your translations in exercise \ref{ex:translate-sdl} were incorrect. Explain.
  (This puzzle is due to Roderick Chisholm.)
\end{exercise}
\begin{solution}
  (c) can obviously be translated as $\Ob p$, (f) as $\neg p$.

  You probably translated (d) as either $p \to \Ob q$ or as $\Ob(p \to q)$.
  $p \to \Ob q$ is entailed by (f). The translation can't be
  right because it is easy to think of a scenario in which (f) is true but (d)
  false. Assume then that (d) is translated as $\Ob(p \to q)$.
  % This is equivalent to $\Ob(q/p)$ in the presence of $\Ob(p)$ or even $\Pe(p)$. We only need the conditional operator to talk about what should be the case under non-ideal conditions.

  The most obvious translations for (e) are $\neg p \to \Ob\neg q$ and
  $\Ob(\neg p \to \neg q)$. The latter is entailed by (c). But it is easy to
  think of a scenario in which (c) is true but (e) false. If (e) is translated
  as $\neg p \to \Ob \neg q$, then (c)--(f) constitute a deontic dilemma: (e)
  and (f) would entail $\Ob \neg t$, but (c) and (d) would entail $\Ob t$.
\end{solution}

% \begin{exercise}
%   What if we analyse (d) and (e) in terms of a necessity modal $\Box$ in
%   addition to the deontic modal $\Ob$? $\Box(p \to Oq)$ where box is some
%   other kind of necessity? Still monotonic.
% \end{exercise}

\begin{exercise}
  The dual of conditional obligation is conditional permission. Spell out
  truth-conditions for $\Pe(B/A)$ that parallel the truth-conditions I have
  given for $\Ob(B/A)$, so that $\Pe(B/A)$ is equivalent to
  $\neg \Ob(\neg B/A)$.
  
  % Explain why `if you have a disability, you can park in front of the
  % entrance' is not adequately translated as either $d \to \Pe(p)$ or
  % $\Pe(d \to p)$.

  % Need better example? Maybe there is none?!

  % Here we don't need to assume SDL to see that the wide-scope transaction is
  % wrong. Intuitively, what's permitted is the conjunction of having a
  % disability and parking at the entrance, not the disjunction of having no
  % disability and parking at the entrance.
\end{exercise}
\begin{solution}
  Simply replace `all' in the semantics for $\Ob(B/A)$ with `some'.
\end{solution}

% An argument against the p->P(q) analysis, from Rescher:
% 1. P(q/p) is equivalent to -O(-q/p).
% 2. p->P(q) entails (p&r)->P(q) [even if -> is strict].
% 3. By the analysis, -O(-q/p) would entail -O(-q/p&r). 
% 4. So O(-q/p&r) would entail O(-q/p), which is absurd.

% Also, strengthening the "antecedent" seems problematic. As Rescher mentions,
% if we allow P(p/q) to entail P(p/q&r), then "before we can assert that an
% action is permitted in some particular circumstance, we must specify this
% circumstance in such a way as to exclude all imaginable countervailing
% conditions". -- The Kratzer account suggests that you're permitted to p if q
% provided that there's one specific way of doing p that's allowed if q.
% Intuitively, many ways are allowed. Tthis is Lewis's "problem about
% permission" aka the paradox of free choice.

% There is also `if ... better'. Dreier discusses a case like BRTD: ``if there
% is war, it's better to disarm''; ``if there is peace, it's better to disarm'';
% but it doesn't follow that it's better to disarm.

% Such cases provide a powerful argument against wide-scoping. How is a
% wide-scope analysis even supposed to go? The only plausible candidate is:
% 'Better(if w than d, if w than -d)'. But that surely gets the TCs wrong.

% Similarly for quasi-cardinal modals like `it would be great that', or `it is
% highly praiseworthy that'.

% TODO: spell out this argument. Can we prove triviality? Can we refute the
% defeasible wide-scopers?

% How should `if ... better' be analysed? Roughly, among the antecedent worlds,
% the A worlds are better than the B worlds. But not every A world is better
% than every B world. We need to take the weighted average.

  
\section{Further challenges}

Many apparent problems for standard deontic logic arise from the dependence of
obligations on circumstances. We can avoid these problems by using deontic
ordering models and formalizing conditional obligation statements with the
binary $\Ob(\cdot/\cdot)$ operator. There are, however, other problems and
``paradoxes'' for which this move doesn't help. I will mention three.

First, we already saw that standard deontic logic does not allow for conflicting
obligations. Suppose you have promised your family to be home for dinner and
your friends to join them at the pub. You are under conflicting \emph{prima
  facie} obligations. It is not clear that one of them overrides the other.
Legal systems can also contain contradictory rules, without any higher-level
rules for how to resolve such contradictions.

We can, of course, drop principle \pr{D}. But even in the minimal logic K,
$\Ob p$ and $\Ob \neg p$ entail $\Ob A$, for any sentence $A$. Intuitively,
however, the fact that you have given incompatible promises does not entail that
you are obligated to, say, kill the Prime Minister.

% Chellas argues that we should at least be able to distinguish
% $\neg (O(A) \land O(\neg A))$ from $\neg \Box \bot$, which are equivalent in
% K. He suggests we should look for a weaker logic in which obligation is still
% closed under consequence (because we have the rule
% $\vdash A \to B \therefore \vdash \Box A \to \Box B$) and there are no
% impossible obligations -- $\neg \Box \bot$ -- but in which we don't have
% $\Box \top$, so that there can be worlds without obligations, and we don't
% have $(\Box A \land \Box B) \to \Box (A \land B)$, giving up a unique standard
% of obligation (the ideal worlds) and allowing for deontic dilemmas. But then a
% weaker semantics must be given.

% Chellas's logic retains $\Box A \to \Box (A \lor B)$. This may seem
% problematic.

Another family of problems arises from the fact that in any logic defined in
terms of Kripke models, $\Ob$ is closed under logical consequence, meaning that
if $\Ob A$ is true and $A$ entails $B$, then $\Ob B$ is true. Since logical
truths are logically entailed by everything, it follows that all logical truths
come out as obligatory. (This is easy to see semantically. A logical truth is
true at all worlds; so it is true at all deontically accessible worlds.) But ought it to be the case that it either rains or doesn't rain?

In response, one might argue that the relevant statements sound wrong not
because they are false, but because their utterance would violate a pragmatic
norm of cooperative communication. A basic norm of pragmatics is that utterances
should make a helpful contribution to the relevant conversation. In a normal
conversational context, it would be pointless to say that something ought (or
ought not) to be the case if it is logically guaranteed to be the case anyway.
An utterance of `it ought to be that $p$' is pragmatically appropriate only if
$p$ could be false. This might explain why it sounds wrong to say that it ought
to either rain or not rain.

Note also that by duality, $\neg\!\Ob(p \lor \neg p)$ entails
$\Pe \!\neg(p \lor \neg p)$. If we deny that it ought to either rain or not rain,
and we accept the duality of obligation and permission, we have to say that it
is permissible that it neither rains nor doesn't rain. That sounds even worse.

The problem of closure under entailment has special bite when obligation
statements are restricted by circumstances. Return to the Samaritan
puzzle. Suppose the victim is bleeding, and Jones ought to stop the blood
flow. It is logically impossible to stop a blood flow if no blood is flowing. In
all the deontic logics we have so far considered, the claim that Jones ought to
stop the victim's blood flow therefore entails that the victim ought to be
bleeding. But wouldn't it be better if the victim weren't bleeding?

Here, too, one might appeal to a pragmatic explanation. When we say that Jones
ought to stop the blood flow, we take for granted that the victim is bleeding.
We are interested in what should be done \emph{given} the state in which Jones
found the victim. Worlds where the victim isn't injured are set aside; they are
not circumstantially accessible. But circumstantial accessibility can shift with
conversational context. The claim that the victim ought to be bleeding is
pointless if we hold fixed the victim's state of injury. So when we evaluate
\emph{this} claim, we naturally assume that the relevant circumstantial
accessibility relation does not hold fixed the injuries. Intuitively, we are no
longer considering what should be done given the state in which Jones found the
victim, but whether that state itself should have obtained. Worlds in which
the state doesn't obtain become circumstantially accessible.

% Similarly: $OK_ap$ entails $Op$ by factivity of K. If Gladys ought
% to know that there's a fire then there ought to be a fire. (Aqvist 1967)

% And similarly for conditional obligations: $\Ob(p/p)$ is valid.

A third family of problems arises from disjunctive statements of permission and
obligation. Consider (1).
\begin{itemize}[leftmargin=10mm]
\itemsep-1mm
\item[(1)] You ought to either mail the letter or burn it.
\end{itemize}
Intuitively, (1) suggests that both mailing the letter and burning it are
permitted. In standard deontic logic, however, $\Ob(A \lor B)$ does not entail
$\Pe A \land \Pe B$. (This puzzle was first noticed by Alf Ross and is known as
``Ross's Paradox''.)

A similar puzzle arises for permissions. (This one is known as the ``Paradox of
Free Choice''.)
\begin{itemize}[leftmargin=10mm]
\itemsep-1mm
\item[(2)] You may have beer or wine.
\end{itemize}
Intuitively, (2) implies that beer and wine are both permitted. But in standard
deontic logic, $\Pe(A \lor B)$ does not entail $\Pe A \land \Pe B$.

We could add the missing principles.
%
\begin{principles}
  \pri{R}{\Ob(A \lor B) \to (\Pe A \land \Pe B)}\\
  \pri{FC}{\Pe(A \lor B) \to (\Pe A \land \Pe B)}
\end{principles}
%
But both of these have unacceptable consequences when added to the minimal modal
logic K. With the help of \pr{R}, we could show that $\Ob A$ entails $\Pe B$:
$\Ob A$ entails $\Ob (A \lor B)$, which by \pr{R} entails $\Pe B$. But clearly
`you ought to mail the letter' does not entail `you may burn the letter'.
Similarly for \pr{FC}. In K, $\Pe A$ entails $\Pe(A \lor B)$; by \pr{FC},
$\Pe(A \lor B)$ entails $\Pe B$. But `you may have beer' does not entail `you
may have wine'.

% \pr{R} is not valid in any class of Kripke frames. But \pr{FC} is valid e.g.
% in frames with only dead ends, where every diamond sentence is false.

% There are also independent reasons to doubt the validity of \pr{R} and
% \pr{FC}. For example, suppose you are \emph{not} permitted to have beer or
% wine. Intuitively, this means that you are not permitted to have beer and you
% are not permitted to have wine. So the following principle also looks
% plausible.
% %
% \principle{FC$'$}{\neg \Pe(A \lor B) \to (\neg \Pe A \land \neg \Pe B)} 
% %
% But \pr{FC} and \pr{FC$'$} can't both be valid. Otherwise we could show that
% if anything is forbidden, then everything whatsoever is forbidden:
% %
% \begin{alignat*}{2}
%   1.\quad& \text{Suppose  $\neg \Pe A$.} &\quad& \\ 
%   2.\quad& \text{Then $\neg(\Pe A \land \Pe B)$.} &\quad& \text{(By propositional logic)}\\
%   3.\quad& \text{Then $\neg\Pe (A \lor B)$.}  &\quad& \text{(By \pr{FC} and modus tollens)}\\
%   4.\quad& \text{Then $\neg\Pe A \land \neg \Pe B$.}  &\quad& \text{(By \pr{FC$'$})}\\
%   5.\quad& \text{So $\neg\Pe B$.}  &\quad& \text{(By propositional logic)}
% \end{alignat*}

\begin{exercise}
  Analogous puzzles to those raised by Ross's Paradox and the Paradox of Free
  Choice arise for epistemic `must' and `might'. Can you give examples?
\end{exercise}
\begin{solution}
  Ross's Paradox: `Alice must be in the office or in the library'
  seems to imply that Alice might be in the office and that she might
  be in the library.

  The Paradox of Free Choice: `Alice might be in the office or in the
  library' seems to imply that Alice might be in the office and that
  she might be in the library.
\end{solution}


\section{Neighbourhood semantics}\label{sec:neighbourhood}

In reaction to apparent problems for standard deontic logic, some have argued
that we should not interpret obligation and permission in terms of
quantification over possible worlds. If we give up this core tenet of Kripke
semantics, we can define ``non-normal'' logics weaker than K. (A \textbf{normal}
modal logic is a modal logic that can be defined in terms of classes of Kripke
frames.)

% In C.I.\ Lewis's 1932 list of modal logics, S4 and S5 were normal, but S1--S3
% were non-normal.

A popular alternative to Kripke semantic is \textbf{neighbourhood semantics},
also known as Scott-Montague semantics, after its inventors Dana Scott and
Richard Montague.

Models in neighbourhood semantics still involve possible worlds. Validity is
still defined as truth at all worlds in all (suitable) models. But the box and
the diamond are no longer interpreted as quantifiers over accessible
worlds. Instead, we simply assume that at every world, some propositions are
``necessary'' and others are not.  $\Box A$ is true at a world if $A$ expresses
one of the necessary propositions at that world.

Formally, the accessibility relation in Kripke models is replaced by a
\textbf{neighbourhood function} $N$ that associates each world in a model with
the propositions that are necessary relative to $w$. Propositions are identified
with sets of possible worlds. Thus $N(w)$ is a set of sets of worlds. Each set
of world in $N(w)$ is necessary at $w$.

% We could equivalently think of N as a relation relating each world to the
% propositions in its neighbourhood.

\begin{definition}{}{neighbourhoodmodel}
  A \textbf{neighbourhood model} consists of
  \vspace{-3mm}
  \begin{itemize*}
  \item a non-empty set $W$,
  \item a function $N$ that assigns to each member of $W$ a set of subsets of
  $W$, and
  \item a function $V$ that assigns to each sentence letter of $\L_M$
    a subset of $W$.
  \end{itemize*}
\end{definition}

The interpretation of non-modal sentences at neighbourhood models works just as
in Kripke semantics (definition \ref{def:kripkesemantics}). To state the
semantics for modal sentences, let $[A]^M$ be the set of worlds in model $M$ at
which $A$ is true. This is our proxy for the proposition expressed by $A$. Then:
%
\begin{align*}
  M,w \models \Box A &\text{ \;iff\; $[A]^M$ is in $N(w)$}.\\
  M,w \models \Diamond A &\text{ \;iff\; $[\neg A]^M$ is not in $N(w)$.}
\end{align*}
%
Intuitively, the clause for the box says that $\Box A$ is true at $w$ iff the
proposition expressed by $A$ is one of those that are necessary at $w$. The
clause for the diamond ensures that the box and the diamond are duals.

In neighbourhood semantics, the modal operators are not closed under logical
consequence. The neighbourhood function $N$ can easily make $p$ necessary at a
world without making $p\lor q$ necessary, even thought $p$ entails $p \lor q$.
If we interpret $\Ob$ and $\Pe$ as the box and the diamond in neighbourhood
semantics, we can therefore say that Jones ought to tend to the victim's
injuries even thought it is not the case that someone ought to be injured.

% \begin{exercise}
%   What formal condition on the neighbourhood function would ensure that $\Box$
%   is closed under logical consequence?
% \end{exercise}
% \begin{solution}
%   Whenever $X \in N(w)$ then all sets that have $X$ as a subset are in
%   $N(w)$.
% \end{solution}

We can also allow for conflicting obligations. If the laws at $w$ require both
$p$ and $\neg p$, we simply have $[p]^M \in N(w)$ and $[\neg p]^M \in N(w)$. It
longer follows that any proposition whatsoever is obligatory.

We may further hope to escape the problems from section \ref{sec:oblig-circ}
that led us to introduce a primitive conditional obligation operator. I argued
that the wide-scope translation $\Ob(A \to B)$ of conditional obligation
sentences is problematic because $\Ob(A \to B)$ is entailed by $\Ob(\neg A)$. In
neighbourhood semantics, this entailment fails. 

Bare neighbourhood semantics determines a very weak logic called \textbf{E}. It
is axiomatized by \pr{Dual}, \pr{CPL}, and a rule (called ``RN'') that allows
inferring $\Box A \leftrightarrow \Box B$ from $A \leftrightarrow B$. We can get
stronger logics, with more validities, by imposing conditions on the
neighbourhood function $N$.

For example, suppose we want to maintain that if something is logically
guaranteed to be true, then it can't be forbidden. Equivalently, any logically
necessary truth should be permitted. By the neighbourhood semantics for $\Pe$,
$A$ is permitted at a world $w$ in a model $M$ iff $[\neg A]^M$ is not in
$N(w)$. If $A$ is a logical truth, then $A$ is true at all worlds; in that case,
$\neg A$ is true at no worlds, and $[\neg A]^M$ is the empty set. If we want
logical truths to be permitted, we therefore have to stipulate that $N(w)$
never contains the empty set.

In Kripke semantics, the assumption that logically necessary truths are
permitted is equivalent to the assumption that (every instance of) the
\pr{D}-schema $\Ob A \to \Pe A$ is valid. Both assumptions correspond to
seriality of the accessibility relation. In neighbourhood semantics, we can
distinguish between the two assumptions. While the permissibility of logical
truths requires that $N(w)$ doesn't contain the empty set, the validity of
$\Ob A \to \Pe A$ requires that $N(w)$ doesn't contains contradictory
propositions $[A]^M$ and $[\neg A]^M$.

If we assume that the neighbourhood function is closed under intersection, in
the sense that whenever two sets $X$ and $Y$ are in $N(w)$ then so is their
intersection $X\cap Y$, then $(\Box A \land \Box B) \to \Box (A \land B)$
becomes valid. If we also require the converse, that whenever $X\cap Y \in N(w)$
then $X \in N(w)$ and $Y\in N(w)$, and in addition that $W \in N(w)$, we get
back the minimal normal logic K.

% Humberstone 160: [](A & B) -> []A requries that whenever X in N(w) and Y is a
% superset of X then Y is in N(w).

\begin{exercise}
  Can you find a condition on the neighbourhood function that renders the
  \pr{T}-schema valid?
\end{exercise}
\begin{solution}
  For every world $w$, every member of $N(w)$ contains $w$.
\end{solution}

For some purposes, even the minimal logic of neighbourhood semantics is too
strong. Return to the intuitive ``Free Choice'' principle from the previous
section:
%
\principle{FC}{\Pe(A \lor B) \to (\Pe A \land \Pe B)}
%
We have seen that this principle is untenable in Kripke semantics. It is
still untenable in neighbourhood semantics.

To see why, note first that whenever two sentences $A$ and $B$ are logically
equivalent, then in neighbourhood semantics $\Pe A$ and $\Pe B$ are also
equivalent. The reason is that the modal operators in neighbourhood semantics
operate on the set of worlds at which the embedded sentence is true. If $A$ and
$B$ are logically equivalent, then in any model $M$, the set $[A]^M$ is the same
set as $[B]^M$, and so $[A]^M$ is in $N(w)$ iff $[B]^M$ is in $N(w)$. Likewise,
$[\neg A]^M$ is in $N(w)$ iff $[\neg B]^M$ is in $N(w)$.
% (That's why the RN rule preserves validity.)

Now any sentence $A$ is logically equivalent to
$(A \land B) \lor (A \land \neg B)$, for any $B$. In the logic E, $\Pe A$
therefore entails $\Pe ((A \land B) \lor (A \land \neg B))$. By \pr{FC},
$\Pe ((A \land B) \lor (A \land \neg B))$ entails $\Pe (A \land B)$. We could
still reason from `you may have a cookie' to `you may have a cookie and burn
down the house'.

\begin{exercise}
  Rational beliefs come in degrees, which are often assumed to satisfy the
  formal rules of probability. Suppose we say that someone believes $A$ iff
  their degree of belief in $A$ is above a certain threshold -- say, 0.9.
  Explain why one can't give a Kripke semantics for this concept of belief.
  (Although one can give a neighbourhood semantics.) \emph{Hint}: One rule of
  probability says that if $p$ and $q$ are independent propositions, then the
  probability of their conjunction $p \land q$ is the product of their
  individual probabilities.
\end{exercise}
\begin{solution}
  In Kripke semantics, $\Box p$ and $\Box q$ together entail $\Box(p \land
  q)$. But if the probability of $p$ is above the threshold and the probability
  of $q$ is above the threshold, it does not follow that the probability of
  $p\land q$ is above the threshold. For example, we could have Pr$(p)=0.95$,
  Pr$(q)=0.94$, and Pr$(p \land q) = 0.95 \times 0.94 = 0.893$.
\end{solution}

% \begin{exercise}
%   Some have argued that the logic of ability is weaker than K, on the grounds
%   that there are cases in which an agent is able to make $p \lor q$ true, but
%   not able to make $p$ true and not able to make $q$ true -- which would provide a counterexample to the K-valid schema $\Diamond(A \lor B) \to (\Diamond A \lor \Diamond B)$. Can you describe a case of this type?
% \end{exercise}
% \begin{solution}
%   A bad dart player may have the ability to hit the dart board but lack the
%   ability to hit the left half of the board and also the ability to hit the
%   right half of the board.  
% \end{solution}



%%% Local Variables: 
%%% mode: latex
%%% TeX-master: "logic2.tex"
%%% End:

\chapter{Temporal Logic}\label{ch:time}

\section{Reasoning about time}\label{sec:time-intro}

It is currently raining in Edinburgh. But it wasn't raining yesterday, and
perhaps it won't rain tomorrow. Let's introduce some operators to formalize
reasoning about the unfolding of events through time.

If we read $r$ as `it is raining', we will use $\tF\! r$ to express that is will be
raining at some point in the future. We will use $\tP\! r$ to express that it has been
raining at some point in the past. In general:
%
\begin{quote}
  $\tF A$ is true at a time $t$ iff $A$ is true at some time after $t$.\\
  $\tP A$ is true at a time $t$ iff $A$ is true at some time before $t$.
\end{quote}

The operators $\tF$ and $\tP$ can be nested. We can use $\tF \tP r$ to express
that at some point it will have rained, $\tP\tF r$ to say that it was once going
to rain, $\tP\tP r$ to say that there was a time before which it rained, and
$\tF\tF r$ to say that there will come a time after which it will rain.

Unlike $\Box$ and $\Diamond$, $\tF$ and $\tP$ are not duals of each other:
$\neg \!\tP\! A$ is not equivalent to $\tF \!\neg A$, and $\neg \!\tF\! A$ is not
equivalent to $\tP \!\neg A$. But it is useful to have duals of $\tF$ and $\tP$.
We therefore introduce two more operators. $\tG$ will be the dual of $\tF$, and
$\tH$ the dual of $\tP$.

Intuitively, $\tG A$ means that $A$ is always \emph{going} to be the case.
(Hence the symbol `G'.) If it is not the case that at some point in the future
it will not rain ($\neg \!\tF\! \neg r$), then it is always going to be the case
that it will rain ($\tG r$). Similarly, $\tH A$ means that $A$ \emph{has} always
been the case. If it is not the case that at some point in the past it was not
raining ($\neg \!\tP\! \neg r$), then it has always been raining ($\tH r$).

We can state the truth-conditions of $\tG A$ and $\tH A$ in parallel to the
above truth-conditions for $\tF A$ and $\tP A$:
\begin{quote}
  $\tG A$ is true at a time $t$ iff $A$ is true at all times after $t$.\\
  $\tH A$ is true at a time $t$ iff $A$ is true at all times before $t$.
\end{quote}

The language of standard propositional logic, extended by the four operators
$\tF,\tP,\tG,\tH$ is known as the \textbf{language of basic temporal logic}. We
will sometimes call it $\L_t$.

% $\L_t$ is only a crude approximation to the means by which we talk about time in
% natural language. In English, for example, there is a past tense -- as in `it
% rained' -- but no future tense; to express that something happens in the future
% we typically use the modal verb `will' applied to an infinitive: `it will rain'.
% In many languages, there is a complex system of tenses and aspects to indicate
% an event's position in time and its status as finished or ongoing. $\L_t$ lacks
% most of these subtleties, but it proves sufficient to formalise many interesting
% hypotheses and arguments about time. We will look at some extensions of the
% language in section \ref{sec:2d}.

\begin{exercise}
  Translate the following sentences into the language of basic temporal logic.
  \begin{exlist}
  \item It has never been warm. %H-w, -Pw
  \item There will be a sea battle. %Fc (note English may not be existential)
  \item There will not have been a sea battle. % F-Pc or -FPc (clarify difference)
  \item At some point, it will be warm or it will have been warm. % F(w v Pw)
  \item If you haven't studied, you won't pass the exam. % -Ps -> -Fp
  \item I was having tea when the door bell rang. % P(t & r)
  \end{exlist}
\end{exercise}
\begin{solution}
  \begin{sollist}
  \item $\tH \neg p$\\
    $p$: It is warm\\[-2mm]
  \item $\tF p$\\
    $p$: There is a sea battle\\[-2mm]
  \item $\neg \tF \tP p$ or, perhaps, $\tF\neg \tP p$\\
    $p$: There is a sea battle\\[-2mm]
  \item  $\tF(p \lor \tP q)$ or $\tF(\tF p \lor \tF\tP q)$\\
    $p$: It is warm\\[-2mm]
  \item $\neg \tP p \to \neg \tF q$ or $\tG(\neg\tP p \to \neg q)$\\
    $p$: You study, $q$: you pass the exam\\[-2mm]
  \item $\tP(p \land q)$\\
    $p$: I am having tea, $q$: the door bell rings
  \end{sollist}
\end{solution}


\section{Temporal models}

A complete scenario for temporal logic needs to tell us what times there are,
how they are ordered, and what is going on at each of them. We can represent
such a scenario, together with an interpretation of $\L_{t}$'s non-logical
vocabulary, by a structure that settles (a) what times there are, (b) which
times come before or after which others, and (c) which sentence letters are true
at which times. This is enough to determine, for every $\L_{t}$-sentence and
every time, whether the sentence is true at that time.

\begin{definition}{Temporal Model}{temporalmodel}
  A \textbf{temporal model} consists of
  \vspace{-3mm}
  \begin{itemize*}
  \item a non-empty set $T$ (of ``times''),
  \item a binary relation $<$ on $T$ (the \textbf{precedence relation}),
  \item a function $V$ that assigns to each sentence letter of $\L_T$
    a subset of $T$.
  \end{itemize*}
\end{definition}

% In a sense, a time is a world. A world is a maximally specific way things
% might be. Now here's a way things might be: it might be Monday. Ignorance of
% time plays an important role in our life, so the worlds of epistemic logic,
% say, must not just specify what is the case from an eternal, God's eye
% perspective. If worlds have a time coordinate, then past times are worlds that
% stand in a certain relationship to our world. They share the same untensed
% truths.

We use `$M, t \models A$' as a short-hand notation to express that sentence $A$
is true at time $t$ in model $M$. The following definition formally specifies
the truth-value of every $\L_T$-sentence at every time in every model.

\begin{definition}{Standard Temporal Semantics}{temporalsemantics}
  If $\Mfr = \t{T,<,V}$ is a temporal model, $t$ is a member of $T$, $P$ is
  any sentence letter, and $A,B$ are any $\L_T$-sentences, then

  \medskip
  \begin{tabular}{lll}
    (a) & $M,t \models P$ &iff $t$ is in $V(P)$.\\
    (b) & $M,t \models \neg A$ &iff $M,t \not\models A$.\\
    (c) & $M,t \models A \land B$ &iff $M,t \models A$ and $M,t \models B$.\\
    (d) & $M,t \models A \lor B$ &iff $M,t \models A$ or $M,t \models B$.\\
    (e) & $M,t \models A \to B$ &iff $M,t \not\models A$ or $M,t \models B$.\\
    (f) & $M,t \models A \leftrightarrow B$ &iff $M,t \models (A\to B)$ and $M,t \models (B\to A)$.\\
    (g) & $M,t \models \tF A$ &iff $M,s \models A$ for some $s\in T$ such that $t<s$.\\
    (h) & $M,t \models \tG A$ &iff $M,s \models A$ for all $s\in T$ such that $t<s$.\\
    (i) & $M,t \models \tP A$ &iff $M,s \models A$ for some $s\in T$ such that $s<t$.\\
    (j) & $M,t \models \tH A$ &iff $M,s \models A$ for all $s\in T$ such that $s<t$.
  \end{tabular}
\end{definition}
Clause (a) says that a sentence letter is true at a time in a model iff the
model's interpretation function specifies that the sentence letter is true at
that time. Clauses (b)--(f) say that the truth-functional connectives have their
normal truth-table meaning at each time. Clauses (g)--(j) formalize the
truth-conditions for temporal sentences from the previous section.

% We'll say that an $\L_T$-sentence is \textbf{valid} iff it is true at every time
% in every (suitable) temporal model; an $\L_T$-sentence $B$ is a \textbf{logical
%   consequence} of sentences $A_1,\ldots,A_n$ iff $B$ is true at every time in
% every (suitable) model at which $A_1,\ldots,A_n$ are all true. I say
% ``suitable'' because we will want to put some constraints on the precedence
% relation $<$. More on that in a moment.

All this should remind you of our Kripke semantics for $\L_M$ in chapter
\ref{ch:accessibility}. In fact, temporal models \emph{are} Kripke models, as
defined on page \pageref{def:kripkemodel}. I have merely relabelled the set
`$W$' as `$T$', and the relation `$R$' as `$<$'. Definition
\ref{def:temporalsemantics} resembles definition \ref{def:kripkesemantics} from
page \pageref{def:kripkesemantics}, except that we have two box-like operators
$\tG$ and $\tH$, and two diamond-like operators $\tF$ and $\tP$. The language of
basic temporal logic is bi-modal, with forward-looking operators ($\tF$ and
$\tG$) and backward-looking operators ($\tP$ and $\tH$). Unlike ordinary models
for multi-modal languages (definition \ref{def:multikripkemodel}), temporal
models have only a single accessibility relation. That's because
the accessibility relation for $\tP$ and $\tH$ is definable from the
accessibility relation for $\tF$ and $\tG$: a time $s$ is earlier than a time
$t$ iff $t$ is later than $s$.

% Exercise: define operators that look back on the epistemic or doxastic
% accessibility relation. ...

Let's look at an example of a temporal model. For the set of times $T$, we use
the set of natural numbers 0,1,2, etc. Let's say that the precedence relation
$<$ holds between $t$ and $s$ iff $t$ is smaller than $s$. So $0<1$ and
$1 < 25$. (We could just as well have stipulated that $<$ holds between $t$ and
$s$ iff $t$ is greater than $s$; we would then have $1<0$ and $25<1$. In
temporal logic, the symbol `$<$' means `earlier than', not `smaller than'.)
Finally, let's say that the interpretation function assigns to $p$ the set of
all even numbers.

Let's call this model $M$. By definition \ref{def:temporalsemantics}, we can
figure out the following facts, among others.
\begin{itemize}[leftmargin=10mm]
\itemsep-1mm
\item $M,0 \models p$ (because 0 is even);
\item $M,0 \models \tF p$ (because there are even numbers greater than 0);
\item $M,0 \models \tG\tF p$ (because for every number there is a greater number that is even);
\item $M,0 \models \neg\tF\tG p$ (because there is no number for which all greater numbers are even).
\end{itemize}

\begin{exercise}
  Now let $M$ be the following model. As before, $T$ is the set of natural
  numbers $\{ 0,1,2,\ldots \}$, and $t < s$ iff $t$ is smaller than $s$. This
  time, $V(p)$ is the set of numbers smaller than 10. Which of the following
  statements are true?
  \begin{exlist}
  %\item $M,0 \models \tF (p \land \neg p)$
  \item $M,0 \models \tF p \land \tF \neg p$ 
  \item $M,0 \models \tG \neg p$
  \item $M,0 \models \tF\tG \neg p$
  \item $M,0 \models \tG\tF p$
  \item $M,0 \models \tG(\tF p \to \tF\tF p)$
  \item $M,0 \models \tF\tH p$
  \item $M,0 \models \neg \tP(p \lor \neg p)$
  \item $M,0 \models \tH p$
  \end{exlist}
\end{exercise}
\begin{solution}
  (a), (c), (f), (g), and (h) are true, (b), (d), and (e) are false. 
\end{solution}

Real times are, of course, not numbers. When I say that `it is raining' is true
now, I don't mean that the sentence is true at a number. It isn't
obvious what kinds of things times are. Fortunately, this doesn't matter for us,
just as the nature of possible worlds doesn't matter for the logic of
possibility and necessity. As long as the formal structure of the times in a
scenario matches the structure of the natural numbers, it does no harm to use
numbers as times in a model of the scenario.

The formal structure of time in a temporal model is captured by the relevant
frame: the pair $\t{T,<\!}$ of the set of times and the precedence relation.
Frames in temporal logic are also called \textbf{flows of time}. Different
applications of temporal logic often come with different assumptions about the
flow of time.

In computer science, for example, the ``times'' $T$ are often understood as
possible states of a computational process; the precedence relation holds
between states $t$ and $s$ iff the computation can lead from $t$ to $s$. If
the computation is indeterministic, so that a given state can have different
successors, the relevant flow of time will involve forks towards the future: we
can have different ``times'' $s$ and $r$ such that $t<s$ and $t<r$ but neither
$s<r$ nor $r<s$. Here the precedence relation cannot be modelled by the
less-than relation on the natural numbers, because the structure of the
less-than relation does not include forks.

In other applications, we may be interested in how the weather changes from day
to day. Here we might identify the relevant times with days and the precedence
relation with the earlier-relation between days -- even though intuitively a day
is not a single time, but an interval comprising many times. For this
application, the natural numbers might have the right formal structure.

For yet other applications, we may want to assume that time is \textbf{dense},
meaning that whenever $t < s$ then there is another point of time lying in
between $t$ and $s$. This assumption is common in physics. The natural numbers,
by contrast, have a \textbf{discrete} structure. There is no natural number in
between 2 and 3. For dense models, we could use real or rational numbers
(fractions) instead of natural numbers.

% Humberstone 199f. distinguishes two notions of discreteness that come apart in models of branching time. The easier one of them corresponds to $(A \land H A) \to FH A$.

If we want to take seriously what physics tells us about time, it is not enough
to assume that time is dense. We also need to reconceptualize the set $T$.
According to the theory of special relativity, whether a point in time is
earlier or later than another is relative to a spatial frame of reference. An
adequate model of relativistic time must therefore include a representation of
space. In these \textbf{spacetime models} (or \emph{Minkowski models}), the set
$T$ consists of spacetime points $\t{x_1,x_2,x_3,t}$ with three spatial and one
temporal coordinate; $(x_1,x_2,x_3,t) < (y_1,y_2,y_3,s)$ holds iff the second
point can be reached from the first without travelling faster than the speed of
light.

\section{Logics of time}

Let's define the minimal temporal logic K$_t$ as the set of $\L_{t}$-sentences
that are true at all times in all temporal models. Since temporal models are
just Kripke models, proof methods for the minimal modal logic K are easily
adapted to K$_{t}$. The main novelty is that the rules for the box and the
diamond can be used twice over, once for the forward-looking operators $\tF$ and
$\tG$, and once for the backward-looking $\tP$ and $\tH$.

In the tree method for K$_{t}$, we have all the K-rules, with $\tG$ as the box
and $\tF$ as the diamond. In addition, we have rules for $\tH$ as the box and
$\tP$ as the diamond with a reversed perspective on the accessibility (or
precedence) relation:

\bigskip\hspace{-3mm}
\begin{minipage}{0.26\textwidth} \centering
  \tree{
    \nnode{12}{}{$\tG A$}{\omega}{}\\
    \dotbelownode{12}{}{$\omega<\nu$}{}{}\\
    \\
    \nnode{12}{}{$A$}{\nu}{}\\
    \Kk[12]{0}{}\\
    \Kk[12]{0}{}
}
\vspace{8mm}


  \tree{
    \dotbelownode{12}{}{$\neg\!\tG A$}{\omega}{}\\
    \\
    \nnode{12}{}{$\omega<\nu$}{}{}\\
    \nnode{12}{}{$\neg A$}{\nu}{}\\
    \Kk[12]{0}{$\uparrow$}\\
    \Kk[12]{0}{\small new}
}
\end{minipage}
\begin{minipage}{0.26\textwidth}\centering
\tree{
    \dotbelownode{12}{}{$\tF A$}{\omega}{}\\
    \\
    \nnode{12}{}{$\omega<\nu$}{}{}\\
    \nnode{12}{}{$A$}{\nu}{}\\
    \Kk[12]{0}{$\uparrow$}\\
    \Kk[12]{0}{\small new}
}

\vspace{8mm}

  \tree{
    \nnode{12}{}{$\neg\!\tF A$}{\omega}{}\\
    \dotbelownode{12}{}{$\omega<\nu$}{}{}\\
    \\
    \nnode{12}{}{$\neg A$}{\nu}{}\\
    \Kk[12]{0}{}\\
    \Kk[12]{0}{}
}
\end{minipage}
\begin{minipage}{0.26\textwidth} \centering
\tree{
    \nnode{12}{}{$\tH A$}{\omega}{}\\
    \dotbelownode{12}{}{\color{red}{$\nu<\omega$}}{}{}\\
    \\
    \nnode{12}{}{$A$}{\nu}{}\\
    \Kk[12]{0}{}\\
    \Kk[12]{0}{}
}

\vspace{8mm}


\tree{
    \dotbelownode{12}{}{$\neg\! \tH A$}{\omega}{}\\
    \\
    \nnode{12}{}{\color{red}{$\nu<\omega$}}{}{}\\
    \nnode{12}{}{$\neg A$}{\nu}{}\\
    \Kk[12]{0}{$\uparrow$}\\
    \Kk[12]{0}{\small new}
  }
  
\end{minipage}
\begin{minipage}{0.26\textwidth}\centering
\tree{
    \dotbelownode{12}{}{$\tP A$}{\omega}{}\\
    \\
    \nnode{12}{}{\color{red}{$\nu<\omega$}}{}{}\\
    \nnode{12}{}{$A$}{\nu}{}\\
    \Kk[12]{0}{$\uparrow$}\\
    \Kk[12]{0}{\small new}
  }
  
\vspace{8mm}

\tree{
    \nnode{12}{}{$\neg\!\tP A$}{\omega}{}\\
    \dotbelownode{12}{}{\color{red}{$\nu<\omega$}}{}{}\\
    \\
    \nnode{12}{}{$\neg A$}{\nu}{}\\
    \Kk[12]{0}{}\\
    \Kk[12]{0}{}
}
\end{minipage}
\bigskip

In the axiomatic approach, we have two versions of the \pr{K} schema, one for
the forward-looking box $\tG$ and one for the backward-looking box $\tH$:%
%
\begin{principles}
   \pri{GK}{\tG(A\to B) \to (\tG A \to \tG B)}\\
   \pri{HK}{\tH(A\to B) \to (\tH A \to \tH B)}
\end{principles}
%
We also have two versions of Necessitation, and two versions of \pr{Dual}:
%
\begin{principles}
  \pri{GDl}{\neg\tF A \leftrightarrow \tG\neg A}\\
  \pri{HDl}{\neg\tP A \leftrightarrow \tH\neg A}\\
  \pri{GNec}{\text{If $A$ occurs in a proof, $\tG A$ may be appended.}}\\
  \pri{HNec}{\text{If $A$ occurs in a proof, $\tH A$ may be appended.}}
\end{principles}
%
In addition, we need two interaction principles, reflecting the fact that the
accessibility relation for $\tF$ and $\tG$ is the inverse of the accessibility
relation for $\tP$ and $\tH$:
%
\begin{principles}
\pri{Con1}{A \to \tG\tP A}\\
\pri{Con2}{A \to \tH\tF A}
\end{principles}

These axioms and rules, added to those of classical propositional logic, define
an axiomatic calculus that is sound and complete for K$_{t}$. (Completeness is
easily proved with the canonical model technique.)

\begin{exercise}
  Show with the help of definition \ref{def:temporalsemantics} that all
  instances of \pr{Con1} and \pr{Con2} are true at all times in all temporal
  models. % delete?
\end{exercise}
\begin{solution}
  (Con1): Suppose some sentence of the form $A \to \tG\tP A$ is false at some time $t$ in some temporal model. By clause (e) of definition \ref{def:temporalsemantics}, this means that $A$ is true at $t$ and $\tG\tP A$ is false at $t$. By clause (h), the latter means that there is a time $s$ with $t<s$ such that $\tP A$ is not true at $s$. By clause (i), it follows that $A$ is not true at $t$. Contradiction.

  The argument for (Con2) is analogous.
\end{solution}

% \begin{exercise}
%   Explain why \pr{Con1} and \pr{Con2} could be equivalently expressed as
%   $\tF\tH A \to A$ and $\tP\tG A \to A$.
% \end{exercise}

\begin{exercise}
  Give K$_{t}$-tree proofs for the following schemas.
  \begin{exlist}
    \item $A \to \tG \tP A$
    \item $A \to \tH \tF A$
    \item $\tF A \to \tH\tF\tF A$ 
    \item $\tP\tG A \to \tP\tF A$ 
    \item $\tH A \leftrightarrow \tH\tF\tH A$ 
  \end{exlist}
\end{exercise}
\begin{solution}
  \begin{sollist}
    
    \item \tree[4]{%
        \nnode{19}{1.}{$\neg(A \to \tG\tP A)$}{t}{(Ass.)} \\
        \nnode{19}{2.}{$A$}{t}{(1)} \\
        \nnode{19}{3.}{$\neg \tG\tP A$}{t}{(1)}  \\
        \nnode{19}{4.}{$t<s$}{}{(3)}  \\
        \nnode{19}{5.}{$\neg \tP A$}{s}{(3)}  \\
        \nnodeclosed{19}{6.}{$\neg A$}{t}{(4,5)}  \\
    }
    \medskip

    \item \tree[4]{%
        \nnode{19}{1.}{$\neg(A \to \tH\tF A)$}{t}{(Ass.)} \\
        \nnode{19}{2.}{$A$}{t}{(1)} \\
        \nnode{19}{3.}{$\neg \tH\tF A$}{t}{(1)}  \\
        \nnode{19}{4.}{$s<t$}{}{(3)}  \\
        \nnode{19}{5.}{$\neg \tF A$}{s}{(3)}  \\
        \nnodeclosed{19}{6.}{$\neg A$}{t}{(4,5)}  \\
   }
    \medskip

   \item \tree[4]{%
        \nnode{24}{1.}{$\neg(\tF A \to \tH\tF\tF A)$}{t}{(Ass.)} \\
        \nnode{24}{2.}{$\tF A$}{t}{(1)} \\
        \nnode{24}{3.}{$\neg \tH\tF\tF A$}{t}{(1)}  \\
        \nnode{24}{4.}{$s<t$}{}{(3)}  \\
        \nnode{24}{5.}{$\neg \tF\tF A$}{s}{(3)}  \\
        \nnodeclosed{24}{6.}{$\neg \tF A$}{t}{(4,5)}  \\
      }
    \medskip

    \item \tree[4]{%
        \nnode{22}{1.}{$\neg(\tP\tG A \to \tP\tF A)$}{t}{(Ass.)} \\
        \nnode{22}{2.}{$\tP\tG A$}{t}{(1)} \\
        \nnode{22}{3.}{$\neg \tP\tF A$}{t}{(1)}  \\
        \nnode{22}{4.}{$s<t$}{}{(2)}  \\
        \nnode{22}{5.}{$\tG A$}{s}{(2)}  \\
        \nnode{22}{6.}{$A$}{t}{(4,5)}  \\
        \nnode{22}{7.}{$\neg \tF A$}{s}{(3,4)}  \\
        \nnodeclosed{22}{8.}{$\neg A$}{t}{(4,7)}  \\
    }
    \medskip
      
   \item \tree[4]{%
        & \bnode{24}{1.}{$\neg(\tH A \leftrightarrow \tH\tF\tH A)$}{t}{(Ass.)} & \\
        &&\\
        \nnode{15}{2.}{$\tH A$}{t}{(1)} && \nnode{15}{4.}{$\neg \tH A$}{t}{(1)} \\
        \nnode{15}{3.}{$\neg\tH\tF\tH A$}{t}{(1)} && \nnode{15}{5.}{$\tH\tF\tH A$}{t}{(1)} \\
        \nnode{15}{6.}{$s<t$}{}{(3)} && \nnode{15}{9.}{$s<t$}{}{(4)} \\
        \nnode{15}{7.}{$\neg\tF\tH A$}{s}{(3)} && \nnode{15}{15.}{$\neg A$}{s}{(4)} \\
        \nnodeclosed{15}{8.}{$\neg\tH A$}{t}{(6,7)} && \nnode{15}{11.}{$\tF\tH A$}{s}{(5,9)} \\
        && \nnode{15}{12.}{$s<r$}{}{(11)} \\
        && \nnode{15}{13.}{$\tH A$}{r}{(11)} \\
        && \nnodeclosed{15}{14.}{$A$}{s}{(12,13)} \\
   }
   
  \end{sollist}
  
\end{solution}

For most applications, K$_{t}$ is too weak. We will want to impose further
restrictions on the relevant temporal models. For example, definition
\ref{def:temporalmodel} allows for cases in which $t < s$ and $s < r$ without
$t < r$. But if a time $t$ is earlier than $s$, and $s$ is earlier than $r$,
then surely $t$ must be earlier than $r$. For almost every application of
temporal logic, we assume that the precedence relation is transitive. This
corresponds to the \pr{4}-schema for $\tG$. It also corresponds to the
\pr{4}-schema for $\tH$.
\begin{principles}
  \pri{4G}{\tG A \to \tG\tG A}\\
  \pri{4H}{\tH A \to \tH\tH A}
\end{principles}

\begin{exercise}
  Explain why, if a relation $<$ is transitive, then so is its converse. The
  converse $>$ of $<$ is the relation that holds between $x$ and $y$ iff $y<x$.
\end{exercise}
\begin{solution}
  Suppose < is transitive, and $x > y$ and $y > z$. Equivalently, $y < x$ and
  $z < y$.By transitivity of <, we have $z < x$. So $x > z$.
\end{solution}

Another plausible condition is that no time is earlier than itself. Formally,
$<$ should be \emph{irreflexive}, so that no element of $T$ is $<$-related to
itself. We know that reflexivity corresponds to the \pr{T}-schema, whose
(forward-looking) temporal analogue would be $\tG A \to A$. What corresponds to
irreflexivity? The following observation reveals the answer: nothing.

\begin{observation}{noirrefl}
  A sentence is valid in the class of irreflexive frames iff it is valid in the
  class of all frames.
\end{observation}
%
\begin{proof}
  \emph{Proof sketch:} The right-to-left direction is obvious. The left-to-right direction is implied by the answer to exercise \ref{ex:acyclical}.
  % In that exercise we showed that every sentence that is satisfiable in a
  % cyclical model is satisfiable in an acyclical model. Suppose X is true at
  % all worlds in all irreflexive models, but that it is not true at all worlds
  % in all reflexive models. So ~X is true at some world in some reflexive
  % model, but not at any world in any irreflexive model. This contradicts the
  % exercise (because every acyclical model is irreflexive).
  But we can give a more direct argument.

  Suppose that some sentence $A$ is not valid in the class of all frames. We
  show that $A$ is not valid in the class of irreflexive frames. That $A$ is not
  valid in the class of all frames means that there is some world $w$ in some
  model $M = \t{W,R,V}$ at which $A$ is false. We will show that there is some
  world in some irreflexive model at which $A$ is false.

  To this end, we will construct an irreflexive model $M^i = \t{W',R',V'}$ from
  $M$ in which the same sentences are true at $w$ as in $M$. Since $A$ is true
  at $w$ in $M$, it follows that $A$ is true at $w$ in $M^i$.

  Initially, $M^i$ has the same worlds, the same accessibility relation, and the
  same interpretation function as $M$. Now for any world $w$ in $M$ that can see
  itself, we add a new world $w'$ to $M^i$ so that
  % 
  \begin{itemize}[leftmargin=10mm]
    \itemsep-1mm
    \item $w'$ verifies the same sentence letters as $w$: if $w \in V(P)$ then $w' \in V(P)$;
    \item $w'$ can see the same worlds as $w$: whenever $wR'v$ then $w'R'v$; and
    \item $w'$ can be seen from the same worlds as $w$: whenever $vR'w$ then
          $vR'w'$.
  \end{itemize}
  % Since $w$ can see itself, this means that $wR'w'$ and $w'R'w$.
  Finally, we make $w$ inaccessible from itself in $M^i$. A simple proof
  by induction on complexity shows that if a sentence is true at a
  world $w$ in $M$ then it is also true at $w$ in $M^i$.
  \qed

  % Can we adjust this proof for an arbitrary class of frames? I.e., $A$ is
  % valid in the class of irreflexive C-frames iff it is valid in all
  % C-frames. No: If $A$ is false at some world in some C-model $M$, we can
  % construct an irreflexive model $M^i = \t{W',R',V'}$ from $M$ in which the
  % same sentences are true at $w$ as in $M$.  But we also need to show that
  % $M^i$ still satisfies the condition C. That won't be the case if C is, for
  % example, the condition that there is exactly one world, or that now world
  % can see more than one world.
\end{proof}

% Observation \ref{obs:noirrefl} tells us that there is no modal principle that is
% valid in all and only the irreflexive frames. So the logic of irreflexive frames
% is the same as logic of all frames. The proof carries over to many
% other classes of frames. For example, the logic of irreflexive and transitive
% frames is the same as the logic of transitive frames (namely, K4).

% We can easily add a tree rule for irreflexivity: simply allow any branch to be
% closed that contains $\omega < \omega$. But while that rule may help to find
% irreflexive countermodels, it won't allow us to prove anything we couldn't
% prove without the rule.

% From Humberstone 84: Why is the class of universal frames not modally
% definable? Take any two universal frames. Their disjoint union is not
% universal. Similar arguments show that the class of finite frames is not
% definable, nor is the class of frames in which each world is accessible from
% some world. We could here mention that we've implicitly seen something like
% this in ch.3, where we saw that a sentence is valid in universal frames iff it
% is valid in equivalence frames. It follows that there's no sentence that
% corresponds to universal frames.

Given transitivity, irreflexivity is closely related to asymmetry. Recall from
the previous chapter that $<$ is asymmetric if whenever $t<s$ then not $s<t$.
There is no modal schema that corresponds to asymmetry.

\begin{exercise}\label{ex:partialorder}
  Show that a transitive relation is irreflexive iff it is asymmetric.
\end{exercise}
\begin{solution}
  Suppose $R$ is transitive. If there are points $x$ and $y$ for which $xRy$ and
  $yRx$ then $xRx$ by transitivity. So if $R$ isn't asymmetric then it isn't
  irreflexive. If $R$ isn't irreflexive then there is a point $x$ with $xRx$.
  This violates asymmetry, because asymmetry demands that if $xRx$ then not
  $xRx$.
\end{solution}

\begin{exercise}
  A popular idea in many cultures is that time is circular. Does this
  cast doubt on asymmetry? What about irreflexivity? 
\end{exercise}
\begin{solution}
  If time is transitive and circular, then it is neither asymmetric nor
  irreflexive.
\end{solution}

\begin{wrapfigure}{r}{3cm}
  \quad
  \begin{tikzpicture}[modal, world/.append style={minimum size=0.5cm}]
    \node[world] (w1) [label=above:{$t$}] {};
    \node[world] (w2) [label=above:{$s$}, above right=5mm and 15mm of w1] {};
    \node[world] (w3) [label=below:{$r$}, below right=5mm and 15mm of w1] {};
    \draw[->] (w1) -- (w2);
    \draw[->] (w1) -- (w3);
  \end{tikzpicture}
\end{wrapfigure}
In the previous chapter, I mentioned that transitive and irreflexive relations
are called (strict) partial orders. The name reflects the fact that such orders
need not order everything. In a model of branching time, for example, we can
have $t<s$ and $t<r$ but neither $s<r$ nor $r<s$; in that case, $r$ and $s$ are
not ordered by the precedence relation.

We can rule out such cases by imposing the requirement of
\textbf{connectedness}, also known as \emph{completeness} or \emph{totality}.
This demands that for any points $t$ and $s$, either $t < s$ or $t=s$ or
$s < t$.  An irreflexive, transitive, and connected relation is called a
\textbf{(strict) linear order} (or a \emph{strict total order}).

For some applications, we may want linearity in only one direction. Many
philosophers have been attracted to a branching-future conception of time,
where a point in time may have more than one future, but only one past. In
such models, we would only require \textbf{left-linearity}: that \emph{if
  $s < t$ and $r < t$}, then either $s < r$ or $s=r$ or $r < s$.

An axiom schema corresponding to left-linearity is \pr{LL}:
%
\principle{LL}{\tF\tP A \to (\tF A \lor A \lor \tP A)}
%
Right-linearity -- the assumption that if $t < s$ and $t < r$, then either
$s < r$ or $s=r$ or $r < s$ -- corresponds to \pr{RL}:
%
\principle{RL}{\tP\tF A \to (\tP A \lor A \lor \tF A)}
%
% These are from Mueller, p.333. HC (143) use
% $\tG((A \land \tG A) \to B) \lor \tG((B \land \tG B) \to A)$, which looks like
% a redundant disjunction; but apparently this really is the axiom Lem_0.
%
The conjunction of \pr{LL} and \pr{RL} is valid on a frame iff the frame's
precedence relation does not branch in either direction. This is not quite the
same as connectedness, because it allows for frames with parallel time lines.
There is no schema that corresponds to connectedness.
% Gore mentions this. It seems obvious.

% Humberstone 77: K4.3 is determined by the class of transitive, irreflexive,
% and connected frames. .3 is []([]A & A) -> B) v []([]B->A). Humberstone 195
% says this disallows branching.

The tree rules for left-linearity and right-linearity directly reflect the
definition of the two properties.
% These rules are from Priest. Girle's rules are wrong.
\bigskip
\begin{center}

  \begin{minipage}[t]{0.4\textwidth} \centering
    Left-Linearity
    
    \tree[2]{
      & \barenode{$\nu < \omega$} & \\
      & \dotbelowbaretribnode{$\upsilon < \omega$} & \\
      && \\
      && \\
      \barenode{$\nu < \upsilon$} & \barenode{$\nu = \upsilon$} & \barenode{$\upsilon < \nu $}
    }
  \end{minipage}
  \begin{minipage}[t]{0.4\textwidth} \centering
    Right-Linearity

    \tree[2]{
      & \barenode{$\omega < \nu$} & \\
      & \dotbelowbaretribnode{$\omega < \upsilon$} & \\
      && \\
      && \\
      \barenode{$\nu < \upsilon$} & \barenode{$\nu = \upsilon$} & \barenode{$\upsilon < \nu $}
    }
  \end{minipage}
\end{center}
\bigskip%
These rules create \emph{three} branches. They also create ``identity nodes'' of
the form $\nu = \upsilon$, stating that two world/time labels refer to the same
thing. (This must be taken into account when we read off a countermodel from an
open branch.) We need two further rules to deal with identity nodes. Both of
these rules are called `Identity'.%
\bigskip
\begin{center}
  \begin{minipage}[t]{0.3\textwidth} \centering
    
    \tree[2]{
      \nnode{10}{}{$A$}{\omega}{} \\
      \dotbelowbarenode{$\omega = \nu$} \\
      && \\
      \nnode{10}{}{$A$}{\nu}{}
    }
  \end{minipage}
  \begin{minipage}[t]{0.3\textwidth} \centering

    \tree[2]{
      \nnode{10}{}{$A$}{\omega}{} \\
      \dotbelowbarenode{$\nu = \omega$} \\
      && \\
      \nnode{10}{}{$A$}{\nu}{}
    }
  \end{minipage}
\end{center}

\begin{exercise}
  Use the tree method to check which of the following sentences are valid,
  assuming time is linear (i.e., using the Transitivity, Left-Linearity,
  Right-Linearity, and Identity rules).
  \begin{exlist}
  \item $(\tF p \land \tF q) \to \tF(p \land q)$
  \item $\tP\tG\tG p \to \tG\tG p$
  \item $\tP\tF p \to (\tP p \lor (p \lor \tF p))$
  \item $\tP\tH p \to \tH p$
  \item $\tF\tG p \to \tG\tF p$
  \item $\tF (\tG q \land \neg p) \to \tG(p \to (\tG p \to q))$ 
  % \item $(\tF p \land \tF q) \to (\tF(p \land q) \lor (\tF (p \land \tF q) \lor \tF(\tF p \land q)))$ % add
  \end{exlist}
\end{exercise}
\begin{solution}
  (a), (d), and (e) are invalid. Here are trees for (b), (c), and (f):

  \bigskip
  \begin{sollist}

    \item[(b)] \tree[4]{%
        \nnode{25}{1.}{$\neg(\tP\tG\tG p \to \tG\tG p)$}{t}{(Ass.)} \\
        \nnode{25}{2.}{$\tP\tG\tG p$}{t}{(1)} \\
        \nnode{25}{3.}{$\neg \tG\tG p$}{t}{(1)}  \\
        \nnode{25}{4.}{$s<t$}{}{(2)}  \\
        \nnode{25}{5.}{$\tG\tG p$}{s}{(2)}  \\
        \nnode{25}{6.}{$t<r$}{}{(3)}  \\
        \nnode{25}{7.}{$\neg \tG p$}{r}{(3)}  \\
        \nnode{25}{8.}{$s<r$}{}{(3,6)}  \\
        \nnodeclosed{25}{9.}{$\tG p$}{r}{(5,8)}  \\
    }
    \medskip

    \item[(c)] \tree[8]{%
        & \nnode{35}{1.}{$\neg(\tP\tF p \to (\tP p \lor (p \lor \tF p)))$}{t}{(Ass.)} & \\
        & \nnode{35}{2.}{$\tP\tF p$}{t}{(1)} & \\
        & \nnode{35}{3.}{$\neg(\tP p \lor (p \lor \tF p))$}{t}{(1)} & \\
        & \nnode{35}{4.}{$\neg \tP p$}{t}{(3)} & \\
        & \nnode{35}{5.}{$\neg (p \lor \tF p)$}{t}{(3)} & \\
        & \nnode{35}{6.}{$\neg p$}{t}{(5)} & \\
        & \nnode{35}{7.}{$\neg \tF p$}{t}{(5)} & \\
        & \nnode{35}{8.}{$s < t$}{}{(2)} & \\
        & \nnode{35}{9.}{$\tF p$}{s}{(2)} & \\
        & \nnode{35}{10.}{$s < r$}{}{(9)} & \\
        & \tribnode{35}{11.}{$p$}{r}{(9)} & \\
        && \\
        \nnode{10}{12.}{$t<r$}{}{} &    \nnode{10}{13.}{$t=r$}{}{} & \nnode{10}{14.}{$r<t$}{}{} \\
        \nnodeclosed{10}{15.}{$\neg p$}{r}{(7,12)} &  \nnodeclosed{10}{16.}{$\neg p$}{r}{(6,13)} & \nnodeclosed{10}{17.}{$\neg p$}{r}{(4,16)}  \\
    }
    \medskip

    \item[(f)] \tree[8]{%
      & \nnode{38}{1.}{$\neg(\tF (\tG q \land \neg p) \to \tG(p \to (\tG p \to q)))$}{t}{(Ass.)} & \\
      & \nnode{38}{2.}{$\tF (\tG q \land \neg p)$}{t}{(1)} & \\
      & \nnode{38}{3.}{$\neg\tG(p \to (\tG p \to q))$}{t}{(1)} & \\
      & \nnode{38}{4.}{$t < s$}{}{(2)} & \\
      & \nnode{38}{5.}{$\tG q \land \neg p$}{s}{(2)} & \\
      & \nnode{38}{6.}{$\tG q$}{s}{(5)} & \\
      & \nnode{38}{7.}{$\neg p$}{s}{(5)} & \\
      & \nnode{38}{8.}{$t < r$}{}{(3)} & \\
      & \nnode{38}{9.}{$\neg(p \to (\tG p \to q))$}{r}{(3)} & \\
      & \nnode{38}{10.}{$p$}{r}{(9)} & \\
      & \nnode{38}{11.}{$\neg(\tG p \to q)$}{r}{(9)} & \\
      & \nnode{38}{12.}{$\tG p$}{r}{(11)} & \\
      & \tribnode{38}{13.}{$\neg q$}{r}{(11)} & \\
      && \\
      \nnode{10}{14.}{$s<r$}{}{} &    \nnode{10}{15.}{$s=r$}{}{} & \nnode{10}{16.}{$r<s$}{}{} \\
      \nnodeclosed{10}{17.}{$q$}{r}{(6,14)} &  \nnodeclosed{10}{18.}{$p$}{s}{(10,15)} & \nnodeclosed{10}{19.}{$p$}{s}{(12,16)}  \\
    }
    \medskip
  \end{sollist}
\end{solution}

% While we can define non-branchingness, we can't define branchingness.

The precedence relation in relativistic spacetime is neither left-linear nor
right-linear. But it has a weaker property: convergence.
%
% Haven't really talked about convergence so far. Draw picture like in HC 134?
%
A spacetime point $p_1$ can precede two points $p_2$ and $p_3$ neither of which
precedes the other, but these two points will always precede a common later
point $p_4$. Convergence corresponds to the \pr{G}-schema. In temporal
logic, we have one \pr{G}-schema for future convergence and one for past
convergence:
%
\begin{principles}
\pri{FG}{\tF\tG A \to \tG\tF A}\\
\pri{PG}{\tP\tH A \to \tH\tP A}
\end{principles}

% Goldblatt proved that S4.2 is the temporal logic of diodorean necessity in
% Minkowski spacetime. Diodorean has F mean 'true from now on'. Is 0.2 implied by
% no branching?

% The logic of Minkowski spacetime with an irreflexive relation is D4.2+(<>p \land
% <>q \to <>(<>p \land <>q)), see \cite[955]{kracht07logically}.

\begin{exercise}
  Can you find schemas that correspond to the following frame properties?
  \begin{exlist}
  \item There is no last time. (That is, every time precedes some time.)
  \item There is no first time. 
  \item There is a last time.
  \item There is a first time.
  \end{exlist}
% HC 131 say that []<>p & []<>q -> <>(p v q) characterises the idea that time
% has an end, or rather: that every point can see a final point. In the presence
% of S4, they say this simplifies to the McKinsey axiom []<>p -> <>[]p.
\end{exercise}
\begin{solution}
  \begin{sollist}
    \item For example, $\tG A \to \tF A$.
    % Suppose there's a last time $t$. Let $p$ be true there. Then $\tG p \to \tF p$ is false at $g$. Conversely, suppose there's no last time and $\tG A$ is true at some time. Then $\tF A$ is true there as well.
  \item For example, $\tH A \to \tP A$.
  \item No schema corresponds to the class of frames with a last time. If we
  also assume linearity,  $\tG(A \land \neg A) \lor \tF\tG(A \land \neg A)$ works.
  % Mueller and Garanko and Humberstone say this. (But why does Garanko say
  % "assuming irreflexivity"?) Girle says FA -> F-FA = GA v FGA. But that
  % doesn't seem right. Without transitivity, there can be a last time, but
  % $G\bot v FG\bot$ is false at any world that's at least two steps away from a
  % last time. Transitivity gets around this. But it's not enough: consider a
  % frame with one eternal time and a disconnected lonely further time t. Here
  % we have a last time, but $G\bot v FG\bot$ is false at all times on the line.
  \item No schema corresponds to the class of frames with a first time. If we
    assume linearity, then
    $\tH(A \land \neg A) \lor \tP\tH(A \land \neg A)$ works.
  \end{sollist}
\end{solution}
\vspace{-2mm}

\begin{exercise}
  Show that the schema $\tF A \to \tF\tF A$ corresponds to density. (You
  have to show that (a) whenever a frame is dense then $\tF A \to \tF\tF A$ is
  valid on the frame, and (b) whenever $\tF A \to \tF\tF A$ is valid on a frame
  then the frame is dense.)
\end{exercise}
\begin{solution}
  Assume a frame is dense. Suppose for reductio that some instance of
  $\tF A \to \tF\tF A$ is false at some point $t$ in some model $M$ based on
  that frame. Then $\tF A$ is true at $t$ and $\tF\tF A$ is false. Since $\tF A$
  is true at $t$, it follows by definition \ref{def:temporalsemantics} that $A$
  is true at some point $s$ such that $t<s$. By density, there is a point $r$
  such that $t<r<s$. But since $A$ is true at $s$, $\tF A$ is true at $r$, and
  so $\tF\tF A$ is true at $t$; contradiction.

  In the other direction, we have to show that if a frame isn't dense then some
  instance of $\tF A \to \tF\tF A$ is false at some point $t$ in some model $M$
  based on that frame. We take the simplest instance $\tF p \to \tF\tF p$. If a
  frame isn't dense then there are points $t,s$ such that $t<s$ and no point
  lies in between $t$ and $s$. Let $V$ be an interpretation function that makes
  $p$ true at $s$ and false everywhere else. Then $\tF p$ is true at $t$ but
  $\tF\tF p$ is false. So $\tF p \to \tF\tF p$ is false at $t$.
\end{solution}
\vspace{-2mm}

\begin{exercise}
  Can you find an $\L_T$-expression stating that $p$ is true at all times? Can
  you do so if you make assumptions about the precedence relation?
\end{exercise}
\begin{solution}
  Without assumptions about the flow of time there is no way to express in
  $\L_T$ that $p$ is true at all times (or at some time). In linear flows,
  $p \land \tH p \land \tG p$ does the job. 
  % Does $GA\land HGA$ work with negatively transitive time (no parallel lines)?
\end{solution}

% Adding \pr{LL} and \pr{RL} yields a sound and complete logic of linear time.
% The completeness proof is non-trivial because the canonical model is not
% linear. We need techniques such as bulldozing. See Venema p.10f.

% \begin{exercise}
%   Can you find a frame condition that corresponds to the \pr{GL}-schema
%   $\Box(\Box A \to A) \to \Box A$? (Hints: (a) In Kripke semantics, \pr{GL}
%   entails \pr{4}; (b) The dual of \pr{GL} is
%   $\Diamond A \to \Diamond(A \land \neg\Diamond A$.)
% \end{exercise}
% \begin{solution}
%   Informally, $\Diamond A \to \Diamond(A \land \neg\Diamond A$ says that if it
%   will ever be the case that $A$ then it will at some time be the case that $A$
%   for the last time. So time will come to an end.
% \end{solution}



\section{Branching time}\label{sec:branching}

In section \ref{sec:systems} we looked at the idea that the future is ``open''
while the past is ``settled'', insofar as we can still influence (say)
whether we will exercise tomorrow, but not whether we have exercised
yesterday. Some have argued that this calls for a non-linear model of time, with
multiple branches into the future. On one branch, we would exercise tomorrow, on
another we would not.

This line of thought appears to conflate temporal and modal considerations.
The precedence relation in models of time is normally understood as a purely
temporal relation -- as the earlier-later relation. The fact that we can bring
about a world in which we exercise tomorrow and a world in which we don't
exercise does not entail that both kinds of tomorrow take place here in the
actual world.

If we want to make explicit the connections between settledness and time, it is
better to use a multi-modal language with circumstantial operators for
settledness and openness in addition to the purely temporal operators
$\tF, \tG, \tP, \tH$. We could then say things like $\tP p \to \Box \tP p$ to
formalize the claim that if $p$ has happened then it is settled that $p$ has
happened.

% The accessibility relation for the box would have to hold fixed the past, so
% that a world $v$ is accessible from a world $w$ only if the past of $v$
% coincides with the past of $w$.

\begin{exercise}
  Suppose we endorse all instances of the schema (S1) $\tP A \to \Box \tP A$.
  Suppose we also endorse all instances of (S2)
  $\neg \!\tP A \to \Box \neg \!\tP A$, on the grounds that if something has
  failed to happen then there is nothing we can do that would make it have
  happened. Let's further assume that the present time is not the first, and
  that the box is closed under logical consequence, meaning that if
  $\Box A_{1},\ldots,\Box A_{n}$ are true at a time, and $B$ is entailed by
  $A_{1},\ldots,A_{n}$, then $\Box B$ is true (at the time) as well. Show that
  we can then derive the fatalist conclusion that anything that never actually
  happen is settled to never happen: all instances of
  $(\neg \tP A \land \neg A \land \neg \tF A) \to \Box \neg \tF A$ are true.
  (Hint: use instances of (S1) and (S2) in which $A$ is a statement about the
  future.)
\end{exercise}
\begin{solution}
  Suppose $\neg \tP A \land \neg A \land \neg \tF A$ is true at the present time
  $t$. Then $\neg \tP\tF A$ is true (at $t$). By (S2), we can infer
  $\Box \neg \tP\tF A$. But $\neg \tP\tF A$ $K_{t}$-entails
  $\neg(\tF A \land \tP(A \lor \neg A))$. Since the box is closed under logical
  consequence, this means that $\Box \neg(\tF A \land \tP(A \lor \neg A))$ is
  true at $t$. Since $t$ is not the first time, $\tP(A \lor \neg A)$ is true at
  $t$, and so $\Box \tP(A \lor \neg A)$ is true at $t$ as well, by (S1). 
  $\neg(\tF A \land \tP(A \lor \neg A))$ and $\tP(A \lor \neg A)$ together
  entail $\neg \tF A$. Since the box is closed under logical consequence,
  it follows that $\Box \tF A$ is true at $t$.
\end{solution}

% Exercise: multi-modal time travel -- can we change the past? Can you kill your
% Grandfather? We'll see that in the worlds where you do, the person you kill is
% not your grandfather. Calling the person your grandfather is smuggling in facts
% about the future.

There are nonetheless good reasons to consider branching models of time. I already
mentioned that such models are widely used in computer science, where
the ``times'' represent states of a computational process and the precedence
relation has a semi-modal interpretation, holding between two states iff the
first can lead to the second.
% (The relevant logics of branching time are called ``computational tree
% logics'').
I also mentioned that the precedence relation in relativistic spacetime allows
for branching, although diverging spacetime branches ultimately reconverge. A
more classical form of branching (without reconvergence) has been argued to
follow from a certain interpretation of quantum physics. On this interpretation,
what are normally understood to be chance events are really branching events in
which all possible outcomes actually take place.

Another way to motivate a branching conception of time arises from a
metaphysical view called \emph{presentism}. According to presentism, only the
present is real; all truths that seem to concern other times are reducible to
more fundamental truths about the present. If, for example, it is true that
there was a sea battle yesterday, then according to presentism this must
ultimately be explained by what is true \emph{now}; there must be facts about
the present state of the world that entail (and explain) yesterday's sea battle.
Different forms of presentism disagree over what the relevant facts about the
present might be. On one view, they are particular facts about the distribution
of physical particles and fields etc.\ together with the general laws of nature.
If the laws of nature are deterministic, then the complete truth about the
present distribution of particles and fields etc.\ together with the laws fixes
all truths about the past and about the future. But suppose the laws are
indeterministic towards the future: they merely settle that if the present
physical state of the world is so-and-so, then the future is \emph{either like
  this or like that}. In that case, the presentist will regard both of these
futures as equally actual.

Let's assume, then, that we want to reason about branching time. This is less
straightforward than it might at first appear.

% The models we are interested in are not right-linear. I will, however, assume
% that they satisfy the following weaker property of
% \textbf{quasi-connectedness}:\label{quasiconnected}
% \[
%   \text{if $t < s$ then for any $r$, either $t < r$ or $r<s$}.
% \]
% % I've made up that label.
% Quasi-connectedness is more often called \emph{negative transitivity}, because
% it is equivalent to the assumption that if $t \not< s$ and $s\not<r$ then
% $t\not<r$. It slightly simplifies our models, for example by ruling out entirely
% disconnected parallel time lines.

Two pieces of terminology will be useful. First, let's define a \textbf{history}
in a model $\t{T,<,V}$ as a maximal linearly ordered subset of $T$. That is, a
history is a collection of times $H$ such that
\begin{itemize}[leftmargin=10mm]
  \itemsep0mm
\item[(i)] for all $t$ and $s$ in $H$, either $t<s$ or $t=s$ or $s<t$, and
\item[(ii)] no further member of $T$ could be added to $H$ without
  making (i) false.
\end{itemize}
%
\noindent
\begin{wrapfigure}{r}{4cm}
  \vspace{-8mm}
  \quad
  \begin{tikzpicture}[modal, world/.append style={minimum size=0.5cm}]
    \node[world] (w1) [label=above:{$t_1$}] {};
    \node[world] (w2) [label=above:{$t_2$}, right=10mm of w1] {};
    \node[world] (w3) [label=above:{$t_3$}, above right=5mm and 10mm of w2] {};
    \node[world] (w4) [label=below:{$t_4$}, below right=5mm and 10mm of w2] {};
    \draw[->] (w1) -- (w2);
    \draw[->] (w2) -- (w3);
    \draw[->] (w2) -- (w4);
  \end{tikzpicture}
  \vspace{-10mm}
\end{wrapfigure}

\noindent
The model (or rather, frame) depicted on the right contains two
histories: $\{ t_1, t_2, t_3 \}$ and $\{ t_1, t_2, t_4 \}$.

For the second piece of terminology, let $t$ be any time in any model. Any
maximal linearly ordered set of times \emph{later than $t$} will be called a
\textbf{future of $t$}. In the model on the right, $t_1$ has two futures:
$\{ t_2, t_3 \}$ and $\{ t_2, t_4 \}$.

If you look back at definition \ref{def:temporalsemantics}, you can see that in
the standard semantics for temporal logic, $\tG p$ is true at $t$ iff $p$ is
true at all times \emph{in all futures of $t$}; $\tF p$, on the other hand, is
true at $t$ iff $p$ is true at some time \emph{in at least one future of $t$}.
This ensures that $\tG$ and $\tF$ are duals, but it is often thought to be
problematic if we want $\tF p$ to translate `it will be the case that $p$'.

To illustrate, suppose I'm about to toss a coin. In one future (let's assume),
the coin will land heads, in another it will land tails. By definition
\ref{def:temporalsemantics}, both $\tF h$ and $\tF t$ are true. But should we
say that the coin will land heads and also that it will land tails?

%Which of Con1 and Con2 should we reject?

We could adopt an alternative semantics for $\tF$ according to which $\tF p$
is true at $t$ iff $p$ is true at some time in \emph{all} futures of $t$:
%
\begin{quote}
  $M,t \models \tF A \;\text{iff every future of $t$ contains some $s$ such that $M,s \models A$}.$
\end{quote}
This is known as the \textbf{Peircean interpretation} of $\tF$ (after
Charles S.\ Peirce; the name is due to Arthur Prior).

On the Peircean account, $\tF p$ is false whenever $p$ only takes place in one
of several futures. If we keep the classical interpretation of $\tG$, both
$\tF p$ and $\tG \neg p$ can be false; the two operators are no longer duals.
The dual of $\tF$ is a strange operator that applies to a sentence $A$ iff there
is \emph{some} future in which $A$ is always true.

\begin{exercise}
  Explain why the Peircean interpretation renders $p \to \tH\tF p$, an instance
  of \pr{Con2}, invalid.
\end{exercise}
\begin{solution}
  Consider a model with three times ordered by $s<t$ and $s<r$. Assume $p$ is true at $t$ and not at $r$. Then $p \to \tH\tF p$ is false on the Peircean interpretation.
\end{solution}

A rather different approach is taken by (what Prior called) the
\textbf{Ockhamist} approach. According to Ockhamism, if there are several
futures then it doesn't make sense to say -- without qualification -- that $p$
will be the case, or that $p$ won't be case. To talk about what will or won't be
the case we must specify which future we have in mind.

Formally, in Ockhamist semantics, the truth-value of every sentence is evaluated
at a pair consisting at a time and a history. Histories are linear by
definition, so the problems raised by multiple futures disappear. To say that
$p$ is the case in \emph{some} history, or in \emph{all} histories, Ockhamists
add new operators $\Diamond$ and $\Box$ that quantify over histories. The
Peircean $\tF$ operator is equivalent to $\Box \tF$ in Ockhamism. $\Box \tF p$
says that every future contains a time at which $p$ is true; $\Diamond Fp$, by
contrast, would say that some future contains a time which $p$ is true.

Here is the full Ockhamist semantics.

\begin{definition}{Ockhamist Semantics}{ockhamistsemantics}
  If $M = \t{T,<,V}$ is a temporal model, $H$ is a history in $M$, $t$
  is a member of $H$, $P$ is any sentence letter, and $A,B$ are any
  sentences in the Ockhamist language, then

  \medskip
  \begin{tabular}{lll}
    (a) & $M,H,t \models P$ &iff $t$ is in $V(P)$.\\
    (b) & $M,H,t \models \neg A$ &iff $M,H,t \not\models A$.\\
    (c) & $M,H,t \models A \land B$ &iff $M,H,t \models A$ and $M,H,t \models B$.\\
    (d) & $M,H,t \models A \lor B$ &iff $M,H,t \models A$ or $M,H,t \models B$.\\
    (e) & $M,H,t \models A \to B$ &iff $M,H,t \not\models A$ or $M,H,t \models B$.\\
    (f) & $M,H,t \models A \leftrightarrow B$ &iff $M,H,t \models (A\to B)$ and $M,H,t \models (B\to A)$.\\
    (g) & $M,H,t \models \tF A$ &iff $M,H,s \models A$ for some $s$ in $ H$ such that $t<s$.\\
    (h) & $M,H,t \models \tG A$ &iff $M,H,s \models A$ for all $s$ in $ H$ such that $t<s$.\\
    (i) & $M,H,t \models \tP A$ &iff $M,H,s \models A$ for some $s$ in $ H$ such that $s<t$.\\
    (j) & $M,H,t \models \tH A$ &iff $M,H,s \models A$ for all $s$ in $ H$ such that $s<t$.\\
    (k) & $M,H,t \models \Box A$ &iff $M,J,t \models A$ for all histories $J$ that contain $t$.\\
    (l) & $M,H,t \models \Diamond A$ &iff $M,J,t \models A$ for some history $J$ that contains $t$.
  \end{tabular}
\end{definition}

% ``The Ockhamist semantics for temporal logic was intuitively conceived by Prior
% but formally developed later, in Burgess (1979)''

A sentence is \emph{valid} in Ockhamist semantics if it is true at all times $t$
on all histories $H$ (containing $t$) in all models. As always, we can get
stronger conceptions of validity -- stronger logics -- by adding further
constraints on the precedence relation.

% Axiomatic systems for many such concepts of validity have been found,
% but I am not aware of any provably complete tree method.
% for ax.systems see \cite{kracht07logically}, p.958, who cite Zanardo 1985 and
% Reynolds 2003.

\begin{exercise}
  Which of the following schemas are valid in Ockhamist semantics?
  \begin{exlist}
  \item $\Box A \to A$
  \item $\Box A \to \Box\Box A$
  \item $\Diamond A \to \Box\Diamond A$
  \item $\Box \tF A \to \tF\Box A$
  \item $\tP A \to \Box \tP \Diamond A$%
  \end{exlist}
\end{exercise}
\begin{solution}
  (a)--(d) are valid, (e) is invalid.

  To show that a schema is valid, assume for reductio that there is some time
  $t$ on some history $H$ in some model $M$ at which the schema is false. Then
  (repeatedly) use definition \ref{def:ockhamistsemantics} to derive a
  contradiction.

  For (e), consider a model with three times $t,s,r$ such that $s\prec t$,
  $r\prec t$, and neither $s \prec r$ nor $r\prec s$. Let $q$ be true at $s$ and
  false at the other two times. $\tP q \to \Box \tP \Diamond q$ is false at $t$
  on the history $\t{s,t}$.
\end{solution}

There is something odd about the Ockhamist approach. Consider a scenario in
which there are multiple futures; one future holds a sea battle, another holds
no sea battle. Let $p$ translate `there is a sea battle'. Is $\tF p$ is true in
this scenario (under the given interpretation of $p$)? What about
$\tF(p \lor \neg p)$? Or $Gp \to GGp$?

Ockhamism refuses to give an answer. In Ockhamism, sentences are only true or
false relative to a model and a time \emph{and a history}. A branching-time
scenario, however, does not fix a particular history. We'd like to know which
sentences are true today if there are multiple futures. Ockhamism only tells us
which sentences are true relative to each of the different futures. Relative to
a history that contains a sea battle, $\tF p$ is true. Relative to other
histories, $\tF p$ is false.

% In almost every approach to modal semantics, sentences are evaluated as true
% or false relative to a model and a certain \textbf{evaluation point} taken
% from the model. In standard Kripke semantics, for example, sentences are
% defined as true or false relative to a model and a world (see definition
% \ref{def:kripkesemantics}). Here the world is the evaluation point. In
% Ockhamist semantics, an evaluation point consists of a time and a history:
% definition \ref{def:ockhamistsemantics} settles which sentences are true
% relative to a model $M$, a history $H$, and a time $t$. The problem is that an
% intuitive brnaching-time scenario determines only a model and a time, but not
% a history.

If we insist that logical validity should formalize the idea of truth in all
scenarios under all interpretations of non-logical vocabulary then we can't
accept the official definition of validity in Ockhamist semantics. We have to
extend the Ockhamist semantics to specify under what conditions a sentence is
true \emph{in a model at a time}, without fixing a history. Then we can say that
a sentence is valid iff it is true at all times in all models.

A simple way to do this is to stipulate that a sentence is true at time in an
(Ockhamist) model iff it is true relative to \emph{all} histories that contain
the time:
\begin{align*}
  M,t \models A & \;\text{iff $M,H,t \models A$ for all histories $H$ that contain $t$}.
\end{align*}
This is known as a \textbf{supervaluationist} semantics.

Supervaluationism is often used when a formal semantics defines truth relative
to an ``extra'' parameter that doesn't correspond to any feature of a
conceivable scenario. In Ockhamist semantics, that parameter is $H$. For a
different application, consider vagueness. If $p$ translates `it is warm', and
the temperature is borderline warm, it is not clear what we should say about the
truth-value of $p$, and about various complex sentences containing $p$. One
popular approach to vagueness is to first define truth relative to a
\emph{sharpening} of vague expressions. Relative to a sharpening on which
temperatures above 15.0 degrees Celsius are warm, $p$ has a clear truth-value in
any conceivable scenario, as do complex sentences containing $p$. Since an
actual scenario does not fix a particular sharpening, this semantics contains an
extra parameter. We can define a notion of truth without that parameter by
saying that a sentence is true in a scenario iff it is true in that scenario
relative to every eligible sharpening.

Supervaluationist accounts tend to have some non-classical features. Suppose we
live in a branching world in which one future contains a sea battle and another
doesn't. Let $p$ express that a sea battle takes place. According to
supervaluationist Ockhamism, neither $\tF p$ nor $\neg \tF p$ is true in that
scenario. Both are true relative to some but not relative to all histories.
So neither is simply true. Assuming that a sentence is \emph{false}
if its negation is true, $\tF p$ is neither true nor false!

Logics in which a sentence can have a third status besides (mere) truth and
(mere) falsity are called \textbf{three-valued}. Three-valued approaches to
branching time are sometimes defended by the intuition that if a sea battle
occurs on some but not all branches of the future, then one can't truly assert
that a battle \emph{will} occur nor that it \emph{won't} occur.

The Polish logician Jan \polishL{}ukasiewicz argued that statements about the
future are either true, false, or ``indeterminate''. To accommodate this third
truth-value, he proposed three-valued truth-tables specifying how the
truth-value of complex sentences are determined by the truth-value of their
parts. For example, he suggested that if two sentences $A$ and $B$ are
indeterminate, then their conjunction $A \land B$, disjunction $A \lor B$, and
negations $\neg A, \neg B$ are also indeterminate.

In the sea battle scenario, \polishL{}ukasiewicz's account renders
$\tF s \,\lor \neg\! \tF s$ indeterminate, assuming $\tF s$ is indeterminate. This
is often regarded as problematic: even if we shouldn't assert that there will be
a sea battle, it is argued that we are justified to assert that there either
will or there won't be a sea battle. The supervaluationist form of Ockhamism,
while also three-valued, avoids this problem. On the supervaluationist
interpretation, $\tF s$ and $\neg \!\tF s$ are neither true nor false in the sea
battle scenario, but $\tF s \,\lor \neg \!\tF s$ is true.

\begin{exercise}
  Let's say that a sentence is \emph{super-valid} if it is true at all times in
  all models, where truth at a time in a model is understood in accordance with
  supervaluationist Ockhamism. Explain why the super-valid sentences are
  precisely the sentences that are valid by the original Ockhamist definition of
  validity (just below definition \ref{def:ockhamistsemantics}).
\end{exercise}
\begin{solution}
  A sentence $A$ is super-valid iff $M,t \models A$ for all temporal models $M$
  and times $t$ in $M$. By supervaluationism, this holds iff $M,H,t \models A$
  for all $M,t$, and histories $H$ containing $t$. That's how Ockhamist validity was originally defined.
\end{solution}
\vspace{-2mm}
\begin{exercise}
  Things are more complicated for entailment. Let's say that $A$
  \emph{Ockham-entails} $B$ iff there is no time on any history in any temporal
  model at which $A$ is true and $B$ false. Let's say that $A$
  \emph{super-entails} $B$ iff there is no time in any temporal model at which
  $A$ is true and $B$ false, where truth at a time in a model is defined in
  accordance with supervaluationism. Is Ockham-entailment equivalent to
  super-entailment? Explain.
\end{exercise}
\begin{solution}
  Ockham-entailment is stronger than super-entailment: whenever $A$
  Ockham-entails $B$, then $A$ super-entails $B$, but not the other way around.

  Suppose $A$ Ockham-entails $B$. Let $t$ be any time in any temporal model at
  which $A$ is true, i.e.: true relative to all histories through $t$. Since $A$
  Ockham-entails $B$, $B$ is true at $t$ relative to all histories through
  $t$. So $A$ super-entails $B$.

  But suppose $A$ super-entails $B$. Let $t$ be any time on any history $h$ in
  any temporal model at which $A$ is true. We can't infer that $B$ is true at
  $t$ on $h$, for $A$ may be false at $t$ relative to other histories $h'$. So
  we can't infer that $A$ Ockham-entails $B$. Indeed, $\tF p$ super-entails
  $\Box \tF p$, but $\tF p$ does not Ockham-entail $\Box \tF p$.
\end{solution}

% So is the true logic of time three-valued? I think this is not a good
% question. The way I see it, it's a matter of choice. We can easily speak about
% branching time in a classical, two-valued logic. And we could do this in
% different ways. If it really turns out that the world has a branching time
% structure, it might be reasonable to never say things like 'so-and-so will
% happen tomorrow', and instead say 'it will happen on one branch', or 'it will
% happen on all branches'. But we could also adopt a convention to use 'it will
% happen' as shorthand for 'it will happen on all branches', and then, if the
% convention is spelled out in a certain way, we'll have chosen to speak in a
% three-valued language.


\section{Extending the language}\label{sec:2d}

The expressive resources of standard modal and temporal logic are weak. There
are many things we might want to say about the unfolding of events in time that
can't be said with $\tF, \tG, \tP$, and $\tH$. The Ockhamist history quantifiers
are one way of adding expressive power to the basic language of temporal logic.
In this section, we will look at some others.

A useful operator for logics of discrete and linear time is the ``next''
operator $\tX$ (also written `$\bigcirc$'). Informally, $\tX A$ means that $A$
is true at the next point in time. Formally:
%
\begin{align*}
\hspace{-17mm}
  M,t \models \tX A \;\text{iff} & \;\text{$M,s \models A$ for some $s$ such that (i) $t<s$ and (ii) $s<r$ for all $r$}\\[-0.5mm]
  & \;\text{such that $r\not=s$ and $t<r$.}
\end{align*}

With the help of $\tX$, we can also say that $A$ is true in two units of time
($\tX\tX A$), in three units of time ($\tX\tX\tX A$), and so on. The
corresponding operator for talking about the \emph{previous} point in time is
usually written $\mathsf{Y}$.

A more powerful extension of $\L_T$ adds binary operators for ``since'' and
``until'', which can be used to translate sentences like (1) and (2).
%
\begin{itemize}[leftmargin=10mm]
  \itemsep-1mm
\item[(1)] Ever since we left the house it has been raining.
\item[(2)] It will be raining until we go back inside.
\end{itemize}
%
Informally, $A\tS B$ is true iff $B$ was true at some time in the past and $A$
has always been true since then; $A \tU B$ is true iff $B$ will be true at some
time in the future and $A$ will always be true until then. Formally:

\vspace{-5mm}
\begin{align*}
  \hspace{-17mm}
  M,t \models A \tS B \;\text{iff} &\; \text{there is some $s$ with $s < t$ for which $M,s \models B$, and for all $r$}\\[-0.5mm]
                                    &\;\text{with $s < r < t$, we have $M, r \models A$.}
\end{align*}
\vspace{-12mm}
\begin{align*}
  \hspace{-17mm}
M,t \models A \tU B \;\text{iff} &\; \text{there is some $s$ with $t < s$ for which $M,s \models B$, and for all $r$}\\[-0.5mm]
  &\;\text{with $t < r < s$, we have $M, r \models A$.}
\end{align*}

The operators $\tF, \tG,\tP,$ and $\tH$ can all be defined in terms of $\tS$ and
$\tU$. For example, $\tP A$ is equivalent to $(p\lor \neg p)\tS A$. And $\tF A$
is equivalent to $(p \lor \neg p)\tU A$.

\begin{exercise}
  Define $\tX A$ in terms of $\tU$.
\end{exercise}
\begin{solution}
  $(A \land \neg A) \tU A$.
\end{solution}

Another noteworthy addition to temporal logic is the ``Now'' operator $\tN$.
To see the point of this operator, consider the following multi-modal statement.

\begin{itemize}[leftmargin=10mm]
  \itemsep-1mm
\item[(3)] We already knew yesterday that there would be a test today.
\end{itemize}

Using $\tY$ for `yesterday', we might try to translate (3) as $\tY \Kn p$, where
$p$ translates `there is a test'. But that's wrong. By the semantics for $\tY$,
$\tY \Kn p$ is true today iff $\Kn p$ is true yesterday (using days as temporal
units). Since $\Kn p$ entails $p$, it follows that $\tY \Kn p$ is true today
only if $p$ is true \emph{yesterday}. But the test takes place today, not
yesterday.

Intuitively, the problem is that `today' in (3) refers to the present day, even
though it occurs in the scope of the `yesterday' operator. The same thing
happens in the quantified statement (4).
%
\begin{itemize}[leftmargin=10mm]
  \itemsep-1mm
\item[(4)] One day everyone who is now rich will be poor.
\end{itemize}
% 
Here, `now' refers to the present time, even though it is in the scope
of the $\tF$ operator `one day'.

% Kamp 1971 proved that every $\L_T$ sentence containing Now is
% equivalent to one without Now. Things change in quantified logic.

With the ``Now'' operator $\tN$, we can translate (3) as $\tY \Kn \tN p$, and
(4) as $\tF \forall x (\tN Rx \to Px)$. (We will have a closer look at
quantified modal logic in later chapters.)

Intuitively, the $\tN$ operator allows us to look outside the scope of an
embedding operator. $\tP \tN p$, for example, is true if there is some time in
the past such that $p$ is true not at that time, but at the present. How
does this work formally?

By the semantics of $\tP$,%
\begin{align*}
  M,t \models \tP \tN p \;\text{iff} \;\text{$M,s \models \tN p$ for some time $s < t$}.
\end{align*}
Now we want $M,s\models \tN p$ to be true iff $p$ is true at the original time
$t$. So we need to keep track of the original time at which we evaluate a
sentence, even if a temporal operator shifts the time at which a subsentence is
evaluated.

The simplest way to achieve this is to define truth relative to pairs of
times. One of the times is shifted by the temporal operators, the other is held
fixed.

\begin{definition}{Two-Dimensional Temporal Semantics}{2dtemporalsemantics}
  If $\Mfr = \t{T,<,V}$ is a temporal model, $t,t_0$ are members of $T$, $P$ is
  any sentence letter, and $A,B$ are any $\L_T$-sentences, then

  \medskip
  \begin{tabular}{lll}
    (a) & $M,t_0,t \models P$ &iff $t$ is in $V(P)$.\\
    (b) & $M,t_0,t \models \neg A$ &iff $M,t_0,t \not\models A$.\\
    (c) & $M,t_0,t \models A \land B$ &iff $M,t_0,t \models A$ and $M,t_0,t \models B$.\\
    (d) & $M,t_0,t \models A \lor B$ &iff $M,t_0,t \models A$ or $M,t_0,t \models B$.\\
    (e) & $M,t_0,t \models A \to B$ &iff $M,t_0,t \not\models A$ or $M,t_0,t \models B$.\\
    (f) & $M,t_0,t \models A \leftrightarrow B$ &iff $M,t_0,t \models (A\to B)$ and $M,t_0,t \models (B\to A)$.\\
    (g) & $M,t_0,t \models \tF A$ &iff $M,t_0,s \models A$ for some $s$ in $ T$ such that $t<s$.\\
    (h) & $M,t_0,t \models \tG A$ &iff $M,t_0,s \models A$ for all $s$ in $ T$ such that $t<s$.\\
    (i) & $M,t_0,t \models \tP A$ &iff $M,t_0,s \models A$ for some $s$ in $ T$ such that $s<t$.\\
    (j) & $M,t_0,t \models \tH A$ &iff $M,t_0,s \models A$ for all $s$ in $ T$ such that $s<t$.\\
    (k) & $M,t_0,t \models \tN A$ &iff $M,t_0,t_0 \models A$.
  \end{tabular}
\end{definition}

Like the Ockhamist semantics from the previous section, this semantics has an
extra parameter. An ordinary scenario is represented by a single time in a
model, not by a pair of times. So we need to specify under what
conditions a sentence is true at a (single) time. Here, the standard approach is
not supervaluation but ``diagonalization'':
\begin{align*}
  M,t \models A & \;\text{iff $M,t,t \models A$}.
\end{align*}

This ``two-dimensional'' semantics correctly predicts that $\tP\tN p$ entails $p$.
\begin{enumerate}[leftmargin=10mm]
  \itemsep-1mm
\item Assume $M,t \models \tP\tN p$.
\item Then $M,t,t \models \tP\tN p$, by the definition of truth at a time in a model.
\item Then $M,t,s \models \tN p$ for some $s<t$, by clause (i) of definition \ref{def:2dtemporalsemantics}.
\item Then $M,t,t \models p$, by clause (k) of definition \ref{def:2dtemporalsemantics}.
\item Then $M,t \models p$, by the definition of truth at a time in a model.
\end{enumerate}

The presence of a ``Now'' operator has far-reaching consequences for the logic
of time. For example, $\tN p \to p$ is valid, in the sense that it is true at
all times in all models. But $\tG(\tN p \to p)$ is invalid. If $p$ is true at
$t$ and false at some time after $t$, then $\tG(\tN p \to p)$ is false at $t$.
So we must give up the forward and backward Necessitation rules. The fact that
something is logically true does not entail that it will always be true!

\begin{exercise}
  `It might have been that everyone who is actually rich is poor.' This says
  that there is a world $w$ such that everyone who is rich \emph{at the actual
    world} is poor \emph{at $w$}. To formalize statements like these, we need a
  modal operator analogous to $\tN$ that takes us back to the actual world, even
  in the scope of other modal operators. This operator is called the
  \emph{actually} operator. Let's write it as $\tA$ and add it to $\L_{M}$. Can
  you find a sentence $B$ in this language that is logically true but not
  necessarily true, in the sense that $B$ is true at all worlds in all models
  but $\Box B$ is not?
\end{exercise}
\begin{solution}
  $\tA p \to p$.
\end{solution}

% \begin{align}
%   1. & N p \to p & \text{theorem}\\
%   2. & \Kn(Np \to p) & \text{necessitation}\\
%   2. & \tG \Kn(Np \to p) & \text{necessitation}\\
%   3. & \tG (\Kn(Np \to p) \to \Kn(Np \to p))& \text{nec factivity}\\
%   4. & \tG (Np \to p)& \text{2,3}\\
%   5. & p\to \tG Np& \text{theorem}\\
%   6. & p\to \tG p& \text{4,5}
% \end{align}

% From Rini & Cresswell p.20: It is usually assumed that worlds and times can be
% shifted independently. Meyer 2006 "Worlds and Times" considers the possibility
% that what times there are varies from world to world, so that we might need to
% allow a sentence to be true/false at a triple <w,t,p> even if t doesn't exist
% at p. [p is a person index.]


%%% Local Variables: 
%%% mode: latex
%%% TeX-master: "logic2.tex"
%%% End:

\chapter{Conditionals}\label{ch:conditionals}

\section{Material conditionals}\label{sec:material}

We are often interested not just in whether something is in fact the case, but
also in whether it is (or would be) the case \emph{if} something else is (or
would be) the case. We might, for example, wonder in what will happen to the
climate if we don't reduce greenhouse gases, or whether World War 2 could have
been avoided if certain steps had been taken in the 1930s.

A sentence stating that something is (or would be) the case if something else is
(or would be) the case is called a \textbf{conditional}. What exactly, do these
statements mean? What is their logic? Philosophers have puzzled over these
questions for more than 2000 years, with no agreement in sight.

One attractively simple view is that a conditional `if $A$ then $B$' is true iff
the antecedent $A$ is false or the consequent $B$ is true. This would make `if
$A$ then $B$' equivalent to `not $A$ or $B$'. Conditionals with these
truth-conditions are called \textbf{material conditionals}.

The conditionals $A \to B$ of classical logic are material. $A \to B$ is
equivalent to $\neg A \lor B$. The attractively simple view that English
conditionals are material conditionals would mean that we can faithfully
translate English conditionals into $\L_{M}$-sentences of the form $A \to B$. Is
this correct?

There are some arguments for a positive answer. Suppose I make the following
promise.
\begin{itemize}[leftmargin=10mm]
  \item[(1)] If I don't have to work tomorrow then I will help you move.
\end{itemize}
I have made a false promise if the next day I don't have to work and yet I don't
help you move. Under all other conditions, you could not fault me for breaking my
promise. So it seems that (1) is false iff I don't have to work and I don't help
you move. Generalizing, this suggests that `if $A$ then $B$' is true iff $A$ is
false or $B$ is true.

Another argument for analysing English conditionals as material conditionals
starts with the intuitively plausible assumption that `$A$ or $B$' entails the
corresponding conditional `if not-$A$ then $B$'. (This is sometimes called the
\emph{or-to-if} inference.) Suppose I tell you that Nadia is either in Rome or
in Paris. Trusting me, you can infer that if she's not in Rome then she's in
Paris. Now we can reason as follows.

To begin, suppose that $A$ and `if $A$ then $B$' are both true. Plausibly, we
can infer that $B$ is true as well: \emph{modus ponens} is valid for English
conditionals. \label{p:mp} This means that if $A$ is true and $B$ is false, then
`if $A$ then $B$' is false. Now suppose, alternatively, that $A$ is false or $B$
is true. Then `not-$A$ or $B$' is true. By or-to-if, we can infer that `if $A$
then $B$' is true as well. Thus `if $A$ then $B$' is true iff $A$ is false or
$B$ is true.

Despite these arguments, most philosophers and linguists don't think that
English conditionals are material conditionals. Consider these facts about
logical consequence (in classical propositional logic).
%
\label{paradoxes-mat-imp}
\begin{principles}
\pri{M1}{B \models A \to B}\\
\pri{M2}{\neg A \models A \to B}\\
\pri{M3}{\neg (A \to B) \models A}\\
%\item (A \to B) \lor (B \to A)}\\
%\pri (A \to B) \lor (B \to C)}\\
\pri{M4}{A \to B \models \neg B\to \neg A}\\
\pri{M5}{A \to B \models (A\land C) \to  B}
\end{principles}

If English conditionals were material conditionals then the following
inferences, corresponding to (M1)--(M5), would be valid.

\begin{enumerate}[leftmargin=12mm]
  \itemsep1mm
  \item[(E1)] There won't be a nuclear war. Therefore: If Russia attacks the US
        with nuclear weapons then there won't be a nuclear war.
  \item[(E2)] There won't be a nuclear war. Therefore: If there will be a
        nuclear war then nobody will die.
  \item[(E3)] It is not the case that if it will rain tomorrow then the Moon
        will fall onto the Earth. Therefore: It will rain tomorrow.
  \item[(E4)] If our opponents are cheating, we will never find out. Therefore:
        If we will find out that our opponents are cheating, then they aren't
        cheating.
  \item[(E5)] If you add sugar to your coffee, it will taste good. Therefore: If
        you add sugar and vinegar to your coffee, it will taste good.
\end{enumerate}

These inferences do not sound good. If we wanted to defend the view that English
conditionals are material conditionals we would have to explain why they sound
bad even though they are valid. We will not explore this option any further.

\begin{exercise}
  Can you find a different analysis of English conditionals that, like the
  material analysis, would make conditionals truth-functional, but that would
  render all of (E1)--(E5) invalid?
\end{exercise}
\begin{solution}
  (E1)--(E5) are invalid assuming that `if $A$ then $B$' is true iff both $A$
  and $B$ are true. There are, of course, strong reasons against the analysis of
  English conditionals as conjunctions.
\end{solution}

Even those who defend the material analysis of English conditionals admit that
it does not work for all English conditionals. Consider (2).
\begin{enumerate}[leftmargin=10mm]
\item[(2)] If water is heated to $100^\circ$ C, it evaporates.
\end{enumerate}
This shouldn't be translated as $p\to q$. Intuitively, (2) states that \emph{in
  all (normal) cases} where water is heated to $100^\circ$ C, it evaporates. It
is a quantified, or modal claim.

Another important class of conditionals that can't be analysed as material
conditionals are so-called \textbf{subjunctive conditionals}. Compare the
following two statements.

\begin{enumerate}[leftmargin=10mm]
  \itemsep-1mm
\item[(3)] If Shakespeare didn't write \emph{Hamlet}, then someone else did.
\item[(4)] If Shakespeare hadn't written \emph{Hamlet}, then someone else
  would have.
\end{enumerate}
%
(3) seems true. Someone has written \emph{Hamlet}; if it wasn't Shakespeare then
it must have been someone else. But (4) is almost certainly false. After all, it
is very likely that Shakespeare did write \emph{Hamlet}. And it is highly
unlikely that if he hadn't written \emph{Hamlet} -- if he got distracted by
other projects, say -- then someone else would have stepped in to write the
exact same piece.

Sentences like (3) are called \textbf{indicative conditionals}. Intuitively, an
indicative conditional states that something is \emph{in fact} the case on the
assumption that something else is the case. A subjunctive conditional like (4)
states that something \emph{would be} the case if something else \emph{were} the
case. Typically we know that the ``something else'' is not in fact the case. We
know, for example, that Shakespeare wrote \emph{Hamlet} and therefore that the
antecedent of (4) is false. For this reason, subjunctive conditionals are also
called \emph{counterfactual conditionals} or simply \emph{counterfactuals}.

It should be clear that subjunctive conditionals are not material conditionals.
I said that (4) is almost certainly false. But it almost certainly has a false
antecedent. So the corresponding material conditional is almost certainly true.

\section{Strict conditionals}\label{sec:strict-implication}

% On the history see http://hume.ucdavis.edu/mattey/phi134/strict.htm

One apparent difference between material conditionals $A \to B$ and conditionals
in natural language is that $A\to B$ requires no connection between the
antecedent $A$ and the consequent $B$. Consider (1).
\begin{enumerate}[leftmargin=10mm]
  \item[(1)] If we leave after 5, we will miss the train.
\end{enumerate}
Intuitively, someone who utters (1) wants to convey that missing the train
is a \emph{necessary consequence} of leaving after 5 -- that it is \emph{impossible} to
leave after 5 and still make it to the train, given certain facts about the
distance to the station, the time it takes to get there, etc. This suggests that
(1) should be formalized not as $p \to q$ but as $\Box(p \to q)$ or,
equivalently, $\neg \Diamond(p \land \neg q)$.

Sentences that are equivalent to $\Box(A \to B)$ are called \textbf{strict
  conditionals}. The label goes back to C.I.\ Lewis (1918), who also introduced
the abbreviation $A \strictif B$ for $\Box(A \to B)$.

Lewis was not interested in `if \ldots then \ldots' sentences. He introduced
$A \strictif B$ to formalize `$A$ implies $B$' or `$A$ entails $B$'. His
intended use of $\strictif$ roughly matches our use of the double-barred
turnstile `$\models$'. But there are important differences. The turnstile is an
operator in our \emph{meta-language}; Lewis's $\strictif$ is an
\emph{object-language} operator that, like $\land$ or $\to$, can be placed
between any two sentences in a formal language to generate another sentence in
the language. $p \strictif (q \strictif p)$ is well-formed, whereas
$p \models (q\models p)$ is gibberish. Moreover, while $p \models q$ is simply
false -- because there are models in which $p$ is true and $q$ false -- Lewis's
$p \strictif q$ is true on some interpretation of the sentence letters and false
on others. If $p$ means that it raining heavily and $q$ that it is raining, then
$p \strictif q$ is true because the hypothesis that it is raining heavily
implies that it is raining.

Let's set aside Lewis's project of formalizing the concept of implication. Our
goal is to find an object-language construction that functions like `if \ldots
then \ldots' in English. To see whether `$\ldots \strictif\ldots$' can do the job,
let's have a closer look at the logic of strict conditionals.

Since $A \strictif B$ is equivalent to $\Box(A \to B)$, standard Kripke
semantics for the box also provides a semantics for strict conditionals. In
Kripke semantics, $\Box(A \to B)$ is true at a world $w$ iff $A \to B$ is true
at all worlds $v$ accessible from $w$. And $A \to B$ is true at $v$ iff 
$A$ is false at $v$ or $B$ is true at $v$. We therefore have the following
truth-conditions for strict conditionals.

\begin{definition}{Kripke semantics for $\strictif$}{semstrictif}
  If $M = \t{W,R,V}$ is a Kripke model, then\\[1mm]
  $M,w \models A \strictif B$ iff for all $v$ such that $wRv$, either
  $M,v \not\models A$ or $M,v \models B$.
\end{definition}

\begin{exercise}
  $A \strictif B$ is equivalent to $\Box(A \to B)$. Can you fill the blank in:
  `$\Box A$ is equivalent to ---', using no modal operator other than
  $\strictif$?
\end{exercise}
\begin{solution}
  For example: $\neg A \strictif A$ or $(A\lor \neg A) \strictif A$.
\end{solution}

As always, the logic of strict conditionals depends on what constraints we
put on the accessibility relation. Without any constraints, $\strictif$ does
not validate \emph{modus ponens}, in the sense that $A \strictif B$ and $A$
together do not entail $B$. We can see this by translating
$A \strictif B$ back into $\Box(A \to B)$ and setting up a tree. Recall that to
test whether some premises entail a conclusion, we start the tree with the
premises and the negated conclusion.
%
\begin{center}
  \tree[3]{%
    & \nnode{18}{1.}{$\Box (A \to B)$}{w}{(Ass.)} &\\
    & \nnode{18}{2.}{$A$}{w}{(Ass.)} &\\
    & \nnode{18}{3.}{$\neg B$}{w}{(Ass.)} &
  }
\end{center}
%
With the K-rules, where we don't make any assumptions about the accessibility
relation, node 1 can't be expanded, so there is nothing more we can do.

\begin{exercise}
  Give a countermodel in which $p \strictif q$ and $p$ are true at some world
  while $q$ is false.
\end{exercise}
\begin{solution}
  $W = \{ w \}$, $R = \emptyset$, $V(p) = \{ w \} $, $V(q)=\emptyset$.
\end{solution}

If we assume that the accessibility relation is reflexive, the tree closes:
\begin{center}
  \tree[3]{%
    & \nnode{18}{4.}{$wRw$}{}{(Ref.)} &\\
    & \bnode{18}{5.}{$A \to B$}{w}{(1,4)} &\\
    &&\\
    \nnodeclosed{10}{6.}{$\neg A$}{w}{(5)} && \nnodeclosed{10}{7.}{$B$}{w}{(5)}
  }
\end{center}

It is not hard to show that \emph{modus ponens} for $\strictif$ is valid on all
and only the reflexive frames. Reflexivity is precisely what we need to render
\emph{modus ponens} valid. Since \emph{modus ponens} looks plausible for English
conditionals (as I've argued on p.~\pageref{p:mp}), we'll probably want the
relevant Kripke models to be reflexive.

\begin{exercise}\label{ex:sda-import}
  Using the tree method, and translating $A \strictif B$ into $\Box(A \to B)$, confirm that following claims hold, for all $A,B,C$.
  \begin{exlist}
  \item $\models_K A \strictif A$
  \item $A \strictif B \models_K \neg B \strictif \neg A$
  \item $A \strictif B \models_K (A \land C) \strictif B$
  \item $A\strictif B, B \strictif C \models_{K} A \strictif C$
  \item $(A \lor B) \strictif C \models_K (A \strictif C) \land (B \strictif C)$
  \item $A \strictif (B \strictif C) \models_T (A \land B) \strictif C$
%  \item $A \strictif B, \neg B \models_T \neg A$ 
  \item $A\strictif B \models_{S4} C \strictif (A \strictif B)$
    % inspired by Hacking's formula for 4, which is the necessitated version of the corresponding conditional
  \item $((A\strictif B) \strictif C) \strictif (A\strictif B) \models_{S5} A\strictif B$
  \end{exlist}
  \medskip

  Which of these schemas do you think should be valid if we
  assume that $A \strictif B$ translates indicative conditionals `if $A$ then $B$'?
\end{exercise}
\begin{solution}
  Use \href{https://www.umsu.de/trees/}{umsu.de/trees/}.
\end{solution}

% Note: if epistemic, then maybe S4.2; if subjunctive, then maybe 5??

We could now look at other conditions on the accessibility relation and decide
whether they should be imposed, based on what they would imply for the logic of
conditionals. But let's take a shortcut.

I have suggested that sentence (1) might be understood as saying that it is
\emph{impossible} to leave after 5 and still make it to the train. Impossible in
what sense? There are many possible worlds at which we leave after 5 and still
make it to the train. There are, for example, worlds at which the train departs
two hours later, worlds at which we live right next to the station, and so on.
When I say that it is impossible to leave after 5 and still make it to the
train, I arguably mean that it is impossible \emph{given what we know about the
  departure time, our location, etc.}

Generalizing, a tempting proposal is that the accessibility relation that is
relevant for indicative conditionals like (1) is the epistemic accessibility
relation that we studied in chapter \ref{ch:epistemic}, where a world $v$ is
accessible from $w$ iff it is compatible with what is known at $w$. On that
hypothesis, the logic of indicative conditionals is determined by the logic
of epistemic necessity. We don't need to figure out the relevant accessibility
relation from scratch.

Since knowledge varies from agent to agent, the present idea implies that the
truth-value of indicative conditionals should be agent-relative. This seems to
be confirmed by the following puzzle, due to Allan Gibbard.
\begin{quote}
  Sly Pete and Mr.\ Stone are playing poker on a Mississippi riverboat. It is
  now up to Pete to call or fold. My henchman Zack sees Stone’s hand, which is
  quite good, and signals its content to Pete. My henchman Jack sees both hands,
  and sees that Pete's hand is rather low, so that Stone's is the winning hand.
  At this point the room is cleared. A few minutes later, Zack slips me a note
  which says `if Pete called, he won', and Jack slips me a note which says `if
  Pete called, he lost'. % \cite[231]{gibbard1981xxx}
\end{quote}
The puzzle is that Zack's note and Jack's note are intuitively contradictory,
yet they both seem to be true.

We can resolve the puzzle if we understand the conditionals as strict
conditionals with an agent-relative epistemic accessibility relation. Take Zack.
Zack knows that Pete knows Stone's hand. He also knows that Pete would not call
unless he has the better hand. So among the worlds compatible with Zack's
knowledge, all worlds at which Pete calls are worlds at which Pete wins. If $p$
translates `Pete called' and $q$ `Pete won', then $p \strictif q$ is true
relative to Zack's information state. Relative to Jack's information state,
however, the same sentence is false. Jack knows that Stone's hand is better than
Pete's, but he doesn't know that Pete knows Stone's hand. Among the worlds
compatible with Jack's knowledge, all worlds at which Pete calls are therefore
worlds at which Pete loses. Relative to Jack's information state,
$p \strictif \neg q$ is true.

Another advantage of the ``epistemically strict'' interpretation is that it
might explain why indicative conditionals with antecedents that are known to be
false seem defective. For example, imagine a scenario in which Jones has gone to
work. In that scenario, is (2) true or false?
\begin{enumerate}[leftmargin=10mm]
  \item[(2)] If Jones has not gone to work then he is helping his neighbours.
\end{enumerate}
The question is hard to answer -- and not because we lack information about the
scenario. Once we are told that Jones has gone to work, it is unclear how we are
meant to assess whether Jones is helping his neighbours \emph{if} he has not
gone to work. On the epistemically strict interpretation, (2) says that Jones is
helping his neighbours at all epistemically accessible worlds at which Jones
hasn't gone to work. Since we know that Jones has gone to work, there are no
epistemically accessible worlds at which he hasn't gone to work. And if there
are no $A$-worlds then we naturally balk at the question whether all $A$-worlds
are $B$-worlds. (In logic, we resolve to treat `all $A$s are $B$' as true if
there are no $A$s. Accordingly, (2) comes out true on the epistemically strict
analysis. But we can still explain why it seems defective.)

We have found a promising alternative to the hypothesis that indicative
conditionals are material conditionals. According to the present alternative,
they are epistemically strict conditionals -- strict conditionals with an
epistemic accessibility relation.

What about subjunctive conditionals? Return to the two Shakespeare conditionals
from the previous section. When we evaluate the indicative sentence -- `If
Shakespeare didn't write \emph{Hamlet}, then someone else did' -- we hold fixed
our knowledge that \emph{Hamlet} exists; worlds where the play was never written
are inaccessible. That's why the conditional is true. At all accessible worlds
at which Shakespeare didn't write \emph{Hamlet}, someone else wrote the play.
When we evaluate the subjunctive conditional -- `If Shakespeare hadn't written
\emph{Hamlet}, then someone else would have' -- we do consider worlds at which
\emph{Hamlet} was never written, even though we know that the actual world is
not of that kind. If subjunctive conditionals are strict conditionals, then
their accessibility relation does not track our knowledge or information.
Unfortunately, as we are going to see in the next section, it is hard to say
what else it could track.

This is one problem for the strict analysis of natural-language conditionals.
Another problem lies in the logic of strict conditionals. Remember (E1)--(E5)
from page \pageref{paradoxes-mat-imp}. If English conditionals are strict
conditionals, then (E1)--(E3) are invalid. For example, while $q$ entails
$p \to q$, it does not entail $p \strictif q$. But the strict analogs of
\pr{M4} and \pr{M5} still hold, no matter what we say about accessibility (see
exercise \ref{ex:sda-import}):
\begin{gather*}
  A \strictif B \models \neg B\strictif \neg A;\\
  A \strictif B \models (A\land C) \strictif  B.
\end{gather*}
So we still predict that the inferences (E4) and (E5) are valid.

\begin{enumerate}[leftmargin=12mm]
  \itemsep1mm
  \item[(E4)] If our opponents are cheating, we will never find out. Therefore:
        If we will find out that our opponents are cheating, then they aren't
        cheating.
  \item[(E5)] If you add sugar to your coffee, it will taste good. Therefore: If
        you add sugar and vinegar to your coffee, it will taste good.
\end{enumerate}

\begin{exercise}
  The badness of (E4) and (E5) suggests that indicative conditionals can't be
  analysed as strict conditionals. Can you give a similar argument suggesting
  that \emph{subjunctive} conditionals can't be analysed as strict conditionals?
\end{exercise}
\begin{solution}
  (E1)--(E5) all work equally well in the subjunctive mood. For \pr{E4} and
  \pr{E5}:
  \begin{itemize}
  \item If our opponents had been cheating, we would never have found out.
    Therefore: If we had found out that our opponents are cheating, then
    they wouldn't have been cheating.
  \item If you had added sugar to your coffee, it would have tasted
    good. Therefore: If you had added sugar and vinegar to your coffee, it would
    have tasted good.
  \end{itemize}
  Both of these inferences are valid if subjunctive conditionals are strict
  conditionals. But they don't sound good.
\end{solution}

% Another problem: subjunctive mights. Might if A then B doesn't seem to mean
% $\Diamond\Box(A \to B)$, nor $\Box(A \to \Diamond B)$.

% \begin{exercise}
%   Some have argued that problems similar to those raised by
%   \pr{M1}--\pr{M3} remain for strict implication, because the
%   following turn out to be true:
%   \begin{enumerate*}
%   \item $\Box B \models A \strictif B$
%   \item $\Box \neg A \models A \strictif B$
%   \item $\neg (A \strictif B) \models \Diamond A$
%   \end{enumerate*}
  
%   Note analogy to wide-scope analysis of conditional obligation.
% \end{exercise}

% Arguably, natural language conditionals lie in between strict and material. For
% if $A \to B$ is false, then $A$ is true and $B$ false, and then `if $A$ then
% $B$' is surely false. And if $A\to B$ is necessary, so at all worlds where $A$
% holds $B$ holds, then plausibly if $A$ holds (or were to hold) at the actual
% world, then so does (would) $B$.

% Every SC model (as defined here) is a VC model (as defined in next sec). So
% everything that's true in all VC models is true in all SC models; i.e. the
% logic SC is strictly stronger than VC.

\begin{exercise}
  A plausible norm of pragmatics is that a sentence should only be asserted if
  it is known to be true. Let's call a sentence \emph{assertable} if it is known
  to be true. Show that if the logic of knowledge is at least S4, then an
  epistemically strict conditional $A \strictif B$ is assertable iff the
  corresponding material conditional $A \to B$ is assertable.
\end{exercise}
\begin{solution}
  Suppose $A \to B$ is assertable. Then $A\to B$ is known. So $\Kn (A \to
  B)$. In S4, it follows that $\Kn\Kn(A \to B)$. So the epistemically strict
  conditional $\Kn(A \to B)$ is assertable. Conversely, if $\Kn(A \to B)$ is
  assertable, then it is known; so $\Kn\Kn(A \to B)$. In S4, it follows that
  $\Kn(A \to B)$. So $A \to B$ is assertable.
\end{solution}

\begin{exercise}
  Explain why the `or-to-if' inference from `$p$ or $q$' to `if not $p$ then
  $q$' is invalid on the assumption that the conditional is epistemically
  strict. How could a friend of this assumption explain why the inference
  nonetheless looks reasonable, at least in normal situations? (Hint: Remember
  the previous exercise.)
\end{exercise}
\begin{solution}
  The `or-to-if' inference is not valid on the assumption that the conditional
  is epistemically strict. For example, if $p$ and $q$ are both true at the actual world and both false at some epistemically accessible world, then `$p$ or $q$' is true but `if $p$ then $q$' is false (on the strict analysis).

  The inference might nonetheless look reasonable because it would normally be
  inappropriate to assert a disjunction `$p$ or $q$' unless the disjunction is
  known -- unless it is true at all epistemically accessible worlds. And if
  $p \lor q$ is true at all epistemically accessible worlds then $\neg p \to q$
  is also true at all epistemically accessible worlds, and so $\Box(\neg p \to q)$ is
  true. Thus the conclusion of or-to-if is true in any situation in which the
  premise is \emph{assertable}. If the logic of knowledge validates the
  \pr{4}-schema, we can go further and say that the conclusion is assertable in
  any situation in which the premise is assertable.
\end{solution}

  
\section{Variably strict conditionals}

Let's have a closer look at subjunctive conditionals. As I am writing these
notes, I am sitting in Coombs Building, room 2228, with my desk facing
the wall to Al H\'ajek's office in room 2229. In light of these facts, (1) seems
true.
\begin{enumerate}[leftmargin=10mm]
\item[(1)] If I were to drill a hole through the wall behind my desk,
  the hole would come out in Al's office.
\end{enumerate}
% This example is meant to get around some of Al's worries.
There is no logical connection between the antecedent of (1) and the consequent.
There are many possible worlds at which I drill a hole through the wall behind
my desk and don't reach Al's office -- for example, worlds at which my desk
faces the opposite wall, worlds at which Al's office is in a different room, and
so on. If (1) is a strict conditional then all such worlds must be inaccessible.

Now consider (2).
\begin{enumerate}[leftmargin=10mm]
\item[(2)] If the office spaces had been randomly reassigned yesterday then Al's
  office would (still) be next to mine.
\end{enumerate}
(2) seems false, or at least very unlikely. But if (2) is a strict conditional,
and worlds at which Al is not in room 2229 or I am not in 2228 are inaccessible
-- as they seem to be for (1) -- then (2) should be true. Among worlds at which
I am in 2228 and Al is in 2229, all worlds at which the office spaces have been
randomly reassigned yesterday are worlds at which Al's office is next to mine.
When we evaluate (2), it looks like we no longer hold fixed who is in which
office. Worlds that were inaccessible for (1) are accessible for (2).

So the accessibility relation, at least for subjunctive conditionals, appears to
vary from conditional to conditional. As David Lewis put it, subjunctive
conditionals seem to be not strict, but ``variably strict''.

Let's try to get a better grip on how this might work. (What follows is a
slightly simplified version of an analysis developed by Robert Stalnaker and
David Lewis in the 1960s.)

Intuitively, when we ask what would have been the case if a certain event had
occurred, we are looking at worlds that are much like the actual world up to the
time of the event. Then these worlds deviate in some minimal way to allow the
event to take place. Afterwards the worlds unfold in accordance with the general
laws of the actual world.

For example, if we wonder what would have happened if Shakespeare hadn't written
\emph{Hamlet}, we are interested in worlds that are like the actual world until
1599, at which point some mundane circumstances prevent Shakespeare from writing
\emph{Hamlet}. We are not interested in worlds at which Shakespeare was never
born, or in which the laws of nature are radically different from the laws at
our world. One might reasonably judge that Shakespeare would have been a famous
author even if he hadn't written \emph{Hamlet}, although we would hardly be
famous in worlds in which he was never born.

Likewise for (1). Here we are considering worlds that are much like the actual
world up to now, at which point I decide to drill a hole and find a suitable
drill. These changes do not require my office to be in a different room. Worlds
where I'm not in room 2228 can be ignored. Figuratively speaking, such worlds
are ``too remote'': they differ from the actual world in ways that are not
required to make the antecedent true.

This suggests that a subjunctive conditional is true iff the consequent is true
at the ``closest'' worlds at which the antecedent is true -- where ``closeness''
is a matter of similarity in certain respects. The closest worlds (to the actual
world) at which Shakespeare didn't write \emph{Hamlet} are worlds that almost
perfectly match the actual world until 1599, then deviate a little so that
Shakespeare didn't write Hamlet, and afterwards still resemble the actual world
with respect to the general laws of nature. We will not try to spell out in full
generality what the relevant closeness measure should look like.

Let `$v \prec_w u$' mean that $v$ is closer to $w$ than $u$, in the sense that
$v$ differs less than $u$ from $w$ in whatever respects are relevant to the
interpretation of subjunctive conditionals.

We make the following structural assumptions about the world-relative ordering
$\prec$.

\begin{enumerate}[leftmargin=8mm]
  \itemsep-1mm
  \item If $v \prec_w u$ then $u \nprec_w v$. (Asymmetry)
  \item If $v \prec_w u$, then for all $t$ either $v \prec_w t$ or
        $t \prec_w u$. (Quasi-connectedness)
% \item For all $w$ and $v$, $v \nprec_w w$. (\textbf{Weak centring})
  \item For any non-empty set of worlds $X$ and world $w$ there is a $v$ in $X$
        such that there is no $u$ in $X$ with $u \prec_w v$.
  % There are different formulations of the Limit Assumption. This is what
  % Kaufmann 2017 calls PLA.
\end{enumerate}

Asymmetric and quasi-connected relations are known as \textbf{weak orders}.
Asymmetry is self-explanatory. Quasi-connectedness is more often called
\emph{negative transitivity}, because it is equivalent to the assumption that if
$t \not< s$ and $s\not<r$ then $t\not<r$. It ensures that the ``equidistance''
relation that holds between $v$ and $u$ if neither $v \prec_w u$ nor
$u \prec_w v$ is an equivalence relation. With these two assumptions, we can
picture each world $w$ as associated with nested spheres of worlds;
$v \prec_w u$ means that $v$ is in a more narrow $w$-sphere than $u$.

Assumption 3 is known as the \textbf{Limit Assumption}. It ensures that for any
consistent proposition $A$ and world $w$, there is a set of closest $A$-worlds.
Without the Limit Assumption, there could be an infinite chain of ever closer
$A$-worlds, with no world being maximally close.

% Maybe express similarity models in terms of spheres, like neighbourhood models,
% i.e. N(w) = { { w }, { w, v }, ... }? This makes the frames easier to draw and
% to write. Students don't know how to write that two worlds are tied in terms
% of <_{w}.

\begin{exercise}
  Show that asymmetry and quasi-connectedness imply transitivity.
\end{exercise}
\begin{solution}
  Assume that $R$ is asymmetric and quasi-connected. We want to show that
  $R$ is transitive. So assume we have $xRy$ and $yRz$. By quasi-connectedness,
  $yRz$ implies that either $yRx$ or $xRz$. By asymmetry, we can't have $yRx$,
  since we have $xRy$. So $xRz$.
\end{solution}

\begin{exercise}
  Define $\preceq_{w}$ so that $v \preceq_w u$ iff $u \nprec_w v$ (that is, iff
  it is not the case that $u \prec_{w} v$). Informally, $v \preceq_w u$ means
  that $v$ is at least as similar to $w$ in the relevant respects as $u$. Many
  authors use $\preceq$ rather than $\prec$ as their basic notion. Can you
  express the above three conditions on $\prec$ in terms of $\preceq$?
\end{exercise}
\begin{solution}
  We have the following equivalences (using `$\Leftrightarrow$' to mean that the
  expressions on either side are equivalent):
  \[
    u \npreceq_w v \Leftrightarrow \neg (u \preceq_w v) \Leftrightarrow \neg(v \nprec_w u) \Leftrightarrow v \prec_w u.
  \]
  So you can simply replace every instance of $\omega \prec_{w} \nu$ in the
  conditions by $\nu \npreceq_{w} \omega$, and every instance of
  $\omega \nprec_{w} \nu$ by $\nu \preceq_{w} \omega$.

  Asymmetry thereby turns into: if $u \npreceq_{w} v$ then $v \preceq_{w} u$.
  Equivalently: either $u \preceq_{w} v$ or $v \preceq_{w} u$. This
  property of relations is called \textbf{completeness}. Notice that it
  entails reflexivity.

  Quasi-connectedness turns into: if $u \npreceq_{w} v$ then for all $t$, either
  $t \npreceq_{w} v$ or $u \npreceq_{w} t$. This is equivalent to transitivity
  for $\preceq$.

  % Weak centring turns into: for all $w$ and $v$, $w \preceq_{w} v$.

  The Limit Assumption turns into: for any non-empty set of worlds $X$ and world
  $w$ there is a $v\in X$ such that there is no $u \in X$ with
  $v \npreceq_{w} u$. Equivalently, for any non-empty set of worlds $X$
  and world $w$ there is a $v\in X$ such that $v \preceq_{w} u$ for all $u\in X$.
\end{solution}

We are going introduce a variably strict operator $\boxright$ so that
$A\boxright B$ is true at a world $w$ iff $B$ is true at the closest worlds to
$w$ at which $A$ is true. Models for a language with the $\boxright$ operator
must contain closeness orderings $\prec$ on the set of worlds.

\begin{definition}{}{similaritymodel}
  A \textbf{similarity model} consists of
  \vspace{-3mm}
  \begin{itemize*}
  \item a non-empty set $W$,
  \item for each $w$ in $W$ a weak order $\prec_w$ that satisfies the Limit
  Assumption, and
  \item a function $V$ that assigns to each sentence letter a subset of $W$.
  \end{itemize*}
  % Notice that this is a special kind of neighbourhood model: the propositions
  % necessary at w are the nested spheres around w.
\end{definition}

To formally state the semantics of $\boxright$, we can re-use a concept from
section \ref{sec:oblig-circ}. Let $S$ be an arbitrary set of worlds, and let $w$
be some world (that may or may nor be in $S$). It will be useful to have an
expression that picks out the most similar worlds to $w$, among all the worlds in
$S$. This expression is $\mathrm{Min}^{\prec_w}(S)$, which we have defined as follows in
section \ref{sec:oblig-circ}:
\[
  \mathrm{Min}^{\prec_w}(S) =_\text{def} \{ v: v \in S \land \neg\exists u (u \in S \land u \prec_w v) \}.
\]

Now $\{ u : M,u\models A \}$ is the set of worlds (in model $M$) at which $A$ is
true. So $\mathrm{Min}^{\prec_w}(\{ u : M,u\models A \})$ is the set of those
$A$-worlds that are closest to $w$. We want $A \boxright B$ to be true at $w$
iff $B$ is true at the closest $A$-worlds to $w$.

\begin{definition}{Similarity semantics for $\boxright$}{similaritysemantics}
  If $M$ is a similarity model and $w$ a world in $M$, then\\[1mm]
  $M,w \models A \boxright B$ iff $M,v \models B$ for all
  $v$ in $\mathrm{Min}^{\prec_w}(\{ u: M,u \models A \})$.
\end{definition}

You may notice that $A \boxright B$ works almost exactly like $\Ob(B/A)$ from
section \ref{sec:oblig-circ}. There, I said that for any world $w$ in any deontic
ordering model $M$,

\medskip
\quad$M,w \models \Ob (B/A) \text{ iff } M,v \models B\text{ for all $v$ in $\mathrm{Min}^{\prec_w}(\{ u: wRu $ and $M,u\models A \})$}$.

\medskip \noindent%
The main difference is that conditional obligation is sensitive to an
accessibility relation. If that relation is an equivalence relation then this
makes no difference to the logic.

% Does quasi-connectedness make a difference to the logic? I haven't assumed it
% for deontic ordering models!

Of course, the order $\prec$ in deontic ordering models is supposed to represent
degree of conformity to norms, while the order $\prec$ in similarity models
represents a certain similarity ranking in the evaluation of subjunctive
conditionals. A different type of ordering might be in play when we evaluate
indicative conditionals, which some have argued should also be interpreted as
variably strict. But again, these differences in interpretation don't affect the
logic.

Suppose we add the $\boxright$ operator to the language of standard
propositional logic. The set of sentences in this language that are true at all
worlds in all similarity models is known as \textbf{system V}. There are tree
rules and axiomatic calculi for this system, but they aren't very user-friendly.
We will only explore the system semantically.

To begin, we can check whether \emph{modus ponens} is valid for $\boxright$. That is,
we check whether the truth of $A$ and $A \boxright B$ at a world in a
similarity model entails the truth of $B$.

Assume that $A$ and $A \boxright B$ are true at a world $w$. By definition
\ref{def:similaritysemantics}, the latter means that $B$ is true at all the
closest $A$-worlds to $w$ (at all worlds in
$\mathrm{Min}^{\prec_w}(\{u: M,u\models A\})$). The world $w$ itself is an
$A$-world. If we could show that $w$ is among the closest $A$-worlds to itself
then we could infer that $A$ is true at $w$.

Without further assumptions, however, we can't show this. If we want to validate
\emph{modus ponens}, we must add a further constraint on our models: that every
world is among the closest worlds to itself. More precisely,
\[
  \text{for all worlds $w$ and $v$, $v \nprec_{w} w$.}
\]
This assumption is known as \textbf{Weak Centring}.
% It means that the closeness spheres associated with a world are centred on that
% world: every world is in the sphere of closest worlds around itself.
The logic we get if we impose this constraint is \textbf{system VC}.

\begin{exercise}
  Should we accept Weak Centring for deontic ordering models?
\end{exercise}
\begin{solution}
  No. We don't want $A$ and $\Ob(B/A)$ to entail $B$. Semantically, we don't
  want to assume that every world is among the best worlds relative to its own
  norms.
\end{solution}

\begin{exercise}
  Explain why $A \boxright B$ entails $A \to B$, assuming Weak Centring.
\end{exercise}
\begin{solution}
  Suppose $A \boxright B$ is true at some world $w$ in some model $M$. So $B$ is
  true at all the closest $A$-worlds to $w$. Now either $A$ is true at $w$ or
  $A$ is false at $w$. If $A$ is false at $w$, then $A\to B$ is true at $w$. If
  $A$ is true at $w$, then $w$ is one of the closest $A$-worlds to $w$, by Weak
  Centring; so $B$ is true at $w$; and so $A\to B$ is true at $w$. Either way,
  then, $A\to B$ is true at $w$.
 \end{solution}

\begin{exercise}
  Show that if $A$ is true at no worlds, then $A \boxright B$ is true.
  % This shows how we could define box A.
\end{exercise}
\begin{solution}
  If $A$ is true at no worlds, then $\mathrm{Min}^{\prec_w}(\{u: M,u\models A\})$
  is the empty set. So it is vacuously true that $M,v \models B$ for all
  $v \in \mathrm{Min}^{\prec_w}(\{ u: M,u \models A \})$.
\end{solution}

None of the problematic inferences (E1)--(E5) are valid if the relevant
conditionals are interpreted as variably strict. (E5), for example, would assume
that $p \boxright r $ entails $(p \land q) \boxright r$. But it does not.
We can
give a countermodel with two worlds
\begin{wrapfigure}{r}{5cm}
  \vspace{-5mm}
  \quad
  \begin{tikzpicture}[modal, world/.append style={minimum size=0.5cm}]
    \node[world] (w) [label=above:{$w$}] {$p,r$};
    \node[world] (v) [label=above:{$v$}, right=7mm of w] {$p,q$};
    \node[circle, draw=gray, minimum size=20mm] at (w) (c) {};
    \node[circle, draw=gray, minimum size=50mm] at (w) (c) {};
    % \node[circle, draw=gray, minimum size=20mm] at (v) (c) {};
  \end{tikzpicture}
  \vspace{-10mm}
\end{wrapfigure}
$w$ and $v$; $p$ is true at both worlds, $q$
is true only at $v$, and $r$ only at $w$; if $w$ is closer
to itself than $v$,
then $p \boxright r$ is true at $w$ (because the closest $p$-worlds to $w$ are
all $r$-worlds), but $(p \land q) \boxright r$ is false at $w$ (because the
closest $(p\land q)$-worlds to $w$ aren't all $r$-worlds).

The diagram on the right represents this model. The circles around $w$ depict
the similarity spheres. $w$ is closer to $w$ than $v$ because it is in the
innermost sphere around $w$, while $v$ is only in the second sphere. (If $v$
were also in the innermost sphere then the two worlds would be equally close to
$w$. That's allowed.) In general, we can represent the assumption that a world $v$
is closer to a world $w$ than a world $u$ ($v \prec_w u$) by putting $v$ is in a
closer sphere around $w$ than $u$. I have not drawn any spheres around $v$
because it doesn't matter what these look like.

% Can we turn this approach into a pictorial proof method? 1. start by putting
% the target sentence(s) into a world w. 2. Draw a sphere around w. 3. Expand
% each world as a non-modal tree. 3a. Rule for A []-> B at w: put A->B in all
% worlds outwards from w until and including worlds in spheres within which A is
% true somewhere. 3b. Rule for not(A []-> B): add a world v with A and -B, and
% make it one of the closest A-worlds. I.e., if A at w, then add v to innermost
% sphere. If not-A at w, draw a sphere around just w and put v outside it. ....?

\begin{exercise}
  Draw countermodels showing that (E1)--(E4) are invalid if the conditionals are
  translated as statements of the form $A \boxright B$. (Hint: You never need
  more than two worlds.)
\end{exercise}
\begin{solution}
    (E1) is an inference from $q$ to $p \boxright q$. To show that this is invalid, we need to give a model in which $q$ is true at some world ($w$) while $p \boxright q$ is false (at $w$). 
    
  \begin{tikzpicture}[modal, world/.append style={minimum size=0.5cm}]
    \node[world] (w) [label=above:{$w$}] {$q$};
    \node[world] (v) [label=above:{$v$}, right=7mm of w] {$p$};
    \node[circle, draw=gray, minimum size=20mm] at (w) (c) {};
    \node[circle, draw=gray, minimum size=40mm] at (w) (c) {};
  \end{tikzpicture}
  
  This model also shows that (E2) and (E3) are invalid.
  (E2) is an inference from $\neg p$ to $p \boxright q$. In the model,
  $\neg p$ is true at $w$ but $p \boxright q$ is false.
  (E3) is an inference from $\neg(p \boxright q)$ to $p$. In the model,
  $\neg(p \boxright q)$ is true at $w$ but $p$ is false.

  (E4) is an inference from $p \boxright q$ to $\neg q \boxright \neg p$.
  In the following model, the premise is true at $w$ and the conclusion false.
  
  \begin{tikzpicture}[modal, world/.append style={minimum size=0.5cm}]
    \node[world] (w) [label=above:{$w$}] {$p,q$};
    \node[world] (v) [label=above:{$v$}, right=7mm of w] {$p$};
    \node[circle, draw=gray, minimum size=20mm] at (w) (c) {};
    \node[circle, draw=gray, minimum size=40mm] at (w) (c) {};
  \end{tikzpicture}
\end{solution}

The logic of variably strict conditionals is weaker than the logic of strict
conditionals. Some have argued that it is too weak to explain our
reasoning with conditionals. It is, for example, not hard to see that the
following statements are all false. (The corresponding statements for
$\strictif$ are true; see exercise \ref{ex:sda-import}.)
\begin{enumerate}[leftmargin=10mm]
  \itemsep-1mm
\item $p \boxright q, q \boxright r \models p \boxright r$
\item $((p \lor q) \boxright r) \models (p \boxright r) \land (q \boxright r)$
\item $p \boxright (q \boxright r) \models (p \land q) \boxright r$
\end{enumerate}
If English conditionals are variably strict, this means (for example) that we can't 
infer `if $p$ then $r$' from `if $p$ then $q$' and `if $q$ then $r$'. But isn't
this a valid inference?

Well, perhaps not. Stalnaker gave the following counterexample, using cold-war
era subjunctive conditionals.
\begin{quote}
  If J.\ Edgar Hoover had been born a Russian, he would be a communist.\\
  If Hoover were a communist, he would be a traitor.\\
  Therefore, if Hoover had been born a Russian, he would be a traitor.
\end{quote}

\begin{exercise}
  Can you find a case where `if $p$ or $q$ then $r$' does not appear to entail
  `if $p$ then $r$' and `if $q$ then $r$'? You can use either indicative or
  subjunctive conditionals. (Hint: Try to find a case in which `if $p$ or $q$
  then $p$' sounds acceptable.)
\end{exercise}
\begin{solution}
  Frances has never learnt a foreign language, although she would have loved to
  learn French. If Frances had been given a choice between learning French and
  learning Italian, she would have chosen French. \emph{If Frances had learned
    French or Italian then she would have learned French.} It does not follow that
  if Frances had learned Italian then she would have learned French.

  The same style of example works for indicative conditionals.
\end{solution}

% \begin{exercise}
%   Are \emph{modus tollens} and contraposition valid for $\boxright$? That is,
%   (a) do $A$ and $A \boxright B$ always entail $B$, and (b) does
%   $A \boxright B$ always entail $\neg B \boxright \neg A$?
% \end{exercise}

The semantics I have presented for $\boxright$ is a middle ground between that
of Lewis and Stalnaker. Stalnaker assumes that $\prec_w$ is not just
quasi-connected, but connected: for any $w,v,u$, either $v \prec_w u$ or $v=u$
or $u \prec_w v$. (`$v=u$' means that $v$ and $u$ are the same world.) This
rules out ties in similarity: no sphere contains more than one world.

Stalnaker's logic (called \textbf{C2}) is stronger than Lewis's VC. The
following principle of ``Conditional Excluded Middle'' is C2-valid but not VC-valid:
%
\principle{CEM}{(A \boxright B) \lor (A \boxright \neg B)}

%  NB: Uniqueness alone doesn't entail CEM, only Uniqueness + Limit.

Whether conditionals in natural language satisfy Conditional Excluded Middle is
a matter of ongoing debate. On the one hand, it is natural think that `it is not
the case that if $p$ then $q$' entails `if $p$ then not $q$', which suggests
that the principle is valid. On the other hand, suppose I have a number of coins
in my pocket, none of which I have tossed. What would have happened if I had
tossed one of the coins? Arguably, I might have gotten heads and I might have
gotten tails. Either result is possible, but neither \emph{would} have come
about.

\begin{exercise}
  Explain why the following statements are true, for all $A,B,C$:
  \begin{exlist}
  \item $A \land B \models_{C2} A \boxright B$
  \item $A \boxright (B\lor C) \models_{C2} (A \boxright B) \lor (A \boxright C)$
  \end{exlist}
\end{exercise}
\begin{solution}
  \begin{sollist}
    \item Assume $A\land B$ is true at some world $w$ in some model $M$. By
    Centring, $w$ is among the closest $A$-worlds to $w$. By connectedness, $w$
    is the unique closest $A$-world to $w$. So $B$ is true at all closest
    $A$-worlds to $w$.
    \item Assume $A \boxright (B\lor C)$ is true at some world $w$ in some model
    $M$. So all the closest $A$-worlds to $w$ are $(B\lor C)$-worlds. If there
    are no $A$-worlds then $A \boxright B$ and $A \boxright C$ are both true. If
    there are $A$-worlds then Stalnaker's semantics implies that there is a
    unique closest $A$-world $v$ to $w$. Since $B\lor C$ is true at $v$, either $B$ or $C$ must be true at $v$. So either $B$ is true at all closest $A$-worlds to $w$ or $C$ is true at all closest $A$-worlds to $v$.
  \end{sollist}
\end{solution}

Lewis not only rejects connectedness, but also the Limit Assumption. He argued
that there might be an infinite chain of ever closer $A$-worlds. Definition
\ref{def:similaritysemantics} implies that if there are no closest $A$-worlds
then any sentence of the form $A \boxright B$ is true. That does not seem right.
Lewis therefore gives a more complicated semantics:

\begin{quote}
  $M,w \models A \boxright B$ iff either there is no $v$ for
  which $M,v\models A$ or there is some world $v$ such that
  $M,v\models A$ and for all $u \prec_w v$, $M,w \models A \to B$.
\end{quote}
%
It turns out that it makes no difference to the logic whether we impose the
Limit Assumption and use the old definition or don't impose the Limit Assumption
and use Lewis's new definition. The same sentences are valid either
way. % (Lewis 1973, p.444).

% Rini and Cresswell 58 use a 4-place relation C to evaluate >, where for any
% time t and worlds w1, w2, w3, C(t,w1,w2,w3) iff at t, w2 is closer to w1 than
% w3. I can probably fold the t into w1. RC point out that Aqvist 1973 also
% invented this kind of semantics.

\iffalse

Let's say that for any sentence $A$, a world $v$ is \emph{$A$-accessible} from
$w$ (for short, $wR_Av$) iff $v$ is one of the closest $A$-worlds to $w$; that
is, iff $v \in \mathrm{Min}^{\prec_w}(\{u: M,u \models A\})$. Definition
\ref{def:similaritysemantics} then states that $A \boxright B$ is true at a
world $w$ iff $B$ is true at all worlds $A$-accessible from $w$. These are the
standard truth-conditions for the box, except that accessibility is relativised
to the antecedent $A$.

We can therefore adapt the standard tree rules for the box to reason with
$\boxright$, as follows.

\bigskip
\begin{center}
\begin{minipage}{0.4\textwidth} \centering
\tree{
  \nnode{18}{}{$A \boxright B$}{\omega}{}\\
  \dotbelownode{18}{}{$\omega R_A v$}{}{}\\
  \\
  \nnode{18}{}{$B$}{\nu}{}\\
  \Kk[18]{0}{\color{red}$\uparrow$}\\
  \Kk[18]{0}{\color{red}\small old}
}
\end{minipage}
\begin{minipage}{0.4\textwidth}\centering
\tree{
  \dotbelownode{18}{}{$\neg (A \boxright B)$}{\omega}{}\\
  \\
  \nnode{18}{}{$\omega R_A v$}{}{}\\
  \nnode{18}{}{$\neg B$}{\nu}{}\\
  \Kk[18]{0}{\color{red}$\uparrow$}\\
  \Kk[18]{0}{\color{red}\small new}
}
\end{minipage}
\end{center}
\bigskip

To get a complete tree system, we need further rules. For example, the above two
rules don't account for the fact that the closest $A$-worlds are always
$A$-worlds. They also don't account for the fact that every $A$-world is among
the closest $A$-worlds to itself (by weak centring). We can add two more rules
to fill these gaps.

\bigskip
\begin{center}
  \begin{minipage}[t]{0.4\textwidth} \centering
    \hspace{-10mm}Truth:
    \medskip
    
\tree{
  \dotbelownode{12}{}{$\omega R_A \nu$}{}{}\\
  \\
  \nnode{12}{}{$A$}{\nu}{}\\
}
\end{minipage}
\begin{minipage}[t]{0.4\textwidth} \centering
  \hspace{-10mm}Centring:
    \medskip
  
\tree{
  \dotbelownode{12}{}{$A$}{\omega}{}\\
  \\
  \nnode{12}{}{$\omega R_A \omega$}{}{}\\
}
\end{minipage}
\end{center}
\bigskip

The resulting tree system is still not complete, because it doesn't reflect
interactions between different accessibility relations. For example, if the
worlds that are $A$-accessible from some world $w$ include $B$-worlds, then the
$A\land B$-accessible worlds from $w$ must be contained within the worlds
$A$-accessible from $w$. The complete tree rules for the $\boxright$ operator
turn out to be rather complicated. I will leave it at the above four rules,
which suffice to establish many useful facts about variably strict conditionals.

% Priest points out that for Lewis-Stalnaker we have more conditions than in the
% basic "plural selection" semantics that underlines his tableaux. In particular,
% we have: (3) if $A$ is consistent, then there is always a non-empty set of
% closest $A$-worlds; (4) if the closest $A$-worlds are a subset of the B worlds,
% and the closest B worlds are a subset of the A-worlds, then the closest A-worlds
% area the clsoest B-worlds; (5) if there are some B-worlds among the clsoest
% A-worlds, then the closest A-and-B worlds are among the closest A-wrolds. [p.92]
% This means that e.g. $A > B, B> A \models (A > C) \leftrightarrow (B > C)$ is
% valid in LS, but not provable with a tableaux. "There are presently no known
% tableau systems of the kind used in this book for S [Lewis/Stalnaker] (and its
% extensions that we will meet in the next section)". p.93. So I might mention
% that the rules are incomplete, meaning that anything that can be proved is
% indeed valid, but that caution is needed when concluding that something is
% invalid.

% Zach 2018 points out that the logic and semantics Priest discusses is due to
% Chellas; the tableau rules are Priest's. ``We give additional branch extension
% rules for Priest’s system which result in sound and complete tableau systems for
% CK and [Lewis's] VC. These systems are, however, non-analytic in that the cut
% rule is included. ... These are not the first tableau systems for Lewis’s VC .
% Gent [1992], based on work of de Swart [1983], has given a tableau system for VC
% . More recently, Negri and Sbardolini [2016] have offered a cut-free complete
% sequ ent calculus for VC . These approaches are all based on Lewis’s semantics
% based on rela tive proximity of worlds and incorporate the ordering relation
% between w orlds into the syntax.''

Here is a tree proof to show (once more) that \emph{modus ponens} is valid.

\begin{center}
  \tree[3]{
    & \nnode{18}{1.}{$A \boxright B$}{w}{(Ass.)} & \\
    & \nnode{18}{2.}{$A$}{w}{(Ass.)} & \\
    & \nnode{18}{3.}{$\neg B$}{w}{(Ass.)} & \\
    & \nnode{18}{4.}{$wR_Aw$}{}{\quad\quad(Centring)} & \\
    & \nnodeclosed{18}{5.}{$B$}{w}{(1,4)} & 
  }
\end{center}

\begin{exercise}
  Give tree proofs for the following statements.
  \begin{exlist}
  \item $A, \neg B \models \neg(A\boxright B)$ 
  \item $A\boxright B, A\boxright C \models  A \boxright (B \land C)$ % from Girle
  \item $A \boxright (B \land C) \models (A \boxright B) \land (A\boxright C)$
  \item $A\boxright \neg A \models A\boxright B$ 
  \end{exlist}
\end{exercise}

Since the tree rules I have presented are sound, you can be sure that whenever a
tree closes then the tested entailment or validity holds.  But since the rules
are not complete, care is required when a tree doesn't close. You always need to
check if a model read off from an open tree is an actual countermodel.

Constructing countermodels from open trees is at any rate not entirely
straightforward. By way of illustration, let's show that
$A\boxright B$ and $B\boxright C$ does not entail $A \boxright C$. The
tree starts like this.

\begin{center}
  \tree[3]{
    & \nnode{20}{1.}{$A \boxright B$}{w}{(Ass.)} & \\
    & \nnode{20}{2.}{$B \boxright C$}{w}{(Ass.)} & \\
    & \nnodeticked{20}{3.}{$\neg(A \boxright C)$}{w}{(Ass.)} & \\
    & \nnode{20}{4.}{$wR_Av$}{}{(3)} & \\
    & \nnode{20}{5.}{$\neg C$}{v}{(3)} & \\
    & \nnode{20}{6.}{$B$}{v}{(1,4)} & \\
    & \nnode{20}{7.}{$A$}{v}{\quad\;(4,Truth)} & 
  }
\end{center}
The Centring rule would allow us to add six more lines, but they
wouldn't be useful, so let's stop here.
% & \nnode{18}{2.}{$wR_{\neg C}w$}{}{(5,Centring)} & \\

The open tree suggests that there is a countermodel with two worlds, $w$ and
$v$. At $v$, $A$ and $B$ are true. We also have $wR_Av$, so $v$ is among the
closest $A$-worlds to $w$. Since $wR_A w$ is not in the tree, let's assume that
$w$ is not among the closest $A$-worlds to $w$, which also means that $A$ is
false at $w$. We don't have $wR_B v$ on the tree either. So even though $B$ is
true at $v$, $v$ is not among the closest $B$-worlds to $w$. We can ensure this
by assuming that $B$ is true at $w$ itself, and that $w$ is the unique closest
$B$-world from itself. Now you can verify that $A \boxright B$ and
$B \boxright C$ are both true at $w$ while $A \boxright C$ is false.

% Priest gives some rules for reading off countermodels: For the initial logic:
% ``Counter-models are read off from the tableau in a natural way. If there is
% something of the form A > B or ¬(A > B) on the branch, then RA is as the
% information about rA on the branch specifies. Otherwise, RA may be arbitrary.''
% For the full logic: ``If A does not occur as the antecedent of a conditional or
% negated conditional at a node, we can no longer allow RA to be arbitrary,
% however, since it must satisfy (1) [centring] and (2) [A is true at any
% R(A)-accessible world]. The simplest trick is to let [the set of closest
% A-worlds to w] = [A] [i.e., the set of all A-worlds] (for every w). With this
% definition, (1) and (2) are clearly satisfied.'' He adds a footnote: ``This is
% legitimate, since [the set of closest A-worlds] is not required to define the
% truth value of A at a world. To evaluate the truth value of A at a world, one
% needs to know only [the set of closest B-worlds] for those B that occur as the
% antecedents of conditionals within A.'' I don't get it. Doesn't this treat the
% conditional as strict? Ah. $\models_S \Phi$ implies $\models_C \Phi$; i.e. if
% something is valid for strict implication then it's also valid for variably
% strict implication. (Right? Might be a good exercise.) So $\not\models_C \Phi$
% implies $\not\models_S\Phi$. So any invalidity in $C$ is also an invalidity in
% $S$. And so (??) any countermodel in $C$ is also a countermodel in $S$.

\fi


\section{Restrictors}
\label{sec:restrictor-analysis}

Consider these two statements.
\begin{enumerate}[leftmargin=10mm]
  \itemsep-1mm
  \item[(1)] If it rains we always stay inside.
  \item[(2)] If it rains we sometimes stay inside.
\end{enumerate}
On its most natural reading, (1) says that we stay inside at all times at which
it rains. We can express this in $\L_{M}$, using the box as a universal
quantifier over the relevant times. (So $\Box A$ now means `always $A$'.) The
translation would be $\Box(r \to s)$.

One might expect that (2) should then be translated as $\Diamond(r \to s)$,
where the diamond is an existential quantifier over the relevant times
(`sometimes'). But $\Diamond(r \to s)$ is equivalent to
$\Diamond(\neg r \lor s)$. This is true whenever $\Diamond \neg r$ is true. (2),
however, isn't true simply because it doesn't always rain. On its most salient
reading, (2) says there are times at which it rains \emph{and} we stay inside.
Its correct translation is $\Diamond(r \land s)$.

This is a little surprising, given that (2) seems to contain a conditional. Does
the conditional here express a conjunction?

Things get worse if we look at (3).
\begin{enumerate}[leftmargin=10mm]
  \item[(3)] If it rains we usually stay inside.
\end{enumerate}
Let's introduce an operator $\Mostly$ for `usually', so that $\Mostly A$ is true
at a time iff $A$ is true at \emph{most} times. Can you translate (3)
with the help of $\Mostly$?

You can't. Neither $\Mostly(r \to s)$ nor $\Mostly(r \land s)$ capture the
intended meaning of (3). $\Mostly(r \land s)$ entails that $r$ is usually true.
But (3) doesn't entail that it usually rains. $\Mostly(r \to s)$ is true as long
as $r$ is usually false, even if we're always outside when it is raining.
You could try to bring in some of the new kinds of conditional that we've
encountered in the previous sections. How about $\Mostly(r \boxright s)$, or
$\Mostly(r \strictif s)$, or $r \boxright \Mostly s$, or
$r \strictif \Mostly s$? None of these are adequate.

The problem is that (3) doesn't say, of any particular proposition, that it is
true at most times. It doesn't say that among all times, most are such-and-such.
Rather, it says that \emph{among times at which it rains}, most times are times
at which we stay inside. The function of the `if'-clause in (3) is to
\textbf{restrict the domain} of times over which the `usually' operator
quantifies.

% We can formalize (3) if we treat $\Mostly$ as a binary operator, taking two
% propositions as arguments: $\Mostly(s/r)$. The semantics for the two forms of
% $\Mostly$ would look as follows, assuming that we are quantifying over times.

% \bigskip
% \begin{tabular}{ll}
%   $M,t \models \Mostly A$ &iff $M,s \models A$ for most times $s$.\\
%   $M,t \models \Mostly(A/B)$ &iff $M,s \models A$ for most times $s$ such that
%                                $M,s \models B$.
% \end{tabular}

Now return to (1) and (2). Suppose that here, too, the `if'-clause serves to
restrict the domain of times, so that `always' and `sometimes' only quantify
over times at which it rains. On that hypothesis, (1) says that \emph{among
  times at which it rains}, all times are times at which we stay inside, and (2)
says that \emph{among times at which it rains}, some times are times at which we
stay inside. This is indeed what (1) and (2) mean, on their most salient
interpretation.

As it turns out, `among $r$-times, all times are $s$-times' is equivalent to `all
times are not-$r$-times or $s$-times'. That's why we can formalize (1) as
$\Box(r \to s)$. `Among $r$-times, some times are $s$-times', on the other hand,
is equivalent to `some times are $r$-times and $s$-times'. That's why we can
formalize (2) as $\Diamond(r \land s)$. It would be wrong to think that the
conditional in (1) is material, the conditional in (2) is a conjunction, and the
conditional in (3) is something else altogether. A much better explanation is
that the `if'-clause in (1) does the exact same thing as in (2) and (3). In each
case, it restricts the domain of times over which the relevant operators
quantify.

We can arguably see the same effect in (4) and (5).
\begin{enumerate}[leftmargin=10mm]
  \itemsep-1mm  
  \item[(4)] If the lights are on, Ada must be in her office.
  \item[(5)] If the lights are on, Ada might be in her office.
\end{enumerate}
Letting the box express epistemic necessity, we can translate (4) as
$\Box(p \to q)$. But (5) can't be translated as $\Diamond(p \to q)$, which would
be equivalent to $\Diamond(\neg p \lor q)$. Nor can we translate (5) as
$p \to \Diamond q$, which is entailed by $\Diamond q$. It is easy to think of
scenarios in which (5) is false even though `Ada might be in her office' is
true. The correct translation of (5) is plausibly $\Diamond(p \land q)$. The
sentence is true iff there is an epistemically accessible world at which the
lights are on and Ada is in her office.

As before, we can understand what is going if we assume that the `if'-clause in
(4) and (5) functions as a restrictor. The `if'-clause restricts the domain of
worlds over which `must' and `might' quantify. (4) says that \emph{among
  epistemically possible worlds at which the lights are on}, all worlds are
worlds at which Ada is in her office. (5) says that \emph{among epistemically
  possible worlds at which the lights are on}, some worlds are worlds at which
Ada is in her office.

\begin{exercise}
  Translate `all dogs are barking' and `some dogs are barking' into the language
  of predicate logic. Can you translate `most dogs are barking' if you add a
  `most' quantifier $\Mostly$ so that $\Mostly x Fx$ is true iff most things
  satisfy $Fx$?
\end{exercise}
\begin{solution}
  `All dogs are barking': $\forall x(Dx \to Bx)$\\
  `Some dogs are barking': $\exists x(Dx \land Bx)$\\
  `Most dogs are barking' cannot be translated in terms of $\Mostly x$. We need
  a binary quantifier: $\Mostly x(Bx / Dx)$
  % (There is a strong parallel here to explicit (nominal) quantifiers in
  % natural language. We usually translate `all $F$ are $G$' as
  % $\forall x (Fx \to Gx)$, and `some $F$ are $G$' as
  % $\exists x (Fx \land Gx)$. This works, but it is a bit strange that what
  % appear to be entirely analogous constructions in English get such different
  % translations. Moreover, the approach breaks down for `most'. `Most $F$ are
  % $G$' cannot be analysed in terms of a quantifier $Mx$ that applies to
  % individual open sentences. In \emph{generalised quantifier theory}, all
  % quantifiers are instead treated as two-place functions, so that we can write
  % $\forall x (Gx/Fx)$, $\exists x (Gx/Fx)$, and $Mx (Gx/Fx)$. (Actually, these
  % are more commonly written $[\forall x: Fx]Gx$, \ldots.) Semantically, the
  % extra argument place selects the domain in which it is claimed that
  % all/some/most things are $G$.)
\end{solution}

% ``The data involving modification by only and even, and VP ellipsis phenomena
% provide strong evidence against the view that the antecedent and consequent of
% conditionals are coordinated. These data support the view that if-clauses are
% adverbials, like temporal phrases and clauses. Furthermore, pronominalization by
% then suggests that if-clauses are advervials, since their anaphoric reflex -
% then - is an adverb''. Bhatt and Pancheva 2006

The hypothesis that `if'-clauses are restrictors also sheds light on the problem
of conditional obligation.
\begin{enumerate}[leftmargin=10mm]
  \itemsep-1mm  
  \item[(6)] Jones ought to help his neighbours.
  \item[(7)] If Jones doesn't help his neighbours, he ought to not tell them that he's coming.
\end{enumerate}
In chapter \ref{ch:deontic}, we analyzed `ought' as a quantifier over the best
of the circumstantially accessible worlds. On this approach, (6) says that among
the accessible worlds, all the best ones are worlds at which Jones helps his
neighbours. Suppose the `if'-clause in (7) serves to restrict the domain of
worlds, excluding worlds at which Jones helps his neighbours. We then predict
(7) to state that \emph{among the accessible worlds at which Jones doesn't help
  his neighbours}, all the best worlds are worlds at which Jones doesn't tell
his neighbours that he's coming. This can't be expressed by combining the
monadic $\Ob$ quantifier with truth-functional connectives. Hence we had to
introduce a primitive binary operator $\Ob(\cdot/\cdot)$.

% As Richmond Thomason argued in 1981, A proper theory of conditional
% obligation \ldots will be the product of two separate components: a
% theory of the conditional and a theory of obligation.

The upshot of all this is that we can make sense of a wide range of puzzling
phenomena by assuming that `if'-clauses are restrictors. Their function is to
restrict the domain or worlds or times over which modal operators quantify.

What, then, is the purpose of `if'-clauses in ``bare'' conditionals like (8) and
(9), where there are no modal operators to restrict?
\begin{enumerate}[leftmargin=10mm]
  \itemsep-1mm
  \item[(8)] If Shakespeare didn't write \emph{Hamlet}, then someone else did.
  \item[(9)] If Shakespeare hadn't written \emph{Hamlet}, then someone else
        would have.
\end{enumerate}

Here opinions vary. One possibility, prominently defended by the linguist
Angelika Kratzer, is that even bare conditionals contain modal operators.
Arguably, `would' in (9) functions as a kind of box. If this box is a simple
quantifier over circumstantially accessible worlds, and the `if'-clause in (9)
restricts its domain, then (9) can be formalized as $\Box(p \to q)$. If, on the
other hand, `would' in (9) works more light `ought' -- if it quantifyies over the
\emph{closest} of the accessible worlds --, and the `if'-clause restricts the
domain of accessible worlds, then the resulting truth-conditions are those of
$p \boxright q$. Both the strict and the variably strict analysis of (9) are
therefore compatible with the hypothesis that `if'-clauses are restrictors.

What about (8)? This sentence really doesn't appear to contain a relevant modal.
Kratzer suggests that it contains an unpronounced epistemic `must': (8) says
that if Shakespeare didn't write \emph{Hamlet} then someone else \emph{must}
have written \emph{Hamlet}. Assuming that the `if'-clause restricts the domain
of this operator, bare indicative conditionals would be equivalent to
strict epistemic conditionals.

\begin{exercise}
  Suppose bare indicative conditionals like (8) contain a box operator $\Box$
  whose accessibility relation relates each world to itself and to no other
  world. (This is a redundant operator insofar as $\Box A$ is equivalent to $A$.)
  Assume the `if'-clause restricts the domain of that operator. What are the
  resulting truth-conditions of (8)?
\end{exercise}
\begin{solution}
  On this proposal, bare indicative conditionals like (8) are material
  conditionals. If $p$ is true and $q$ is false then there is an accessible
  $p$-world at which $q$ is false, and so $q$ is not true at all accessible
  worlds at which $p$ is true. In all other cases, $q$ \emph{is} true at all
  accessible worlds at which $p$ is true.
\end{solution}

\begin{exercise}
  Besides ``would counterfactuals'' there are also ``might counterfactuals'' like
  \begin{enumerate}[leftmargin=10mm]
    \item[(10)] If I had played the lottery, I might have won.
  \end{enumerate}
  Suppose `might' is the dual of `would', and suppose the `if'-clause in (10)
  restricts the domain of worlds over which `might' quantifies. It follows that
  `if $A$ then might $B$' is true iff $B$ holds at some of the
  closest/accessible $A$-worlds. (`Closest' or `accessible' depending on how we
  understand the `would'/`might' operators.) Can you see why this casts doubt on
  the validity of Conditional Excluded Middle?
\end{exercise}
\begin{solution}
  Conditional Excluded Middle is valid iff there is never more than one
  closest/accessible $A$-world. On that assumption, `some closest/accessible
  $A$-world is a $B$-world' entails `all closest/accessible $A$-worlds are
  $B$-worlds'. But (10) does not entail `If I had played the lottery, I would
  have won'.
\end{solution}


\iffalse

We don't have a versatile restrictor device in $\L_{M}$. For many purposes it
isn't needed. We can use $\Box(A \to B)$ or $\Diamond(A \land B)$ to express
restricted necessity or possibility claims. Could we also add a conditional
operator $>$ that serves as a general restrictor?

Informally, we want $A > B$ to be true at a world $w$ in a model $M$ iff $B$ is
true at $w$ in a model $M'$ that is like $M$ except that all not-$A$ worlds are
removed. To achieve this, let's make truth relative to an added parameter $U$,
the ``universe of live possibilities''. The truth-functional connectives behave
in the usual way. We have the following clauses for sentence letters, the box,
and the restrictor conditional.

\begin{definition}{Restrictor semantics for $>$}{restrictorsemantics}
  If $\Mfr = \t{W,R,V}$ is a Kripke model, $w$ is a member of $W$, $A$ is a
  sentence letter, and $B,C$ are $\L_M$-sentences, then \medskip
  \begin{tabular}{lll}
    (a) & $M,U,w \models A$ &iff $w$ is in $V(A)$.\\
    (b) & $M,U,w \models \neg B$ &iff $M,U,w\not\models B$.\\
    (c) & $M,U,w \models A \to B$ &iff $M,U,w\not\models B$ or $M,U,w \models C.\\
    % $M,U,w \models A$ iff $M,U,w \models A$ and $w \in V(A)$;\\
    % without this, 'if p q' is equivalent to 'q'.
    (d) & $M,U,w \models \Box B$ &iff $M,v \models B$ for all $v$ in $U$ such that $wRv$.\\
    (e) & $M,U,w \models B > C$ &iff $M,U,w \not\models B$ or $M,\{ v \in U: M,U,v \models B \},w \models C$.
    % "At all worlds, blanking out the A worlds, B" should mean the same as "at all A-worlds, B". So "blanking out the A worlds, B" must be true at w iff w is either a not-A world or a B-world. I.e., w |= A>B iff w |= A->B ?! We say w |= A>B iff [A],w |= B. This should be vacuously true if w is not in [A]. Can we ensure in the compositional semantics that U,w occurs on the LHS of |= only if w \in U? Sure. The condition is trivially satisfied for U=W. So we only need to adjust the clause for >: instead of saying that U,w |= A>B iff U \cap [A],w |= B, we say that U,w |= A>B iff either w \not\in [A] or U \cap [A],w |= B. Equivalently, U,w |= A>B iff U \cap [A],w |= A->B. It might be better to treat w \in U as a presupposition. -- All of that is problematic if we want A > []B to be non-vacuous if A is false! "Blanking out non-A worlds, all worlds are B-worlds" should not be vacuous.
  \end{tabular}
\end{definition}
%
Clause (d) says that $\Box
B$ is true at $w$ iff all accessible worlds among the live possibilities $U$ are $B$-worlds. Clause (e) says that $B
>
C$ is true at $w$ iff restricting the live possibilities $U$ to $B$-worlds renders $C$ true at $w$, provided $w$ is itself a $B$-world; if it is not, then $B
> C$ counts as vacuously true at $w$. 

We define truth relative to a world and a model by stipulating that for any sentence $A$, $M,w
\models A$ iff $M,W,w \models A$.

It is easy to see that $p > \Box q$ is equivalent to $\Box(p \to q)$, in the
sense that $M,w \models p > \Box q$ iff $M,w \models \Box(p \to q)$. Both
sentences are true at a world $w$ iff $q$ is true at all $w$-accessible worlds
at which $p$ is true.
% check limit case: if there's no (accessible) p-world then p > []A is true iff
% A is true at all v in the empty set; so p > []A is vacuously true, just like
% [](p->q).
In the same way, $p > \Diamond q$ is equivalent to $\Diamond(p \land q)$.
% limit case: if there's no (accessible) p-world then p > <>A is true iff A is
% true at some v in the empty set, which is never the case.

If we add the restrictor $>$ to an ordering model, where $\Box A$ quantifies
over the minimal $U$-worlds (or the minimal accessible $U$-worlds), $p > \Box q$
becomes equivalent to $p \boxright q$. In deontic ordering models, this, in
turn, is equivalent to $\Ob(q/p)$.

% What happens in A > <><>B? The restrictor account suggests that this means
% <>(A & <>B). My semantics makes it equivalent to <><>(A & B).

% Covic and Egre discuss a model in which $R$ is shifted: $w,R \models \Box(A/B)$
% iff $w,R' \models \Box A$ where $R'(w)$ is $R(w)$ restricted by the truth-set of
% $B$ relative to $R,\preceq$. Seems easier with modal bases: the if-clause is
% added to the modal base.

\fi

%%% Local Variables: 
%%% mode: latex
%%% TeX-master: "logic2.tex"
%%% End:

\chapter{Towards Modal Predicate Logic}\label{ch:qml}

% Should I do this more like week9.org? Move the whole "modal logic as fragment"
% section into an exercise?

\section{Predicate logic recap}

In these last two chapters, we are going to add the resources of first-order
predicate logic to those of propositional modal logic. Let's begin by reviewing
the syntax and semantics of classical, non-modal predicate logic.

The language $\L_P$ of first-order predicate logic consists of \emph{predicates}
$F^{0},F^1,F^2,\ldots,$ $G^{0},G^1,G^2,\ldots$, \emph{individual constants} (or
\emph{names}) $a,b,c,\ldots$, \emph{individual variables} $x,y,z,\ldots$, the
logical symbols $\neg$, $\land$, $\lor$, $\to$, $\leftrightarrow$, $\forall$,
$\exists$, and the parentheses $($ and $)$. Individual variables and constants
are also called \emph{(singular) terms}.

Atomic sentences of $\L_{P}$ are formed by conjoining a predicate with zero or
more terms. Each predicate takes a fixed number of terms, as indicated by its
numerical superscript: $F^1$ is a \emph{one-place} predicate that combines with
one term to form a sentence, $F^2$ is \emph{two-place}, and so on. In practice,
we usually omit the superscripts, because context makes clear what kind of
predicate is in play. $Fa \lor Gab$, for example, is well-formed only if $F$ is
one-place and $G$ two-place.

In English, a predicate is what is what you get when you remove all names from a
sentence. Removing `Bob' from `Bob is hungry' yields the predicate `-- is
hungry'. From `Bob is in Rome', we get the two-place predicate '-- is in --'.
From `Bob saw Carol's father in Jerusalem', we could get the
three-place-predicate '-- saw --'s father in --'. When we translate from
English, we normally translate English names into $\L_P$-names and (logically
simple) English predicates into $\L_P$-predicates. `Bob is in Rome' might become
$Fab$, where $a$ translates `Bob', $b$ `Rome', and $F$ `-- is in --'.

From atomic sentences, complex sentences are formed in the usual way by means of
the truth-functional operators $\neg$, $\land$, $\lor$, $\to$,
$\leftrightarrow$.

Another way to construct a complex sentence from a simpler sentence is to add a
quantifier in front of the simpler sentence. A \emph{quantifier} is an
expression of the form $\forall \chi$ or $\exists \chi$, where $\chi$ is some
variable. A quantifier is said to \emph{bind} the variable it contains:
$\forall x$ binds $x$, $\exists y$ binds $y$, and so on.

In English, quantifier expressions are usually restricted to a particular
subclass of the things under discussion: `\emph{all whales} are mammals',
`\emph{some students} went home'. The $\L_{P}$-quantifiers $\forall x$ and
$\exists x$ are unrestricted. They roughly correspond to `everything is such
that \ldots' and `something is such that \ldots'. We can translate restricted
quantifiers by combining unrestricted quantifiers with truth-functional
connectives. `All whales are mammals' is equivalent to `Everything is either not
a whale or a mammal'; so it can be translated as $\forall x(Wx \to Mx)$. `Some
students went home' could be translated as $\exists x (Sx \land Hx)$.

Variables are book-keeping devices. They function somewhat like pronouns in
English. $\exists x(Sx \land Hx)$ might be read as `something is such that
\emph{it} is a student and \emph{it} went home'. By using different variables
($x,y,z,\ldots$), we can disambiguate statements with nested quantifiers.
Consider
\begin{quote}
  Every dog barked at a tree.
\end{quote}
This can mean that there is a particular tree at which all the dogs barked, but
it can also mean that each dog found some tree to bark at -- possibly different
trees for different dogs. The first reading could be translated as
\[
  \exists y (Ty \land \forall x (Dx \to Bxy)),
\]
the second as
\[
  \forall x (Dx \to \exists y(Ty \land Bxy)).
\]

Some more terminology. Recall that the \emph{scope} of an operator (token) in a
sentence is the shortest well-formed subsentence in which it occurs. In
$\exists y (Ty \land \forall x(Fx \to Bxy))$, the scope of the quantifier
$\forall x$ is the subsentence $\forall x(Fx \to Bxy)$. If an occurrence of a
variable lies in the scope of a quantifier that binds the variable, then the
occurrence is called \emph{bound}, otherwise it is \emph{free}. In
$\forall x(Fx \to Bxy)$, all occurrences of $x$ are bound, but $y$ is free.

A sentence containing free variables is called \emph{open}. Sentences that aren't
open are \emph{closed}. Intuitively, only closed sentences make complete
statements. For this reason, some authors reserve the word `sentence' for closed
sentences, referring to open sentences as `formulas'. (Others call every
$\L_P$-sentence a `formula'.)

\begin{exercise}
  Translate the following sentences into $\L_P$.
  \begin{exlist}
  \item Keren and Keziah are sisters of Jemima.
  \item All myriapods are oviparous.
  \item Fred has a new car.
  \item Not every student loves logic.
  \item Every student who loves logic loves something.
  \end{exlist}
\end{exercise}
\begin{solution}
  \begin{sollist}
  \item $Srj \land Skj$; \qquad $r$: Keren, $k$: Keziah, $j$: Jemima, $S$: -- is a sister of --
  \item $\forall x (Mx \to Ox)$; \qquad $M$: -- is a myriapod, $O$: -- is oviparous 
  \item $\exists x (Cx \land Nx \land Hfx)$; \qquad $f$: Fred, $C$: -- is a car, $N$: -- is new, $H$: -- has --
  \item $\neg\forall x (Sx \to Lxl)$; \qquad $l$: logic; $S$: -- is a student, $L$: -- loves --
  \item $\forall x ((Sx \land Lxl) \to \exists y Lxy)$; \qquad $l$: logic; $S$: -- is a student, $L$: -- loves --
  \end{sollist}
\end{solution}

Like sentences of modal propositional logic, sentences of predicate logic are
interpreted relative to a model. A model of predicate logic first of all
specifies an \emph{individual domain} $D$ over which the quantifiers are said to
range. If we read $\forall x$ as `everything is such that' and $\exists x$ as
`something is such that' then the relevant ``somethings'' are the members of the
domain $D$.

The remainder of a model is an \emph{interpretation function $V$} that assigns
\begin{itemize}[leftmargin=10mm]
  \itemsep-1mm
  \item[(a)] to each name a member of $D$,
  \item[(b)] to each zero-place predicate a truth-value,
  \item[(c)] to each one-place predicate a subset of $D$, and
  \item[(d)] to each $n$-place predicate with $n>1$ a set of $n$-tuples from
        $D$.
\end{itemize}
An ``$n$-tuple from $D$'' is simply a list of length $n$, all elements of which
are in $D$. Repetitions are allowed, so if Bob is a member of $D$, then
$\t{\text{Bob, Bob}}$ counts as a 2-tuple from $D$. (2-tuples are more commonly
called \emph{pairs}.) We can subsume condition (c) under condition (d) by
assuming that a 1-tuple from $D$ is a member of $D$. We can subsume (b) under
(d) by identifying the truth-value False with the empty tuple $\emptyset$ and
the truth-value True with $\{ \emptyset \}$. (Don't worry if you find this confusing or objectionable. We won't be using zero-ary predicates.)

% V(p) = 1 means V(p) is true of all zero-tuples

\begin{definition}{}{predicatemodel}
  A \textbf{(classical) first-order model} is a pair $\t{D,V}$ consisting of
  \vspace{-3mm}
  \begin{itemize*}
    \item a non-empty set $D$, and
    \item a function $V$ that assigns to each name a member of $D$ and to each
    $n$-place predicate a set of $n$-tuples from $D$.
  \end{itemize*}
\end{definition}

As always, the purpose of a model is to represent a conceivable scenario together
with an interpretation of the non-logical vocabulary. The non-logical vocabulary
of $\L_P$ are the names and predicates, which is why these are interpreted by
$V$. 

We assume that in any relevant scenario there are some things we want to talk
about; these things are represented by the domain. The members of $D$ are often
called \emph{individuals}, but this should not be taken to imply anything about
their nature. An individual might be a rock, a person, a symphony, a sentence, a
number, or a possible world. Every $\L_P$-name is assumed to pick out one of
these individuals. (Different names can pick out the same individual, and there
can be individuals that aren't picked out by any name.)

Intuitively, a predicate expresses a property or relation that may be
instantiated by the individuals in the domain. In order to determine the
truth-value of a sentence like $Fa$ or $\exists x Fx$ in a given scenario,
however, we only need to know which individuals in the domain have the property
expressed by $F$. Similarly, to determine the truth-value of sentences like
$Rab$ or $\forall x \exists y Rxy$, we only need to know which pairs of
individuals stand in the relation expressed by $R$. That's why the
interpretation function in a first-order model simply assigns sets of
individuals or $n$-tuples of individuals to predicates. $Fa$ is true in a given
model iff the individual assigned to $a$ (in the model) is a member of the set
assigned to $F$; that is, iff $V(a) \in V(F)$. Likewise, $Rab$ is true in a
model iff the pair of individuals assigned to $a$ and $b$ -- the pair
$\t{V(a),V(b)}$ -- is in the set assigned to $R$.

In this way, the truth-value of every closed atomic sentences is determined. For
truth-functionally complex sentences, the standard rules apply: a negated
sentence $\neg A$ is true iff the corresponding sentence $A$ is not true;
$A \land B$ is true iff $A$ and $B$ are both true; and so on.

When we turn to quantified sentences, we face a problem. We can't define the
truth-value of $\forall x Fx$ in terms of the truth-value of $Fx$, because an
open sentence like $Fx$ doesn't have a truth-value. Interpretation functions
interpret names and predicates; they say nothing about variables. Even if we
changed this and said that $x$ should also be interpreted as picking out a
member of the domain, we would have to ignore this interpretation when we
evaluate $\forall x Fx$. We want $\forall x Fx$ to be true iff $Fx$ is true
\emph{no matter which individual is assigned to $x$}. We therefore define truth
not just relative to a model, but relative to a model \emph{and an assignment of
  individuals to variables}.

To illustrate, consider a model with just two individuals, Alice and Bob, which
are picked out by the names $a$ and $b$ respectively. Let $V(F)$ be the set $\{$
Alice $\}$, a set that only contains Alice. So $Fa$ is true and $Fb$ false. The
sentence $Fx$ is neither true nor false, for the variable $x$ does not refer to
any particular individual. All we can say is that $Fx$ is ``true of'' Alice and
``false of'' Bob. That is, $Fx$ is true if we assign Alice to $x$ and false if
we assign Bob to $x$. $\exists x Fx$ is true because there is an individual
(Alice) of which $Fx$ is true. Equivalently, $\exists x Fx$ is true because
there is some assignment of individuals to variables relative to which $Fx$ is
true. $\forall x Fx$ is false because it is not the case that every assignment
of individuals to variables renders $Fx$ true.

So we'll define truth relative to a model $M = \t{D,V}$ and a variable
assignment $g$. A \emph{variable assignment} is a function that maps variables
to members of $D$. When we have nested quantifiers, as in
$\forall x \exists y Gxy$, we need to consider variable assignments that differ
from other assignments with respect to a particular variable.
$\forall x \exists y Gxy$ is true iff, no matter what individual is assigned to
$x$, there is some assignment of an individual to $y$ (but holding fixed the
assignment to $x$) that makes $Gxy$ true. Equivalently: $\forall x\exists y Gxy$
is true iff for every variable assignment $g$, there is some variable assignment
$g'$ that differs from $g$ at most in what it assigns to $y$ such that $Gxy$ is
true relative to $g'$.

Let's say that (for any variable $\chi$) a variable assignment $g'$ is an
\emph{$\chi$-variant} of a variable assignment $g$ iff $g'$ differs from $g$ at
most in the value it assigns to $\chi$. Let's also introduce $[\tau]^{M,g}$ as
shorthand for the individual picked out by a term $\tau$ in a model
$M = \t{D,V}$ relative to assignment $g$:
\[
  [\tau]^{M,g} =_\text{def} \begin{cases} \;V(\tau) & \text{ if $\tau$ is a name}\\
    \;g(\tau) & \text{ if $\tau$ is a variable}.
  \end{cases}
\]
This is a compact way of saying that (1) for any variable $\chi$, $[\chi]^{M,g}$
is the individual assigned to $\chi$ by $g$, and (2) for any name $\eta$,
$[\eta]^{M,g}$ is the individual assigned to $\eta$ by the interpretation function
of $M$.

Now we can state the standard semantics of first-order predicate logic.
(`$M,g \models A$' is pronounced `$A$ is true in $M$ relative to $g$').

\begin{definition}{Semantics of first-order predicate logic}{predicatesemantics}
  If $M = \t{D,V}$ is a first-order model, $\phi^{n}$ is an $n$-place predicate
  (for $n\geq 0$), $\tau_1,\ldots,\tau_n$ are terms, $\chi$ is a variable, and
  $g$ is a variable assignment, then
  
  \medskip
  \begin{tabular}{lll}
    (a) & $M,g \models \phi^{n} \tau_1\ldots \tau_n$ &iff $\t{[\tau_1]^{M,g},\ldots,[\tau_n]^{M,g}} \in V(\phi)$.\\
    % (b) & $M,g \models \tau_1=\tau_2$ &iff $[\tau_1]^{M,g} = [\tau_2]^{M,g}$.\\
    (b) & $M,g \models \neg A$ &iff $M,g \not\models A$.\\
    (c) & $M,g \models A \land B$ &iff $M,g \models A$ and $M,g \models B$.\\
    (d) & $M,g \models A \lor B$ &iff $M,g \models A$ or $M,g \models B$.\\
    (e) & $M,g \models A \to B$ &iff $M,g \not\models A$ or $M,g \models B$.\\
    (f) & $M,g \models A \leftrightarrow B$ &iff $M,g \models A\to B$ and $M,g \models B\to A$.\\
    (g) & $M,g \models \forall \chi A$ &iff $M,g' \models A$ for all $\chi$-variants $g'$ of $g$.\\
    (h) & $M,g \models \exists \chi A$ &iff $M,g' \models A$ for some $\chi$-variant $g'$ of $g$.
  \end{tabular}
\end{definition}

Clause (a) says that, for example, $Fa$ is true in a model $M$ relative to an
assignment $g$ iff in that model, the predicate $F$ applies to the individual
picked out by $a$. Clauses (b)-(f) say that the truth-functional operators are
interpreted in the standard fashion. Clauses (g) and (h) tell us how quantified
sentences are interpreted. $\exists x Fx$, for example, is true relative to $M$
and $g$ iff $Fx$ is true relative to some assignment function $g'$ that differs
from $g$ at most in what it assigns to $x$.

Definition \ref{def:predicatesemantics} settles the truth-value of every
$\L_P$-sentence in every (first-order) model, relative to any assignment
function.

We can also define a concept of truth relative to a model, without reference to
an assignment function. Let's say that an $\L_P$-sentence is \textbf{true in a
  model} $M$ iff it is true in $M$ relative to \emph{every} assignment function
$g$ for $M$.

Finally, we say that an $\L_P$-sentence is \textbf{valid} (in classical
first-order logic) iff it is true in all (classical, first-order)
models. Equivalently: An $\L_P$ sentence is valid iff it is true in all models
relative to all assignment functions.

On the present definition, $Fx \to Fx$ is valid, even though it does not make a
complete statement, due to the free variable $x$. To avoid this, many authors
restrict the concept of validity to closed sentences.

% We will do so below, because our tree rules don't allow proving Fx v ~Fx. But
% we don't officially commit to it here. The problem is that if we don't define
% validity for open sentences then modal logic isn't a fragment of predicate
% logic: Fx v \neg Fx should be valid.

\begin{exercise}
  Define a first-order model in which $\exists x Fx \to \forall x Fx$ is
  false. Demonstrate that the sentence is false in your model by applying all
  relevant clauses from definition \ref{def:predicatesemantics}.
\end{exercise}
\begin{solution}
  Let the model $M$ be given by $D = \{ \text{Rome}, \text{Paris} \}$ and
  $V(F) = \{ \text{Rome} \}$. By clause (a) of definition
  \ref{def:predicatesemantics}, $M,g' \models Fx$ holds for every assignment
  function $g'$ that maps $x$ to Rome, because then $g'(x) \in V(F)$. By clause
  (h) it follows that $M,g \models \exists x Fx$ for every assignment function
  $g$. By clause (a) again, $M, g' \not\models Fx$ for every assignment function
  $g'$ that maps $x$ to Paris. By clause (g), it follows that
  $M,g \not\models \forall x Fx$ for every assignment function $g$. So
  $\exists x Fx$ is true (in $M$) relative to every assignment function while
  $\forall x Fx$ is false relative to every assignment function. By clause (e)
  it follows that $\exists x Fx \to \forall x Fx$ is false in $M$ relative to
  every assignment function.
\end{solution}

\begin{exercise}
  The definition of truth in a model uses the method of supervaluation
  that we met in section \ref{sec:branching}. Give examples to illustrate the
  following claims.
  
  \begin{exlist}
  \item If a sentence $A$ is not true in a model, it does not follow that
    $\neg A$ is true in the model.
  \item A disjunction $A \lor B$ can be true in a model even though neither $A$
    nor $B$ is true in the model. 
  \end{exlist}
\end{exercise}
\begin{solution}
  For both cases, use $Fx$ as the sentence $A$, and $\neg Fx$ as $B$, and
  consider a model in which $F$ applies to some but not to all individuals. Both
  $Fx$ and $\neg Fx$ are then true relative to some assignment functions and
  false relative to others. So neither sentence is true in the model. But
  $Fx \lor \neg Fx$ is true relative to every assignment function.
\end{solution}

\section{Modal fragments of predicate logic}
\label{sec:fragment}

Much of the power and complexity of predicate logic comes from its ability to
handle nested quantifiers with different variables. For some applications, these
complexities aren't needed, and we can simplify the semantics.

Consider a fragment $\L_P^1$ of $\L_P$ with only one variable $x$, no names, and
only one-place predicates. In $\L_P^1$, we have sentences like $Fx$,
$\forall x Gx$, $\forall x \exists x (Fx \to Gx)$, but not $Fa$ or
$\forall x \exists y(Fx \to Gy)$.

Following definition \ref{def:predicatemodel}, a model for $\L_P^1$ consists of
a non-empty set $D$ and an interpretation function $V$ that assigns to each
predicate a subset of $D$. That is, for $\L_{P}^{1}$ definition
\ref{def:predicatemodel} can be simplified as follows:

\begin{justabox}
  A \textbf{model of $\L_P^1$} is a pair $\t{D,V}$
  consisting of \vspace{-1mm}
  \begin{itemize*}
  \item a non-empty set $D$, and
  \item a function $V$ that assigns to every $\L_P^1$-predicate a subset of $D$.
  \end{itemize*}
\end{justabox}

We can also simplify definition \ref{def:predicatesemantics}. Since $\L_P^1$ has
only one variable $x$, an assignment function for $\L_P^1$ only needs to tell us
which individual in $D$ is picked out by $x$. So we can represent an entire
assignment function for $\L_P^1$ by a member of $D$. This leaves us with the
following semantics.

\begin{justabox}
  If $M = \t{D,V}$ is a model for $\L_P^1$, $d$ is a member of $D$, and $\phi$
  is an $\L_{P}^{1}$-predicate, then
  
  \medskip
  \begin{tabular}{lll}
    (a) & $M,d \models \phi x$ &iff $d \in V(\phi)$.\\
    (b) & $M,d \models \neg A$ &iff $M,d \not\models A$.\\
    (c) & $M,d \models A \land B$ &iff $M,d \models A$ and $M,d \models B$.\\
    (d) & $M,d \models A \lor B$ &iff $M,d \models A$ or $M,d \models B$.\\
    (e) & $M,d \models A \to B$ &iff $M,d \not\models A$ or $M,d \models B$.\\
    (f) & $M,d \models A \leftrightarrow B$ &iff $M,d \models A\to B$ and $M,d \models B\to A$.\\
    (g) & $M,d \models \forall x A$ &iff $M,d' \models A$ for all $d' \in D$.\\
    (h) & $M,d \models \exists x A$ &iff $M,d' \models A$ for some $d' \in D$.
  \end{tabular}
\end{justabox}

These definitions look a lot like definitions \ref{def:basicmodel} and
\ref{def:basicsemantics} from chapter \ref{ch:worlds}. The only difference is
that the sentence letters from chapter \ref{ch:worlds} are now called predicates
and written in uppercase, the box is written $\forall x$, the diamond
$\exists x$, and we always append the letter $x$ to sentence letters: we write
$\forall x Fx$, not $\forall x F$. But it doesn't really matter how a symbol is
called or how it is written.

The upshot is that propositional modal logic, interpreted as in chapter
\ref{ch:worlds}, can be regarded as a disguised \emph{fragment of first-order
  predicate logic}. The sentence letters of $\L_M$ are disguised (one-place)
predicates, the box and the diamond are disguised quantifiers. If we adopted the
orthographic convention to write the box as $\forall x$, the diamond as
$\exists x$, and to always append the letter $x$ to (capitalised) sentence
letters, $\L_M$ would look just like $\L_P^1$, and it would have the same
semantics.

If we use chapter \ref{ch:accessibility}'s Kripke semantics rather than the
simple semantics from chapter \ref{ch:worlds} to interpret $\L_{M}$, we get a
different fragment of first-order predicate logic. The box and the diamond are
still disguised quantifiers, but this time they are restricted by the
accessibility relation. We could drop the disguise by writing $\Box p$ as
$\forall y(Rxy \to Py)$ and $\Diamond p$ as $\exists y(Rxy \land Py)$. The
fragment of $\L_P$ that now corresponds to $\L_M$-sentences has two variables
$x$ and $y$ and one two-place predicate `$R$' in addition to the one-place
predicates; it no longer has unrestricted quantifiers.

What's the point of the disguise? Why didn't we write boxes and diamonds as
$\L_P$-quantifiers all along? There are several reasons.

One is that we often use the box and the diamond to formalize pre-theoretic
concepts of which it is not obvious that they can be understood as a quantifiers
over worlds. Some hold that the correct semantics for obligation and permission,
for example, is not Kripke semantics, but neighbourhood semantics. The language
of modal propositional logic is neutral on this disagreement. Or think of
provability logic, where the box formalizes mathematical provability. As it
turns out, one can give a Kripke semantics for provability, but nobody thinks
this somehow reveals what provability really means. In provability logic,
$\Box A$ means that $A$ is derivable from the axioms and rules of (say) ZFC; it
would not be illuminating to write this as $\forall y(Rxy \to Ay)$.

One might also argue that the syntax of modal logic conveniently resembles the
surface form of English statements that we may want to formalize. In `Bob knows
that it is raining', for example, the object of Bob's knowledge is specified by
`it is raining'. It seems appropriate to formalize the sentence in terms of an
operator $\Kn$ that applies to a sentence, $p$. If we ``dropped the disguise'',
the formalization would be $\forall y(Rxy \to Py)$. The sentence `it is raining'
would have to be translated by a predicate $P$ -- a predicate that applies to
all and only the worlds at which it is raining.

There is a deeper point here. Sentences of modal logic are interpreted \emph{at
  a world} in a model. Modal logic looks at models ``from the inside'', from the
perspective of a particular world. Predicate logic, by contrast, describes
models ``from the outside'', from a God's eye perspective. If we want to say
that a particular individual has a property $P$ in predicate logic, we need to
pick out that individual among all the elements of the domain, perhaps by a
name. We can then say $Pa$. In modal logic, we can simply say $p$ to express
that the internal point from which we're looking at the model has the relevant
property.

For many applications, this internal perspective is very natural. When we think
about what is possible or about what the future will bring, our thinking takes
place at a particular time, in a particular world. We are looking at the
structure of times and worlds from the inside. When I say that it is raining, I
mean that it is raining \emph{here and now}, in \emph{this world}. I don't need
to pick out the relevant time and place and world from a God's eye perspective.
I can pick them out simply as the time and place and world at which I currently
find myself.

There are other, more pragmatic reasons to use the modal language $\L_M$ rather
than $\L_P$. The language of boxes and diamonds is simpler than the language of
first-order predicate logic. It has a simpler syntax, a simpler semantics, and
allows for simpler proofs. For almost all the conceptions of validity we have
studied (K-validity, S4-validity, etc.), there are efficient mechanical
procedures to determine whether an arbitrary $\L_M$-sentence is valid or
invalid, By contrast, there is no mechanical procedure at all to determine, for
an arbitrary $\L_P$-sentence, whether it is valid or invalid.

You may wonder how this is possible given that $\L_M$-sentence are just
$\L_P$-sentences in disguise. The reason is that while every $\L_M$-sentence is
a disguised $\L_P$-sentence, not every $\L_P$-sentence can be disguised as an
$\L_M$-sentence. There are many things one can say in $\L_P$ that can't be said
in $\L_M$. The $\L_{P}$-sentence $\forall x Rxx$, for example, states that $R$
is reflexive. No sentence of $\L_M$ has this meaning: there is no
$\L_{M}$-sentence that is true at a world in a model iff the model's
accessibility relation is reflexive.

That's why modal propositional logic, interpreted as in chapter \ref{ch:worlds}
or \ref{ch:accessibility}, is a disguised \emph{fragment} of predicate logic. It
is a simple and computationally attractive fragment that takes an ``internal''
perspective on models.

\begin{exercise}
  Since $\Box A \to A$ corresponds to reflexivity, one might think that
  $\Box p \to p$ is true at a world in a model iff the model's accessibility
  relation is reflexive. (a) Explain why this is not correct. (b) Can you also
  show that there is no $\L_{M}$-sentence that is true at a world in a model iff
  the model's accessibility is reflexive?
\end{exercise}
\begin{solution}
  There are many non-reflexive models in which $\Box p \to p$ is true at some
  world -- for example, any non-reflexive model in which $p$ is false at all
  worlds.

  For the more general question, let $M_1$ be a model with a single world that
  can see itself. Let $M_{2}$ be a model with two worlds, each of which can see
  the other but not itself. In both models, all sentence letters are false at
  all worlds. The very same $\L_M$-sentences are true at all worlds in these
  models (as a simple proof by induction shows). But the first model is
  reflexive and the second isn't. So there is no $\L_{M}$-question that is true
  at a world in a model iff the model's accessibility relation is reflexive.
\end{solution}

% \begin{exercise}
%   Can you give another example of an $\L_P$-sentence (about a Kripke model) for
%   which there is no equivalent $\L_M$-sentence?
% \end{exercise}
% \begin{solution}
%   For example: $\forall w \neg Rww$ states that $R$ is reflexive; as the answer
%   to the previous exercise shows, this can't be expressed in $\L_M$. 
% \end{solution}

% \begin{exercise}
%   standard translation? 
% \end{exercise}


\section{Predicate logic proofs}

If we want to know whether a given $\L_{P}$-sentence is valid or invalid, we
could in principle work through definition \ref{def:predicatesemantics}. Various
proof systems for classical predicate logic offer a more streamlined approach.

Let's look at the tree method for classical predicate logic. Suppose we want to
test whether $\exists x(Fx \land Gx) \to \exists x Fx$ is valid. As always, we
start the tree with the negation of the target sentence:

\medskip
\begin{center}
  \tree{
    \nnode{26}{1.}{$\neg(\exists x (Fx \land Gx) \to \exists x Fx)$}{}{(Ass.)}
  }
\end{center}
\medskip%
There is no world label because we're not doing modal logic. Next, we apply the
standard rule for negated conditionals:%
\medskip
\begin{center}
  \tree{
    \nnode{26}{2.}{$\exists x (Fx \land Gx)$}{}{(1)}\\
    \nnode{26}{3.}{$\neg\exists x Fx$}{}{(1)}
  }
\end{center}

\medskip\noindent
Node 2 says that $Fx \land Gx$ is true of some individual. To
expand this node, we introduce a new name $a$ for that individual, and infer
$Fa \land Ga$.
\medskip
\begin{center}
  \tree{
    \nnode{26}{4.}{$Fa \land Ga$}{}{(2)}\\
  }
\end{center}

\medskip\noindent%
We expand the conjunction on node 4.
\medskip
\begin{center}
  \tree{
    \nnode{26}{5.}{$Fa$}{}{(4)}\\
    \nnode{26}{6.}{$Ga$}{}{(4)}
  }
\end{center}
Next, we expand node 3, which says that $Fx$ is true of nothing. In particular
then, $Fx$ can't be true of $a$. So we add $\neg Fa$:

\medskip
\begin{center} \tree{
    \nnodeclosed{26}{7.}{$\neg Fa$}{}{(3)}
  }
\end{center}

\medskip\noindent%
The tree is closed because the sentence on node 7 is the negation of the
sentence on node 5. The target sentence is valid.

To state the general rules, we need some more notation. If $A$ is a sentence,
$\chi$ is a variable, and $\eta$ is a name, let $A[\eta/\chi]$ be the sentence
obtained from $A$ by replacing all free occurrences of $\chi$ with $\eta$. So
$Fx[a/x]$ is $Fa$, but $\forall x Fx[a/x]$ is $\forall x Fx$ because this
sentence contains no free occurrences of $x$.

The general rule for expanding nodes of type $\exists \chi A$ is that you add a
node $A[\eta/\chi]$, where $\eta$ is a ``new'' name that does not already occur
on the relevant branch. If this node has been added to every open branch below
$\exists \chi A$ then the $\exists \chi A$ node can be ticked off.
$\forall \chi A$ nodes can be expanded multiple times, once for each ``old''
name. So if $\forall x A$ occurs on a branch, and the branch contains the names
$a$ and $b$ then we can add both $A[a/x]$ and $A[b/x]$. If there is no old name
on a branch, we are allowed to expand $\forall \chi A$ with a new name.
$\forall \chi A$ nodes are never ticked off.

Here is a summary of the quantifier rules; `old or first' means that the
relevant name either already occurs on the branch or it is introduced as the
first name on the branch.

\bigskip

\begin{minipage}{0.24\textwidth} \centering
\tree{
  \dotbelownode{12}{}{$\forall \chi A$}{}{}\\
  \\
  \nnode{12}{}{$A[\eta/\chi]$}{}{}\\
  \Kk[-1]{0}{\color{red}$\uparrow$}\\
  \Kk[0]{0}{\color{red}\small old or first}
}
\end{minipage}
\begin{minipage}{0.24\textwidth}\centering
\tree{
  \dotbelownode{12}{}{$\exists \chi A$}{}{}\\
  \\
  \nnode{12}{}{$A[\eta/\chi]$}{}{}\\
  \Kk[-1]{0}{\color{red}$\uparrow$}\\
  \Kk[0]{0}{\color{red}\small new}
}
\end{minipage}
\begin{minipage}{0.24\textwidth}\centering
\tree{
  \dotbelownode{12}{}{$\neg \forall \chi  A$}{}{}\\
  \\
  \nnode{12}{}{$\neg A[\eta/\chi]$}{}{}\\
  \Kk[0]{0}{\color{red}$\uparrow$}\\
  \Kk[1]{0}{\color{red}\small new}
}
\end{minipage}
\begin{minipage}{0.24\textwidth} \centering
\tree{
  \dotbelownode{12}{}{$\neg \exists \chi A$}{}{}\\
  \\
  \nnode{12}{}{$\neg A[\eta/\chi]$}{}{}\\
  \Kk[0]{0}{\color{red}$\uparrow$}\\
  \Kk[1]{0}{\color{red}\small old or first}
}
\end{minipage}

\medskip

\begin{exercise}
  Give tree proofs for the following sentences.
  \begin{exlist}
  \item $\forall x Fx \to Fa$
  \item $\forall x (Fx \to Gx) \to (\forall x Fx \to \forall x Gx)$
  \item $\forall x (Fx \land Gx) \leftrightarrow (\forall x Fx \land \forall x Gx)$
  \item $\exists x\forall y Gxy \to \forall y \exists x Gxy$
  \item $\exists y \forall x(Fy \to Fx)$
  \end{exlist}
\end{exercise}
\begin{solution}
  Use \href{https://www.umsu.de/trees/}{umsu.de/trees/}.
\end{solution}

There are also axiomatic calculi for predicate logic. We can, for example, use
the following axiom schemas:
%
\begin{principles}
  \pri{$\forall\exists$}{\neg\exists \chi A \leftrightarrow \forall \chi \neg A}\\
  \pri{UI}{\forall \chi A \to A[\eta/\chi]}\\
  \pri{DI}{\forall \chi (A \to B) \to (A \to \forall \chi B),\text{ if $\chi$ is not free in $A$}}
\end{principles}
%
To these we would add the following rules. As in earlier chapters, $\Gamma \models_{P} A$ means that $A$ is a truth-functional consequence of (the sentences in) $\Gamma$.
\begin{principles}
  \pri{CPL}{\text{If }\Gamma \models_{P} A\text{ and all members of }\Gamma\text{ are on a proof, then one may add $A$.}}\\
  \pri{Gen}{\text{If $A$ occurs on a proof, then one may add $\forall \chi A[\eta/\chi]$.}}
\end{principles}

% These are based on Bostock, pp.220f.. Bostock doesn't have the duality rule
% because he takes \exists as defined. His GEN rule says that if |- A then |-
% \forall\chi A[\chi/\nu]. I.e., it converts provable closed sentences into
% quantifies sentences. His UI rule (A4) says \forall \chi A \to A[\nu/\chi],
% i.e. it doesn't allow deriving (x)Fx -> Fx. He mentions that if open sentences
% are permitted then my version must be used. I'm not sure why we need the
% substitution in GEN. Suppose we can prove the closed sentence phi(a).
% Bostock's GEN allows inferring (x)phi(x). If in every such case we can also
% prove phi(x) then my simpler rule (Gen) suffices. I think all my axioms and
% rules are neutral between names and variables, insofar as any proof ending
% with phi(a) can be converted into a proof of phi(x) by simply replacing all
% occurrences of a with x (for some x). For how could the name have been
% introduced? Either by (CPL), e.g. through the tautology Fa v ~Fa, in which
% case one could just as well use a variable, or by (UI), where again one can
% just as well use a variable. All inferences from sentences with the name can
% still be made if the name is replaced by a variable.

The calculus is sound and complete: everything that can be proved is valid, and
every valid (closed) sentence can be proved. The above tree rules are also sound
and complete.

\begin{exercise}
  The completeness proof for first-order trees (like the proof in chapter
  \ref{ch:proofs}) shows that if a sentence is valid then any fully expanded
  tree for that sentence will close, provided the tree rules are applied in a
  sensible order. Why doesn't this contradict the claim I made in the previous
  section. that there is no mechanical procedure to determine, for an arbitrary
  $\L_P$-sentence, whether the sentence is valid? (Tree proofs count as
  ``mechanical'', so that's not the problem.)
\end{exercise}
\begin{solution}
  If a sentence is valid (in first-order predicate logic) then a fully expanded
  tree for the sentence will close and show that the sentence is valid. But if a
  sentence is not valid, the tree might grow forever. There is no algorithm for
  detecting whether a tree will grow forever. 
\end{solution}


\section{Modality de dicto and de re}

% A student: "Page 179 of the Modal Predicate Logic handout. I don't
% understand the de re sentence Ex◇Fx. ◇Fx means that there is some x
% at v which is F, that's fine. But what does the existential
% qualifier mean in this context? That there is some x at world w that
% can see Fx at v? If so, surely all x at w can see v, in which case
% any existential qualifier followed by a modal operator would be
% either Ex or Vx?"

We are now ready to add boxes and diamonds to the language of first-order
predicate logic. This gives us the \textbf{standard language of first-order
  modal logic}, or $\L_{M\!P}$. The sentences of $\L_{M\!P}$ are defined as
follows.
%
\begin{enumerate}[leftmargin=9mm]
  \itemsep-1mm
  \item An $n$-place predicate followed by $n$ terms is an $\L_{M\!P}$-sentence.
  \item If $A$ is an $\L_{M\!P}$-sentence, then so are $\neg A$, $\Diamond A$,
        and $\Box A$.
  \item If $A$ and $B$ are $\L_{M\!P}$-sentences, then so are $(A \land B)$,
        $(A \lor B)$, $(A \to B)$ and $(A \leftrightarrow B)$.
  \item If $A$ is an $\L_{M\!P}$-sentence and $\chi$ is a variable, then
        $\forall \chi A$ and $\exists \chi A$ are $\L_{M\!P}$-sentence.
  \item Nothing else is an $\L_{M\!P}$-sentence.
\end{enumerate}

We continue to interpret the box and the diamond as (disguised) quantifiers. So
$\L_{M\!P}$ effectively has two kinds of quantifiers: overt quantifiers of the
form $\forall \chi$ and $\exists \chi$, and the disguised quantifiers $\Box$ and
$\Diamond$. This is only useful if the two kinds of quantifiers range over
different things. In applications of modal predicate logic, the box and the
diamond usually range over possible worlds or times, while the overt
quantifiers range over things like people, rocks, ghosts, etc., which are
assumed to inhabit the worlds or times.

To illustrate, consider the following inference, in which I've written the box as `$\Kn$'.

\bigskip
\begin{tabular}{ll}
  Bob knows that all humans are mortal. & $\Kn\forall x (Hx \to Mx)$\\
  Socrates is human. & $Hs$\\
  Therefore: Socrates is mortal. & $Ms$
\end{tabular}
\bigskip

\noindent%
The knowledge operator $\Kn$ is a quantifier over the worlds compatible with
Bob's (implicit) knowledge. $\Kn\forall x (Hx \to Mx)$ says that
$\forall x (Hx \to Mx)$ is true at every world compatible with Bob's knowledge.
$\forall x (Hx \to Mx)$ is assumed to quantify not over worlds, but over things
that exist relative to a world. $\forall x (Hx \to Mx)$ is true at a world $w$
iff $Hx \to Mx$ is true of every inhabitant of $w$, meaning that every
inhabitant of $w$ is either not human or mortal. The inference is valid because
the accessibility relation for knowledge is reflexive.

Imagine a lottery. Let's read the box as `it is certain that' and $W$ as `-- is
a winning ticket'. Can you see what is expressed by the following two
statements?
\begin{enumerate}[leftmargin=14mm]
  \itemsep-1mm
  \item[(1)] $\Box \exists x Wx$
  \item[(2)] $\exists x \Box Wx$
\end{enumerate}
%
(1) says that it is certain that some ticket wins: at every epistemically
accessible world there is a winning ticket. (2) says that there is a particular
ticket of which we are sure that it will win: there is an individual such that
at every epistemically accessible world, \emph{it} is the winning ticket.
(2) is only true if we know which ticket is the (or a) winning ticket.

Sentences like $\exists x \Box Wx$ are called \textbf{de re}, Latin for `of a
thing'. Intuitively, $\exists x \Box Wx$ assert \emph{of} a particular ticket
that it has a modal property, namely the property of being the certain winner.
By contrast, $\Box \exists x Fx$, merely states that the proposition (Latin,
\emph{dictum}) $\exists x Fx$ is certain. Sentences like this are called
\textbf{de dicto}.

In general, an $\L_{M\!P}$-sentence is \emph{de re} whenever it contains a
variable that is free in the scope of some modal operator. To determine
whether a sentence $A$ is \emph{de re}, first identify all subsentences of $A$
that constitute the scope of a modal operator. (In $\exists x \Box Wx$, there is
one such subsentence: $\Box Wx$.) Next, check if at least one of these
subsentences contains a free variable. ($\Box Wx$ contains the free variable
$x$.) If yes, the sentence $A$ is \emph{de re}.

If a sentence contains a modal operator and is not \emph{de re}, then it is
\emph{de dicto}. So $\forall x (Fx \to \Box Gx)$ and
$\exists y\Box (\forall x Fx \to Fy)$ are \emph{de re}, but
$\Box \forall x Fx \to Fa$ is \emph{de dicto}. $\forall x Fx \to Fa$ is neither
\emph{de dicto} nor \emph{de re}, because it isn't modal.

There is no consensus on how to classify sentences like $\Box Fa$ that contain
a name, but no free variable, in the scope of a modal operator. One might argue
that $\Box Fa$ is \emph{de dicto} because it attributes a modal status -- say,
necessity -- to the proposition $Fa$. But one might also interpret the sentence
as attributing a modal property to the individual $a$: the property of being
necessarily $F$. The sentence should then be classified as \emph{de re}. Which
of these two perspectives is more adequate depends on the precise semantics of
$\L_{M\!P}$. We therefore have to postpone the question until the next chapter,
where we will consider some options for developing a semantics of $\L_{M\!P}$.

% I don't actually return to the question, do I?

Many natural-language sentences are ambiguous between a \emph{de re} reading and
a \emph{de dicto} reading. Consider `something necessarily exists'. This can
mean either that there is an object which could not have failed to exist
($\exists x \Box Ex$); but it can also mean that it is necessary that something
or other exists ($\Box \exists x Ex$). The first reading is \emph{de re}, the
second \emph{de dicto}.

\begin{exercise}
  Translate the following sentences into modal predicate logic. (Some of them
  are ambiguous.)
  \begin{exlist}
  \item John must be hungry.
  \item Anyone who is a cyclist must have legs.
  \item Every day might be our last.
  \item If anyone wants to leave early, they should do so quietly.
  \item Everyone who bought a ticket is allowed to enter.
  % \item One day, all those who are rich will be poor.
  \end{exlist}
\end{exercise}
\begin{solution}

  \begin{sollist}
    \item $\Box Fa$ \\ $a$: John, $F$: -- is hungry.\\[1mm]
    (Might be classified as either \emph{de re} or \emph{de dicto}.)\\[-2mm]
    
  \item $\Box \forall x(Fx \to Gx)$ \\ $F$: -- is a cyclist, $G$: -- has legs.\\[1mm]
    This is \emph{de dicto}. Also correct (but different in meaning) is the \emph{de re} translation $\forall x (Fx \to \Box Gx)$.
    Close but incorrect (and \emph{de re}): $\forall x \Box(Fx \to Gx)$.\\[-2mm] 
    
  \item $\forall x (Fx \to \Diamond Gx)$ \\ $F$: -- is a day, $G$: -- is our last day.\\[1mm]
    This is \emph{de re}. The English sentence could also be understood \emph{de dicto}, as 
    $\Diamond \forall x (Fx \to Gx)$, but that would be a very strange
    thing to say.\\[-2mm]
    
  \item %If anyone wants to leave early, they should do so quietly.
    $\forall x \Ob(Fx \to Gx)$\\
    $F$: -- wants to leave early, $G$: -- leaves quietly.\\[1mm]
    Even better, if we can use the conditional obligation operator: $\forall x \Ob(Gx / Fx)$. These aren't too far off either:
    $\forall x (Fx \to \Ob Gx)$, $\Ob\forall x(Fx \to Gx)$ .

    All of these are \emph{de re}. \\[-2mm]
    
  \item % Everyone who bought a ticket is allowed to enter.

    $\forall x (\exists y (Fy \land Hxy) \to \Pe Gx)$\\
    $F$: -- is a ticket, $G$: -- enters, $H$: -- bought --.\\[1mm]
    Perhaps even better: $\forall x \Pe(Gx/ \exists y (Fy \land Hxy))$.
    Both of these are \emph{de re}.\\[1mm]
    You could translate `bought a ticket' as a simple predicate here;
    you could also use a temporal operator to account for the past
    tense of `bought' (but it's confusing to use two different kinds
    of `$\tP$' in one sentence).
    \\[-2mm]

   % \item $\tF \forall x (Fx \to Gx$)\\
   %  $F$: -- is rich, $G$: -- is poor.\\[1mm]
   %  Assuming that `rich' and `poor' are incompatible, this is equivalent to $\tF \forall x \neg Fx$.\\
   %  The more natural reading of the sentence can only be captured with the `now' operator:
   %  $\tF \forall x (\tN Fx \to Gx)$.\\[1mm]
   %  Both translations are \emph{de re}.
  \end{sollist}
\end{solution}

\begin{exercise}
  Which of your translations from the previous exercise are \emph{de
    re} and which are \emph{de dicto}?
\end{exercise}
\begin{solution}
  See the previous answer.
\end{solution}

On some interpretations of the modal operators, one may question whether
\emph{de re} sentences are intelligible. Suppose we interpret the box as `it is
analytic that' or `it is provable that'. The things that are analytic or
provable are sentences or propositions. That 2+2=4, for example, is provable in
ZFC, and `all vixens are female foxes' is analytic in English. (Remember that a
sentence is analytic if it is true in virtue of its meaning.) It is not clear
what it could mean to say that something is provable or analytic \emph{of} a
particular thing.

To illustrate the problem, let's introduce the name `Julius' for whoever
invented the zip. The sentence `Julius invented the zip' is analytic. (In fact,
`Julius invented the zip' entails that someone invented the zip, which is not
analytic. We should really use `If anyone invented the zip, then Julius invented
the zip'. Let's ignore this complication.) But is it analytic \emph{of} the
person who invented the zip that they invented the zip? The problem is that this
person has multiple names, and depending on which name we plug into the schema
`--- invented the zip', we sometimes get an analytic truth and sometimes not.
For `Julius', the sentence is analytic; for whatever name the inventor of the
zip was given by his or her parents, the sentence is not analytic.

This kind of worry was prominently raised by W.V.O.\ Quine in the 1940s.  It has
since faded, mostly because philosophers have turned their attention away from
analyticity to other interpretations of the box for which the problem is thought
not to arise. But we will return to the matter in section \ref{sec:twi}.

% factoid: ``(Ex)[]A is not equivalent to any de dicto sentence even in
% S5+BF and so not in any weaker system.'' Cresswell 149.

\section{Identity and descriptions}
\label{sec:identity}

In applications of modal and non-modal predicate logic, it is often useful to
have a special predicate for identity. Let's assume that $\L_P$ and $\L_{M\!P}$
have the two-place predicate `='. The identity predicate is conventionally
placed between its two arguments: we write `$a=b$', not `$=\!ab$'. We also write
`$a\not=b$' instead of `$\neg(a\!=\!b)$'.

Unlike the other predicates of $\L_P$ and $\L_{M\!P}$, the identity predicate
counts as a logical symbol. Its meaning is held fixed. In any model, $a=b$
means that the individual picked out by $a$ is the very same thing as the
individual picked out by $b$. This is reflected by the following clause, which
we add to the semantics of predicate logic:
\[
  M,g \models \tau_1\!=\!\tau_2\text{ \; iff }[\tau_1]^{M,g} = [\tau_2]^{M,g}.
\]

It is easy to see that the sentence $a=a$ is now valid, because $a$ and $a$ are
guaranteed to pick out the same individual. More interestingly, since the
function of a name in classical predicate logic is just to pick out an
individual, it never matters which of two names we use if they pick out the same
individual. That is, if $a=b$ is true, then replacing some or all occurrences of
$a$ in a sentence with $b$ never affects whether that sentence is true. This
principle is known as \textbf{Leibniz' Law}.

To reflect these facts, the tree method for (non-modal) predicate logic must be
extended by two new rules. First, if $\eta$ is an ``old'' name (that already
occurs on a branch) then we can always add a node $\eta=\eta$ to the branch.
Second, if an identity statement $\eta_1=\eta_2$ occurs on a branch, and some
sentence $A$ on the branch contains $\eta_1$, then we may add a new node with
the same sentence $A$ except that one or more occurrences of $\eta_1$ in $A$ are
replaced by $\eta_2$, or one or more occurrences of $\eta_2$ by $\eta_1$. Let
$A[\eta_2//\eta_1]$ stand for any sentence that results from $A$ by replacing
one or more occurrences of $\eta_1$ by $\eta_2$. The new rules can then be
summarized as follows.

\bigskip
\begin{center}
  \begin{minipage}[t]{0.3\textwidth} \centering

    Self-Identity
    
    \tree{
      \dotbelowbarenode{}\\
      \\
      \barenode{$\eta=\eta$}\\
      \Kk[0]{0}{\color{red}$\uparrow$\hspace{6mm}}\\
      \Kk[0]{0}{\color{red}\small old\hspace{6mm}}
    }
  \end{minipage}
  \begin{minipage}[t]{0.3\textwidth} \centering
    Leibniz' Law
    \bigskip

    \tree{
      \nnode{20}{}{$\eta_1=\eta_2$}{}{}\\
      \dotbelownode{20}{}{$A$}{}{}\\
      \\
      \nnode{20}{}{$A[\eta_2//\eta_1]$}{}{}
    }
  \end{minipage}
  \begin{minipage}[t]{0.3\textwidth} \centering
    Leibniz' Law
    \bigskip

    \tree{
      \nnode{20}{}{$\eta_1=\eta_2$}{}{}\\
      \dotbelownode{20}{}{$A$}{}{}\\
      \\
      \nnode{20}{}{$A[\eta_1//\eta_2]$}{}{}
    }
  \end{minipage}
  
\end{center}


\bigskip

Here is a tree for $(Raa \land a\!=\!b) \to Rab$, using Leibniz's Law.

\medskip
\begin{center}
  \tree{
    \nnode{30}{1.}{$\neg((Raa \land a\!=\!b) \to Rab)$}{}{(Ass.)}\\
    \nnode{30}{2.}{$Raa \land a\!=\!b$}{}{(1)}\\
    \nnode{30}{3.}{$\neg Rab$}{}{(1)}\\
    \nnode{30}{4.}{$Raa$}{}{(2)}\\
    \nnode{30}{5.}{$a\!=\!b$}{}{(2)}\\
    \nnodeclosed{30}{6.}{$Rab$}{}{\quad(4, 5, LL)}
  }
\end{center}

\begin{exercise}
  Use the tree method to check which of the following sentences are valid.
  \begin{exlist}
  \item $\forall x (x\!=\!x)$
  \item $\forall x \forall y(x\!=\!y \to y\!=\!x)$
  \item $(a=b \land b=c) \to a=c$
  \item $Rab \to \forall x(x=a \leftrightarrow Rxb)$
  \item $\forall x \forall y\forall z(x\not= y \land y\not= z \to x \not= z)$
  \end{exlist}
\end{exercise}
\begin{solution}
  Use \href{https://www.umsu.de/trees/}{umsu.de/trees/}.
\end{solution}

\begin{exercise}
  Show that the second version of the Leibniz' Law rule is redundant: we could
  reach $A[\eta_1//\eta_2]$ from $\eta_1=\eta_2$ and $A$ with the other rules.
\end{exercise}
\begin{solution}
  We assume that some branch on a tree contains nodes $b=c$ and $A$. We have to
  show that we can add $A[b//c]$ without using the second version of Leibniz'
  Law.
  \tree{
    \nnode{15}{k.}{$b=c$}{}{}\\
    \nnode{15}{n.}{$A$}{}{}\\ % e.g. Fc v Hcb, want to get Fb v Hcb
    \nnode{15}{m.}{$b=b$}{}{(SI)}\\
    \nnode{15}{m+1.}{$c=b$}{}{\hspace{32mm}(k, m, LL (first version))}\\
    \nnode{15}{m+2.}{$A[b//c]$}{}{\hspace{32mm}(m+1, n, LL (first version))} }
\end{solution}

In the axiomatic approach, the two facts about identity are often represented by
the following axiom schemas:
%
\begin{principles}
  \pri{SI}{\eta=\eta}\\
  \pri{LL}{\eta_1=\eta_2 \to (A \to A[\eta_2//\eta_1])}
\end{principles}

Once we add boxes and diamonds to the language of predicate logic, the seemingly
harmless axioms and rules for identity become problematic. Consider the
following inference:
%
\begin{quote}
  It is analytic that Julius invented the zip.\\
  Julius = Whitcomb L.\ Judson.\\
  Therefore: It is analytic that Whitcomb L.\ Judson invented the zip.
\end{quote}
%
The conclusion clearly doesn't follow from the premises, but the inference seems
to be licensed by Leibniz's law. Another well-known example:
%
\begin{quote}
  Lois Lane believes that Superman can fly.\\
  Superman = Clark Kent.\\
  Therefore: Lois Lane believes that Clark Kent can fly.
\end{quote}

\begin{exercise}
  (a) Give an axiomatic proof of $\Box \exists x\, x=a$, using (SI), (UI), (CPL),
  ($\forall\exists$), (CPL), and (Nec), in this order. (b) Can you see why we might
  not want to count $\Box\exists x\, x=a$ as a logical truth in some
  applications of modal logic? At which point do you think the proof goes wrong?
\end{exercise}
\begin{solution}
  \begin{sollist}
    
    \item 
\begin{align*}
   1. \quad & a=a &&\text{ (SI)}\\
   2. \quad & \forall x\, x\not= a \to a\not= a &&\text{ (UI)}\\
   3. \quad & \neg \forall x\, x\not=a &&\text{ (1, 2, CPL)}\\
   4. \quad & \neg \exists x\, x=a \leftrightarrow \forall x \, x\not=a &&\text{ ($\forall\exists$)}\\
   5. \quad & \exists x\, x=a &&\text{ (3, 4, CPL)}\\
   6. \quad & \Box\exists x\, x=a &&\text{ (5, Nec)}
\end{align*}

\item There are many correct answers. For example: historians debate whether
Homer ever existed. If $a$ translates `Homer' then $\exists x\, x=a$ is arguably false if Homer isn't a real person. Since the available evidence is compatible with  $\neg \exists x\, x=a$, the sentence $\Box\exists x\, x=a$ is false on an epistemic interpretation of the box.

Where does the proof go wrong? Each of steps 1, 2, and 6 might be blamed.
\end{sollist}
\end{solution}


We will return to these issues in section \ref{sec:twi}. In the remainder of the
present section, I want to highlight some other things we can do with the
identity predicate, apart from making claims about identity.

You have already encountered one other use in earlier chapters. Suppose we want
to express that some relation $R$ is connected, meaning that for any two things,
either the first is $R$-related to the second or the second is $R$-related to
the first. This can't be expressed without an identity predicate. With an
identity predicate, it is easy:
\[
  \forall x\forall y(Rxy \lor x\!=\!y \lor Ryx).
\]

We can also use identity to express numerical quantifiers. For example, we can
express `there are at least two $F$s' as
\[
  \exists x(Fx \land \exists y(Fy \land x\not=y)).
\]
`There is exactly one $F$' can be expressed as
\[
  \exists x(Fx \land \forall y(Fy \to x\!=\!y)).
\]

\begin{exercise}
  Can you express the following in $\L_P$ with identity?
  \begin{exlist}
  \item There are exactly two $F$s.
  \item There are no more than three $F$s.
  \end{exlist}
\end{exercise}
\begin{solution}
  \begin{sollist}
  \item $\exists x \exists y(Fx \land Fy \land x\not=y \land \forall z(Fz \to (z=x \lor z=y)))$
  \item $\forall x \forall y\forall z\forall v(Fx \land Fy \land Fz \land Fv \to (x=y \lor x=z \lor x=v \lor y=z \lor y=v \lor z=v))$
  \end{sollist}
\end{solution}

Another important use of the identity predicate is to formalise statements
involving definite descriptions. A definite description is a complex noun
phrase, typically of the form `the $F$', that purports to pick out a particular
object. `The current Prime Minister', `the highest mountain in Scotland', and
`Carol's father' are definite descriptions.

The standard language of predicate logic does not have a definite article
(`the'). The only way to pick out an individual in $\L_P$ is by a name. But
there are good reasons not to translate descriptions as names.

One reason is that we would thereby miss logical connections between
descriptions and predicates. `The current Prime Minister is not Prime Minister'
is a logical contradiction, but this can't be brought out if we translate `the
current Prime Minister' as a simple name.

% Another reason not to translate descriptions as names is that descriptions can
% fail to refer. Since France is a republic, `the present King of France' does not
% refer to anybody. But names in classical predicate logic always have a
% reference: they pick out some member of the domain $D$. --- Well, names can also
% fail to refer.

Another reason not to translate descriptions as names is that descriptions often
give rise to a \emph{de re}/\emph{de dicto} ambiguity. Consider the following
sentence:
\begin{quote}
  The Pope might have been Italian.
\end{quote}
This has two readings. It can mean either that the actual Pope, Jorge Mario
Bergoglio, might have been Italian (\emph{de re}). Alternatively, it can mean
that the following might have been the case: some Italian person is Pope
(\emph{de dicto}). There is no way to account for these two readings in
$\L_{M\!P}$ if we translate `the Pope' as a name. 

A better translation for statements involving definite descriptions was proposed
by Bertrand Russell in 1905. Russell argued that a statement of the form `the
$F$ is $G$' is true just in case there is exactly one (relevant) $F$, and
this one $F$ is also $G$. If we have an identity predicate, we can easily
express this in the language of predicate logic:
\[
  \exists x(Fx \land \forall y(Fy \to x\!=\!y) \land Gx). 
\]

Following Russell, we might translate `The current Prime Minister is not Prime
Minister' as
\[
  \exists x(Px \land \forall y(Py \to x\!=\!y) \land \neg Px).
\]
This is indeed a contradiction: it is true in no model.

We can also account for the two readings of `the Pope might have been Italian'.
The \emph{de re} reading is
\[
  \exists x (Px \land \forall y(Py \to x\!=\!y) \land \Diamond Ix).
\]
The \emph{de dicto} reading is
\[
  \Diamond \exists x (Px \land \forall y(Py \to x\!=\!y) \land Ix).
\]

% \begin{exercise}
%   People sometimes translate DDs with a term-form iota. Why is that problematic?
% \end{exercise}

\begin{exercise}
  Give two translations for each of the following sentences, one \emph{de re} and one \emph{de dicto}.
  \begin{exlist}
  \item Hillary Clinton might have been the 45th US President.
  \item Smith's murderer could have been a woman.
  \item Alice believes that the student representative is rude.
  \end{exlist}
\end{exercise}
\begin{solution}
  The \emph{de dicto} reading of (a) can be translated as
  \begin{equation*}
    \Diamond \exists x (Px \land \forall y(Py \to x\!=\!y) \land x\!=\!c),%
  \end{equation*}
  where `$P$' translates `-- is 45th US President' and `$c$' denotes Hillary
  Clinton. The \emph{de re} reading can be translated as
  \begin{equation*}
    \exists x (Px \land \forall y(Py \to x\!=\!y) \land \Diamond x\!=\!c).
  \end{equation*}
  The answers to (b) and (c) are analogous.
\end{solution}



%%% Local Variables: 
%%% mode: latex
%%% TeX-master: "ml.tex"
%%% End:

\chapter{Semantics for Modal Predicate Logic}\label{ch:qml2}

\section{Constant domain semantics}\label{sec:constantdomainsemantics}

We have met the language $\L_{M\!P}$ of (first-order) modal predicate logic. It
is time to think about how this language should be interpreted. This will tell
us which sentences and inferences in the language are valid.

As in modal propositional logic, we will assume that the box and the diamond are
quantifiers over accessible worlds, where ``accessibility'' is a placeholder
whose meaning depends on the application. If we want to reason about knowledge,
a world $v$ might be accessible from a world $w$ iff $v$ is compatible with what
is known at $w$. If we're interested in metaphysical modality then a world $v$
might be accessible from a world $w$ iff it is compatible with the nature of
things at $w$. Here we might, for example, read $\Diamond Fa$ as saying that
Aristotle could have been a sailor, assuming that $a$ picks out Aristotle and
$F$ the property of being a sailor.

Our topic in logic is not whether a particular claim about Aristotle is true. We
want to know which statements are \emph{logically true} or \emph{valid}, meaning
that they are true in any conceivable scenario, under any interpretation of the
non-logical expressions (but holding fixed the meaning of the modal operators).

As always, we use models to represent a scenario together with an interpretation
of the non-logical vocabulary. A model for $\L_{M\!P}$ contains just enough
information about a scenario and an interpretation to determine, for every
$\L_{M\!P}$-sentence and every world, whether the sentence is true at that
world.

The non-logical vocabulary of $\L_{M\!P}$ are the names and the predicates (with
the exception of the identity predicate `='). Let's assume, for now, that the
purpose of a name is simply to pick out an individual. Intuitively, a predicate
picks out a property or relation. In non-modal predicate logic, we could
represent these properties or relations by their extension -- by the sets of
individuals (or tuples of individuals) to which they apply. In modal predicate
logic, however, we typically want to allow for scenarios in which an individual
has different properties at different worlds. In one world, Aristotle might be a
sailor, in another he might be a shoemaker. If $F$ expresses the property of
being a sailor, then the set of individuals to whom $F$ applies will differ from
world to world. To determine the truth-value of $Fa$ at a world, we need to know
to which individuals $F$ applies \emph{at that world}. A model's interpretation
function will therefore assign a set of (tuples of) individuals to each
predicate \emph{relative to each world}.

Consider a model with two worlds $w$ and $v$. Both worlds, let's assume, are
accessible from $w$ and neither is accessible from $v$. The model's
interpretation function tells us that the name $a$ picks out, say, Aristotle. It
also tells us that the predicate $F$ applies to Aristotle and Boethius at $w$
and only to Boethius at $v$. We can write this as follows:
%
\begin{quote}
  $V(a) = \text{Aristotle}$\\
  $V(F,w) = \{ \text{Aristotle, Boethius} \}$\\
  $V(F,v) = \{ \text{Boethius} \}$
\end{quote}
%
\noindent%
We don't know what property is expressed by $F$, nor which properties Aristotle
and Boethius have at $w$ and $v$. Nonetheless, we can figure out that $Fa$ is
true at $w$, because the predicate $F$ applies to Aristotle at $w$. We can also
figure out that $Fa$ is false at $v$, and that $\Box Fa$ is false at $w$.

To determine the truth-value of arbitrary $\L_{M\!P}$-sentences, we need some
more information. As it stands, we can't tell whether (say) $\forall x Fx$ is
true at $w$. Informally, $\forall x Fx$ says that every individual is $F$. We
know that Aristotle and Boethius are $F$ at $w$. But we don't know if there are
other individuals besides Aristotle and Boethius. If yes, then $\forall x Fx$ is
false at $w$. If no, the sentence is true. We therefore assume that a model for
$\L_{M\!P}$ also specifies a domain of individuals.

\begin{definition}{}{VDM}
  A \textbf{constant-domain Kripke model} for $\L_{M\!P}$ is a structure $M$
  consisting of%
  \medskip
  \begin{compactenum}
    \item a non-empty set $W$ (the ``worlds''),
    \item a binary (``accessibility'') relation $R$ on $W$,
    \item a non-empty set $D$ (of ``individuals''), and
    \item an interpretation function $V$ that assigns%
    \vspace{-1mm}
    \begin{itemize}
      \itemsep-1mm
      \item to each $\L_{M\!P}$-name a member of $D$, and
      \item to each $n$-place predicate of $\L_{M\!P}$ and world $w \in W$ a set of
            $n$-tuples from $D$.
    \end{itemize}
  \end{compactenum}
\end{definition}

Models of this type are called ``constant-domain models'' because the domain of
individuals is the same for each world. This may seem questionable -- and we are
soon going to question it -- but it simplifies the semantics. Let’s stick
with it for the moment.

Having defined a concept of a model, we can lay down the rules that determine
whether any given $\L_{M\!P}$-sentence is true at a world in a model.

In fact, truth will be defined relative to three parameters: a model, a world,
and an assignment function. The assignment function plays the same role as in
non-modal predicate logic. $\forall x \Diamond Fx$, for example, is true at a
world $w$ in a model iff there is some assignment of an individual to $x$ that
renders $\Diamond Fx$ true at $w$. We continue to use $[\tau]^{M,g}$ for the
individual picked out by a term (name or variable) $\tau$ relative to a model
$M = \t{D,W,R,V}$ and an assignment function $g$:
\[
  [\tau]^{M,g} =_\text{def} \begin{cases} \;V(\tau) & \text{ if $\tau$ is a name}\\
    \;g(\tau) & \text{ if $\tau$ is a variable}.
  \end{cases}
\]

\begin{definition}{Constant-domain Kripke semantics}{constantdomainsemantics}
  If $\Mfr = \t{W,R,D,V}$ is a constant-domain Kripke model, $w$ is a member of
  $W$, $\phi$ is an $n$-place predicate (for $n\geq 0$),
  $\tau_1,\tau_{2},\ldots,\tau_{n}$ are terms, $\chi$ is a variable, and $g$ is a
  variable assignment, then
  
  \medskip\hspace{-4mm}
  \begin{tabular}{lll}
    (a) & $M,w,g \models \phi \tau_1\ldots \tau_n$ &iff $\t{[\tau_1]^{M,g},\ldots,[\tau_n]^{M,g}} \in V(\phi,w)$.\\
    (b) & $M,w,g \models \tau_1=\tau_2$ &iff $[\tau_1]^{M,g} = [\tau_2]^{M,g}$.\\
    (c) & $M,w,g \models \neg A$ &iff $M,w,g \not\models A$.\\
    (d) & $M,w,g \models A \land B$ &iff $M,w,g \models A$ and $M,w,g \models B$.\\
    (e) & $M,w,g \models A \lor B$ &iff $M,w,g \models A$ or $M,w,g \models B$.\\
    (f) & $M,w,g \models A \to B$ &iff $M,w,g \not\models A$ or $M,w,g \models B$.\\
    (g) & $M,w,g \models A \leftrightarrow B$ &iff $M,w,g \models (A\to B)$ and $M,w,g \models (B\to A)$.\\
    (h) & $M,w,g \models \forall \chi A$ &iff $M,w,g' \models A$ for all $\chi$-variants $g'$ of $g$.\\
    (i) & $M,w,g \models \exists \chi A$ &iff $M,w,g' \models A$ for some $\chi$-variant $g'$ of $g$.\\
    (j) & $M,w,g \models \Box A$ &iff $M,v,g \models A$ for all $v\in W$ such that $wRv$.\\
    (k) & $M,w,g \models \Diamond A$ &iff $M,v,g \models A$ for some $v\in W$ such that $wRv$.
  \end{tabular}

  \medskip
  $A$ is \textbf{true at $w$ in $M$} iff $M,w,g \models A$ for
  every assignment function $g$ for $M$.

\end{definition}

Let's return to the model from above, and let's add the information that the
domain of individuals consists of just Aristotle and Boethius. That is, let $M$
be the following model:
%
\begin{quote}
  $W = \{ w,v \}$\\
  $R = \{ \t{w,w}, \t{w,v} \}$\\
  $D = \{ \text{Aristotle, Boethius} \}$\\
  $V(a) = \text{Aristotle}$\\
  $V(F,w) = \{ \text{Aristotle, Boethius} \}$\\
  $V(F,v) = \{ \text{Boethius} \}$
\end{quote}
%
This isn't a complete specification of a model because I haven't assigned a
meaning to names and predicates other than $a$ and $F$, but we have enough
information to determine the truth-value of any $\L_{M\!P}$-sentence whose only
non-logical vocabulary are $a$ and $F$.

We can, for example, verify that $Fa$ is true at $w$ in $M$. A sentence is true
at $w$ in $M$ iff it is true at $w$ in $M$ relative to every assignment function
$g$. By clause (a) of definition \ref{def:constantdomainsemantics}, $Fa$ is true
at $w$ in $M$ relative to $g$ iff $[a]^{M,g}$ is a member of $V(F,w)$. Since $a$
is a name, $[a]^{M,g}$ is $V(a)$. And $V(a)$ is Aristotle. So $Fa$ is true at
$w$ relative to $g$ iff Aristotle is a member of $V(F,w)$. We know that $V(F,w)$
is $\{ \text{Aristotle, Boethius} \}$. Aristotle evidently is a member of
$\{ \text{Aristotle, Boethius} \}$. So $Fa$ is true at $w$ in $M$, relative to
any assignment $g$.

We can also verify that $\Box Fa$ is false at $w$. By clause (j) of definition
\ref{def:constantdomainsemantics}, $\Box Fa$ is true at $w$ (in $M$ relative to
$g$) iff $Fa$ is true (in $M$ relative to $g$) at all worlds accessible from
$w$. And $Fa$ is false at $v$ because Aristotle is not a member of $\{ \text{Boethius} \}$.


% Here is a (partial) picture of a constant domain model, with three worlds and
% three individuals.

% \begin{center}
%   \begin{tikzpicture}[modal, world/.append style={minimum size=22mm}, node distance=15mm]
%     \node[world] (w) [label=above:{$w$}] {};
%     \draw[gray,looseness=1] (w.south west) -- (w.north east);
%     \node[gray] at ([xshift=3mm,yshift=-2mm]w.north west){\small $F$};
%     \node[gray] at ([xshift=-3mm,yshift=2mm]w.south east){\small $\neg F$};
%     \node at ([yshift=15mm]w.south){\small $a$};
%     \node at ([yshift=7mm]w.south){\small $b$};
%     \node at ([yshift=10mm]w.south east){\small $c$};
%     %
%     \node[world] (v) [label=above:{$v$}, right=of w] {};
%     \draw[gray,looseness=1] (v.south west) -- (v.north east);
%     \node[gray] at ([xshift=3mm,yshift=-2mm]v.north west){\small $F$};
%     \node[gray] at ([xshift=-3mm,yshift=2mm]v.south east){\small $\neg F$};
%     \node at ([yshift=15mm]v.south){\small $a$};
%     \node at ([xshift=2mm,yshift=8mm]v.south west){\small $b$};
%     \node at ([yshift=6mm]v.south){\small $c$};
%     %
%     \node[world] (u) [label=above:{$u$}, right=of v] {};
%     \draw[gray,looseness=1] (u.south west) -- (u.north east);
%     \node[gray] at ([xshift=3mm,yshift=-2mm]u.north west){\small $F$};
%     \node[gray] at ([xshift=-3mm,yshift=2mm]u.south east){\small $\neg F$};
%     \node at ([yshift=17mm,xshift=3mm]u.south){\small $a$};
%     \node at ([xshift=-2mm,yshift=12mm]u.south){\small $b$};
%     \node at ([xshift=-7mm,yshift=8mm]u.south){\small $c$};
%     \path[->] (w) edge (v);
%     \path[->] (v) edge (u);
%   \end{tikzpicture}
% \end{center}
% %
% Each world is inhabited by $a,b$, and $c$. At world $w$, only $a$ is $F$; at
% $v$, $a$ and $b$ are $F$; at $u$, all three individuals are $F$. So $Fa$ is true
% at $w$. $Fx$ is true at $w$ relative to an assignment $g$ that maps $x$
% to $a$. By definition \ref{def:constantdomainsemantics}, this means that
% $\exists x Fx$ is true at $w$. Along the same lines, we can figure out that
% $\exists x Fx$ is true at $v$. Since $w$ can see $v$, it follows that
% $\Diamond \exists x Fx$ is true at $w$. The \emph{de re} sentence
% $\exists x \Diamond Fx$ is also true at $w$, because $\Diamond Fx$ is true at
% $w$ relative to an assignment $g$ that maps $g$ to (say) $b$.

\begin{exercise}
  Which of the following sentences are true at $w$ in $M$? 
  \begin{exlist}
  \item $\neg Fa \to Fa$ % true
  \item $\Box \exists x Fx$ % true
  \item $\Box \forall x Fx$ % false
  \item $\exists x \Box Fx$ % true
  \item $\forall x \Box Fx$ % false
  \item $\forall x (\Box Fx \to \Box\Box Fx)$ % true
  \end{exlist}
\end{exercise}
\begin{solution}
  (a), (b), (d), and (f) are true; (c) and (e) are false.
\end{solution}

Validity is truth at all worlds in all models of a certain kind. A sentence is
\textbf{CK-valid} iff it is true at all worlds in all constant-domain Kripke
models. `C' comes from `constant domains'; `K' indicates that we have put no
constraints on the accessibility relation. We get stronger concepts of validity
-- stronger logics -- if we require the accessibility relation to be reflexive,
or transitive, or euclidean, etc.

It is not hard to see that every sentence that is valid in classical predicate
logic is CK-valid. Similarly, every K-valid sentence is CK-valid. We also get
some new interaction principles between modal operators and quantifiers. For
example, consider the following schema, known as the \emph{Barcan Formula},
after Ruth Barcan Marcus.
%
\principle{BF}{\forall x \Box A \to \Box \forall x A}

\begin{observation}{BFinCDM}
  All instances of \pr{BF} are CK-valid.
\end{observation}
\begin{proof}
  \emph{Proof.} Suppose a sentence $\forall x \Box A$ is true at some world $w$
  in some constant-domain model $M$ relative to some assignment $g$. By clause
  (h) of definition \ref{def:constantdomainsemantics}, it follows that $\Box A$
  is true at $w$ relative to every $x$-variant $g'$ of $g$. By clause (j) of
  definition \ref{def:constantdomainsemantics}, it follows that $A$ is true at
  every world $v$ accessibility from $w$ relative to every $x$-variant $g'$ of
  $g$. By clause (h), this means that $\forall x A$ is true relative to $g$ at
  every world $v$ accessible from $w$. So by clause (j), $\Box \forall x A$ is
  true at $w$ relative to $g$.

  We've shown that whenever $\forall x \Box A$ is true at some world $w$ in some
  model $M$ relative some assignment $g$, then $\Box A \forall x A$ is also true
  at $w$ in $M$ relative to $g$. By clause (f) of definition
  \ref{def:constantdomainsemantics}, it follows that
  $\forall x \Box A \to\Box A \forall x A$ is true at every world in every model
  relative to every assignment. \qed
\end{proof}

Instead of working through definition \ref{def:constantdomainsemantics}, we can
use trees to test if a sentence is CK-valid. The tree rules for CK are all the
rules for K (from chapter 3) together with all the rules for standard predicate
logic, with an added world parameter on each node that is held fixed when
applying a rule from standard predicate logic. (In the predicate logic rules, a
name counts as `old' if it already occurs on the relevant branch, no matter at which world.)

To get a complete proof system, we need one further identity rule, reflecting
the fact that the reference of a name does not vary from world to world:

\medskip
\begin{center}
  \begin{minipage}[t]{0.3\textwidth} \centering

    Identity Invariance
    
    \bigskip
    \tree{
      \dotbelownode{15}{}{$\eta_1 = \eta_2$}{\omega}{}\\
      \\
      \nnode{15}{}{$\eta_1 = \eta_2$}{\nu}{}\\
      \Kk[15]{0}{\color{red}$\uparrow$}\\
      \Kk[15]{0}{\color{red}\small old}
    }
  \end{minipage}
\end{center}
\bigskip

% Can we prove the Necessity of Identity? Girle p.111 suggests we can't. So does
% https://softoption.us/content/node/647. But here's a proof (which I can run
% even on https://softoption.us/content/node/647, where it says the tree won't
% close):
%
% 1. a=b    (w)
% 2. -[]a=b (w)
% 3. -[]a=a (w)  (1,2,LL)
% 4. wRv         (3)
% 5. -a=a   (v)  (3)
%
% Girle says LL is restricted to literals, which is probably enough for
% predicate logic, but blocks this proof.
%
% Without Identity Invariance, we can't prove the Necessity of Distinctness, even with the liberal form of LL, and even if we have the S5 rules:
%
% 1. -a=b    (w)
% 2. -[]-a=b (w)
% 3. wRv         (2)
% 4. a=b     (v) (2) 
% 5. vRw         (3,Symmetry)
%
% At this point, we'd like to say that <>-a=b is true at v, because -a=b is true
% at the accessible w; then the tree could easily be closed. But there's no such
% rule.
%
% If we have a global LL rule that allows substituting corefering names at
% arbitrary other worlds then we can infer -a=a from lines 4 and 1, so we're
% done, even without Symmetry. That's equivalent to my version of LL and
% Identity Invariance. To mimick the axiomatic situation perhaps we should use a
% version of LL which says that we can substitute only at accessible worlds. Equivalently, we could say that

Here is a tree proof for a simple instance of the Barcan Formula,
$\forall x \Box Fx \to \Box \forall x Fx$.

\medskip
\begin{center}
  \tree{
    \nnode{30}{1.}{$\neg(\forall x \Box Fx \to \Box\forall x Fx)$}{w}{(Ass.)}\\
    \nnode{30}{2.}{$\forall x \Box Fx$}{w}{(1)}\\
    \nnode{30}{3.}{$\neg \Box\forall x Fx$}{w}{(1)}\\
    \nnode{30}{4.}{$wRv$}{}{(3)}\\
    \nnode{30}{5.}{$\neg \forall x Fx$}{v}{(3)}\\
    \nnode{30}{6.}{$\neg Fa$}{v}{(5)}\\
    \nnode{30}{7.}{$\Box Fa$}{w}{(2)}\\
    \nnodeclosed{30}{8.}{$Fa$}{v}{(7,4)}\\
  }
\end{center}
\medskip

And here is a proof of $\forall x \forall y(x\!=\!y \to \Box\, x\!=\!y)$, the
``necessity of identity'':
%
\medskip
\begin{center}
  \tree{
    \nnode{30}{1.}{$\neg\forall x\forall y(x\!=\!y \to \Box\, x\!=\!y)$}{w}{(Ass.)}\\
    \nnode{30}{2.}{$\neg\forall y(a\!=\!y \to \Box \,a\!=\!y)$}{w}{(1)}\\
    \nnode{30}{3.}{$\neg(a\!=\!b \to \Box \,a\!=\!b)$}{w}{(2)}\\
    \nnode{30}{4.}{$a=b$}{w}{(3)}\\
    \nnode{30}{5.}{$\neg \Box \,a\!=\!b$}{w}{(3)}\\
    \nnode{30}{6.}{$\neg \Box \,b\!=\!b)$}{w}{\qquad(4, 5, LL)}\\
    \nnode{30}{7.}{$wRv$}{}{(6)}\\
    \nnode{30}{8.}{$b\not=b$}{v}{(6)}\\
    \nnodeclosed{30}{9.}{$b=b$}{v}{(SI)}
  }
\end{center}

\begin{exercise}\label{ex:CKexamples}
  Use the tree method to show that the following sentences are
  CK-valid. 
  \begin{exlist}
  \item $ \Box \forall x Fx \to \forall x \Box Fx$
  \item $ \exists x \Box Fx \to \Box \exists x Fx$
  \item $ \forall x \Box (Fx \land Gx) \to \Box \forall x Fx$
  \item $ \Box\Diamond \exists x Fx \to \Box \exists x \Diamond(Fx \lor Gx)$ % Priest p.327
  \item $\forall x \Box \exists y\, y\!=\!x$
  \item $ \forall x \forall y(x\!\not=\!y \to \Box x\!\not=\!y)$
  \end{exlist}
\end{exercise}
\begin{solution}
  Use \href{https://www.umsu.de/trees/}{umsu.de/trees/}.
  Note that the website uses slightly different identity rules: instead of the
  Self-Identity rule, it has a rule for closing any branch that contains a
  statement of the form $\tau \not= \tau$.
\end{solution}

\begin{exercise}
  The following sentences are CK-invalid. Can you describe a countermodel for
  each? (It may help to construct a tree and inspect its open branches.)
  \begin{exlist}
    \item $\Diamond \exists x Fx \to \Diamond\exists x(Fx \land Gx)$
    \item $\Box \exists x Fx \to \exists x \Box Fx$
    \item
    $\forall x \forall y ((\Diamond Fx \land \Diamond \neg Fy) \to x\!\not=\!y)$
    \item $\forall x \Box (Px \to Qx) \to \forall x (Px \to \Box Qx)$
  \end{exlist}
\end{exercise}
\begin{solution}
  \begin{sollist}
    \item $W=\{ w \}$, $wRw$, $D = \{ \text{Alice} \}$,
    $V(F,w) = \{ \text{Alice} \}$, $V(G,w) = \emptyset$
    \item $W=\{ w,v \}$, $wRw$ and $wRv$, $D = \{ \text{Alice}, \text{Bob} \}$,
    $V(F,w) = \{ \text{Alice} \}$, $V(F,v) = \{ \text{Bob} \}$
    \item $W=\{ w,v \}$, $wRw$ and $wRv$, $D = \{ \text{Alice}, \text{Bob} \}$,
    $V(F,w) = \{ \text{Alice} \}$, $V(F,v) = \emptyset$
    \item $W=\{ w,v \}$, $wRw$ and $wRv$, $D = \{ \text{Alice}, \text{Bob} \}$,
    $V(P,w) = \{ \text{Alice} \}$, $V(P,v) = \emptyset$,
    $V(Q,w) = \{ \text{Alice} \}$, $V(Q,v) = \emptyset$
  \end{sollist}
\end{solution}

There are also axiomatic calculi for CK. We can, for example, combine the axiom
schemas and rules of classical predicate logic with those of K, and add two new
schemas: the Barcan Formula \pr{BF} and the ``necessity of distinctness'',
%
\principle{ND}{\forall x \forall y(x\!\not=\!y \to \Box x\!\not=\!y).}

% (BF) is provable if we have the B-schema. The proof (due to Prior) is
% surprisingly difficult.
% \begin{align*}
%    (1) \quad & \forall x \Box A \to \Box A &&\text{ by first-order logic}\\
%    (2) \quad & \Diamond\forall x \Box A \to \Diamond\Box A &&\text{ from (1) by K\ }\\
%    (3) \quad & \neg A \to \Box\Diamond \neg A &&\text{ (B)\ }\\ 
%    (4) \quad & \Diamond\Box A \to A &&\text{ from (3)\ }\\ 
%    (5) \quad & \Diamond\forall x \Box A \to A &&\text{ from (2) and (4)\ }\\ 
%    (6) \quad & \Diamond\forall x \Box A \to \forall x A &&\text{ from (5) by first-order logic\ }\\
%    (7) \quad & \Box\Diamond\forall x \Box A \to \Box\forall x A &&\text{ from (6) by K\ }\\  
%    (8) \quad & \forall x \Box A \to \Box\Diamond\forall x \Box A &&\text{ (B)}\\
%    (9) \quad & \forall x \Box A \to \Box \forall x A &&\text{ from (7) and (8)}
% \end{align*}

% Like the Barcan formula, \pr{ND} is provable if we add the axioms or rules for
% the modal logic B or S5, but it is not provable in the minimal combination of
% predicate logic with K. Proof in B:
%
% 1. <>-(x=y) -> -(x=y)      (NI)
% 2. []<>-(x=y) -> []-(x=y)  (1,K,M\!P)
% 3. -(x=y) -> []<>-(x=y)    (B)
% 4. -(x=y) -> []-(x=y)      (2,3)

% \begin{exercise}\label{ex:nni}
%   Show that \pr{ND} is CK-valid. 
% \end{exercise}
% \begin{solution}
%   Suppose for reductio that some instance of \pr{ND} is false at
%   some world $w$ in some constant domain model $M$. Then there is some
%   assignment $g$ such that $M,w,g \models \neg(x=y)$ and
%   $M,w,g \not\models \Box\neg(x=y)$. The latter means that
%   $M,v,g \not\models \neg(x=y)$ for some world $v$. But
%   $M,w,g \models \neg(x=y)$ holds only if $g(x)\not=g(y)$, and 
%   $M,v,g \not\models \neg(x=y)$ holds only if $g(x)=g(y)$. Contradiction. 
% \end{solution}

As I mentioned above, stronger logics can be defined by putting constraints on
the accessibility relation. For example, the system \textbf{CT} is the set of
$\L_{M\!P}$-sentences that are valid in the class of constant-domain Kripke
models with a reflexive accessibility relation. \textbf{CS4} is the set of
$\L_{M\!P}$-sentences that are valid in the class of constant-domain Kripke
models with a reflexive and transitive accessibility relation. And so on.

Properties of the accessibility relation still correspond to modal schemas, just
as in chapter \ref{ch:accessibility}: \pr{T} corresponds to reflexivity, \pr{4}
to transitivity, \pr{G} to convergence, etc. Recall that a schema
\emph{corresponds} to a property of the accessibility relation if the schema is
valid in all and only the frames in which the accessibility relation has that
property. A \emph{frame} is a model without an interpretation function. In the
present context, a frame therefore consists of two non-empty sets $W$ and $D$ and a
relation $R$ on $W$.

We can still use the tree method or the axiomatic method to test for validity in
logics stronger than CK. To test for CT-validity, for example, we would add the
Reflexivity rule to the tree rules for CK. To test for CS4-validity, we would
add the Reflexivity and Transitivity rules. We can get an axiomatic calculus for
CT by adding the \pr{T}-schema to the calculus for CK; for CS4, we can add
\pr{T} and \pr{4}. And so on for other systems.

But there are exceptions. Remember S4.2 -- the set of $\L_M$-sentences valid in
the class of reflexive, transitive, and convergent Kripke models. Reflexivity
corresponds to \pr{T}, transitivity to \pr{4}, and convergence to \pr{G}. If we
add these schemas to the axiomatic calculus for system K, we get a sound and
complete calculus for S4.2. But if we add the schemas to the calculus for CK,
the resulting calculus is \emph{not} complete for CS4.2. There are
$\L_{MP}$-sentences that are valid in the class of reflexive, transitive, and
convergent constant-domain models that can't be derived.

% In particular, we can't prove
% $\neg(\Diamond(\exists x Ax \land \forall x(Ax \to \Box Bx) \land \Box \neg\forall x Bx) \,\land\, \Diamond\forall x(Ax \lor \Box Bx)\,\land\, \forall x (\Diamond Ax \to \Box (\exists x Ax \to Ax)))$.)
%
% Similarly for S4M: QS4M+(BF) cannot prove
% $\Box \exists x Ax \to \Diamond \exists x \Box Ax$, which is valid in the
% class of all its frames.
%
% For the proofs, see \cite{cresswell95incompleteness}.
%
% In either case, the canonicity proof does not carry over to the first-order
% case because first-order canonicity requires that certain formula sets have
% not only a maximally consistent extension, but a maximally consistent
% \emph{and witnessed} extension.
%
% Does a problem like this also arise for the tree rules?

% The calculus we get if we add T+4+G is not complete with respect to any class
% of constant-domain frames.

\section{Quantification and existence}

We have assumed that the domain of individuals is the same for every world. This
may seem problematic.

Earlier today I was baking bread. Let's call the loaf of bread that I made
Loafy. Intuitively, Loafy could have failed to exist. I could have decided not
to bake bread. Even if determinism is true, we can consider worlds at which the
laws of nature or the origin of the universe are different. In many of these
worlds, there are no humans, and no loafs of bread. So we should allow for
worlds at which Loafy doesn't exist.

If we use $b$ as a name for Loafy, we can arguably express Loafy's existence as
\[
  \exists x \,x\!=\!b.
\]
Why might this express that Loafy exists? Consider a scenario in which Loafy does
exist. In that scenario, there is some thing $x$ which is identical to Loafy
(namely, Loafy). Conversely, consider a scenario in which Loafy does not exist.
In that scenario, there is no thing $x$ which is identical to Loafy. So
$\exists x\, x\!=\!b$ is true in all and only the scenarios in which Loafy exists.

Now we can sharpen the above worry. Intuitively, it could have been the case
that Loafy doesn't exist. So $\Diamond \neg \exists x\, x\!=\!b$ is true, on a
suitable understanding of the diamond. But in constant-domain semantics, that
sentence is a contradiction: it is false at every world in every model.

A converse problem arises if we think that something could have existed that
doesn't actually exist. For example, let's assume that there could have been
unicorns. If we interpret the predicate $U$ as `-- is a unicorn' and the box as
a suitable kind of circumstantial necessity, $\Box \forall x \neg Ux$ should
then be false. But let's also assume that no individual in our world could have
been a unicorn. So $\forall x \Box \neg Ux$ is true. We then have a
counterexample to the Barcan Formula $\forall x \Box A \to \Box \forall x A$.
And all instances of the Barcan Formula are valid in constant-domain semantics.

% This problem is a little harder to bring out because there is no direct way to
% say, in $\L_{M\!P}$, that something could have existed that doesn't actually
% exist. We could express it if we add an `actually' operator:
% $\neg \Diamond\exists x \neg \Always \exists y(y=x)$ comes out valid.

\begin{exercise}
  The \textbf{Converse Barcan Formula} is the schema
  $\Box \forall x A \to \forall x \Box A$. All instances of the Converse Barcan
  Formula are CK-valid. Explain why Loafy's possible non-existence seems to
  provide a counterexample to the Converse Barcan Formula.
\end{exercise}
\begin{solution}
  $\Box \forall x \exists y\, x\!=\!y \to \forall x \Box\exists y\, x\!=\!y$ is an
  instance of the Converse Barcan Formula. If we read the box as a relevant kind
  of circumstantial necessity, and Loafy could have failed to exist, then the
  consequent of this conditional is false. But the antecedent is true.
\end{solution}

% Here is a (schematic) proof of \pr{CBF} in the combined axiomatic
% calculus for predicate logic and the modal logic K.
% \begin{align*}
%    1. \quad & \forall x A \to A[c/x] &&\text{ (UI)}\\
%    2. \quad & \Box(\forall x A \to A[c/x]) &&\text{ (from 1 by Nec)}\\
%    3. \quad & \Box(\forall x A \to A[c/x]) \to (\Box\forall x A \to \Box A[c/x]) &&\text{ (K)}\\
%    4. \quad & \Box\forall x A \to \Box A[c/x] &&\text{ (from 2 and 3 by M\!P)}\\
%    5. \quad & \Box \forall x A \to \forall x \Box A &&\text{ (from 4 by UG)}.
% \end{align*}

\begin{exercise}
  Consider the following four schemas.
  \begin{enumerate}[leftmargin=14mm]
    \itemsep-1mm
  \item[(1)] $\Diamond \exists x A \to \exists x \Diamond A$
  \item[(2)] $\Box \exists x A \to \exists x \Box A$
  \item[(3)] $\exists x \Box A \to \Box \exists x A$
  \item[(4)] $\exists x \Diamond A \to \Diamond \exists x A$
  \end{enumerate}
  \vspace{-3mm}
  \begin{exlist}
    \item Are any of (1)--(4) equivalent to the Barcan Formula or the Converse
    Barcan Formula (given the duality of $\Box$ and $\Diamond$, of $\forall x$
    and $\exists x$, and the standard truth-tables for propositional
    connectives)?
    \item Which of these schemas do you think are intuitively valid on a
    metaphysical interpretation of the box and the diamond?
  \end{exlist}
\end{exercise}
\begin{solution}
  (1) is equivalent to the Barcan Formula, (4) to the Converse Barcan Formula.
  (2) is highly implausible. (1) and (4) are often regarded as implausible, for
  the reasons I discuss in the text. Like the Converse Barcan Formula, the
  validity of (3) rules out scenarios in which individuals at one world may fail
  to exist at an accessible world.
  % (3) isn't equivalent to CBF though: (3) can be valid even without increasing domains, provided that everything at w exists at /some/ accessible world.
\end{solution}  
  
% \begin{exercise}
%   The \pr{K}-like principle
%   $\forall x (A \to B) \to (\forall x A \to \forall x B)$ is easily provable
%   in classical predicate logic. Can you see how the strict version
%   $\forall x (A \strictif B) \to (\forall x A \strictif \forall x B)$ is
%   related to the Barcan Formula?
% \end{exercise}

An obvious response to these problems is to replace constant-domain semantics
with a semantics in which the domain of individuals can vary from world to
world. We will explore this option in the following section. First I want to
mention two other lines of response.

Some philosophers have argued that we should bite the bullet: we are simply
mistaken when we judge that Loafy could have failed to exist, or that anything
could have existed that doesn't actually exist. In temporal logic, biting the
bullet means to accept that anything that has ever existed still exists today,
and that anything that exists today has always existed and is always going to
exist. In epistemic logic, biting the bullet means to accept that nobody can be
unsure or ignorant about which individuals exists: if something exists, nobody
can fail to know that it exists, nor can anyone believe that an individual
exists that doesn't really exist.

% It seems odd to say that if I'm unsure whether there are dragons then I'm
% actually sure the relevant objects exist, just not whether they are dragons.

A different response is to break the link between quantification and
existence. $\exists x$ is traditionally called an ``existential'' quantifier,
and pronounced `there is an $x$' or `there exists an $x$'. But $\L_{M\!P}$ is a
made-up language. We can make its symbols mean whatever we want. We can give a
different interpretation of $\exists x$ so that `Loafy exists' can't be
translated as $\exists x\, x\!=\!b$.

One alternative to the standard interpretation of quantifiers is associated with
the Austrian philosopher Alexius Meinong. Meinong observed that when we describe
beliefs, plans, hopes, or fears, we often seem to refer to non-existent objects.
We might say that someone is afraid of \emph{a ghost}, or that they are
searching for \emph{a golden mountain} -- even though there are no ghosts or
golden mountains. According to Meinong, people who are searching for a golden
mountain are really searching for \emph{something}. That something is a golden
mountain. But it is not an existent golden mountain. Meinong concluded that
besides existent mountains, there are also non-existent mountains.

Quantifiers that range over both existent and non-existent individuals are
called \emph{Meinongian}. If the $\L_{M\!P}$-quantifiers are Meinongian, then
clearly $\exists x\, x\!=\!b$ does not translate `Loafy exists'.

Meinong's postulation of non-existent individuals is widely rejected as
incoherent. It certainly raises difficult questions. Suppose you are
searching for a golden mountain. You probably don't have any firm views about
the mountain's height. You are not looking for a mountain that is exactly 2000
meters tall, nor are you looking for a mountain that is exactly 2100 meters
tall. On the Meinongian account, there is a genuine mountain that you
are looking for. It is a mountain that is not 2000 meters tall, not 2100 meters
tall, and doesn't have any other particular height either. But how could there
be a mountain without any particular height? Besides, it also doesn't seem right
to say that you are looking for a peculiar ``mountain'' that doesn't have any
height and doesn't exist. Intuitively, you are looking for an \emph{existent}
mountain that \emph{does} have a height.

% Even if we accept Meinongian quantification as coherent, it is not clear whether
% it fully avoids the problem of constant domains. For example, couldn't I be
% unsure about how many things there are, even in the Meinongian, extended sense
% of `there are'? To model my uncertainty in terms of accessible worlds, the
% domain of the Meinongian quantifier would have to vary from world to world.

% One motivation for this move is that in natural language, we can apparently
% quantify over non-existent objects: We seem to have names for them, like
% `Pegasus', and we can quantify over them, as when I say that there's a strange
% house I often see in my dreams, made of chocolate. The question is whether we
% want to say such things in our formal language, and if so, how we want to say
% them. The purpose of our language is to avoid confusion and aid clear
% reasoning. And for that purpose, many think it advisable to not quantify over
% things that don't exist.

% Consider that house in my dream. If someone made an inventory of all houses,
% should that house really be included? If someone says that \emph{there are no}
% houses made of chocolate, are they wrong? What other properties does that house
% have? When was it built? Does someone live in it? My dreams don't give an
% answer. It is hard to believe that there is a fact of the matter. So should we
% accept that there are houses that are neither inhabited nor uninhabited?

% Also, as Lycan 1994, p.5 points out, if `the city 100 km South of Edinburgh
% and 100 km North of London' refers, doesn't it follow that Edinburgh is 200 km
% North of London?

% Another obvious problem that arises if we allow for non-existent objects is
% that we must now explain both the quantifier $\exists$ and the predicate
% ``existing'', in a way that doesn't trivialize the issue. (For example, it
% won't do to say that `existing' means being concrete.) Lewis has an answer:
% ``existing'' means `being located in our world (and at the present)'.

% Barcan Marcus proposed that Meinongian quantification is substitutional. That
% looks plausible for `there are things that don't exist'.

A more straightforward alternative to the standard interpretation of quantifiers
is the \emph{possibilist} interpretation. Here we assume that $\forall x$ and
$\exists x$ range not only over things that exist at the world at which the
quantifiers are interpreted, but over everything that exists at any possible
world. On this interpretation, too, $\exists x\, x\!=\!b$ no longer states that Loafy
exists. It merely states that Loafy could have existed, in an unrestricted sense
of `could'. Constant-domain semantics then only assumes that the set of
individuals that exist at some world or other does not vary from world to world.

% Now, it seems rather odd to say that something at $w$ is red at $w$ and in a
% box at $w$ even though no existing object is red at $w$ and even though the
% box is empty at $w$. So most predicates are existence-entailing. The things
% that don't exist aren't red or in a box. They are possibly red and possibly in
% a box.

% I must not conflate the view that actuality contains non-concrete individuals
% that are dragons in other worlds with the view that quantifiers directly range
% over non-actual individuals. Rini and Cresswell say that `there could have
% been a unicorn' means that there is a world $w'$ at which something $u$ is a
% unicorn; ``If \emph{unicorn} is a natural kind term then presumably nothing
% actual \emph{could} be a unicorn, and so $u$ is a non-actual possible.''
% (p.72)

One  downside of the possibilist interpretation is that it goes against the
``internalist'' spirit of modal logic. As we saw in section \ref{sec:fragment},
one of the key features of modal logic is that it looks at the structure of
worlds from the inside, from the perspective of a particular world, with only
the modal operators providing (incomplete) access to other worlds.  Possibilist
quantifiers would provide unrestricted access to the inhabitants of other
worlds.

Let's set aside these alternatives and see how constant-domain semantics could
be changed to allow for variable domains.

\section{Variable-domain semantics}\label{sec:variabledomains}

In variable-domain models, every world $w$ is associated with its own individual
domain $D_w$. Loafy the bread may be a member of $D_w$ but not of $D_v$.
Quantifiers range over the individuals in the local domain of the world at which
they are interpreted: $\exists x Fx$ is true at $w$ iff $Fx$ is true (at $w$) of
some individual in $D_w$.

Here is our revised definition of an $\L_{M\!P}$-model.

\begin{definition}{}{variabledomainmodel}
  A \textbf{variable-domain Kripke model} for $\L_{M\!P}$ is a structure $M$
  consisting of%
  \medskip
  \begin{compactenum}
  \item a non-empty set $W$ (the ``worlds''),
  \item a binary (``accessibility'') relation $R$ on $W$,
  \item for each world $w$, a non-empty set $D_w$ (of ``individuals''), and
  \item an interpretation function $V$ that assigns
    \vspace{-1mm}
    \begin{itemize}
      \itemsep-1mm
      \item to each name a member of some domain $D_w$, and
      \item to each $n$-place predicate and world $w$ a set of $n$-tuples from
            $D_w$.
    \end{itemize}
  \end{compactenum}
\end{definition}

To complete the semantics, we need to explain how $\L_{M\!P}$-sentences are
interpreted relative to any given world in a variable-domain model. This raises
a problem.

Since Loafy could have failed to exist, we want to have models in which
$\Diamond \neg \exists x\, {x\!=\!b}$ is true at some world $w$. It follows that
$\neg \exists x\, x\!=\!b$ is true at some world $v$ accessible from $w$.
Intuitively, $v$ is a world at which Loafy doesn't exist. The problem is that we
need to explain how a sentence that contains a name (here, $b$) should be
interpreted at a world (here, $v$) where the thing that's picked out by the name
doesn't exist.

In the case of $\neg \exists x\, x\!=\!b$, the sentence should come out true.
Other cases are less clear. What about $b\!=\!b$? Is Loafy identical to Loafy at
$v$, where Loafy doesn't exist? What about $Fb$, $\neg Fb$, or
$Fb \lor \neg Fb$? Is Loafy delicious at $v$? Is Loafy not delicious at $v$? Is
Loafy either delicious or not delicious at $v$?

These questions are discussed not just in modal logic, but also in a branch of
non-modal logic called \textbf{free logic}. Free logic differs from classical
predicate logic by dropping the assumption that every name has a referent. The
assumption is, after all, not true for names in natural language.

Consider the story of `Vulcan'. In the 19th century, it was observed that
Mercury's path around the Sun conforms to Newton's laws only if there is
another, smaller planet between Mercury and the Sun. With the help of Newton's
laws, astronomers calculated the size and position of that planet, and called it
Vulcan. But Vulcan was never discovered. Eventually, Mercury's path was
explained by Einstein's theory of relativity, without assuming any new planets.
The name `Vulcan' turned out to be \emph{empty}: it doesn't refer to anything.

% Before applying classical logic one would have to determine which names refer
% and which don't, which often requires substantial empirical information. Free
% logic avoids these problems. It can be applied even if the emptiness of names
% is unknown, and it can capture valid inference with empty names.

How should we formalize reasoning with empty names? The orthodox answer is that
we shouldn't: the function of a name is to pick out an individual; if there is
no individual to be picked out, we shouldn't use a name. Proponents of free
logic disagree. They hold that we can perfectly well reason with empty names. We
then need to answer the same questions that I posed above: if $b$ is an empty
name, how should we interpret $b=b$, $Fb$, $\neg Fb$, and $Fb \lor \neg Fb$?

Within free logic, there are broadly three approaches.

The first is Meinongian. It assumes that apparently empty names are not really
empty after all; they merely pick out a non-existent individual. Statements with
such names are then interpreted as usual: $Fb$ may be true or false, depending
on whether the (non-existent) individual picked out by $b$ has the property
expressed by $F$.

Non-Meinongian versions of free logic usually assume that \emph{atomic}
sentences with empty names are never true: if $b$ is empty, then $Fb$ can't be
true. The idea is that predicates express properties, and if something doesn't
exist then it doesn't have any properties. For example, it is not true that
Vulcan is a planet -- as you can see from the fact that Vulcan would not occur
on a list of all planets. Nor is it true that Vulcan orbits the sun, or that
Vulcan has any particular mass.

% To be sure, Vulcan was \emph{believed} to be a planet and to orbit the Sun.
% But one could argue that this should be formalized as something like
% $\Bel(Pa \land Oas)$, not $Fa$.

What shall we say about $\neg Fb$ then, if $b$ is an empty name? In some
versions of free logic, the standard semantic rules for complex sentences are
applied: since $Fb$ is not true, $\neg Fb$ is true, and so is $Fb \lor \neg Fb$.
Other versions of free logic assume that if $b$ doesn't refer then neither $Fb$
nor $\neg Fb$ is true. Since a sentence is called false iff its negation is
true, this means that $Fb$ and $\neg Fb$ are neither true nor false. We get a
three-valued semantics that can be spelled out in different ways, with different
verdicts on sentences like $Fb \lor \neg Fb$.

Each version of free logic can be used to give a semantics for modal predicate
logic with variable domains. I am going to use the two-valued non-Meinongian
approach, mainly because it is the simplest. We will assume that at worlds where
Loafy doesn't exist, every atomic sentence involving a name for Loafy is false:
$b=b$ is false, $Fb$ is also false, but $\neg Fb$ and $Fb \lor \neg Fb$ are
true.

% (see \cite{lambert03philosophical} or \cite{nolt07free} for a more in-depth
% overview of all this) and e.g. \cite[para 2.3]{sainsbury05reference} for a
% defense of my choice.

\begin{definition}{Variable-domain Kripke semantics}{variabledomainsemantics}
  If $\Mfr = \t{W,R,D,V}$ is a variable-domain Kripke model, $w$ is a member of
  $W$, $\phi$ is an $n$-place predicate (for $n\geq 0$), $\tau_1,\ldots,\tau_n$
  are terms, $\chi$ is a variable, and $g$ is a variable assignment, then
  
  % Note that [\tau]^{M,g} is never empty. We've assumed that names have a
  % referent.
  
  \medskip\hspace{-4mm}
  \begin{tabular}{lll}
    (a) & $M,w,g \models \phi \tau_1\ldots \tau_n$ &iff $\t{[\tau_1]^{M,g},\ldots,[\tau_n]^{M,g}} \in V(\phi,w)$.\\
    (b) & $M,w,g \models \tau_1=\tau_2$ &iff $[\tau_1]^{M,g} = [\tau_2]^{M,g}$ and $[\tau_1]^{M,g} \in D_w$.\\
    (c) & $M,w,g \models \neg A$ &iff $M,w,g \not\models A$.\\
    (d) & $M,w,g \models A \land B$ &iff $M,w,g \models A$ and $M,w,g \models B$.\\
    (e) & $M,w,g \models A \lor B$ &iff $M,w,g \models A$ or $M,w,g \models B$.\\
    (f) & $M,w,g \models A \to B$ &iff $M,w,g \not\models A$ or $M,w,g \models B$.\\
    (g) & $M,w,g \models A \leftrightarrow B$ &iff $M,w,g \models (A\to B)$ and $M,w,g \models (B\to A)$.\\
    (h) & $M,w,g \models \forall \chi A$ &iff $M,w,g' \models A$ for all $\chi$-variants $g'$ of $g$ for\\[-1mm]
        && which $g'(\chi)\in D_w$.\\
    (i) & $M,w,g \models \exists \chi A$ &iff $M,w,g' \models A$ for some $\chi$-variant $g'$ of $g$ for\\[-1mm]
        && which $g'(\chi)\in D_w$.\\
    (j) & $M,w,g \models \Box A$ &iff $M,v,g \models A$ for all $v\in W$ such that $wRv$.\\
    (k) & $M,w,g \models \Diamond A$ &iff $M,v,g \models A$ for some $v\in W$ such that $wRv$.
  \end{tabular}

  \medskip
  $A$ is \textbf{true at $w$ in $M$} iff $M,w,g \models A$ for all assignments
  $g$ for $M$.
\end{definition}

A sentence is \textbf{VK-valid} (`V' for `variable-domain') iff it is true at all
worlds in all variable-domain models.

The system VK is weaker than classical predicate logic. Not everything
that is valid in classical predicate logic is CK-valid. For example, both $b=b$
and $\exists x\, x\!=\!b$ are valid in classical predicate logic, but they are
not true at every world in every variable-domain model. If $V(b)$ is not a
member of $D_w$, then $b=b$ and $\exists x \, x\!=\!b$ are false at $w$.

On the other hand, you can check that $\forall x\, x\!=\!x$ is VK-valid. So we
don't just have to revise the rules for identity. We also need to revise the
rule of ``universal instantiation'': from the fact that a universal
generalisation like $\forall x\, x\!=\!x$ is true (at a world, or at all worlds),
we can't infer that all its instances are true: $b=b$ may be false. For another
example, consider a world $w$ where everything is made of chocolate. Let $F$
express the property of being made of chocolate. $\forall x Fx$ is true at $w$.
But we can't infer that Loafy the bread is made of chocolate ($Fb$) at $w$, for
Loafy may not exist at $w$.

In the type of free logic we have adopted, the rule of universal instantiation
requires another premise: from $\forall x A$ we can infer $A[b/x]$ only if we
also know that $b$ exists -- which can be expressed as $\exists x \, x\!=\!b$, or
even simpler as $b\!=\!b$, given our assumption that atomic sentences with empty
names are always false.

Here are the revised tree rules for VK. I only give the quantifier rules for
$\forall \chi A$ and $\exists \chi A$. You can find the rules for
$\neg \forall \chi A$ and $\neg\exists \chi A$ by converting these into
$\exists \chi \neg A$ and $\forall \chi \neg A$, respectively.

\bigskip

\begin{minipage}{0.6\textwidth} \centering
\tree[2]{
  & \dotbelowbnode{12}{}{$\forall \chi A$}{\omega}{} &\\
  && \\
  && \\
  \nnode{10}{}{$\eta\!\not=\!\eta$}{\omega}{} && \nnode{12}{}{$A[\eta/\chi]$}{\omega}{}\\
  \Kk[3]{0}{\color{red}$\uparrow$} &&\\
  \Kk[3]{0}{\color{red}\small old} &&
}
\end{minipage}
\begin{minipage}{0.4\textwidth}\centering
\tree{
  \dotbelownode{12}{}{$\exists \chi A$}{\omega}{}\\
  \\
  \nnode{12}{}{$\eta=\eta$}{\omega}{}\\
  \nnode{12}{}{$A[\eta/\chi]$}{\omega}{}\\
  \Kk[-1]{0}{\color{red}$\uparrow$}\\
  \Kk[0]{0}{\color{red}\small new}
}
\end{minipage}

% \begin{minipage}{0.24\textwidth}\centering
% \tree{
%   \dotbelownode{12}{}{$\neg\forall \chi A$}{\omega}{}\\
%   \\
%   \nnode{12}{}{$\eta=\eta$}{\omega}{}\\
%   \nnode{12}{}{$\neg A[\eta/\chi]$}{\omega}{}\\
%   \Kk[-1]{0}{\color{red}$\uparrow$}\\
%   \Kk[0]{0}{\color{red}\small new}
% }
% \end{minipage}
% \begin{minipage}{0.24\textwidth} \centering
% \tree[2]{
%   & \dotbelowbnode{12}{}{$\neg\exists \chi A$}{\omega}{} &\\
%   && \\
%   && \\
%   \nnode{15}{}{$\eta\!\not=\!\eta$}{\omega}{} && \nnode{15}{}{$\neg A[\eta/\chi]$}{\omega}{}\\
%   \Kk[4]{0}{\color{red}$\uparrow$} &&\\
%   \Kk[4]{0}{\color{red}\small old or first} &&
% }
% \end{minipage}


\bigskip

We keep the rule for Leibniz's Law. But we replace the Self-Identity and Identity Invariance rules by the following three rules.

\bigskip

\begin{minipage}{0.32\textwidth}\centering
    Existence
    
    \tree{
      \dotbelowbarenode{}\\
      \\
      \nnode{10}{}{$\eta=\eta$}{\omega}{}\\
      \Kk[0]{0}{\color{red}$\uparrow$\hspace{6mm}}\\
      \Kk[0]{0}{\color{red}\small new\hspace{6mm}}
    }
\end{minipage}
\begin{minipage}{0.32\textwidth}\centering
    Identity Invariance
    
    \bigskip
    \tree{
      \nnode{15}{}{$\eta_1 = \eta_2$}{\omega}{}\\
      \dotbelownode{15}{}{$\eta_1 = \eta_1$}{\nu}{}\\
      \\
      \nnode{15}{}{$\eta_1 = \eta_2$}{\nu}{}
    }
\end{minipage}
\begin{minipage}{0.32\textwidth}\centering
\tree{
  \dotbelownode{16}{}{$\Phi\eta_1\ldots\eta_n$}{\omega}{}\\
  \\
  \nnode{16}{}{$\eta_1=\eta_1$}{\omega}{}\\
  \nnode{16}{}{$\eta_2=\eta_2$}{\omega}{}\\
  \nnode{16}{}{$\vdots$}{}{}\\
  \nnode{16}{}{$\eta_n=\eta_n$}{\omega}{}
}
\end{minipage}

% Shouldn't I turn Identity Invariance into a branching rule, like FUI?

\bigskip

The Existence rule reflects our assumption that the domain of individuals is
never empty. The unnamed last rule is a rule for expanding atomic nodes. From the
assumption that $Fb$ is true at a world, for example, the rule allows us to
infer that $b$ exists at that world, which can be expressed as $b\!=\!b$. We
then don't need a separate rule of Self-Identity.

% These rules are adapted from Priest, pp. 331, 337, and 353. The main
% differences are these:
%
% (1) in NFL, we can use a=a instead of Ea. This makes Priest's Self-Identity
% rule trivial.
%
% (2) Priest allows all inner domains to be empty, I require them all to be
% non-empty. This makes a difference. From $\forall x Fx$, I can infer that $Fx$
% is true of something. Priest's rules don't allow that. I assume this can be
% fixed by adding a new rule that allows us to introduce a new name and say that
% it exists. This is the new Self-Identity rule. When expanding universal
% quantifiers, one then never needs to introduce a new (first) name.

\begin{exercise}
  Use the tree method to show that the following sentences are VK-valid.
  \begin{exlist}
  \item $\exists x \Box Fx \to \Box \exists x Fx$
  \item $\Box\forall x(Fx \to Gx) \to (\Box\forall x Fx \to \Box \forall x Gx)$ % Priest p.332
  \item $\Box \exists x \, x\!=\!x$
  \item $\Diamond Fa \to \Diamond \exists x Fx$
  \item $a\!=\!b \to \Box(a\!=\!a \to a\!=\!b)$
  \end{exlist}
\end{exercise}
\begin{solution}
  \begin{sollist}
    \item \tree[4]{%
        & \nnode{22}{1.}{$\exists x \Box Fx \to \Box \exists x Fx$}{w}{(Ass.)} & \\
        & \nnode{22}{2.}{$\exists x \Box Fx$}{w}{(1)} & \\
        & \nnode{22}{3.}{$\neg \Box \exists x Fx$}{w}{(1)} & \\
        & \nnode{22}{4.}{$\Box Fa$}{w}{(2)} & \\
        & \nnode{22}{5.}{$wRv$}{}{(3)} & \\
        & \nnode{22}{6.}{$\neg \exists x Fx$}{v}{(3)} & \\
        & \nnode{22}{7.}{$Fa$}{v}{(4,5)} & \\
        & \bnode{22}{8.}{$a\!=\!a$}{v}{(7)} & \\
        && \\
        \nnodeclosed{12}{9.}{$a\!\not=\!a$}{v}{(6)} &&  \nnodeclosed{12}{9.}{$\neg Fa$}{v}{(6)}\\
    }
    \medskip

    \item DIY. The tree has four branches. I can't typeset it.
    \medskip
    
    \item \tree[4]{%
        & \nnode{25}{1.}{$\neg\Box \exists x\, x\!=\!x$}{w}{(Ass.)} &\\
        & \nnode{25}{2.}{$wRv$}{}{(1)} &\\
        & \nnode{25}{3.}{$\neg \exists x\, x\!=\!x$}{v}{(1)} &\\
        & \bnode{25}{4.}{$a\!=\!a$}{v}{(Ex.)} &\\
        && \\
        \nnodeclosed{12}{9.}{$a\!\not=\!a$}{v}{(3)} &&  \nnodeclosed{12}{9.}{$a\!\not=\!a$}{v}{(3)}\\
    }
    \medskip

    \item \tree[4]{%
        & \nnode{25}{1.}{$\neg(\Diamond Fa \to \Diamond \exists x Fx)$}{w}{(Ass.)} &\\
        & \nnode{25}{2.}{$\Diamond Fa$}{w}{(1)} &\\
        & \nnode{25}{3.}{$\neg \Diamond \exists x\, Fx$}{w}{(1)} &\\
        & \nnode{25}{4.}{$wRv$}{}{(2)} &\\
        & \nnode{25}{5.}{$Fa$}{v}{(2)} &\\
        & \nnode{25}{6.}{$a\!=\!a$}{v}{(5)} &\\
        & \bnode{25}{7.}{$\neg \exists x\, Fx$}{v}{(3,4)} &\\
        && \\
        \nnodeclosed{12}{9.}{$a\!\not=\!a$}{v}{(3)} &&  \nnodeclosed{12}{10.}{$\neg Fa$}{v}{(3)}\\
    }
    \medskip

    \item \tree[4]{%
        & \nnode{38}{1.}{$\neg(a\!=\!b \to \Box(a\!=\!a \to a\!=\!b))$}{w}{(Ass.)} &\\
        & \nnode{38}{2.}{$a\!=\!b$}{w}{(1)} &\\
        & \nnode{38}{3.}{$\neg\Box(a\!=\!a \to a\!=\!b)$}{w}{(1)} &\\
        & \nnode{38}{4.}{$wRv$}{}{(3)} &\\
        & \nnode{38}{5.}{$\neg(a\!=\!a \to a\!=\!b)$}{v}{(3)} &\\
        & \nnode{38}{6.}{$a\!=\!a$}{v}{(5)} &\\
        & \nnode{38}{7.}{$\neg a\!=\!b$}{v}{(5)} &\\
        & \nnodeclosed{38}{8.}{$a\!=\!b$}{v}{(2,6)} &\\
    }
    \medskip
  \end{sollist}
\end{solution}

% What about the axiomatic approach?
%
% The \pr{UI} principle $\forall x A \to A[a/x]$ has become invalid. It is
% replaced by a weaker principle of ``free universal instantiation'', \pr{FUI}:
% 
% \principle{FUI}{\forall \chi A \to (\eta\!=\!\eta \to A[\eta/\chi])}
% 
% What is a complete axiomatisation?
%
% For soundness and completeness of typical systems on variable domains see HC
% 294--302. The completeness technique for constant domain QML generally fails
% because we don't have the BF which allows us to construct maximally consistent
% and witnessed sets.

It is easy to check that the Barcan Formula,
$\forall x \Box A \to \Box \forall x A$, and its converse,
$\Box \forall x A \to \forall x \Box A$, are invalid in variable-domain
semantics. (By this I mean that not all their instances are valid.) In fact, we
can now prove that the Barcan formula corresponds to the assumption that
whatever exists at an accessible world also exists at the original world, while
its converse corresponds to the assumption that whatever exists at a world also
exists at all accessible worlds.

\begin{observation}{barcancorr}
  \vspace{-1mm}
  \begin{enumerate}[leftmargin=10mm]
    \itemsep0mm
  \item[(i)] \pr{CBF} is valid on a variable-domain frame iff the frame has
    \emph{increasing domains}, meaning that whenever $wRv$, then
    $D_w \subseteq D_v$.
    
  \item[(ii)] \pr{BF} is valid on a variable-domain frame iff the frame has
    \emph{decreasing domains}, meaning that whenever $wRv$ then
    $D_v \subseteq D_w$.
  \end{enumerate}
  \vspace{-2mm}
\end{observation}
%
\begin{proof}
  \emph{Proof of (i).} Suppose some variable-domain frame $F$ does not have
  increasing domains. Then $F$ has a world $w$ whose domain $D_w$ contains an
  individual $d$ that does not exist at some $w$-accessible world $v$. Let $V$
  be an interpretation function on $F$ so that $V(F,w) = D_w$ and
  $V(F,v) = D_v$. In the model composed of $F$ and $V$, $\Box \forall x Fx$ is
  true at $w$, but $\forall x \Box Fx$ is false, since $d$ is not in $V(F,v)$.
  So \pr{CBF} is not true at all worlds in all models based on $F$.

  In the other direction, suppose \pr{CBF} is not valid on a frame $F$. This
  means that there is a world $w$ in some model $M$ based on $F$ at which some
  instance of $\Box \forall x A$ is true while $\forall x \Box A$ is false. If
  $\forall x \Box A$ is false at $w$, then there is some $w$-accessible world
  $v$ at which $A$ is false of some individual $d$ in $D_w$. But since
  $\Box \forall x A$ is true at $w$, $A$ is true of all members of $D_v$. So $d$
  is not in $D_v$. And so $F$ does not have increasing domains.

  The proof of (ii) is similar.\qed
  
\end{proof}

% In modal propositional logic, we saw that many interesting modal schemas
% corresponded to properties of the accessibility relation. In variable-domain
% semantics, interaction principles like \pr{BF} and \pr{CBF} correspond
% to frame conditions that link the accessibility relation with the domain of
% individuals.

% It is also easy to prove that (CBF) is valid in a VD frame iff (NE) is valid
% there. For (BF), the equivalent principle stating the necessity of
% non-existence cannot easily put in a single formula. But one can put it in
% terms of a schema: $\neg E!x \to \Box \neg E!x$, or contrapositively,
% $\Diamond E!x \to E!x$.

% In models with identity, the same can be proved for $\forall x \Box
% \exists y (y=x)$ in place of (CBF) and $\Diamond \exists x (x=y) \to
% \exists x (x=y)$ in place of (BF). This can be useful because unlike
% (BF) and (CBF) these are single formulas rather than schemas (see
% \cite[181f.]{fitting98first}).

% We've remarked that (CBF) is provable if we merge the axiomatic approach to
% first-order logic with K. $\Box (\forall x A \to A[c/x])$ is derivable from UI
% and Nec. If we want to hold on to classical rules, the most conservative
% variable domain systems are therefore the \emph{increasing domain systems}
% studied in \citey[ch.10]{hughes68introduction} (and \citey[15]{hughes96new} or
% e.g. \cite[44ff.]{gabbay76investigations}). (CBF) is valid in these systems,
% but not (BF). Recall that proving the Barcan Formula,
% $\forall x\Box A \to \Box\forall x A$, requires the Brouwerian axiom (B). So
% if we are willing to give up (B), we might consider relaxing the requirement
% of fixed domains. An \emph{expanding domain model} is a variable-domain model
% with the Increasing Domains requirement.

% \begin{exercise}
%   Suppose we interpret the domain of a constant domain model to
%   include all possible individuals (that is, everything that exists
%   relative to some possible world). It is then useful to introduce an
%   existence predicate $E$ that applies to an individual at a world
%   only if the individual exists at that world.
%   \begin{exlist}
%   \item Can you define an expression that functions like $E$ xxx
%   \item Show that it's free...?
%   \end{exlist}
% \end{exercise}

\begin{exercise}
  Definition \ref{def:variabledomainmodel} requires that every name in every
  model picks out a possible individual. In that sense, the definition does not
  allow for genuinely empty names. How could we change definitions
  \ref{def:variabledomainmodel} and \ref{def:variabledomainsemantics} if we
  wanted to allow for names that don't pick out anything?
\end{exercise}
\begin{solution}
  In the definition of a model, we could allow the interpretation function to be
  undefined for some names. We might also allow the sets $D_{w}$ to be empty. In
  the truth definition \ref{def:variabledomainsemantics}, we only need to
  clarify that $M,w,g \not\models A$ for every atomic sentence $A$ that contains a
  term $\tau$ for which $[\tau]^{M,g}$ is undefined.
\end{solution}

\section{Trans-world identity}\label{sec:twi}

In section \ref{sec:identity} I mentioned an apparent problem with Leibniz' Law.
The Law allows us to reason from $\Box Fa$ and $a\!=\!b$ to $\Box Fb$. On some
interpretations of the box, however, the inference looks problematic. In the
Superman stories, Lois Lane knows that Superman can fly, and Superman is
identical to Clark Kent. Can we infer that Lois knows that Clark Kent can fly?

If we can, we would have to conclude that Lois Lane has inconsistent beliefs,
since she also believes that Clark Kent \emph{cannot} fly. She would believe
that Clark Kent can't fly, but also that he can fly. Intuitively, however,
Lois's beliefs are perfectly consistent. What she lacks is information, not
logical acumen. Her belief worlds are not worlds at which someone can both fly
and not fly. Rather, they are worlds at which one person plays the Superman role
and a different person plays the Clark Kent role.

Consider also the case of Julius. When we introduce the name `Julius' for
whoever invented the zip, we can be sure that Julius invented the zip. But it
would be absurd to think that we have found out who invented the zip merely by
making a linguistic stipulation. If before introducing the name `Julius', we
were unsure whether the zip was invented by Benjamin Franklin or Whitcomb L.\
Judson, the introduction of the new name does nothing to remove our ignorance.
There are still epistemically accessible worlds at which the zip was invented by
Franklin and others at which it was invented by Judson. Knowing that Julius
invented the zip is not the same thing as knowing that Judson invented the zip,
even if in fact Julius = Judson.

Similar problems have been argued to arise in the logic of metaphysical
modality. Imagine a clay statue, standing on a shelf. Let's call it
Goliath. Since Goliath is made of clay, there is also a piece of clay on the
shelf, at the exact same spot as the statue. Let's call that piece of clay
Lumpl. How is Lumpl related to Goliath? We might want to say that they are one
and the same thing: Lumpl = Goliath. After all, there is only \emph{one}
statue-shaped object on the shelf, not two. But we might also want to say that
Lumpl could have had the shape of a bowl, while Goliath could not: if the clay
had been formed into a bowl rather than a statue, then Lumpl would have been a
bowl, but Goliath, the statue, would not have existed. Goliath is necessarily
not a bowl, but Lumpl is not necessarily not a bowl. We have $\Box \neg Bg$ but
not $\Box \neg Bl$, even though $l\!=\!g$.

\begin{exercise}
  Explain why the three examples I just presented also cast doubt on the
  ``necessity of identity'', $\forall x\forall y(x\!=\!y \to \Box\, x\!=\!y)$.
\end{exercise}
\begin{solution}
  In the Superman case, Clark Kent and Superman are the same person, but Lois
  Lane doesn't know that they are. So we appear to have $s\!=\!c$ but not
  $\Box\, s\!=\!c$. Similarly, in the Julius case, Julius and Whitcomb L.\ Judson
  are the same person, but one may well not know that they are. In the Goliath
  case, we have Lumpl = Goliath without it being metaphysically necessary that
  Lumpl = Goliath, as there are worlds in which Lumpl is a bowl and Goliath is
  not.
\end{solution}

Semantically, Leibniz' Law corresponds to the assumption that names are
\textbf{directly referential}, meaning that the only contribution a name makes
to the truth-value of a sentence is its referent. If names are directly
referential, and two names have the same referent, then it makes no difference
which of them we use: replacing one by the other never affects the truth-value
of a sentence.

So far, we have assumed direct reference in both constant-domain and
variable-domain semantics. On either account, names are interpreted as simply
picking out an individual. It is a matter of debate whether names in ordinary
language are directly referential. Some hold that Lois Lane really has
inconsistent beliefs. Others hold that Lois neither believes that Superman can
fly nor that Clark Kent cannot fly, because the objects of belief or knowledge
are never adequately represented by statements involving ordinary names. (This
also gets around the Julius problem.) With respect to Lumpl and Goliath, some
simply deny that Lumpl is identical to Goliath.

We will not descend into these debates. Instead, let's explore how we could
change our semantics for $\L_{M\!P}$ to block the relevant applications of
Leibniz' Law. There are several ways to achieve this. We will only look at one.

The approach we will explore drops the assumption that names are rigid. A name
is \textbf{rigid} if it picks out the same individual relative to any possible
world. Earlier, we assumed that no matter at which world the sentence $Fa$ is
interpreted, the name $a$ always picks out the same individual, $V(a)$. A name
like `Julius', however, seems to be non-rigid. It picks out different
individuals relative to different (epistemically) possible worlds. Relative to a
world where Benjamin Franklin invented the zip, `Julius' picks out Benjamin
Franklin. Relative to a world where Whitcomb L.\ Judson invented the zip, the
name picks out Whitcomb L.\ Judson.

Let's assume, then, that a model's interpretation function assigns an individual
to each name \emph{relative to each world}. This is equivalent to assuming that
each name is interpreted as expressing a \emph{function from worlds to
  individuals}, telling us which individual the name picks out relative to any
given world. Functions from worlds to individuals are known as
\textbf{individual concepts}, which is why the present approach is often called
\textbf{individual concept semantics}.

To motivate this label, return to Lois Lane. When Lois is thinking about
Superman, she is thinking about the audacious hero whose superhuman powers she
has witnessed on several occasions. When she is thinking about Clark Kent, she
is thinking about her shy and awkward colleague. Lois has distinct ``concepts''
for Superman and Clark Kent, one associated with the Superman role, the other
with the Clark Kent role. The two concepts actually pick out the same person
because one and the same person plays both the Superman role and the Clark Kent
role. We can model each of these roles as a function from worlds to individuals.
The Superman role is represented by a function that maps every world to whoever
plays the Superman role at that world. The Clark Kent role is represented by a
function that maps every world to whoever plays the Clark Kent role at that
world. For the world of the Superman stories, both functions return the same
individual. For Lois Lane's belief worlds, they return different individuals.

\begin{exercise}
  What individual concepts might be associated with the names `Lumpl' and
  `Goliath'?
\end{exercise}
\begin{solution}
  `Lumpl' might express a concept that maps every world $w$ to a certain piece
  of clay at $w$, where that piece is perhaps individuated by its matter or
  origin. The piece's shape doesn't matter. `Goliath' might instead express a concept that maps every world $w$ to a certain statue at $w$, where the statue is perhaps individuated by its shape and origin.
\end{solution}

We can easily convert our earlier constant-domain and variable-domain semantics
into an individual concept semantics. We first need  to change the definition of a
model, so that $V$ assigns individual concepts to names. In variable-domain
semantics, we might stipulate that an individual concept never maps a world to
an individual that doesn't exist at the world. We might also want to allow for
``partial concepts'': individual concepts that don't return any value for
certain worlds.

It is advisable to give a parallel treatment for names and variables. So we'll
also assume that an assignment function $g$ interprets each variable as
expressing an individual concept. In the truth definition, we replace
$[\tau]^{M,g}$, by $[\tau]^{M,w,g}$, which is defined as the referent of $\tau$
in $M$ \emph{at $w$}, relative to $g$. (That is, if $\tau$ is a name, then
$[\tau]^{M,w,g} = V(\tau)(w)$; if $\tau$ is a variable, then
$[\tau]^{M,w,g} = g(\tau)(w)$.) Finally, we adjust the definition of an
$x$-variant so that $g'$ is an $x$-variant of $g$ iff $g'$ differs from $g$ at
most in the individual concept it assigns to $x$.

The resulting logic of individual concepts has some unexpected features. For
example, all instances of the following schema become valid:
\[
  \Box \exists x A \to \exists x \Box A
\]
To see why, consider the instance $\Box \exists x Fx \to \exists x \Box Fx$.
Suppose the antecedent is true at some world in some model. This means that at
every accessible world $v$, there is at least one individual that is $F$. In
this case, there are functions that map every accessible world to some
individual that is $F$. Let $g'(x)$ be some such function. Relative to $g'$,
$\Box Fx$ is true at $w$. So $\exists x \Box Fx$ is true at $w$.

This is widely regarded as problematic. It would suggest that the two readings
of `something necessarily exists' are actually equivalent: it is necessary that
something or other exists just in case there is something that necessarily
exists.

% Also conceptually unsatisfactory that names effectively pick out trans-world
% individuals; ``intensional objects''. Note that when we're interpreting
% predicates, variables contribute regular "extensional" objects, members of the
% domain: g(x)(w) = I(w) = o. But quantifiers range over intensional objects.
% That seems odd. If we allow for intensional predicates (as one might think we
% should), we have gained nothing.

Another problematic feature of individual concept semantics is that the
resulting logic has no sound and complete proof procedure. There are no tree
rules, or natural deduction rules, or axioms and inference rules that would
allow proving all and only the sentences that are true at all worlds in all
models of individual concept semantics (no matter if we assume constant or
variable domains). It's not just that no-one has yet found a suitable proof
method. One can prove that no such method exists.

Both of these problems can be avoided by putting further constraints on models.
We have assumed that any function from worlds to individuals is a candidate
interpretation for a name or a variable. Relative to a given assignment
function, a variable may pick out Donald Trump in one world, the Eiffel tower in
another, a fried egg in a third, and so on. Ordinary concepts are not that
gerrymandered. We might therefore identify a certain subset of all individual
concepts as ``eligible'' for being expressed by names or variables. If this is
done sensibly, $\Box \exists x A \to \exists x \Box A$ becomes invalid, and
complete proof methods become available.

\begin{exercise}
  The following line of thought may be attributed to Descartes. ``I am certain
  that I exist, but not that my body exists. [After all, it could turn out that
  I am a disembodied soul.] Therefore: I am not my body.'' Translate the
  argument into $\L_{M\!P}$. Is it CK-valid? Is it VK-valid? Do you find it
  convincing?
\end{exercise}
\begin{solution}
  The premises are $\Box \exists x\, x\!=\!i$ and
  $\neg\Box \exists x\, x\!=\!b$. The conclusion is $i\!\not=\!b$. The argument
  is CK-valid and VK-valid. (It is not valid in individual concept semantics.)
\end{solution}

\begin{exercise}
  The following sentence sounds contradictory.
  \begin{quote}
    Some ticket will win, but I don't know if it will win.
  \end{quote}
  Translate the sentence into $\L_{M\!P}$. Explain why its apparent
  contradictoriness poses a problem for accounts on which variables are treated
  as directly referential.
\end{exercise}
\begin{solution}
  Translation: $\exists x (Tx \land Wx \land \neg K Wx \land \neg K \neg Wx)$,
  where $T$ translates `-- is a ticket' and `-- will win'.

  If variables are directly referential, then this sentence is true in any
  scenario in which I don't know which ticket will win.
\end{solution}

\begin{exercise}
  In individual concept semantics, both the necessity of identity and the
  necessity of distinctness are invalid. How could we change the semantics to
  make the necessity of identity valid, but not the necessity of distinctness?
  (Assume constant domains.)
\end{exercise}
\begin{solution}
  To render $\forall x \forall y (x\!=\!y \to \Box x\!=\!y)$ valid, we can
  restrict the eligible individual concepts in a model as follows. For any
  individual concepts $f$ and $g$ and worlds $w$ and $v$, if $wRv$ and
  $f(w)=g(w)$ then $f(v)=g(w)$. (We do not stipulate that if $wRv$ and
  $f(v)=g(v)$ then $f(w)=g(w)$, which would render the necessity of distinctness
  valid.)
\end{solution}

\iffalse

An alternative to individual concept semantics is \textbf{counterpart
  semantics}. Here the guiding idea is that when a name occurs in the scope of a
modal operator, then at the relevant accessible worlds it picks out whichever
individual resembles its actual referent in certain respects. Consider Lois
Lane's belief worlds -- the worlds in which things are as Lois believes they
are. In these worlds, Lois has a shy colleague called `Clark Kent' who can't
fly; there is also a superhero called `Superman' who can fly; the two are
different people. The shy colleague in Lois's belief worlds resembles the actual
Clark Kent (by which I mean the Clark Kent of the Superman stories) in certain
respects, which is why he is picked out by the name `Clark Kent' when that name
is interpreted relative to Lois's belief worlds. The superhero in Lois's belief
worlds resembles the actual Clark Kent in other respects, respects that are
associated with the name `Superman'. The difference between the two names is
that they invoke different criteria for trans-world resemblance.

% If we think of the object on the shelf as Lumpl, and we consider a world where a
% copper hat has been added, we judge that Lumpl is only the clay part of the
% resulting statue; if we think of the same object on the shelf as Goliath, we
% judge that it includes the copper hat in the counterfactual scenario. In
% counterpart semantics, different ways of tracking objects across worlds are
% represented by different \emph{counterpart relations}. The object on the shelf
% at our world stands in different counterpart relations to different objects at
% the world where the copper hat has been added.

If an individual at some world $v$ sufficiently resembles an individual at $w$
in relevant respects, then the first individual is called a \textbf{counterpart}
of the second (relative to $v$ and $w$, and relative to the given resemblance
criteria). The shy colleague in each of Lois Lane's belief worlds is a
counterpart of Clark Kent relative to some resemblance criteria; the superhero
in her belief worlds is a counterpart of Clark Kent relative to other criteria.

In counterpart semantics, we now assume that names are associated with an
individual, but also with a counterpart relation which determines how the name
should be interpreted when modal operators shift the point of evaluation to
other worlds. $\Box Fa$ is true at a world $w$ iff at every accessible world
$v$, every individual that stands in the $a$-relevant counterpart relation to
$a$ at $w$ is $F$.

Counterpart relations play a similar role as individual concept, but they are
more liberal. For example, we can allow for cases in which an individual has
multiple counterparts at an accessible world, relative to the same counterpart
relation. Suppose Bob lives next door to Alice, and has sometimes met her on the
stairwell. For some reason, Bob believes that Alice's flat is inhabited by
identical twins, so when he sees Alice, he thinks he sees one of the
twins. Alice has two counterparts in Bob's belief worlds, but the two
counterparts don't correspond to different roles or different similarity
standards.

Counterpart relations also needn't be transitive: a counterpart of a counterpart
of an individual need not itself be a counterpart of the individual. This might
help with the following puzzle about metaphysical modality.

Intuitively, my bicycle could have been composed of somewhat different parts. If
you exchange the lights or the seatposts on a bike, you aren't destroying the
old bike and creating a new one. So my bike could have had different lights, or
a different seatpost. On the other hand, arguably my bike could not have been
composed of \emph{entirely} different parts. A bike composed of entirely
different parts would be a different bike. Now let $b$ denote my actual bike;
let $F_1$ be a predicate that gives a detailed description of my bike as it
actually is. Let $F_2$ give a detailed description of a bike that is just like
mine except for the seatpost. We want to say that my bike could have fit that
description. So $\Diamond F_2b$ should be true. But as we make more and more
changes to $F_1$, we reach a point -- say $F_{32}$ -- where the description could
no longer have applied to my bike. So $\Diamond F_{32}b$ is false. But now
consider what would have been the case if $F_{31}b$ had been true. In that case,
it seems that $F_{32}b$ could have been true. After all, an $F_{32}$ bike differs
from an $F_{31}$ bike only by a single part. It would be strange if the bike in a
world where $F_{31}b$ is true could not possibly have had any different parts. So
while $\Diamond F_{32}b$ is false at the actual world, it seems to be true at any
world where $F_{31}b$ is true. Since $\Diamond F_{31}b$ is true at the
actual world, it follows that $\Diamond\Diamond F_{32}b$ is true as well. So we
have $\Diamond\Diamond F_{32}b$ but not $\Diamond F_{32}b$.

This is puzzling, because metaphysical possibility is often assumed to be an
absolute kind of possibility, with a universal accessibility relation. One would
then expect the logic of metaphysical possibility to be S5. Yet in S5,
$\Diamond\Diamond A$ entails $\Diamond A$.

In counterpart semantics, modal schemas like
$\Diamond \Diamond A \to \Diamond A$ correspond not just to properties of the
accessibility relation but to combined properties of the accessibility relation
and the counterpart relations. So one can explain what's going on in the puzzle
of the bike without giving up the assumption that the accessibility relation for
metaphysical modality is universal. $\Diamond \Diamond A \to \Diamond A$ is
invalid because a counterpart of a counterpart of my bike need not be a
counterpart of my bike.

\fi


%%% Local Variables: 
%%% mode: latex
%%% TeX-master: "logic2.tex"
%%% End:


\ifcompilesolutions
\chapter{Answers to the Exercises}\label{ch:answers}

\section*{Chapter \ref{ch:operators}}

\printsolutions[chapter=1]

\newpage
\section*{Chapter \ref{ch:worlds}}

\printsolutions[chapter=2]

\newpage
\section*{Chapter \ref{ch:accessibility}}

\printsolutions[chapter=3]

\newpage
\section*{Chapter \ref{ch:proofs}}

\printsolutions[chapter=4]

\newpage
\section*{Chapter \ref{ch:epistemic}}

\printsolutions[chapter=5]

\newpage
\section*{Chapter \ref{ch:deontic}}

\printsolutions[chapter=6]

\newpage
\section*{Chapter \ref{ch:time}}

\printsolutions[chapter=7]

\newpage
\section*{Chapter \ref{ch:conditionals}}

\printsolutions[chapter=8]

\newpage
\section*{Chapter \ref{ch:qml}}

\printsolutions[chapter=9]

\newpage
\section*{Chapter \ref{ch:qml2}}

\printsolutions[chapter=10]

%%% Local Variables: 
%%% mode: latex
%%% TeX-master: "ml.tex"
%%% End:

\fi


% {
% \small
% \bibliographystyle{../wobib2en}
% \bibliography{../bib}
% }



\end{document}
