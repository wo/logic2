\chapter*{Preface}\label{ch:preface}
\addcontentsline{toc}{chapter}{Preface}

These notes are aimed at philosophy students who have taken an introductory
course in formal logic. They provide an introduction to modal logic, with plenty
of philosophical applications. I also use the opportunity to introduce general
ideas that might be taught in an intermediate logic course: the concept of a
model, soundness and completeness, compactness, three-valued logics, free
logics, supervaluation, properties of relations and orders, etc.

Chapters 1--3 introduce the standard toolkit of modal propositional logic:
Kripke models, frame correspondence, some popular systems, the tableau method
and axiomatic calculi. Chapter 4 goes through soundness and completeness.
Chapters 5--8 turn to philosophical applications. Each of these chapters also
extends the toolkit from chapter 3. Chapter 5 introduces multi-modal logics,
chapter 6 ordering models and neighbourhood semantics, chapter 6 two-dimensional
semantics and supervaluationism, chapter 7 conditional logics and
Lewis-Stalnaker models. Chapters 9 and 10 then introduce some of the
complexities that arise in first-order modal logic.

Apart from chapter 9, which sets the stage for chapter 10, every chapter after
chapter 3 can be skipped or skimmed without affecting the accessibility of later
chapters.

The best way to learn logic is by solving problems. That's why the text is
frequently interrupted by exercises. As a student, you should try to do the
exercises as soon as you reach them, before continuing with the text.

\bigskip
--- Wolfgang Schwarz, \today.


% And remember: It is more effective to learn a little every day than to cram 

% ``If you rely on passively rereading your course materials, you’ll tend to end up using your memory to try to reproduce the author’s understanding of the subject rather than generating your own. So, what is the best catalyst for generating your own understanding of what you read? The answer is to question what you read as you’re reading it.''

% Don't cram. space out.

% - focus is on understanding, not on rote learning of rules


%%% Local Variables: 
%%% mode: latex
%%% TeX-master: "ml.tex"
%%% End:
