\chapter{Towards Modal Predicate Logic}\label{ch:qml}

% Should I do this more like week9.org? Move the whole "modal logic as fragment"
% section into an exercise?

\section{Predicate logic recap}

In these last two chapters, we are going to add the resources of first-order
predicate logic to those of propositional modal logic. Let's begin by reviewing
the syntax and semantics of classical, non-modal predicate logic.

The language $\L_P$ of first-order predicate logic consists of \emph{predicates}
$F^{0},F^1,F^2,\ldots,$ $G^{0},G^1,G^2,\ldots$, \emph{individual constants} (or
\emph{names}) $a,b,c,\ldots$, \emph{individual variables} $x,y,z,\ldots$, the
logical symbols $\neg$, $\land$, $\lor$, $\to$, $\leftrightarrow$, $\forall$,
$\exists$, and the parentheses $($ and $)$. Individual variables and constants
are also called \emph{(singular) terms}.

Atomic sentences of $\L_{P}$ are formed by conjoining a predicate with zero or
more terms. Each predicate takes a fixed number of terms, as indicated by its
numerical superscript: $F^1$ is a \emph{one-place} predicate that combines with
one term to form a sentence, $F^2$ is \emph{two-place}, and so on. In practice,
we usually omit the superscripts, because context makes clear what kind of
predicate is in play. $Fa \lor Gab$, for example, is well-formed only if $F$ is
one-place and $G$ two-place.

In English, a predicate is what is what you get when you remove all names from a
sentence. Removing `Bob' from `Bob is hungry' yields the predicate `-- is
hungry'. From `Bob is in Rome', we get the two-place predicate `-- is in --'.
From `Bob saw Carol's father in Jerusalem', we could get the
three-place-predicate `-- saw --'s father in --'. When we translate from
English, we normally translate English names into $\L_P$-names and (logically
simple) English predicates into $\L_P$-predicates. `Bob is in Rome' might become
$Fab$, where $a$ translates `Bob', $b$ `Rome', and $F$ `-- is in --'.

From atomic sentences, complex sentences are formed in the usual way by means of
the truth-functional operators $\neg$, $\land$, $\lor$, $\to$,
$\leftrightarrow$.

Another way to construct a complex sentence from a simpler sentence is to add a
quantifier in front of the simpler sentence. A \emph{quantifier} is an
expression of the form $\forall \chi$ or $\exists \chi$, where $\chi$ is some
variable. A quantifier is said to \emph{bind} the variable it contains:
$\forall x$ binds $x$, $\exists y$ binds $y$, and so on.

In English, quantifier expressions are usually restricted to a particular
subclass of the things under discussion: `\emph{all whales} are mammals',
`\emph{some students} went home'. The $\L_{P}$-quantifiers $\forall x$ and
$\exists x$ are unrestricted. They roughly correspond to `everything is such
that \ldots' and `something is such that \ldots'. We can translate restricted
quantifiers by combining unrestricted quantifiers with truth-functional
connectives. `All whales are mammals' is equivalent to `Everything is either not
a whale or a mammal'; so it can be translated as $\forall x(Wx \to Mx)$. `Some
students went home' could be translated as $\exists x (Sx \land Hx)$.

Variables are book-keeping devices. They function somewhat like pronouns in
English. $\exists x(Sx \land Hx)$ might be read as `something is such that
\emph{it} is a student and \emph{it} went home'. By using different variables
($x,y,z,\ldots$), we can disambiguate statements with nested quantifiers.
Consider
\begin{quote}
  Every dog barked at a tree.
\end{quote}
This can mean that there is a particular tree at which all the dogs barked, but
it can also mean that each dog found some tree to bark at -- possibly different
trees for different dogs. The first reading could be translated as
\[
  \exists y (Ty \land \forall x (Dx \to Bxy)),
\]
the second as
\[
  \forall x (Dx \to \exists y(Ty \land Bxy)).
\]

Some more terminology. Recall that the \emph{scope} of an operator (token) in a
sentence is the shortest well-formed subsentence in which it occurs. In
$\exists y (Ty \land \forall x(Fx \to Bxy))$, the scope of the quantifier
$\forall x$ is the subsentence $\forall x(Fx \to Bxy)$. If an occurrence of a
variable lies in the scope of a quantifier that binds the variable, then the
occurrence is called \emph{bound}, otherwise it is \emph{free}. In
$\forall x(Fx \to Bxy)$, all occurrences of $x$ are bound, but $y$ is free.

A sentence containing free variables is called \emph{open}. Sentences that aren't
open are \emph{closed}. Intuitively, only closed sentences make complete
statements. For this reason, some authors reserve the word `sentence' for closed
sentences, referring to open sentences as `formulas'. (Others call every
$\L_P$-sentence a `formula'.)

\begin{exercise}
  Translate the following sentences into $\L_P$.
  \begin{exlist}
  \item Keren and Keziah are sisters of Jemima.
  \item All myriapods are oviparous.
  \item Fred has a new car.
  \item Not every student loves logic.
  \item Every student who loves logic loves something.
  \end{exlist}
\end{exercise}
\begin{solution}
  \begin{sollist}
  \item $Srj \land Skj$; \qquad $r$: Keren, $k$: Keziah, $j$: Jemima, $S$: -- is a sister of --
  \item $\forall x (Mx \to Ox)$; \qquad $M$: -- is a myriapod, $O$: -- is oviparous 
  \item $\exists x (Cx \land Nx \land Hfx)$; \qquad $f$: Fred, $C$: -- is a car, $N$: -- is new, $H$: -- has --
  \item $\neg\forall x (Sx \to Lxl)$; \qquad $l$: logic; $S$: -- is a student, $L$: -- loves --
  \item $\forall x ((Sx \land Lxl) \to \exists y Lxy)$; \qquad $l$: logic; $S$: -- is a student, $L$: -- loves --
  \end{sollist}
\end{solution}

Like sentences of modal propositional logic, sentences of predicate logic are
interpreted relative to a model. A model of predicate logic first of all
specifies an \emph{individual domain} $D$ over which the quantifiers are said to
range. If we read $\forall x$ as `everything is such that' and $\exists x$ as
`something is such that' then the relevant ``somethings'' are the members of the
domain $D$.

The remainder of a model is an \emph{interpretation function $V$} that assigns
\begin{itemize}[leftmargin=10mm]
  \itemsep-1mm
  \item[(a)] to each name a member of $D$,
  \item[(b)] to each zero-place predicate a truth-value,
  \item[(c)] to each one-place predicate a subset of $D$, and
  \item[(d)] to each $n$-place predicate with $n>1$ a set of $n$-tuples from
        $D$.
\end{itemize}
An ``$n$-tuple from $D$'' is simply a list of length $n$, all elements of which
are in $D$. Repetitions are allowed, so if Bob is a member of $D$, then
$\t{\text{Bob, Bob}}$ counts as a 2-tuple from $D$. (2-tuples are more commonly
called \emph{pairs}.) We can subsume condition (c) under condition (d) by
assuming that a 1-tuple from $D$ is a member of $D$. We can subsume (b) under
(d) by identifying the truth-value False with the empty tuple $\emptyset$ and
the truth-value True with $\{ \emptyset \}$. (Don't worry if you find this confusing or objectionable. We won't be using zero-ary predicates.)

% V(p) = 1 means V(p) is true of all zero-tuples

\begin{definition}{}{predicatemodel}
  A \textbf{(classical) first-order model} is a pair $\t{D,V}$ consisting of
  \vspace{-3mm}
  \begin{itemize*}
    \item a non-empty set $D$, and
    \item a function $V$ that assigns to each name a member of $D$ and to each
    $n$-place predicate a set of $n$-tuples from $D$.
  \end{itemize*}
\end{definition}

As always, the purpose of a model is to represent a conceivable scenario together
with an interpretation of the non-logical vocabulary. The non-logical vocabulary
of $\L_P$ are the names and predicates, which is why these are interpreted by
$V$. 

We assume that in any relevant scenario there are some things we want to talk
about; these things are represented by the domain. The members of $D$ are often
called \emph{individuals}, but this should not be taken to imply anything about
their nature. An individual might be a rock, a person, a symphony, a sentence, a
number, or a possible world. Every $\L_P$-name is assumed to pick out one of
these individuals. (Different names can pick out the same individual, and there
can be individuals that aren't picked out by any name.)

Intuitively, a predicate expresses a property or relation that may be
instantiated by the individuals in the domain. In order to determine the
truth-value of a sentence like $Fa$ or $\exists x Fx$ in a given scenario,
however, we only need to know which individuals in the domain have the property
expressed by $F$. Similarly, to determine the truth-value of sentences like
$Rab$ or $\forall x \exists y Rxy$, we only need to know which pairs of
individuals stand in the relation expressed by $R$. That's why the
interpretation function in a first-order model simply assigns sets of
individuals or $n$-tuples of individuals to predicates. $Fa$ is true in a given
model iff the individual assigned to $a$ (in the model) is a member of the set
assigned to $F$; that is, iff $V(a) \in V(F)$. Likewise, $Rab$ is true in a
model iff the pair of individuals assigned to $a$ and $b$ -- the pair
$\t{V(a),V(b)}$ -- is in the set assigned to $R$.

In this way, the truth-value of every closed atomic sentences is determined. For
truth-functionally complex sentences, the standard rules apply: a negated
sentence $\neg A$ is true iff the corresponding sentence $A$ is not true;
$A \land B$ is true iff $A$ and $B$ are both true; and so on.

When we turn to quantified sentences, we face a problem. We can't define the
truth-value of $\forall x Fx$ in terms of the truth-value of $Fx$, because an
open sentence like $Fx$ doesn't have a truth-value. Interpretation functions
interpret names and predicates; they say nothing about variables. Even if we
changed this and said that $x$ should also be interpreted as picking out a
member of the domain, we would have to ignore this interpretation if we
evaluate $\forall x Fx$. We want $\forall x Fx$ to be true iff $Fx$ is true
\emph{no matter which individual is assigned to $x$}. We therefore define truth
not just relative to a model, but relative to a model \emph{and an assignment of
  individuals to variables}.

To illustrate, consider a model with just two individuals, Alice and Bob, which
are picked out by the names $a$ and $b$ respectively. Let $V(F)$ be the set $\{$
Alice $\}$, a set that only contains Alice. So $Fa$ is true and $Fb$ false. The
sentence $Fx$ is neither true nor false, for the variable $x$ does not refer to
any particular individual. All we can say is that $Fx$ is ``true of'' Alice and
``false of'' Bob. That is, $Fx$ is true if we assign Alice to $x$ and false if
we assign Bob to $x$. $\exists x Fx$ is true because there is an individual
(Alice) of which $Fx$ is true. Equivalently, $\exists x Fx$ is true because
there is some assignment of individuals to variables relative to which $Fx$ is
true. $\forall x Fx$ is false because it is not the case that every assignment
of individuals to variables renders $Fx$ true.

So we'll define truth relative to a model $M = \t{D,V}$ and a variable
assignment $g$. A \emph{variable assignment} is a function that maps variables
to members of $D$. If we have nested quantifiers, as in
$\forall x \exists y Gxy$, we need to consider variable assignments that differ
from other assignments with respect to a particular variable.
$\forall x \exists y Gxy$ is true iff, no matter what individual is assigned to
$x$, there is some assignment of an individual to $y$ (but holding fixed the
assignment to $x$) that makes $Gxy$ true. Equivalently: $\forall x\exists y Gxy$
is true iff for every variable assignment $g$, there is some variable assignment
$g'$ that differs from $g$ at most in what it assigns to $y$ such that $Gxy$ is
true relative to $g'$.

Let's say that (for any variable $\chi$) a variable assignment $g'$ is an
\emph{$\chi$-variant} of a variable assignment $g$ iff $g'$ differs from $g$ at
most in the value it assigns to $\chi$. Let's also introduce $[\tau]^{M,g}$ as
shorthand for the individual picked out by a term $\tau$ in a model
$M = \t{D,V}$ relative to assignment $g$:
\[
  [\tau]^{M,g} =_\text{def} \begin{cases} \;V(\tau) & \text{ if $\tau$ is a name}\\
    \;g(\tau) & \text{ if $\tau$ is a variable}.
  \end{cases}
\]
This is a compact way of saying that (1) for any variable $\chi$, $[\chi]^{M,g}$
is the individual assigned to $\chi$ by $g$, and (2) for any name $\eta$,
$[\eta]^{M,g}$ is the individual assigned to $\eta$ by the interpretation function
of $M$.

Now we can state the standard semantics of first-order predicate logic.
(`$M,g \models A$' is pronounced `$A$ is true in $M$ relative to $g$').

\begin{definition}{Semantics of first-order predicate logic}{predicatesemantics}
  If $M = \t{D,V}$ is a first-order model, $\phi^{n}$ is an $n$-place predicate
  (for $n\geq 0$), $\tau_1,\ldots,\tau_n$ are terms, $\chi$ is a variable, and
  $g$ is a variable assignment, then
  
  \medskip
  \begin{tabular}{lll}
    (a) & $M,g \models \phi^{n} \tau_1\ldots \tau_n$ &iff $\t{[\tau_1]^{M,g},\ldots,[\tau_n]^{M,g}} \in V(\phi)$.\\
    % (b) & $M,g \models \tau_1=\tau_2$ &iff $[\tau_1]^{M,g} = [\tau_2]^{M,g}$.\\
    (b) & $M,g \models \neg A$ &iff $M,g \not\models A$.\\
    (c) & $M,g \models A \land B$ &iff $M,g \models A$ and $M,g \models B$.\\
    (d) & $M,g \models A \lor B$ &iff $M,g \models A$ or $M,g \models B$.\\
    (e) & $M,g \models A \to B$ &iff $M,g \not\models A$ or $M,g \models B$.\\
    (f) & $M,g \models A \leftrightarrow B$ &iff $M,g \models A\to B$ and $M,g \models B\to A$.\\
    (g) & $M,g \models \forall \chi A$ &iff $M,g' \models A$ for all $\chi$-variants $g'$ of $g$.\\
    (h) & $M,g \models \exists \chi A$ &iff $M,g' \models A$ for some $\chi$-variant $g'$ of $g$.
  \end{tabular}
\end{definition}

Clause (a) says that, for example, $Fa$ is true in a model $M$ relative to an
assignment $g$ iff in that model, the predicate $F$ applies to the individual
picked out by $a$. Clauses (b)-(f) say that the truth-functional operators are
interpreted in the standard fashion. Clauses (g) and (h) tell us how quantified
sentences are interpreted. $\exists x Fx$, for example, is true relative to $M$
and $g$ iff $Fx$ is true relative to some assignment function $g'$ that differs
from $g$ at most in what it assigns to $x$.

Definition \ref{def:predicatesemantics} settles the truth-value of every
$\L_P$-sentence in every (first-order) model, relative to any assignment
function.

We can also define a concept of truth relative to a model, without reference to
an assignment function. Let's say that an $\L_P$-sentence is \textbf{true in a
  model} $M$ iff it is true in $M$ relative to \emph{every} assignment function
$g$ for $M$.

Finally, we say that an $\L_P$-sentence is \textbf{valid} (in classical
first-order logic) iff it is true in all (classical, first-order)
models. Equivalently: An $\L_P$-sentence is valid iff it is true in all models
relative to all assignment functions.

On the present definition, $Fx \to Fx$ is valid, even though it does not make a
complete statement, due to the free variable $x$. To avoid this, some authors
restrict the concept of validity to closed sentences.

% We will do so below, because our tree rules don't allow proving Fx v ~Fx. But
% we don't officially commit to it here. The problem is that if we don't define
% validity for open sentences then modal logic isn't a fragment of predicate
% logic: Fx v \neg Fx should be valid.

\begin{exercise}
  Define a first-order model in which $\exists x Fx \to \forall x Fx$ is
  false. Demonstrate that the sentence is false in your model by applying all
  relevant clauses from definition \ref{def:predicatesemantics}.
\end{exercise}
\begin{solution}
  Let the model $M$ be given by $D = \{ \text{Rome}, \text{Paris} \}$ and
  $V(F) = \{ \text{Rome} \}$. By clause (a) of definition
  \ref{def:predicatesemantics}, $M,g' \models Fx$ holds for every assignment
  function $g'$ that maps $x$ to Rome, because then $g'(x) \in V(F)$. By clause
  (h) it follows that $M,g \models \exists x Fx$ for every assignment function
  $g$. By clause (a) again, $M, g' \not\models Fx$ for every assignment function
  $g'$ that maps $x$ to Paris. By clause (g), it follows that
  $M,g \not\models \forall x Fx$ for every assignment function $g$. So
  $\exists x Fx$ is true (in $M$) relative to every assignment function while
  $\forall x Fx$ is false relative to every assignment function. By clause (e)
  it follows that $\exists x Fx \to \forall x Fx$ is false in $M$ relative to
  every assignment function.
\end{solution}

\begin{exercise}
  The definition of truth in a model uses the method of supervaluation
  that we met in section \ref{sec:branching}. Give examples to illustrate the
  following claims.
  
  \begin{exlist}
  \item If a sentence $A$ is not true in a model, it does not follow that
    $\neg A$ is true in the model.
  \item A disjunction $A \lor B$ can be true in a model even though neither $A$
    nor $B$ is true in the model. 
  \end{exlist}
\end{exercise}
\begin{solution}
  For both cases, use $Fx$ as the sentence $A$, and $\neg Fx$ as $B$, and
  consider a model in which $F$ applies to some but not to all individuals. Both
  $Fx$ and $\neg Fx$ are then true relative to some assignment functions and
  false relative to others. So neither sentence is true in the model. But
  $Fx \lor \neg Fx$ is true relative to every assignment function.
\end{solution}

\section{Modal fragments of predicate logic}
\label{sec:fragment}

Much of the power and complexity of predicate logic comes from its ability to
handle nested quantifiers with different variables. For some applications, these
complexities aren't needed, and we can simplify the semantics.

Consider a fragment $\L_P^1$ of $\L_P$ with only one variable $x$, no names, and
only one-place predicates. In $\L_P^1$, we have sentences like $Fx$,
$\forall x Gx$, $\forall x \exists x (Fx \to Gx)$, but not $Fa$ or
$\forall x \exists y(Fx \to Gy)$.

Following definition \ref{def:predicatemodel}, a model for $\L_P^1$ consists of
a non-empty set $D$ and an interpretation function $V$ that assigns to each
predicate a subset of $D$. That is, for $\L_{P}^{1}$ definition
\ref{def:predicatemodel} can be simplified as follows:

\begin{justabox}
  A \textbf{model of $\L_P^1$} is a pair $\t{D,V}$
  consisting of \vspace{-1mm}
  \begin{itemize*}
  \item a non-empty set $D$, and
  \item a function $V$ that assigns to every $\L_P^1$-predicate a subset of $D$.
  \end{itemize*}
\end{justabox}

We can also simplify definition \ref{def:predicatesemantics}. Since $\L_P^1$ has
only one variable $x$, an assignment function for $\L_P^1$ only needs to tell us
which individual in $D$ is picked out by $x$. So we can represent an entire
assignment function for $\L_P^1$ by a member of $D$. This leaves us with the
following semantics.

\begin{justabox}
  If $M = \t{D,V}$ is a model for $\L_P^1$, $d$ is a member of $D$, and $\phi$
  is an $\L_{P}^{1}$-predicate, then
  
  \medskip
  \begin{tabular}{lll}
    (a) & $M,d \models \phi x$ &iff $d \in V(\phi)$.\\
    (b) & $M,d \models \neg A$ &iff $M,d \not\models A$.\\
    (c) & $M,d \models A \land B$ &iff $M,d \models A$ and $M,d \models B$.\\
    (d) & $M,d \models A \lor B$ &iff $M,d \models A$ or $M,d \models B$.\\
    (e) & $M,d \models A \to B$ &iff $M,d \not\models A$ or $M,d \models B$.\\
    (f) & $M,d \models A \leftrightarrow B$ &iff $M,d \models A\to B$ and $M,d \models B\to A$.\\
    (g) & $M,d \models \forall x A$ &iff $M,d' \models A$ for all $d' \in D$.\\
    (h) & $M,d \models \exists x A$ &iff $M,d' \models A$ for some $d' \in D$.
  \end{tabular}
\end{justabox}

These definitions look a lot like definitions \ref{def:basicmodel} and
\ref{def:basicsemantics} from chapter \ref{ch:worlds}. The only difference is
that the sentence letters from chapter \ref{ch:worlds} are now called predicates
and written in uppercase, the box is written $\forall x$, the diamond
$\exists x$, and we always append the letter $x$ to sentence letters: we write
$\forall x Fx$, not $\forall x F$. But it doesn't really matter how a symbol is
called or how it is written.

The upshot is that propositional modal logic, interpreted as in chapter
\ref{ch:worlds}, can be regarded as a disguised \emph{fragment of first-order
  predicate logic}. The sentence letters of $\L_M$ are disguised (one-place)
predicates, the box and the diamond are disguised quantifiers. If we adopted the
orthographic convention to write the box as $\forall x$, the diamond as
$\exists x$, and to always append the letter $x$ to (capitalised) sentence
letters, $\L_M$ would look just like $\L_P^1$, and it would have the same
semantics.

If we use chapter \ref{ch:accessibility}'s Kripke semantics rather than the
simple semantics from chapter \ref{ch:worlds} to interpret $\L_{M}$, we get a
different fragment of first-order predicate logic. The box and the diamond are
still disguised quantifiers, but this time they are restricted by the
accessibility relation. We could drop the disguise by writing $\Box p$ as
$\forall y(Rxy \to Py)$ and $\Diamond p$ as $\exists y(Rxy \land Py)$. The
fragment of $\L_P$ that now corresponds to $\L_M$-sentences has two variables
$x$ and $y$ and one two-place predicate `$R$' in addition to the one-place
predicates; it no longer has unrestricted quantifiers.

What's the point of the disguise? Why didn't we write boxes and diamonds as
$\L_P$-quantifiers all along? There are several reasons.

One is that we often use the box and the diamond to formalize pre-theoretic
concepts of which it is not obvious that they can be understood as a quantifiers
over worlds. Some hold that the correct semantics for obligation and permission,
for example, is not Kripke semantics, but neighbourhood semantics. The language
of modal propositional logic is neutral on this disagreement. Or think of
provability logic, where the box formalizes mathematical provability. As it
turns out, one can give a Kripke semantics for provability, but nobody thinks
that this somehow reveals what provability really means. In provability logic,
$\Box A$ means that $A$ is derivable from the axioms and rules of (say) ZFC; it
would not be illuminating to write this as $\forall y(Rxy \to Ay)$.

One might also argue that the syntax of modal logic conveniently resembles the
surface form of English statements that we may want to formalize. In `Bob knows
that it is raining', for example, the object of Bob's knowledge is specified by
`it is raining'. It seems appropriate to formalize the sentence in terms of an
operator $\Kn$ that applies to a sentence, $p$. If we ``dropped the disguise'',
the formalization would be $\forall y(Rxy \to Py)$. The sentence `it is raining'
would have to be translated by a predicate $P$ -- a predicate that applies to
all and only the worlds at which it is raining.

There is a deeper point here. Sentences of modal logic are interpreted \emph{at
  a world} in a model. Modal logic looks at models ``from the inside'', from the
perspective of a particular world. Predicate logic, by contrast, describes
models ``from the outside'', from a God's eye perspective. If we want to say
that a particular individual has a property $P$ in predicate logic, we need to
pick out that individual among all the elements of the domain, perhaps by a
name. We can then say $Pa$. In modal logic, we can simply say $p$ to express
that the internal point from which we're looking at the model has the relevant
property.

For many applications, this internal perspective is natural. If we think
about what is possible or  what the future will bring, our thinking takes
place at a particular time, in a particular world. We are looking at the
structure of times and worlds from the inside. When I say that it is raining, I
mean that it is raining \emph{here and now}, in \emph{this world}. I don't need
to pick out the relevant time and place and world from a God's eye perspective.
I can pick them out simply as the time and place and world at which I currently
find myself.

There are other, more pragmatic reasons to use the modal language $\L_M$ rather
than $\L_P$. The language of boxes and diamonds is simpler than the language of
first-order predicate logic. It has a simpler syntax, a simpler semantics, and
allows for simpler proofs. For almost all the conceptions of validity we have
studied (K-validity, S4-validity, etc.), there are efficient mechanical
procedures to determine whether an arbitrary $\L_M$-sentence is valid or
invalid, By contrast, there is no mechanical procedure at all to determine, for
an arbitrary $\L_P$-sentence, whether it is valid or invalid.

You may wonder how this is possible given that $\L_M$-sentence are just
$\L_P$-sentences in disguise. The reason is that while every $\L_M$-sentence is
a disguised $\L_P$-sentence, not every $\L_P$-sentence can be disguised as an
$\L_M$-sentence. There are many things one can say in $\L_P$ that can't be said
in $\L_M$. The $\L_{P}$-sentence $\forall x Rxx$, for example, states that $R$
is reflexive. No sentence of $\L_M$ has this meaning: there is no
$\L_{M}$-sentence that is true at a world in a model iff the model's
accessibility relation is reflexive.

That's why modal propositional logic, interpreted as in chapter \ref{ch:worlds}
or \ref{ch:accessibility}, is a disguised \emph{fragment} of predicate logic. It
is a simple and computationally attractive fragment that takes an ``internal''
perspective on models.

\begin{exercise}
  Since $\Box A \to A$ corresponds to reflexivity, one might think that
  $\Box p \to p$ is true at a world in a model iff the model's accessibility
  relation is reflexive. (a) Explain why this is not correct. (b) Can you 
  show that there is no $\L_{M}$-sentence that is true at a world in a model iff
  the model's accessibility is reflexive?
\end{exercise}
\begin{solution}
  (a) There are many non-reflexive models in which $\Box p \to p$ is true at some
  world -- for example, any non-reflexive model in which $p$ is false at all
  worlds.

  (b) Let $M_1$ be a model with a single world that can see itself. Let $M_{2}$
  be a model with two worlds, each of which can see the other but not itself. In
  both models, all sentence letters are false at all worlds. The very same
  $\L_M$-sentences are true at all worlds in these models (as a simple proof by
  induction shows). But the first model is reflexive and the second isn't. So
  there is no $\L_{M}$-question that is true at a world in a model iff the
  model's accessibility relation is reflexive.
\end{solution}

% \begin{exercise}
%   Can you give another example of an $\L_P$-sentence (about a Kripke model) for
%   which there is no equivalent $\L_M$-sentence?
% \end{exercise}
% \begin{solution}
%   For example: $\forall w \neg Rww$ states that $R$ is reflexive; as the answer
%   to the previous exercise shows, this can't be expressed in $\L_M$. 
% \end{solution}

% \begin{exercise}
%   standard translation? 
% \end{exercise}


\section{Predicate logic proofs}

If we want to know whether an $\L_{P}$-sentence is valid or invalid, we
could in principle work through definition \ref{def:predicatesemantics}. Various
proof systems for classical predicate logic offer a more streamlined approach.

Let's look at the tree method for classical predicate logic. Suppose we want to
test whether $\exists x(Fx \land Gx) \to \exists x Fx$ is valid. As always, we
start the tree with the negation of the target sentence:

\medskip
\begin{center}
  \tree{
    \nnode{26}{1.}{$\neg(\exists x (Fx \land Gx) \to \exists x Fx)$}{}{(Ass.)}
  }
\end{center}
\medskip%
There is no world label because we're not doing modal logic. Next, we apply the
standard rule for negated conditionals:%
\medskip
\begin{center}
  \tree{
    \nnode{26}{2.}{$\exists x (Fx \land Gx)$}{}{(1)}\\
    \nnode{26}{3.}{$\neg\exists x Fx$}{}{(1)}
  }
\end{center}

\medskip\noindent
Node 2 says that $Fx \land Gx$ is true of some individual. To
expand this node, we introduce a new name $a$ for that individual, and infer
$Fa \land Ga$.
\medskip
\begin{center}
  \tree{
    \nnode{26}{4.}{$Fa \land Ga$}{}{(2)}\\
  }
\end{center}

\medskip\noindent%
We expand the conjunction on node 4.
\medskip
\begin{center}
  \tree{
    \nnode{26}{5.}{$Fa$}{}{(4)}\\
    \nnode{26}{6.}{$Ga$}{}{(4)}
  }
\end{center}
Next, we expand node 3, which says that $Fx$ is true of nothing. In particular
then, $Fx$ can't be true of $a$. So we add $\neg Fa$:

\medskip
\begin{center} \tree{
    \nnodeclosed{26}{7.}{$\neg Fa$}{}{(3)}
  }
\end{center}

\medskip\noindent%
The tree is closed because the sentence on node 7 is the negation of the
sentence on node 5. The target sentence is valid.

To state the general rules, we need some more notation. If $A$ is a sentence and $\tau_{1}, \tau_{2}$ terms, let $A[\tau_{2}/\tau_{1}]$ be the sentence
obtained from $A$ by replacing all free occurrences of $\tau_{1}$ with $\tau_{2}$. So
$Fx[a/x]$ is $Fa$, but $\forall x Fx[a/x]$ is $\forall x Fx$ because this
sentence contains no free occurrences of $x$.

The general rule for expanding nodes of type $\exists \chi A$ is that you add a
node $A[\eta/\chi]$, where $\eta$ is a ``new'' name that does not already occur
on the relevant branch. If such a node has been added to every open branch below
$\exists \chi A$ then the $\exists \chi A$ node can be ticked off.
$\forall \chi A$ nodes can be expanded multiple times, once for each ``old''
name. So if $\forall x A$ occurs on a branch, and the branch contains the names
$a$ and $b$ then we can add both $A[a/x]$ and $A[b/x]$. If there is no old name
on a branch, we are allowed to expand $\forall \chi A$ with a new name.
$\forall \chi A$ nodes are never ticked off.

Here is a summary of the quantifier rules; `old or first' means that the
relevant name either already occurs on the branch or it is introduced as the
first name on the branch.

\bigskip

\begin{minipage}{0.24\textwidth} \centering
\tree{
  \dotbelownode{12}{}{$\forall \chi A$}{}{}\\
  \\
  \nnode{12}{}{$A[\eta/\chi]$}{}{}\\
  \Kk[-1]{0}{\color{red}$\uparrow$}\\
  \Kk[0]{0}{\color{red}\small old or first}
}
\end{minipage}
\begin{minipage}{0.24\textwidth}\centering
\tree{
  \dotbelownode{12}{}{$\exists \chi A$}{}{}\\
  \\
  \nnode{12}{}{$A[\eta/\chi]$}{}{}\\
  \Kk[-1]{0}{\color{red}$\uparrow$}\\
  \Kk[0]{0}{\color{red}\small new}
}
\end{minipage}
\begin{minipage}{0.24\textwidth}\centering
\tree{
  \dotbelownode{12}{}{$\neg \forall \chi  A$}{}{}\\
  \\
  \nnode{12}{}{$\neg A[\eta/\chi]$}{}{}\\
  \Kk[0]{0}{\color{red}$\uparrow$}\\
  \Kk[1]{0}{\color{red}\small new}
}
\end{minipage}
\begin{minipage}{0.24\textwidth} \centering
\tree{
  \dotbelownode{12}{}{$\neg \exists \chi A$}{}{}\\
  \\
  \nnode{12}{}{$\neg A[\eta/\chi]$}{}{}\\
  \Kk[0]{0}{\color{red}$\uparrow$}\\
  \Kk[1]{0}{\color{red}\small old or first}
}
\end{minipage}

\medskip

If you want to read off a countermodel from an open branch, you can simply take
the domain $D$ to consist of all names that occur on the branch. For the
interpretation function $V$, you then stipulate that each name picks out itself
-- so that, for example, $V(a) = a$ -- and that a predicates $P$ applies to a
tuple of names $\t{a,b,\ldots}$ iff $Pab\ldots$ occurs on the branch.

\begin{exercise}
  Give tree proofs for the following sentences.
  \begin{exlist}
  \item $\forall x Fx \to Fa$
  \item $\forall x (Fx \to Gx) \to (\forall x Fx \to \forall x Gx)$
  \item $\forall x (Fx \land Gx) \leftrightarrow (\forall x Fx \land \forall x Gx)$
  \item $\exists x\forall y Gxy \to \forall y \exists x Gxy$
  \item $\exists y \forall x(Fy \to Fx)$
  \end{exlist}
\end{exercise}
\begin{solution}
  Use \href{https://www.umsu.de/trees/}{umsu.de/trees/}.
\end{solution}

There are also axiomatic calculi for predicate logic. We can, for example, use
the following axiom schemas:
%
\begin{principles}
  \pri{$\forall\exists$}{\neg\exists \chi A \leftrightarrow \forall \chi \neg A}\\
  \pri{UI}{\forall \chi A \to A[\eta/\chi]}\\
  \pri{DI}{\forall \chi (A \to B) \to (A \to \forall \chi B),\text{ if $\chi$ is not free in $A$}}
\end{principles}
%
To these we would add the following rules. As in earlier chapters, $\Gamma \models_{0} A$ means that $A$ is a truth-functional consequence of (the sentences in) $\Gamma$.
\begin{principles}
  \pri{CPL}{\text{If }\Gamma \models_{0} A\text{ and all members of }\Gamma\text{ are on a proof, then one may add $A$.}}\\
  \pri{Gen}{\text{If $A$ occurs on a proof, then one may add $\forall \chi A[\chi/\eta]$.}}
\end{principles}

% These are based on Bostock, pp.220f.. Bostock doesn't have the duality rule
% because he takes \exists as defined. His GEN rule says that if |- A then |-
% \forall\chi A[\chi/\nu]. I.e., it converts provable closed sentences into
% quantified sentences. His UI rule (A4) says \forall \chi A \to A[\nu/\chi],
% i.e. it doesn't allow deriving (x)Fx -> Fx. He mentions that if open sentences
% are permitted then my version must be used. I'm not sure why we need the
% substitution in GEN. Suppose we can prove the closed sentence phi(a).
% Bostock's GEN allows inferring (x)phi(x). If in every such case we can also
% prove phi(x) then my simpler rule (Gen) suffices. I think all my axioms and
% rules are neutral between names and variables, insofar as any proof ending
% with phi(a) can be converted into a proof of phi(x) by simply replacing all
% occurrences of a with x (for some x). For how could the name have been
% introduced? Either by (CPL), e.g. through the tautology Fa v ~Fa, in which
% case one could just as well use a variable, or by (UI), where again one can
% just as well use a variable. All inferences from sentences with the name can
% still be made if the name is replaced by a variable.

These axioms and rules are sound and complete: everything that can be proved is valid, and
every valid (closed) sentence can be proved. The above tree rules are also sound
and complete.

\begin{exercise}
  The completeness proof for first-order trees (like the proof in chapter
  \ref{ch:proofs}) shows that if a sentence is valid then any fully expanded
  tree for that sentence will close, provided the tree rules are applied in a
  sensible order. Why doesn't this contradict the claim I made in the previous
  section. that there is no mechanical procedure to determine, for an arbitrary
  $\L_P$-sentence, whether the sentence is valid? (Tree proofs count as
  ``mechanical'', so that's not the problem.)
\end{exercise}
\begin{solution}
  If a sentence is valid (in first-order predicate logic) then a fully expanded
  tree for the sentence will close and show that the sentence is valid. But if a
  sentence is not valid, the tree might grow forever. There is no algorithm for
  detecting whether a tree will grow forever. 
\end{solution}


\section{Modality de dicto and de re}

% A student: "Page 179 of the Modal Predicate Logic handout. I don't
% understand the de re sentence Ex◇Fx. ◇Fx means that there is some x
% at v which is F, that's fine. But what does the existential
% qualifier mean in this context? That there is some x at world w that
% can see Fx at v? If so, surely all x at w can see v, in which case
% any existential qualifier followed by a modal operator would be
% either Ex or Vx?"

We are now ready to add boxes and diamonds to the language of first-order
predicate logic. This gives us the \textbf{standard language of first-order
  modal logic}, or $\L_{M\!P}$. The sentences of $\L_{M\!P}$ are defined as
follows.
%
\begin{enumerate}[leftmargin=9mm]
  \itemsep-1mm
  \item An $n$-place predicate followed by $n$ terms is an $\L_{M\!P}$-sentence.
  \item If $A$ is an $\L_{M\!P}$-sentence, then so are $\neg A$, $\Diamond A$,
        and $\Box A$.
  \item If $A$ and $B$ are $\L_{M\!P}$-sentences, then so are $(A \land B)$,
        $(A \lor B)$, $(A \to B)$ and $(A \leftrightarrow B)$.
  \item If $A$ is an $\L_{M\!P}$-sentence and $\chi$ is a variable, then
        $\forall \chi A$ and $\exists \chi A$ are $\L_{M\!P}$-sentence.
  \item Nothing else is an $\L_{M\!P}$-sentence.
\end{enumerate}

We continue to interpret the box and the diamond as (disguised) quantifiers. So
$\L_{M\!P}$ effectively has two kinds of quantifiers: overt quantifiers of the
form $\forall \chi$ and $\exists \chi$, and the disguised quantifiers $\Box$ and
$\Diamond$. This is only useful if the two kinds of quantifiers range over
different things. In applications of modal predicate logic, the box and the
diamond usually range over possible worlds or times, while the overt
quantifiers range over things like people, rocks, ghosts, etc., which are
assumed to inhabit the worlds or times.

For example, consider the following inference, in which I've written the box as `$\Kn$'.

\bigskip
\begin{tabular}{ll}
  Bob knows that all humans are mortal. & $\Kn\forall x (Hx \to Mx)$\\
  Socrates is human. & $Hs$\\
  Therefore: Socrates is mortal. & $Ms$
\end{tabular}
\bigskip

\noindent%
The knowledge operator $\Kn$ is a quantifier over the worlds compatible with
Bob's (implicit) knowledge. $\Kn\forall x (Hx \to Mx)$ says that
$\forall x (Hx \to Mx)$ is true at every world compatible with Bob's knowledge.
$\forall x (Hx \to Mx)$ is assumed to quantify not over worlds, but over things
that exist relative to a world. $\forall x (Hx \to Mx)$ is true at a world $w$
iff $Hx \to Mx$ is true of every inhabitant of $w$, meaning that every
inhabitant of $w$ is either not human or mortal. The inference is valid because
the accessibility relation for knowledge is reflexive.

Imagine a lottery. Let's read the box as `it is certain that' and $W$ as `-- is
a winning ticket'. Can you see what is expressed by the following two
statements?
\begin{enumerate}[leftmargin=14mm]
  \itemsep-1mm
  \item[(1)] $\Box \exists x Wx$
  \item[(2)] $\exists x \Box Wx$
\end{enumerate}
%
(1) says that it is certain that some ticket wins: at every epistemically
accessible world there is a winning ticket. (2) says that there is a particular
ticket of which we are sure that it will win: there is an individual such that
at every epistemically accessible world, \emph{it} is the winning ticket.
(2) is only true if we know which ticket is the (or a) winning ticket.

Sentences like $\exists x \Box Wx$ are called \textbf{de re}, Latin for `of a
thing'. Intuitively, $\exists x \Box Wx$ assert \emph{of} a particular ticket
that it has a modal property, namely the property of being the certain winner.
By contrast, $\Box \exists x Fx$, merely states that the proposition (Latin,
\emph{dictum}) $\exists x Fx$ is certain. Sentences like this are called
\textbf{de dicto}.

In general, an $\L_{M\!P}$-sentence is \emph{de re} whenever it contains a
variable that is free in the scope of some modal operator. To determine
whether a sentence $A$ is \emph{de re}, first identify all subsentences of $A$
that constitute the scope of a modal operator. (In $\exists x \Box Wx$, there is
one such subsentence: $\Box Wx$.) Next, check if at least one of these
subsentences contains a free variable. ($\Box Wx$ contains the free variable
$x$.) If yes, the sentence $A$ is \emph{de re}.

If a sentence contains a modal operator and is not \emph{de re}, then it is
\emph{de dicto}. So $\forall x (Fx \to \Box Gx)$ and
$\exists y\Box (\forall x Fx \to Fy)$ are \emph{de re}, but
$\Box \forall x Fx \to Fa$ is \emph{de dicto}. $\forall x Fx \to Fa$ is neither
\emph{de dicto} nor \emph{de re}, because it isn't modal.

There is no consensus on how to classify sentences like $\Box Fa$ that contain
a name, but no free variable, in the scope of a modal operator. One might argue
that $\Box Fa$ is \emph{de dicto} because it attributes a modal status -- say,
necessity -- to the proposition $Fa$. But one might also interpret the sentence
as attributing a modal property to the individual $a$: the property of being
necessarily $F$. The sentence should then be classified as \emph{de re}. Which
of these two perspectives is more adequate depends on the precise semantics of
$\L_{M\!P}$. We therefore have to postpone the question until the next chapter,
where we will consider some options for developing a semantics of $\L_{M\!P}$.

% I don't actually return to the question, do I?

Many natural-language sentences are ambiguous between a \emph{de re} reading and
a \emph{de dicto} reading. Consider `something necessarily exists'. This can
mean either that there is an object which could not have failed to exist
($\exists x \Box Ex$); but it can also mean that it is necessary that something
or other exists ($\Box \exists x Ex$). The first reading is \emph{de re}, the
second \emph{de dicto}.

\begin{exercise}
  Translate the following sentences into modal predicate logic. (Some of them
  are ambiguous.)
  \begin{exlist}
  \item John must be hungry.
  \item Anyone who is a cyclist must have legs.
  \item Every day might be our last.
  \item If anyone wants to leave early, they should do so quietly.
  \item Everyone who bought a ticket is allowed to enter.
  % \item One day, all those who are rich will be poor.
  \end{exlist}
\end{exercise}
\begin{solution}

  \begin{sollist}
    \item $\Box Fa$ \\ $a$: John, $F$: -- is hungry.\\[1mm]
    (Might be classified as either \emph{de re} or \emph{de dicto}.)\\[-2mm]
    
  \item $\Box \forall x(Fx \to Gx)$ \\ $F$: -- is a cyclist, $G$: -- has legs.\\[1mm]
    This is \emph{de dicto}. Also correct (but different in meaning) is the \emph{de re} translation $\forall x (Fx \to \Box Gx)$.
    Close but incorrect (and \emph{de re}): $\forall x \Box(Fx \to Gx)$.\\[-2mm] 
    
  \item $\forall x (Fx \to \Diamond Gx)$ \\ $F$: -- is a day, $G$: -- is our last day.\\[1mm]
  Better:
  $\forall x (Fx \to \Diamond (Hx \land \neg \exists y(Fy \land Lyx \land Hy)))$ \\ $F$: -- is a day, $L$: -- is later than --, $H$: We are alive on --.
  
    Both \emph{de re}. The English sentence could also be understood \emph{de dicto}, as 
    $\Diamond \forall x (Fx \to Gx)$, but that would be a very strange
    thing to say.\\[-2mm]
    
  \item %If anyone wants to leave early, they should do so quietly.
    I would translate this as $\forall x \Ob(Fx \to Gx)$\\
    $F$: -- wants to leave early, $G$: -- leaves quietly.\\[1mm]
    Even better, if we can use the conditional obligation operator: $\forall x \Ob(Gx / Fx)$. Also defensible are
    $\forall x (Fx \to \Ob Gx)$ and $\Ob\forall x(Fx \to Gx)$.
    % One might think that the former is wrong because it is true in any situation
    % where nobody wants to leave early, no matter the norms.
    % But one could argue that the English sentence is also true in such a situation,
    % because it is equivalent to 'Anyone who wants to leave early should do so quietly'.

    All of these are \emph{de re}. \\[-2mm]
    
  \item % Everyone who bought a ticket is allowed to enter.

    $\forall x (\exists y (Fy \land Hxy) \to \Pe Gx)$\\
    $F$: -- is a ticket, $G$: -- enters, $H$: -- bought --.\\[1mm]
    Perhaps even better: $\forall x \Pe(Gx/ \exists y (Fy \land Hxy))$.
    Both of these are \emph{de re}.\\[1mm]
    You could translate `bought a ticket' as a simple predicate here;
    you could also use a temporal operator to account for the past
    tense of `bought' (but it's confusing to use two different kinds
    of `$\tP$' in one sentence).
    \\[-2mm]

   % \item $\tF \forall x (Fx \to Gx$)\\
   %  $F$: -- is rich, $G$: -- is poor.\\[1mm]
   %  Assuming that `rich' and `poor' are incompatible, this is equivalent to $\tF \forall x \neg Fx$.\\
   %  The more natural reading of the sentence can only be captured with the `now' operator:
   %  $\tF \forall x (\tN Fx \to Gx)$.\\[1mm]
   %  Both translations are \emph{de re}.
  \end{sollist}
\end{solution}

\begin{exercise}
  Which of your translations from the previous exercise are \emph{de
    re} and which are \emph{de dicto}?
\end{exercise}
\begin{solution}
  See the previous answer.
\end{solution}

On some interpretations of the modal operators, one may question whether
\emph{de re} sentences are intelligible. Suppose we interpret the box as `it is
analytic that' or `it is provable that'. The things that are analytic or
provable are sentences or propositions. 2+2=4, for example, is provable in
ZFC, and `all vixens are female foxes' is analytic in English. (Remember that a
sentence is analytic if it is true in virtue of its meaning.) It is not clear
what it could mean to say that something is provable or analytic \emph{of} a
particular thing.

To illustrate the problem, let's introduce the name `Julius' for whoever
invented the zip. The sentence `Julius invented the zip' is analytic. (In fact,
`Julius invented the zip' entails that someone invented the zip, which is not
analytic. We should really use `If anyone invented the zip, then Julius invented
the zip'. Let's ignore this complication.) But is it analytic \emph{of} the
person who invented the zip that they invented the zip? The problem is that this
person has multiple names, and depending on which name we plug into the schema
`--- invented the zip', we sometimes get an analytic truth and sometimes not.
For `Julius', the sentence is analytic; for whatever name the inventor of the
zip was given by his or her parents, the sentence is not analytic.

This kind of worry was prominently raised by W.V.O.\ Quine in the 1940s.  It has
since faded, mostly because philosophers have turned their attention away from
analyticity to other interpretations of the box for which the problem is thought
not to arise. But we will return to the matter in section \ref{sec:twi}.

% factoid: ``(Ex)[]A is not equivalent to any de dicto sentence even in
% S5+BF and so not in any weaker system.'' Cresswell 149.

\section{Identity and descriptions}
\label{sec:identity}

In applications of modal and non-modal predicate logic, it is often useful to
have a special predicate for identity. Let's assume that $\L_P$ and $\L_{M\!P}$
have the two-place predicate `='. The identity predicate is conventionally
placed between its two arguments: we write `$a=b$', not `$=\!ab$'. We also sometimes abbreviate `$\neg a\!=\!b$' as `$a\not=b$'. 

Unlike the other predicates of $\L_P$ and $\L_{M\!P}$, the identity predicate
counts as a logical symbol. Its meaning is held fixed. In any model, $a=b$
means that the individual picked out by $a$ is the very same thing as the
individual picked out by $b$. This is reflected by the following clause, which
we add to the semantics of predicate logic:
\[
  M,g \models \tau_1\!=\!\tau_2\text{ \; iff }[\tau_1]^{M,g} = [\tau_2]^{M,g}.
\]

It is easy to see that the sentence $a=a$ is now valid, because $a$ and $a$ are
guaranteed to pick out the same individual. More interestingly, since the
function of a name in classical predicate logic is just to pick out an
individual, it never matters which of two names we use if they pick out the same
individual. That is, if $a=b$ is true, then replacing some or all occurrences of
$a$ in a sentence with $b$ never affects whether the sentence is true. This
principle is known as \textbf{Leibniz' Law}.

To reflect these facts, the tree method for (non-modal) predicate logic must be
extended by two new rules. First, if $\eta$ is an ``old'' name (that already
occurs on a branch) then we can always add a node $\eta=\eta$ to the branch.
Second, if an identity statement $\eta_1=\eta_2$ occurs on a branch, and some
sentence $A$ on the branch contains $\eta_1$, then we may add a new node with
the same sentence $A$ except that one or more occurrences of $\eta_1$ in $A$ are
replaced by $\eta_2$, or one or more occurrences of $\eta_2$ by $\eta_1$. Let
$A[\eta_2/\!/\eta_1]$ stand for any sentence that results from $A$ by replacing
one or more occurrences of $\eta_1$ by $\eta_2$. The new rules can then be
summarized as follows.

\bigskip
\begin{center}
  \begin{minipage}[t]{0.3\textwidth} \centering

    Self-Identity
    
    \tree{
      \dotbelowbarenode{}\\
      \\
      \barenode{$\eta=\eta$}\\
      \Kk[0]{0}{\color{red}$\uparrow$\hspace{6mm}}\\
      \Kk[0]{0}{\color{red}\small old\hspace{6mm}}
    }
  \end{minipage}
  \begin{minipage}[t]{0.3\textwidth} \centering
    Leibniz' Law
    \bigskip

    \tree{
      \nnode{20}{}{$\eta_1=\eta_2$}{}{}\\
      \dotbelownode{20}{}{$A$}{}{}\\
      \\
      \nnode{20}{}{$A[\eta_2/\!/\eta_1]$}{}{}
    }
  \end{minipage}
  \begin{minipage}[t]{0.3\textwidth} \centering
    Leibniz' Law
    \bigskip

    \tree{
      \nnode{20}{}{$\eta_1=\eta_2$}{}{}\\
      \dotbelownode{20}{}{$A$}{}{}\\
      \\
      \nnode{20}{}{$A[\eta_1/\!/\eta_2]$}{}{}
    }
  \end{minipage}
  
\end{center}

% Mention that the $A$ node should not be ticked off?

\bigskip

Here is a tree for $(Raa \land a\!=\!b) \to Rab$, using Leibniz's Law.

\medskip
\begin{center}
  \tree{
    \nnode{30}{1.}{$\neg((Raa \land a\!=\!b) \to Rab)$}{}{(Ass.)}\\
    \nnode{30}{2.}{$Raa \land a\!=\!b$}{}{(1)}\\
    \nnode{30}{3.}{$\neg Rab$}{}{(1)}\\
    \nnode{30}{4.}{$Raa$}{}{(2)}\\
    \nnode{30}{5.}{$a\!=\!b$}{}{(2)}\\
    \nnodeclosed{30}{6.}{$Rab$}{}{\quad(4, 5, LL)}
  }
\end{center}

% When reading off countermodels from an open branch with an identity node $a=b$,
% you ignore one of the names.

\begin{exercise}
  Use the tree method to check which of the following sentences are valid.
  \begin{exlist}
  \item $\forall x (x\!=\!x)$
  \item $\forall x \forall y(x\!=\!y \to y\!=\!x)$
  \item $(a=b \land b=c) \to a=c$
  \item $Rab \to \forall x(x=a \leftrightarrow Rxb)$
  \item $\forall x \forall y\forall z((x\not= y \land y\not= z) \to x \not= z)$
  \end{exlist}
\end{exercise}
\begin{solution}
  Use \href{https://www.umsu.de/trees/}{umsu.de/trees/}.
\end{solution}

\begin{exercise}
  Show that the second version of the Leibniz' Law rule is redundant: we could
  reach $A[\eta_1/\!/\eta_2]$ from $\eta_1=\eta_2$ and $A$ with the other rules.
\end{exercise}
\begin{solution}
      % \nnode{20}{}{$\eta_1=\eta_2$}{}{}\\
      % \dotbelownode{20}{}{$A$}{}{}\\
      % \\
      % \nnode{20}{}{$A[\eta_1/\!/\eta_2]$}{}{}
  We assume that some branch on a tree contains nodes $\eta_1=\eta_2$ and $A$. We have to
  show that we can add $A[\eta_1/\!/\eta_2]$ without using the second version of Leibniz'
  Law.
  \tree{
    \nnode{15}{k.}{$\eta_1=\eta_2$}{}{}\\
    \nnode{15}{n.}{$A$}{}{}\\ % e.g. Fc v Hcb, want to get Fb v Hcb
    \nnode{15}{m.}{$\eta_1=\eta_1$}{}{(SI)}\\
    \nnode{15}{m+1.}{$\eta_2=\eta_1$}{}{\hspace{32mm}(k, m, LL (first version))}\\
    \nnode{15}{m+2.}{$A[\eta_1/\!/\eta_2]$}{}{\hspace{32mm}(m+1, n, LL (first version))} }
\end{solution}

In the axiomatic approach, the two facts about identity are often represented by
the following axiom schemas:
%
\begin{principles}
  \pri{SI}{\eta=\eta}\\
  \pri{LL}{\eta_1=\eta_2 \to (A \to A[\eta_2/\!/\eta_1])}
\end{principles}

Once we add boxes and diamonds to the language of predicate logic, the seemingly
harmless axioms and rules for identity become problematic. Consider the
following inference:
%
\begin{quote}
  It is analytic that Julius invented the zip.\\
  Julius = Whitcomb L.\ Judson.\\
  Therefore: It is analytic that Whitcomb L.\ Judson invented the zip.
\end{quote}
%
The conclusion clearly doesn't follow from the premises, but the inference seems
to be licensed by Leibniz's law. Another well-known example:
%
\begin{quote}
  Lois Lane believes that Superman can fly.\\
  Superman = Clark Kent.\\
  Therefore: Lois Lane believes that Clark Kent can fly.
\end{quote}

\begin{exercise}
  (a) Give an axiomatic proof of $\Box \exists x\, x=a$, using (SI), (UI), (CPL),
  ($\forall\exists$), (CPL), and (Nec), in this order. (b) Can you see why we might
  not want to count $\Box\exists x\, x=a$ as a logical truth in some
  applications of modal logic? At which point do you think the proof goes wrong?
\end{exercise}
\begin{solution}
  \begin{sollist}
    
    \item 
\begin{align*}
   1. \quad & a=a &&\text{ (SI)}\\
   2. \quad & \forall x\, x\not= a \to a\not= a &&\text{ (UI)}\\
   3. \quad & \neg \forall x\, x\not=a &&\text{ (1, 2, CPL)}\\
   4. \quad & \neg \exists x\, x=a \leftrightarrow \forall x \, x\not=a &&\text{ ($\forall\exists$)}\\
   5. \quad & \exists x\, x=a &&\text{ (3, 4, CPL)}\\
   6. \quad & \Box\exists x\, x=a &&\text{ (5, Nec)}
\end{align*}

\item There are many correct answers. For example: historians debate whether
Homer ever existed. If $a$ translates `Homer' then $\exists x\, x=a$ is arguably false if Homer isn't a real person. Since the available evidence is compatible with  $\neg \exists x\, x=a$, the sentence $\Box\exists x\, x=a$ is false on an epistemic interpretation of the box.

Where does the proof go wrong? Each of steps 1, 2, and 6 might be blamed.
\end{sollist}
\end{solution}


We will return to these issues in section \ref{sec:twi}. In the remainder of the
present section, I want to highlight some other things we can do with the
identity predicate, apart from making claims about identity.

You have already encountered one other use in earlier chapters. Suppose we want
to express that some relation $R$ is connected, meaning that for any two things,
either the first is $R$-related to the second or the second is $R$-related to
the first. This can't be expressed without an identity predicate. With an
identity predicate, it is easy:
\[
  \forall x\forall y(Rxy \lor x\!=\!y \lor Ryx).
\]

We can also use identity to express numerical quantifiers. For example, we can
express `there are at least two $F$s' as
\[
  \exists x(Fx \land \exists y(Fy \land x\not=y)).
\]
`There is exactly one $F$' can be expressed as
\[
  \exists x(Fx \land \forall y(Fy \to x\!=\!y)).
\]

\begin{exercise}
  Can you express the following in $\L_P$ with identity?
  \begin{exlist}
  \item There are exactly two $F$s.
  \item There are no more than three $F$s.
  \end{exlist}
\end{exercise}
\begin{solution}
  \begin{sollist}
  \item $\exists x \exists y(Fx \land Fy \land x\not=y \land \forall z(Fz \to (z=x \lor z=y)))$
  \item $\forall x \forall y\forall z\forall v(Fx \land Fy \land Fz \land Fv \to (x=y \lor x=z \lor x=v \lor y=z \lor y=v \lor z=v))$
  \end{sollist}
\end{solution}
% should add more identity translation exercises, as e.g. in Macfarlane p.30?

Another important use of the identity predicate is to formalise statements
involving definite descriptions. A definite description is a complex noun
phrase, typically of the form `the $F$', that purports to pick out a particular
object. `The current Prime Minister', `the highest mountain in Scotland', and
`Carol's father' are definite descriptions.

The standard language of predicate logic does not have a definite article
(`the'). The only way to pick out an individual in $\L_P$ is by a name. But
there are good reasons not to translate descriptions as names.

One reason is that we would thereby miss logical connections between
descriptions and predicates. `The current Prime Minister is not Prime Minister'
is a logical contradiction, but this can't be brought out if we translate `the
current Prime Minister' as a simple name.

% Another reason not to translate descriptions as names is that descriptions can
% fail to refer. Since France is a republic, `the present King of France' does not
% refer to anybody. But names in classical predicate logic always have a
% reference: they pick out some member of the domain $D$. --- Well, names can also
% fail to refer.

Another reason not to translate descriptions as names is that descriptions often
give rise to a \emph{de re}/\emph{de dicto} ambiguity. Consider the following
sentence:
\begin{quote}
  The Pope might have been Italian.
\end{quote}
This has two readings. It can mean either that the actual Pope, Jorge Mario
Bergoglio, might have been Italian (\emph{de re}). Alternatively, it can mean
that the following might have been the case: some Italian person is Pope
(\emph{de dicto}). There is no way to account for these two readings in
$\L_{M\!P}$ if we translate `the Pope' as a name. 

A better translation for statements involving definite descriptions was proposed
by Bertrand Russell in 1905. Russell argued that a statement of the form `the
$F$ is $G$' is true iff there is exactly one (relevant) $F$, and
this one $F$ is also $G$. If we have an identity predicate, we can easily
express this in the language of predicate logic:
\[
  \exists x(Fx \land \forall y(Fy \to x\!=\!y) \land Gx). 
\]

Following Russell, we might translate `The current Prime Minister is not Prime
Minister' as
\[
  \exists x(Px \land \forall y(Py \to x\!=\!y) \land \neg Px).
\]
This is indeed a contradiction: it is true in no model.

We can also account for the two readings of `the Pope might have been Italian'.
The \emph{de re} reading is
\[
  \exists x (Px \land \forall y(Py \to x\!=\!y) \land \Diamond Ix).
\]
The \emph{de dicto} reading is
\[
  \Diamond \exists x (Px \land \forall y(Py \to x\!=\!y) \land Ix).
\]

% \begin{exercise}
%   People sometimes translate DDs with a term-form iota. Why is that problematic?
% \end{exercise}

\begin{exercise}
  Give two translations for each of the following sentences, one \emph{de re} and one \emph{de dicto}.
  \begin{exlist}
  \item Hillary Clinton might have been the 45th US President.
  \item Smith's murderer could have been a woman.
  \item Alice believes that the student representative is rude.
  \end{exlist}
\end{exercise}
\begin{solution}
  The \emph{de dicto} reading of (a) can be translated as
  \begin{equation*}
    \Diamond \exists x (Px \land \forall y(Py \to x\!=\!y) \land x\!=\!c),%
  \end{equation*}
  where `$P$' translates `-- is 45th US President' and `$c$' denotes Hillary
  Clinton. The \emph{de re} reading can be translated as
  \begin{equation*}
    \exists x (Px \land \forall y(Py \to x\!=\!y) \land \Diamond x\!=\!c).
  \end{equation*}
  The answers to (b) and (c) are analogous.
\end{solution}



%%% Local Variables: 
%%% mode: latex
%%% TeX-master: "logic2.tex"
%%% End:
