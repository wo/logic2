\chapter{Deontic Logic}\label{ch:deontic}

\section{Permission and obligation}

Deontic logic studies formal properties of obligation, permission, prohibition,
and related normative concepts. The box in deontic logic is usually written
`$\Ob$' (for `obligation' or `ought'), the diamond `$\Pe$' (for `permission').
If we read $q$ as stating that you cook dinner, we might use $\Ob q$ to express
that you are obligated to cook dinner.

We assume that obligation and permission are duals. You are not obligated to
cook dinner iff you are permitted to not cook dinner; you are not permitted to
cook dinner iff you are obligated to not cook dinner.

There are many kinds of norms: legal norms, moral norms, prudential norms,
social norms, and so on. There may also be overarching norms that combine some
or all of the others. Deontic logic is applicable to norms of all kinds. We do
not have to settle whether $\Ob$ expresses legal obligation or moral obligation
or some other kind of obligation. It is important, however, that we don't
equivocate. If the law requires $q$ and morality $\neg q$, we should not
formalize this as $\Ob q \land \Ob \neg q$. It would be better to use a
multi-modal language with different operators for legal and moral obligation.

% Humberstone 245 discusses the idea of using multi-relational frames in which
% the different relations correspond to different sources of norms or values.
% One may then read OA as true iff /some/ norm says that A is ideal. In this
% semantics, O(A) and O(B) do not entail O(A & B). Compare Lewis on
% compartmentalised beliefs.

Obligations and permissions often vary from agent to agent. If it is your turn
to cook dinner then you are obligated to cook dinner, but I am not. To capture
this agent-relativity, we could add agent subscripts to the operators, as we did
in epistemic logic. We could then express our different obligations as
$\Ob_1 q \land \neg \Ob_2 q$. But what does the sentence letter $q$ stand for?
When I say that you are obligated to cook dinner, the object of the obligation
appears to be a type of act: cooking dinner. In the language of modal
propositional logic, $\Ob$ and $\Pe$ are sentence operators. Unless we want to
say that verb phrases in English (like `cook dinner') should be translated into
sentences of $\L_M$ -- which is possible, but non-standard -- we have to
transform the acts that appear to be the true objects of obligation and
permission into propositions.

Consider sentence (1), which is arguably equivalent to (2).
\begin{itemize}[leftmargin=10mm]
\itemsep-1mm
\item[(1)] You ought to cook dinner.
\item[(2)] You ought to see to it that you cook dinner.
\end{itemize}
In (2), the operator `you ought to see to it that' attaches to a sentence, `you
cook dinner'. So we can translate (1) via (2) as $\Ob_1 q$, where $q$ translates
`you cook dinner', and $\Ob_1$ corresponds to `you ought to see to it that'.

The subject (you) is mentioned twice in (2). A common assumption in deontic
logic is that we can drop the agent subscripts from deontic operators, since the
embedded proposition will tell us upon whom the obligation or permission falls.
Informally, the idea is that (2) is equivalent to (3), with an impersonal
`ought'.
\begin{itemize}[leftmargin=10mm]
\itemsep1mm
\item[(3)] It ought to be the case that you cook dinner.
\end{itemize}

The impersonal `ought' also figures in statements like (4).
\begin{itemize}[leftmargin=10mm]
\itemsep1mm
\item[(4)] Nobody ought to die of hunger.
\end{itemize}
When I say (4), I don't mean that nobody is obligated to die of hunger. Nor do I
mean that everybody is obligated to not die of hunger. Rather, I mean that a
certain state of affairs -- that nobody dies of hunger -- ought to be the case.
By itself, this does not impose any obligations on anyone.

% The reducibility of personal to impersonal ought was suggested by Meinong and
% Chisholm, see Handbook of Modal Logic p 1204.

On closer inspection, the equivalence between agent-relative `ought' statements
like (2) and impersonal `ought' statements like (3) is questionable. Suppose Amy
has promised to play with Betty. Then Amy is obligated to play with Betty. But
Betty is not thereby obligated to play with Amy. Betty may even have promised
not to play with Amy. It is hard to express these facts in terms of impersonal
oughts. If we say that it ought to be the case that Amy plays with Betty, we're
missing the fact that the obligation falls on Amy, not on Betty (who might be
under a contrary obligation). So perhaps it would be better to keep the agent
subscripts after all.

It can also be useful to make the `see to it that' component in statements like
(2) explicit. That Amy ought to play with Betty could then be translated as
$\Ob_a \stit p$, where $\stit$ formalizes `sees to it that'. This allows us to
distinguish between the following three claims.

\bigskip
\begin{tabular}{ll}
  $\Ob_a \stit \neg p$ & Amy ought to see to it that she doesn't play with Betty.\\
  $\Ob_a \neg \stit p$ & Amy ought to not see to it that she plays with Betty.\\
  $\neg \Ob_a \stit p$ & It is not the case that Amy ought to see to it that she plays with\\[-0.5mm]
                       & Betty.
\end{tabular}
\bigskip

% Others have suggested that an adequate deontic logic should include special
% terms for actions in addition to terms for propositions. We could then write
% $\Ob_a \alpha$, where $\alpha$ stands for the (possible) action of Amy playing
% with Betty. This leads to the field of \emph{dynamic deontic logic}. Here,
% action terms are themselves treated as modal operators, associated with their
% own accessibility relation: a world $w'$ is accessible from $w$ through action
% $\alpha$ if performing $\alpha$ in $w$ could bring about $w$. Thus if $p$ is
% the proposition that Betty is happy, one could use $[\alpha]p$ to express that
% Amy's dancing with Betty would definitely lead to Betty being happy, while
% $\langle \alpha \rangle p$ would expresses that the action $\alpha$
% \emph{could} lead to $p$.

% Hilpinen says stit accounts are more sophisticated versions of these.

The $\stit$ operator has proved useful to represent different concepts of rights
and duties. In what follows, we will nonetheless stick to the simplest (and
oldest) approach, without a $\stit$ operator and without agent subscripts. This
approach is sufficient for many applications, but its limitations should be kept
in mind.

% \begin{exercise}
%   Let $\mathsf{F}A$ mean that $A$ is forbidden. Can you define
%   $\mathsf{F}$ in terms of $\Ob$ or $\Pe$ (or both)?
% \end{exercise}
% \begin{solution}
%   For example: $\Ob \neg$, or $\neg \Pe$.
% \end{solution}

\begin{exercise}\label{ex:translate-sdl}
  Translate the following sentences into the standard language of deontic logic (without $\stit$ or agent subscripts).
  \begin{exlist}
  \item You must not go into the garden.
  \item You may not go into the garden.
  \item Jones ought to help his neighbours.
  \item If Jones is going to help his neighbours, then he ought to tell them
    he's coming.
  \item If Jones isn't going to help his neighbours, then he ought to not tell
    them he's coming.
  \end{exlist}
\end{exercise}
\begin{solution}
  \begin{sollist}
    \item $\Ob \neg p$; \quad $p$: You go into the garden.
    \item $\Ob \neg p$; \quad $p$: You go into the garden.
    \item $\Ob p$; \quad $p$: Jones helps his neighbours.
    \item $\Ob (p \to q)$; \quad $p$: Jones helps his
    neighbours, $q$: Jones tells his neighbours that he's coming.
    \item You might try $\Ob (\neg p \to \neg q)$ or $\neg p \to \Ob \neg q$ \quad $p$: Jones helps his neighbours, $q$: Jones tells his neighbours that he's coming.
  \end{sollist}
  See section \ref{sec:oblig-circ}, especially exercise \ref{ex:chisholmsparadox},
  for why neither translation of (e) is fully satisfactory.
\end{solution}


\section{Standard deontic logic}

Think of a possible world as a history of events. For any such history, and any
system of norms, we can ask whether the history conforms to the norms. Let's
call a world \emph{acceptable} relative to some norms if everything that happens
at the world conforms to the norms. That is, a world is acceptable if it
contains no violation of any relevant norm.

By definition, whatever happens at an acceptable world is permitted, in the
sense that it does not violate any (relevant) norms. The converse is plausible
as well: whenever something is permitted then it is the case at some acceptable
world. For example, if it is permitted that Amy plays with Betty, then there
should be a complete history of events in which Amy plays with Betty and no
norms are violated. If there were no such history, then Amy's playing with Betty
would logically entail the violation of some norms; but if an act entails the
violation of some norms, then it is hard to see how the act could be permitted
relative to these norms.

So we have the following connection between permission and acceptable worlds,
which amounts to a possible-worlds analysis of permission:
\begin{quote}
  $A$ is permitted (relative to some norms) iff $A$ is the case at
  some possible world that is acceptable (relative to these norms).
\end{quote}
%
Given the duality of permission and obligation, we also get a possible-worlds
analysis of obligation:
%
\begin{quote}
  $A$ is obligatory (relative to some norms) iff $A$ is the case at
  all worlds that are acceptable (relative to these norms).
\end{quote}

In logic, we are not interested in who is in fact obligated to do what, but in
whether a given deontic statement is logically valid, or whether it logically
follows from other statements.

Validity means truth in every conceivable scenario under every interpretation of
the non-logical vocabulary. A scenario for deontic logic has to specify the
relevant norms. This can be done by specifying which worlds are acceptable
relative to which other worlds.

We can represent a scenario of this type together with an interpretation of the
sentence letters by a Kripke model. A world $v$ in the model is accessible from
a world $w$ if $v$ is acceptable relative to the norms at $w$ -- equivalently,
if everything that ought to be the case at $w$ is the case at $v$. Worlds that
are accessible from $w$ in this sense are called \textbf{ideal} relative to $w$.

Our possible-worlds analysis of obligation and permission is reflected in
definition \ref{def:kripkesemantics}, which settles under what conditions a
sentence is true at a world in a model. Writing the box as `$\Ob$' and the
diamond as $\Pe$', clause (g) of the definition states that $\Ob A$ is true at a
world $w$ in a model $M$ iff $A$ is true at all worlds of $M$ that are ideal
relative to $w$. Clause (h) states that $\Pe A$ is true at $w$ in $M$ iff $A$ is
true at some world that is ideal relative to $w$.

A sentence is valid iff it is true at every world in every suitable model. If we
count all Kripke models as suitable, the logic of obligation and permission will
be the minimal normal modal logic K. We can get stronger logics by imposing
constraints on the accessibility relation. Let's have a look at a few options.

We might stipulate that the deontic accessibility relation is reflexive, so that
every world can see itself. This would make all instances of the \pr{T}-schema
valid:
%
\principle{T}{\Ob A \to A}
%
In deontic logic, the \pr{T}-schema is highly implausible. The fact that
something ought to be the case does not entail that it is the case. Semantically
speaking, many worlds are not ideal relative to themselves. We will not assume
reflexivity.

We might, however, impose the weaker condition of seriality -- that each world
can see some world. This would validate principle \pr{D}:
%
\principle{D}{\Ob A \to \Pe A}
%
Intuitively, \pr{D} says that the norms are consistent: if you're obligated to
do $A$, then you can't also be obligated to do not-$A$. (Remember that $\Pe A$
is equivalent to $\neg \Ob \neg A$.) Semantically, \pr{D} corresponds to the
assumption that there is always at least one world at which all the norms are
satisfied.

Without seriality, we have to allow for worlds from which no world is
accessible. At such a world, all sentences of the form $\Ob A$ are true, and all
sentences of the form $\Pe A$ are false. Everything is obligatory, but nothing
is allowed. That makes little sense. If we use Kripke semantics for deontic
logic, we should rule out inconsistent norms and accept \pr{D} as valid.

Here it may be important to distinguish \emph{prima facie} obligations from
\emph{actual}, or \emph{all-things-considered} obligations. If you've promised
to cook dinner, you are under a \emph{prima facie} obligation to cook dinner.
But the obligation can be overridden by intervening circumstances or contrary
obligations. If your child has an accident and needs urgent medical care, the
right thing to do may well be to not cook dinner and instead bring your child to
the hospital. In a sense, you are under conflicting obligations: you ought to
cook dinner, and you ought to look after your child (and not cook dinner). There
is no world at which you meet both of these obligations. But that is not a
counterexample to \pr{D}, if we understand $\Ob$ as all-things-considered
obligation. You are \emph{prima facie} obligated to cook dinner, but all things
considered, you should not cook dinner.

Let's return to the non-reflexivity of the deontic accessibility relation. Many
things that are not the case nonetheless ought to be the case. Some have argued
that this is only true in non-ideal worlds. In an ideal world, everything that
ought to be the case is the case. By this line of thought, if a world $v$ is
accessible from some world $w$ -- meaning that $v$ is ideal relative to $w$ --
then $v$ should be accessible from itself. This condition is sometimes called
``shift reflexivity'' and corresponds to the following schema \pr{U} (for
``utopia'')
%
\principle{U}{\Ob(\Ob A \to A)}
%
In words: it ought to be the case that whatever ought to be the case is the
case.

The \pr{U} principle is entailed by an alternative way of formalizing obligation
and permission that goes back to Leibniz. Let `$\OK$' be a propositional
constant whose intended meaning is that all norms are satisfied, no obligations
violated. Suppose we add this expression to $\L_M$, and we interpret the box of
$\L_M$ as a suitable kind of circumstantial necessity. Arguably, $\Ob A$ is then
definable as $\Box (\OK \to A)$: it ought to be that $A$ iff, necessarily, $A$
is the case whenever all obligations are met. It is not hard to show that if the
\pr{T}-schema is valid for the circumstantial box, and $\Ob A$ is defined as
$\Box(\OK \to A)$, then the \pr{U}-schema is valid for $\Ob$.

% Kanger uses 'Q' instead of '$\OK$'. Note that the same approach could be used
% in other contexts, e.g. having 'Q' specify that the laws of nature are
% satisfied, or someone's beliefs. See Humberston 257.

% Humberstone 259 shows that if the box satisfies T and we have <>Q, then the
% Q-logic is precisely KDU.

\begin{exercise}
  \beginwithlist
  \begin{exlist}
    \item Translate the \pr{U}-schema into the Leibnizian language just
    proposed.
  \item Give a tree proof for the translated \pr{U}-schema, using the T-rules
    for the box. 
  \end{exlist}
\end{exercise}
\begin{solution}
  (a): $\Box(\OK \to (\Box(\OK \to A) \to A))$. (b): use 
  \href{https://www.umsu.de/trees/}{umsu.de/trees/}.
\end{solution}

% \begin{exercise}
%   Show that if $\Box$ satisfies \pr{4}, then so does $\Ob$.
% \end{exercise}

\begin{exercise}
  How could we define $\Pe$ in terms of $\Box$ and $\OK$, so that $\Pe$ is the
  dual of $\Ob$? 
\end{exercise}
\begin{solution}
  $\Pe A$ could be defined as $\neg\Box(\OK \to \neg A)$, or more simply (and
  equivalently) as $\Diamond (\OK \land A)$.
\end{solution}



% A weakening of $U$ is 4C or density:
% \[
%   \Ob\Ob A \to \Ob A.
% \]
% This means that every deontic alternative is a deontic alternative to some
% deontic alternative.

Turning to more familiar schemas and frame conditions, what shall we say about
transitivity and euclidity, and the corresponding schemas \pr{4} and \pr{5}?
%
\begin{principles}
\pri{4}{\Ob A \to \Ob\Ob A}\\
\pri{5}{\Pe A \to \Ob\Pe A}
\end{principles}
%
If something ought to be the case, ought it to be the case that it ought to be
the case? If something is permitted, is it obligatory that it is permitted?
Iterations of deontic operators sound strange in ordinary language. But they
have a well-defined meaning in our Kripke semantics. The validity of \pr{4}
would mean that whenever something is obligatory at a world, then it is also
obligatory at all ideal alternatives to that world. \pr{5} would mean that if
something is permissible at a world, then it's also permissible at all ideal
alternatives to that world. On the background of \pr{D}, these two assumptions
together imply that for each world there is a class of ideal worlds all of which
are ideal relative to one another.

To get a clearer grip on whether that is plausible, we need to clarify how
obligations and permissions can vary from world to world.

One obvious sense in which norms can vary across worlds is that people subscribe
to different norms at different worlds. At our world, UK traffic law requires
driving on the left and most people think it is morally wrong to torture
animals for fun. At other worlds, the laws and attitudes are different.

Let $v$ be a world at which the traffic laws require driving on the right, and
at which everyone thinks it is fine to torture animals. Suppose Norman at $v$ is
torturing kittens, while driving on the right (in the UK). Is Norman doing
something that's morally wrong? Is he doing something that violates the traffic
laws? The answer depends on whether we evaluate Norman's acts relative to our
norms -- the norms at our world -- or relative to the norms at Norman's world.
Both perspectives are intelligible. They lead to different deontic logics.

On an \textbf{absolutist} conception, the norms do not vary from world to world.
Whichever world we look at, we always assess it relative to the same set of
norms. On this conception, it is natural to assume that the very same worlds are
ideal relative to any world: a world will be accessible from any world just in
case it contains no violation of the (fixed) norms. The resulting logic of
obligation and permission is KD45.

% Strictly speaking, I am here also assuming that not all worlds are ideal, and
% that there is more than one ideal world.

\begin{exercise}
  Explain why the deontic accessibility relation is transitive and euclidean if the same worlds are ideal relative to any world.
\end{exercise}
\begin{solution}
  Transitivity (if $wRv$ and $vRu$ then $wRu$) and euclidity (if $wRv$ and $wRu$
  then $vRu$) both state that if $v$ is ideal and $u$ is ideal then $u$ is
  ideal.
\end{solution}
\begin{exercise}
  Show that euclidity implies shift reflexivity.
  % This is needed later in applying the euclidity rule for trees.
\end{exercise}
\begin{solution}
  $R$ is euclidean if $\forall x \forall y \forall z((xRy \land xRz) \to yRz)$.
  Suppose $wRv$. Instantiating the universal formula with $w$ for $x$ and with
  $v$ for $y$ and $z$, we have $(wRv \land wRv) \to vRv$. So $vRv$.
\end{solution}

% when I draw a diagram for euclidity I shouldn't use a symmetrical arrow from v
% to u: if I draw the reverse arrow, I should also draw the loop at v and u.
% Maybe that's a good point to highlight in the text, by way of explaining that
% the variables don't need to pick out distinct worlds.

On a \textbf{relativist} conception of norms, we evaluate the events at other
worlds relative to the norms at these worlds. Transitivity and euclidity now
become implausible, as does shift reflexivity. To see why, add another world $u$
to the Norman scenario. The laws at $u$ say that one must drive on the right but
everyone nonetheless drives on the left. Nothing that happens at $u$, we may
assume, violates the traffic laws of our world. So $u$ is deontically accessible
from the actual world. But if we evaluate the events at $u$ relative to the laws
at $u$, then much of what happens at $u$ violates the norms, so $u$ is not
deontically accessible from itself. Shift reflexivity fails.

\begin{exercise}
  Explain why deontic accessibility is neither transitive nor euclidean, on
  the relativist conception.
\end{exercise}
\begin{solution}
  Consider the example from the text, where $w$ is the actual world (in the UK)
  and $u$ is a $w$-accessible world at which everyone drives on the left
  although the law says that one must drive on the right. A typical world
  accessible from $u$ will be a world at which people drive on the right. This
  world will not be accessible from $w$. So we have a counterexample to
  transitivity. We also have a counterexample to euclidity because we have $wRu$
  and $wRu$ but not $uRu$. (Euclidity entails shift reflexivity.)
\end{solution}

The relativist conception is more common in deontic logic. So-called
\textbf{standard deontic logic} assumes only that the accessibility relation is
serial, making the system D the complete logic of obligation and permission.

The proposed logics of absolutism and relativism only disagree about sentences
in which a deontic operator occurs in the scope of another deontic operator. Any
sentence that does not contain an $\Ob$ or $\Pe$ operator embedded under another
$\Ob$ or $\Pe$ operator is D-valid iff it is KD45-valid.

% Proof: Clearly everything that's D-valid is KD45-valid, so we need to show
% that if a degree-1 sentence A is KD45-valid, then A is also D-valid. By
% contraposition, suppose A is not D-valid, meaning that there is a world w in a
% serial model M at which A is false. We construct a world w' in a
% serial+euclidean+transitive model M' at which A is false. The worlds of M' are
% w and every world that can be seen from w in M. Every world that can be seen
% from w can see every other such world in M'. The interpretation function in M'
% is like that in M, restricted to the relevant worlds. Trivially, all degree-0
% sentences have the same truth-values at all worlds in M' as they have in M.
% For degree-1 sentences A, we show by induction on complexity that M,w |= A iff
% M',w |= A. Case (1): A is atomic. Trivial. Case (2): A is -B. Trivial by i.h.
% Case (3): A is BvC. Trivial by i.h. Case (4): A is []B. Then B is degree-0. We
% know that B has the same truth-value at any w-accessible world in M and M'. So
% B is true at all w-accessible worlds in M iff it is true at all w-accessible
% worlds in M'. So []B is true at w in M iff []B is true at w in M'. QED.

\begin{exercise}
  Use the tree method to check which of the following sentences are D-valid and
  which are KD45-valid.
  \begin{exlist}
    % \item $\Ob p \to \Ob (p \lor q)$
    \item $\Pe (p \lor q) \to (\Pe p \land \Pe q)$
    % \item $(\Ob p \land \Ob q) \to \Ob(p \land q)$
    % \item $\Pe p \to \Pe(p \lor q)$
    \item $\Ob\Pe p \to \Pe p$
    % \item $\neg (\Ob p \land \Ob \neg p)$
    \item $\neg\Pe(p \lor q) \to (\Pe \neg p \lor \Pe\neg q)$
    \item $\Ob\Pe p \lor \Pe\Ob p$
  \end{exlist}
\end{exercise}
\begin{solution}
  Use
  \href{https://www.umsu.de/trees/}{https://www.umsu.de/trees/}.
  (Write $\Ob$ as a box and $\Pe$ as a diamond. For D, make the accessibility
  relation serial; for KD45, make it serial, transitive, and euclidean.)
\end{solution}

\begin{exercise}
  Consider a world in which there are no sentient beings, and nothing else that
  could introduce norms or laws. Since there are no norms at this world, one
  might hold that nothing is obligatory relative to the world's norms, and
  nothing is permitted. Explain why this casts doubt on the validity of
  \pr{Dual1} and \pr{Dual2} in the logic of relativist obligation and
  permission.
\end{exercise}
\begin{solution}
  \pr{Dual1} says that $\neg \Diamond A$ is equivalent to $\Box \neg A$. If
  nothing is permitted then $\neg \Diamond A$ is true for all $A$. But if
  nothing is forbidden then $\Box \neg A$ is false for all $A$.

  \pr{Dual2} says that $\neg \Box A$ is equivalent to $\Diamond \neg A$. If
  nothing is forbidden then $\neg \Box A$ is true for all $A$. But if
  nothing is permitted then $\Diamond \neg A$ is false for all $A$.
\end{solution}

\begin{exercise}\label{ex:amybettycarla}
  Amy ought to have either promised to help Betty or to help Carla. She hasn't
  made either promise. If she had promised to help Betty, she would be obligated
  to help Betty. If she had promised to help Carla, she would be obligated to
  help Carla. So it ought to be the case that Amy is either obligated to help
  Betty or obligated to help Carla. In fact, since Amy made neither promise, she
  is neither obligated to help Betty nor to help Carla. Explain why this casts
  doubt on the assumption that deontic accessibility is euclidean.
\end{exercise}
\begin{solution}
  In the described situation, it ought to be the case that Amy is either
  obligated to help Betty or obligated to help Carla, but Amy is neither
  obligated to help Betty nor to help Carla. So if $p$ translates `Amy helps
  Betty' and $q$ `Amy helps Carla', we seem to have $\Ob(\Ob p \lor \Ob q)$ and
  $\neg \Ob p$ and $\neg \Ob q$. But these assumptions are inconsistent in K5.
  You can draw a K5-tree (using the K-rules and the Euclidity rule) starting
  with $\Ob(\Ob p \lor \Ob q)$ and $\neg \Ob p$ and $\neg \Ob q$ on which all
  branches close. This shows that there is no world in any euclidean model at
  which the three assumptions are true.
  % The deeper point here is that obligations often depend on the circumstances
  % -- for example on what Amy has promised to do. Even if we hold fixed the
  % underlying norms ("keep your promises"), we shouldn't hold fixed the
  % specific obligations and permissions that arise from these norms together
  % with the circumstances. See section \sec{sec:oblig-circ} for more on this
  % theme. (This is a bank robber paradox type case. If we don't hold fixed what
  % was promised then having promised to p does not entail Op:
  % not-promise-and-not-p worlds are OK.)
\end{solution}

% \begin{exercise}
%   The \pr{C4}-schema $\Ob\Ob A \to \Ob A$ is entailed by the \pr{U}-schema
%   $\Ob(\Ob A \to A)$ in the sense that whenever an instance of \pr{U} is true at
%   a world in a model then so is the corresponding instance of \pr{C4}. What about the other direction? Does the \pr{C4}-schema entail the \pr{U}-schema? 
%   \begin{exlist}
%   \item if all instances of \pr{U} are valid on a frame, then so are all instances of \pr{C4};
%   \item it is not the case that if all instances of \pr{C4} are valid on a frame,
%     then so are all instances of \pr{U}.
%     % Need to give a counterexample frame where C4 is valid but not U.
%     % U invalid means not shift reflexive. But still dense. That's
%     % hard. Obvious example: W = Reals, R = <. Then wherever we can
%     % get in one step we can get in two steps (C4), but we don't have
%     % shift reflexivity. Does a tree help?
%   \end{exlist}
% \end{exercise}
% \begin{solution}
%   \begin{sollist}
%     \item I argue by contraposition. Suppose some sentence $\Ob\Ob A \to \Ob A$
%     is invalid on a frame. This means that at some world $w$ in some model $M$
%     based on the frame, $\Ob\Ob A$ is true while $\Ob A$ is false. It follows
%     that there is a world accessible from $w$ at which $A$ is false and $\Ob A$
%     true. So $\Ob A \to A$ is false at $v$. So $\Ob (\Ob A \to A)$ is false at
%     $w$. (You could also give a tree proof with the K-rules showing that \pr{U}
%     entails \pr{C4}.)
    
%     \item It is not enough to give a model in which some instance of \pr{C4} is
%     true at some world while the corresponding instance of \pr{U} is false. For
%     a counterexample, you need to give a \emph{frame} on which every instance of
%     \pr{C4} is valid but not every instance of \pr{U}. Here is one such frame:
%     $W = \{w,v\}$, $wRw$, $wRv$, and $vRw$.
%   \end{sollist}
% \end{solution}

\section{Norms and circumstances}\label{sec:oblig-circ}

The possible-worlds analysis from the previous section assumes that something
ought to be the case iff it is the case at all ideal worlds, where no norms are
violated. Many ordinary statements about oughts and obligations do not fit this
analysis.

Suppose you are walking past a drowning baby. You ought to save the baby. But
are you saving the baby at every world at which no norms are violated? Clearly
not. There are worlds at which the baby never fell into the pond, and others at
which you are overseas and have no means to rescue the baby. These worlds need
not involve any violations of norms.

Whether something ought to be the case depends not just on the norms but also on
the circumstances. Under circumstances in which you have the opportunity of
saving a drowning baby, you ought to save it. Under other circumstances you do
not.

% The point also applies to impersonal oughts. In worlds where an increase of
% greenhouse gases threatens to destabilise the climate, greenhouses gases ought
% to be reduced; in worlds where greenhouse gases never increased, there is no
% imperative for reduction.

We can account for the dependence of obligations on circumstances by changing
our interpretation of the accessibility relation. Previously, we assumed that a
world $v$ is accessible from $w$ iff all the norms at $w$ are respected at $v$.
On the new interpretation, we also require that the relevant circumstances at
$w$ are preserved at $v$. If $w$ is a world at which you come across a drowning
baby then any accessible world will also be a world at which you come across a
drowning baby.

As a first stab, we might redefine deontic accessibility as follows:

\begin{quote}
  A world $v$ is deontically accessible from a world $w$ iff (a) the relevant
  circumstances at $w$ also obtain at $v$, and (b) no norms from $w$ are
  violated at $v$.
\end{quote}
%
I use `relevant circumstances' as a placeholder for the circumstances we hold
fixed when we consider what ought to be the case. Often we hold fixed everything
that is \emph{settled} in the sense we studied in section \ref{sec:systems} --
everything that can no longer be changed. If the baby has fallen into the pond
at $w$, then there is nothing anyone can do to undo the falling; the falling is
a ``relevant circumstance'' that takes place at every world accessible from $w$.

% Humberstone 238 mentions that an example due to Aqvist reveals that the
% circumstances that are held fixed can't be defined temporally. This point is
% also made in von Fintel & Heim: "this fence should be white".

% The worlds of deontic logic are plausibly centred (or world-time pairs) since
% the truth-value of Op at w depends on what is still open at w.

% There's also the issue of actualism. We may or may not hold fixed that Prof
% Procrastinate won't write the review. Or suppose you ought to go to work, but
% don't. If you had gone, you'd have walked past a drowning baby, and you'd have
% been obligated to save it. It seems that at all accessible worlds, you rescue
% the baby. But it's odd to say that you should save the baby. Why? Are we
% holding fixed that you're not going to work? Then you're not even obligated to
% go to work. Apparently what we hold fixed depends on the prejacent.

Clause (b) in the above definition assumes that no norms are violated at any
accessible world. But if accessibility is restricted by circumstances, then this
is implausible, as the relevant circumstances will often involve violations of
norms.

The problem is brought ought by Arthur Prior's ``Samaritan Paradox''.  Suppose
someone has been injured in a robbery, and Jones has the opportunity to help. We
want to say that Jones ought to help the victim. On the possible-worlds analysis
of `ought', this means that Jones helps the victim at all worlds accessible from
the actual world. It follows that the robbery took place at all these
worlds. (In a world without a robbery, there is no victim to help.)  But then
all the accessible worlds contain a violation of norms. In a truly ideal world,
nobody would have been robbed and injured.

In the Samaritan Paradox, the robbery is settled; it has happened at all worlds
that are compatible with the ``relevant circumstances''. None of these worlds is
ideal. Worlds at which Jones doesn't help the victim are even \emph{worse}, in
terms of norm violations, than worlds at which he helps the victim. Both kinds
of worlds are bad, because the victim got robbed. But our norms don't just
divide the possible worlds into good and bad; they allow for finer distinctions
between bad worlds and even worse worlds. Jones ought to help the victim because
that's what he does in the \emph{best} worlds among those he can bring about,
even though none of these worlds are ideal.

% So we need a "better-than" ordering over the worlds, where a world is "better"
% if the agent comes closer to satisfying her obligations. Naively, this might
% be a matter of counting obligation violations, but we may want to classify two
% shopliftings as better than one murder.

% Lewis 1973:98f. discusses whether, in a relativist approach, one needs to
% account for "abnormal" worlds that induce no betterness order, for example
% because they have no god. If there are no spheres around such a world at all,
% nothing is obligatory and everything is permissible. He suggests that one
% might want W to be the trivial unique sphere, so that tautologies are
% obligatory. /Normality/ means no abnormal worlds. Relatedly, he wonders
% whether each world v is in the order induced by any given world w. (He calls
% this /universality/.) /Absoluteness/ is the idea that betterness is not
% world-relative.
%
% If we interpret the counterfactual in terms of betterness, we get conditional
% obligation; the might counterfactual is conditional permission. [100]
%
% Lewis mentions [102] that several treatments of conditional obligation are
% problematic because they validate inferences from 'if A ought C' to 'if A and
% B ought C' or conversely. He suggests that there are Sobel-type
% counterexamples where 'ought C' alternates with further conjuncts in the
% antecedent: `given that Jesse robbed the bank, he ought to confess, but given
% in addition that his confession would send his mother to an early grave, he
% ought not to; etc.' [Note that like Sobel's sequence, these are hard
% to reverse!]
%
% Lewis also mentions in a footnote [102] the circumstance-relativity of
% ordinary oughts: There is a natural way to construe 'It ought to be that psi'
% so that it does become true when Jesse robs the bank. It can be taken as
% tacitly conditional, meaning something like 'Given those actual circumstances
% that now cannot be helped, it ought to be that psi '. But this tacitly
% conditional and time-dependent construal is not the appropriate one when 'It
% ought to be that psi' is used as a reading for the unconditional obligation
% operator of standard tenseless deontic logic.

So here is a second pass at the revised definition of deontic accessibility.

\begin{quote}
  A world $v$ is deontically accessible from a world $w$ iff (a) the relevant
  circumstances at $w$ are also the case at $v$, and (b) $v$ is one of the best
  worlds, by the norms at $w$, among worlds at which the relevant circumstances
  from $w$ are held fixed.
\end{quote}

The revised accessibility relation combines circumstantial and purely deontic
conditions. It can be useful to separate these two components. To this end,
let's first add a circumstantial accessibility relation to our models. In
addition, a model needs to specify which worlds are better than others, relative
to the norms at any given world (which may be the norms at every world, on an
absolutist approach).

Let `$u \prec_w v$' mean that world $u$ is better than world $v$ relative to the
norms at $w$. (The symbol `$\prec$' hints at the idea that $u$ contains
\emph{fewer} violations of norms than $v$.) We assume that for any world $w$,
the relation $\prec_w$ is transitive. We also assume that it is asymmetric,
meaning that if $u \prec_w v$ then it is not the case that $v \prec_w u$.
Asymmetric and transitive relations are known as \textbf{strict partial orders}.

\begin{definition}{}{orderingmodel}
  A \textbf{deontic ordering model} consists of
  \vspace{-3mm}
  \begin{itemize*}
    \item a non-empty set $W$ (the worlds),
    \item a binary relation $R$ on $W$ (the circumstantial accessibility
    relation),
    \item for each world $w\in W$, a strict partial order $\prec_w$ on $W$ (the
    world-relative ranking of worlds as better or worse), and
    \item a function $V$ that assigns to each sentence letter of $\L_M$ a subset
    of $W$.
  \end{itemize*}
\end{definition}

Now we need to say under what conditions a sentence of the form $\Ob A$ is true
at a world in an ordering model. Informally, $\Ob A$ should be true at $w$ iff
$A$ is true at the best worlds among those that are circumstantially accessible.
Let's introduce one more piece of notation. For any set of worlds $S$ and any
partial order $\prec$, let $\emph{Min}^{\prec}(S)$ be the set of $\prec$-minimal
members of $S$:
\[
  \emph{Min}^{\prec}(S) =_\text{def} \{ v: v \in S \land \neg\exists u(u \in S \land u \prec v) \}.
\]
An expression of the form `$\{ x: \ldots x \ldots \}$' denotes the set of all
things $x$ that satisfy the condition $\ldots x \ldots$. So $Min^{<}(S)$ is the
set of all things $v$ that are members of $S$ and for which there are no members
$u$ of $S$ for which $u \prec v$.

Here, then, are the truth-conditions for $\Ob A$ and $\Pe A$ in deontic ordering
models:
\begin{align*}
  M,w \models \Ob A &\text{ \;iff\; $M,v \models A$ for all $v \in \emph{Min}^{\prec_w}(\{ u: wRu\})$}\\
  M,w \models \Pe A &\text{ \;iff\; $M,v \models A$ for some $v \in \emph{Min}^{\prec_w}(\{ u: wRu\})$}
\end{align*}
This is just a formal way of saying that $\Ob A$ is true at $w$ iff $A$ is true
at the best worlds (by the norms at $w$) among the worlds that are circumstantially accessible at $w$.

If we want the \pr{D}-schema to be valid, we have to assume that there is always
at least one best world among the circumstantially accessible worlds, so that
$\emph{Min}^{\prec_w}(\{ u: wRu\})$ is never empty. Let's make this assumption.

We could now study how the logic of obligation and permission depends on formal
properties of the circumstantial accessibility relation $R$ and the deontic
orderings $\prec_w$. In section \ref{sec:systems}, I argued that the logic of
historical necessity (of what is settled and open) is S5. This suggests that in
normal contexts, $R$ is an equivalence relation. If we adopt an absolutist
approach, on which the orderings $\prec_{w}$ are the same for every world $w$, we
then still get KD45. If we allow the orderings to vary from world to world, we
still get D, unless we impose further restrictions on the orderings.

% Why? If R is an equivalence relation then we can ignore the restriction to
% R-accessible worlds: A is true at w in M iff A is true at w in M*, where W* =
% [w]_R and R* is the universal relation. So []A is true at w iff A is true at
% all the w-best worlds. On the absolutist conception, the w-best worlds are
% best relative to every world. So []A is true at w iff A is true at all worlds
% within some fixed set. This yields KD45. On the relativist conception, the
% w-best worlds may be different from the v-best worlds. Every world has a set
% of best worlds, and the box quantifies over that set. This is just as before.

\begin{exercise}
  Suppose fatalism is true and the only world that is open (circumstantially
  accessible) relative to any world $w$ is $w$ itself. Can you describe the
  resulting deontic logic (on either an absolutist or a relativist approach)?
\end{exercise}
\begin{solution}
  Since we assume that there is always at least one best world among the
  accessible worlds, and the accessible worlds comprise just one world, it
  follows that $\Box A$ is true at $w$ iff $A$ is true at $w$. The logic we get
  is the ``Triv'' logic that is axiomatized by adding the \pr{Triv}-schema
  $\Box A \leftrightarrow A$ to the standard axioms and rules for K. This logic
  is stronger than S5: all S5-valid sentences are Triv-valid. (We also have,
  among other things, all instances of $\Box A \leftrightarrow \Diamond A$.) The
  choice between absolutism and relativism makes no difference.
\end{solution}

% \begin{exercise}
%   Can you describe a deontic ordering model for the scenario from exercise
%   \ref{ex:amybettycarla}? 
% \end{exercise}
% \begin{solution}
%   One possible answer: Let $w$ be the world in which the story takes place. At
%   $w$, Amy doesn't make any promises and isn't helping anyone. Let $v$ be a
%   world at which Amy promises to help Betty and keeps her promise. Let $u$ be a
%   world at which Amy promises to help Carla and keeps her promise. $v$ and $u$
%   are better than $w$, and neither is better than the other. $v$ and $u$ are
%   circumstantially accessible from $w$, but not from each other: we can easily
%   make promises, but if we've made a promise we can't easily dispose of the
%   commitment.

  % This is tricky. Strictly, v and u are open only if the promising at these
  % worlds takes place in the future. Then u is accessible from v: at a world
  % where Amy is about to promise to help Betty she can still promise to help
  % Carla. Likewise, v is accessible from u. Both v and u contain zero norm
  % violations. So we can't explain why Ob is true at v. Indeed, in some sense
  % it isn't true that Amy ought (right now) to be obligated to help either
  % Betty or Carla, if it's OK for her to make the promise in the future. (If
  % she will make the promise to help Carla, she /will/ be obligated to help
  % Carla, but she isn't already under that obligation.)
  %
  % Does the puzzle arise more clearly if we assume that Amy ought to have made
  % the promise yesterday? The problem is that if Amy has promised yesterday to help Betty, and we leave open what promise she made (if any) then it isn't true that at the best worlds she helps Betty: she might instead have promised to help Carla and do that. 
% \end{solution}

% \begin{exercise}
%   It is safe to assume that the relevant circumstances at $w$ are the
%   case at $w$ itself. What does this imply about the ``absolutist''
%   logic of obligation and permission on the revised definition of
%   deontic accessibility, where we hold fixed the norms of the actual
%   world?
% \end{exercise}

Ordering models can also help when we want to formalize statements containing
modal operators and if-clauses, like (1)--(3).

\begin{itemize}[leftmargin=10mm]
\itemsep-1mm
\item[(1)] If you smoke then you must smoke outside.
\item[(2)] If you miss the deadline for tax returns then you must pay a fine.
\item[(3)] If you have promised to call your parents then you must call them.
\end{itemize}
%
With the resources of $\L_M$, we seem to face a choice between (W) and (N) for each of these sentences.
\begin{itemize}[leftmargin=10mm]
\itemsep-1mm
\item[(W)] $\Ob(p \to q)$
\item[(N)] $p \to \Ob q$
\end{itemize}
In (W), the operator $\Ob$ is said to have \textbf{wide scope} because it
applies to the entire conditional $p \to q$. In (N), the operator has
\textbf{narrow scope} because it only applies to the consequent $q$.

On reflection, neither translation is satisfactory. Starting with (N), note that
$p \to \Ob q$ and $\neg \Ob q$ together entail $\neg p$. But from (1), together
with the assumption that you are not required to smoke ($\neg \Ob q$), we surely
can't infer that you do not in fact smoke.

% Also, consider (3*) "If you have promised to call your parents then you must
% kill the Prime Minister". If this is translated as (N) then anyone who is
% unsure about $p$ can't be sure that (3*) is false, for $p \to \Ob k$ is true
% whenever $p$ is false.
%
% Note also that (N) is true whenever $\Ob q$ is true. So the narrow-scope
% approach implies that whenever you ought to do something, then you have a
% conditional obligation to do it under any condition whatsoever. But
% intuitively, the fact that you ought to cook dinner does not imply that if
% your child needs urgent medical care then you ought to cook dinner.

(W) is not much better. For one, in our Kripke-style semantics, $\Ob(p \to q)$
is entailed by $\Ob(\neg p)$. But it is easy to imagine a scenario in which you
must not smoke, or you must submit your tax return before the deadline, and yet
(1) and (2) are false.

% Suppose you should not have promised to call your parents: $\Ob \neg p$. On
% the wide-scope approach, we could infer that if you promised to call your
% parents, then you must kill the Prime Minister.

Another problem with both (N) and (W) is that they would license a problematic
form of ``strengthening the antecedent''. For example, they both suggest that
(3)  entails (4).
\begin{itemize}[leftmargin=10mm]
\itemsep-1mm
\item[(4)] If you have promised to call your parents and you know that someone
  has attached a bomb to your parents' phone that will go off if you call, then
  you must call them.
\end{itemize}

\begin{exercise}
  Give a tree proof with the K-rules to show that $p \to \Ob r$ entails
  $(p \land q) \to \Ob r$, and that $\Ob (p \to r)$ entails
  $\Ob((p \land q) \to r)$.
\end{exercise}
\begin{solution}
  Use \href{https://www.umsu.de/trees/}{umsu.de/trees/}.
\end{solution}

Let's think about what is expressed by statements like (1)--(4). Intuitively,
when we ask what must be done if $p$ is the case, we are limiting our attention
to situations in which $p$ is the case, and consider which of \emph{these}
situations best conform to the relevant norms. It is irrelevant whether $p$ is
in fact the case or whether it ought to be the case. (1) says -- roughly -- that
among worlds where you smoke, the ``best'' worlds are worlds where you smoke
outside. Worlds where you smoke inside are worse than worlds where you smoke
outside. Similarly for (2). A world at which you miss the deadline for tax
returns and pay the fine contains only one violation of the tax rules. Worlds at
which you miss the deadline and don't pay the fine contain two. The ``best''
worlds among those at which you miss the deadline are worlds at which you pay
the fine. Likewise for (3). Among worlds at which you have promised to call your
parents, the ``best'' are worlds at which you keep the promise and call them.

The if-clause in sentences like (1)--(3) therefore seems to \emph{restrict} the
worlds over which the modal operator quantifies. Whereas `ought $q$' alone says
that $q$ is true at the best of the open worlds, `if $p$ then ought $q$' says
that $q$ is true at the best of the open worlds \emph{at which $p$ is true}.

There is no way to express these truth-conditions with the resources of $\L_M$.
But we can introduce a new, binary operator for \textbf{conditional obligation}.
The operator is often written `$\Ob(\cdot/\cdot)$', with a slash separating the
two argument places. Intuitively, $\Ob(B/A)$ means that $B$ ought to be the
case if $A$ is the case.

% The precise semantics of $\Ob(\cdot/\cdot)$ is a matter of debate; I will
% sketch one attractive approach, drawing on ideas from Bengt Hansson and
% Angelika Kratzer.

% Hansson 1981:143 suggests something like Kratzer's account, according to
% Hilpinen 170. In the best worlds among those where -h, we have -t. Express this
% by O(-t/-h). We now assume that for any consistent proposition p there is a
% nonempty set of optimal p-worlds, generalising D.

% Chellas 276 suggests that O(/) can be defined as $A \Rightarrow O(B)$, for a
% Lewis-Stalnaker-selection-type conditional $\Rightarrow$. Others suggest to use
% defeasible conditionals.

% Neighbourhood semantics doesn't help much. In CTD situations, we get conflicting
% obligations. But we want to know more about how the obligations relate.

The formal truth-conditions for $\Ob(B/A)$ are much like those for $\Ob B$, except
that we add the assumption $A$ to the circumstances that are held fixed:

\bigskip

\quad$M,w \models \Ob (B/A) \text{ iff $M,v \models B$ for all
  $v \in \emph{Min}^{\prec_w}(\{ u: wRu \text{ and } M,u\models A \})$}$.

\bigskip\noindent%
%
Here, $\{ u: wRu \text{ and } M,u\models A \}$ is the set of worlds $u$ that are
circumstantially accessible from $w$ and at which $A$ is true.
$\emph{Min}^{\prec_{w}}(\{ u: wRu \text{ and } M,u\models A \})$ is the set that
comprises the best of these worlds. So $\Ob(B/A)$ is true at $w$ iff $B$ is true
at all of the best $A$-worlds that are accessible at $w$.

% Hilpinen: Suppose p entails r, and p is true at some of the r-optimal worlds.
% I.e., some of the best r-worlds are (more strongly) p-worlds. Then the best
% p-worlds are the r-optimal worlds where p is true. I.e., optimality is subject
% to the condition:

% If $[[p]] \subseteq [[r]]$ and $[[p]] \cap Opt(r,w)$ is non-empty, then
% $Opt(p,w) = [[p]] \cap Opt(r,w)$.

% It looks like this is entailed by my ordering semantics.

% Syntactically, this means that $\Ob p$ and $\Ob (q/p)$ entail $\Ob(q)$. I.e., we
% have ``deontic detachment''. But we don't have ``factual detachment'': $p$ and
% $\Ob(q/p)$ does not entail $\Ob q$.
%
\begin{exercise}
  ``Deontic detachment'' is the inference from $\Ob A $ and $\Ob(B/A)$ to
  $\Ob B$. ``Factual detachment'' is the inference from $A$ and $\Ob(B/A)$ to
  $\Ob B$. Which of these are valid on the present semantics?
\end{exercise}
\begin{solution}
  Deontic detachment is valid. Suppose $A$ is true at the best of the
  (circumstantially) accessible worlds, and $B$ is true at the best of the
  accessible worlds at which $A$ is true. Then $B$ is true at the best of the
  accessible worlds.

  Factual detachment is invalid. A counterexample is the ``gentle murder
  puzzle''. Suppose John is determined to kill his grandmother. \emph{If he will
    go ahead and kill her, he ought to do so gently}. Can we conclude that John
  ought to gently kill his grandmother? Arguably not. He shouldn't kill her at
  all! We have $k$ and $\Ob(g/k)$, but not $\Ob(g)$. Formally, $g$ is true at
  the best of the accessible $k$-worlds, but since all the $k$-worlds are quite
  bad, $g$ is not true at the best of the accessible worlds.
\end{solution}

% \begin{exercise}
%   Shelly in standing in front of a burning building. Trapped inside
%   are two small babies. Shelly could enter the building and rescue
%   them, but at the cost of suffering crippling and probably fatal
%   burns. Consider the following three possibilities:
%   \begin{enumerate*}
%   \item[(A)] Shelly stays out and rescues neither baby.
%   \item[(B)] Shelly enters the building and rescues only one of the
%     babies, although he could easily and without any further costs
%     have rescued both.
%   \item[(C)] Shelly enters the building and rescues both the babies.
%   \end{enumerate*}
%   Intuitively, (A) and (C) are permissible but (B) is not.
% \end{exercise}

\begin{exercise}\label{ex:chisholmsparadox}
  In exercise \ref{ex:translate-sdl}, you were asked to translate the following
  statements.
  \begin{exlist}
    \item[(c)] Jones ought to help his neighbours.
    \item[(d)] If Jones is going to help his neighbours, then he ought to tell them
    he's coming.
    \item[(e)] If Jones isn't going to help his neighbours, then he ought to not  tell them he's coming.
  \end{exlist}
  \medskip\noindent%
  Let's add a fourth statement:
  \begin{exlist}
    \item[(f)] Jones is not going to help his neighbours.
  \end{exlist}
  \medskip\noindent%
  These four statements are intuitively consistent and independent: There are
  conceivable scenarios in which all four are true, and none of them is entailed
  by the others. Explain why your translations in exercise
  \ref{ex:translate-sdl} violate either consistency or independence. (This puzzle is due to Roderick Chisholm.)
\end{exercise}
\begin{solution}
  (c) can obviously be translated as $\Ob p$, (f) as $\neg p$.
  
  If (d) is translated as $p \to \Ob q$ then it is entailed by (f). We would
  violate independence. (More directly: the translation can't be right because
  it is easy to think of a scenario in which (f) is true but (d) false.) Assume
  then that (d) is translated as $\Ob(p \to q)$.
  % This is equivalent to $\Ob(q/p)$ in the presence of $\Ob(p)$ or even $\Pe(p)$. We only need the conditional operator to talk about what should be the case under non-ideal conditions.

  If (e) is similarly translated as $\Ob(\neg p \to \neg q)$ then it is entailed
  by (c). We would violate independence. (More directly: it is easy to think of
  a scenario in which (c) is true but (e) false.) If (e) is translated as
  $\neg p \to \Ob \neg q$ then (a)--(c) are logically inconsistent. To show
  this, you can, for example, start a D-tree with the four assumptions $\Ob p$,
  $\neg p$, $\Ob(p \to q)$, and $\neg p \to \Ob \neg q$, all at the same world
  $w$. The tree will close.
\end{solution}

% \begin{exercise}
%   What if we analyse (d) and (e) in terms of a necessity modal $\Box$ in
%   addition to the deontic modal $\Ob$? $\Box(p \to Oq)$ where box is some
%   other kind of necessity? Still monotonic.
% \end{exercise}

\begin{exercise}
  The dual of conditional obligation is conditional permission. Spell out
  truth-conditions for $\Pe(B/A)$ that parallel the truth-conditions I have
  given for $\Ob(B/A)$, so that $\Pe(B/A)$ is equivalent to
  $\neg \Ob(\neg B/A)$.
  
  % Explain why `if you have a disability, you can park in front of the
  % entrance' is not adequately translated as either $d \to \Pe(p)$ or
  % $\Pe(d \to p)$.

  % Need better example? Maybe there is none?!

  % Here we don't need to assume SDL to see that the wide-scope transaction is
  % wrong. Intuitively, what's permitted is the conjunction of having a
  % disability and parking at the entrance, not the disjunction of having no
  % disability and parking at the entrance.
\end{exercise}
\begin{solution}
  Simply replace `all' in the semantics for $\Ob(B/A)$ with `some'.
\end{solution}

% An argument against the p->P(q) analysis, from Rescher:
% 1. P(q/p) is equivalent to -O(-q/p).
% 2. p->P(q) entails (p&r)->P(q) [even if -> is strict].
% 3. By the analysis, -O(-q/p) would entail -O(-q/p&r). 
% 4. So O(-q/p&r) would entail O(-q/p), which is absurd.

% Also, strengthening the "antecedent" seems problematic. As Rescher mentions,
% if we allow P(p/q) to entail P(p/q&r), then "before we can assert that an
% action is permitted in some particular circumstance, we must specify this
% circumstance in such a way as to exclude all imaginable countervailing
% conditions". -- The Kratzer account suggests that you're permitted to p if q
% provided that there's one specific way of doing p that's allowed if q.
% Intuitively, many ways are allowed. Tthis is Lewis's "problem about
% permission" aka the paradox of free choice.

% There is also `if ... better'. Dreier discusses a case like BRTD: ``if there
% is war, it's better to disarm''; ``if there is peace, it's better to disarm'';
% but it doesn't follow that it's better to disarm.

% Such cases provide a powerful argument against wide-scoping. How is a
% wide-scope analysis even supposed to go? The only plausible candidate is:
% 'Better(if w than d, if w than -d)'. But that surely gets the TCs wrong.

% Similarly for quasi-cardinal modals like `it would be great that', or `it is
% highly praiseworthy that'.

% TODO: spell out this argument. Can we prove triviality? Can we refute the
% defeasible wide-scopers?

% How should `if ... better' be analysed? Roughly, among the antecedent worlds,
% the A worlds are better than the B worlds. But not every A world is better
% than every B world. We need to take the weighted average.

  
\section{Further challenges}

Many apparent problems for standard deontic logic arise from the dependence of
obligations on circumstances. We can avoid these problems by using deontic
ordering models and formalizing conditional obligation statements with the
binary $\Ob(\cdot/\cdot)$ operator. There are, however, other problems and
``paradoxes'' for which this move doesn't help. I will mention three.

First, we already saw that standard deontic logic does not allow for conflicting
obligations. Suppose you have promised your family to be home for dinner and
your friends to join them at the pub. You are under conflicting \emph{prima
  facie} obligations. It is not clear that one of them overrides the other.
Legal systems can also contain contradictory rules, without any higher-level
rules for how to resolve such contradictions.

We can, of course, drop principle \pr{D}. But even in the minimal logic K,
$\Ob p$ and $\Ob \neg p$ entail $\Ob A$, for any sentence $A$. Intuitively,
however, the fact that you have given incompatible promises does not entail that
you are obligated to, say, kill the Prime Minister.

% Chellas argues that we should at least be able to distinguish
% $\neg (O(A) \land O(\neg A))$ from $\neg \Box \bot$, which are equivalent in
% K. He suggests we should look for a weaker logic in which obligation is still
% closed under consequence (because we have the rule
% $\vdash A \to B \therefore \vdash \Box A \to \Box B$) and there are no
% impossible obligations -- $\neg \Box \bot$ -- but in which we don't have
% $\Box \top$, so that there can be worlds without obligations, and we don't
% have $(\Box A \land \Box B) \to \Box (A \land B)$, giving up a unique standard
% of obligation (the ideal worlds) and allowing for deontic dilemmas. But then a
% weaker semantics must be given.

% Chellas's logic retains $\Box A \to \Box (A \lor B)$. This may seem
% problematic.

Another family of problems arises from the fact that in any logic defined in
terms of Kripke models, $\Ob$ is closed under logical consequence, meaning that
if $\Ob A$ is true and $A$ entails $B$, then $\Ob B$ is true. Since logical
truths are logically entailed by everything, it follows that all logical truths
come out as obligatory. (This is easy to see semantically. A logical truth is
true at all worlds; so it is true at all deontically accessible worlds.) But ought it to be the case that it either rains or doesn't rain?

In response, one might argue that the relevant statements sound wrong not
because they are false, but because their utterance would violate a pragmatic
norm of cooperative communication. A basic norm of pragmatics is that utterances
should make a helpful contribution to the relevant conversation. In a normal
conversational context, it would be pointless to say that something ought (or
ought not) to be the case if it is logically guaranteed to be the case anyway.
An utterance of `it ought to be that $p$' is pragmatically appropriate only if
$p$ could be false. This might explain why it sounds wrong to say that it ought
to either rain or not rain.

Note also that by duality, $\neg\Ob(p \lor \neg p)$ entails
$\Pe \neg(p \lor \neg p)$. If we deny that it ought to either rain or not rain,
and we accept the duality of obligation and permission, we have to say that it
is permissible that it neither rains nor doesn't rain. That sounds even worse.

The problem of closure under entailment has special bite when obligation
statements are restricted by circumstances. Return to the Samaritan
puzzle. Suppose the victim is bleeding, and Jones ought to stop the blood
flow. It is logically impossible to stop a blood flow if no blood is flowing. In
all the deontic logics we have so far considered, the claim that Jones ought to
stop the victim's blood flow therefore entails that the victim ought to be
bleeding. But wouldn't it be better if the victim weren't bleeding?

Here, too, one might appeal to a pragmatic explanation. When we say that Jones
ought to stop the blood flow, we take for granted that the victim is bleeding.
We are interested in what should be done \emph{given} the state in which Jones
found the victim. Worlds where the victim isn't injured are set aside; they are
not circumstantially accessible. But circumstantial accessibility can shift with
conversational context. The claim that the victim ought to be bleeding is
pointless if we hold fixed the victim's state of injury. So when we evaluate
\emph{this} claim, we naturally assume that the relevant circumstantial
accessibility relation does not hold fixed the injuries. Intuitively, we are no
longer considering what should be done given the state in which Jones found the
victim, but whether that state itself should have obtained. Worlds in which
the state doesn't obtain become circumstantially accessible.

% Similarly: $OK_ap$ entails $Op$ by factivity of K. If Gladys ought
% to know that there's a fire then there ought to be a fire. (Aqvist 1967)

% And similarly for conditional obligations: $\Ob(p/p)$ is valid.

A third family of problems arises from disjunctive statements of permission and
obligation. Consider (1).
\begin{itemize}[leftmargin=10mm]
\itemsep-1mm
\item[(1)] You ought to either mail the letter or burn it.
\end{itemize}
Intuitively, (1) suggests that both mailing the letter and burning it are
permitted. In standard deontic logic, however, $\Ob(A \lor B)$ does not entail
$\Pe A \land \Pe B$. (This puzzle was first noticed by Alf Ross and is known as
``Ross's Paradox''.)

A similar puzzle arises for permissions. (This one is known as the ``Paradox of
Free Choice''.)
\begin{itemize}[leftmargin=10mm]
\itemsep-1mm
\item[(2)] You may have beer or wine.
\end{itemize}
Intuitively, (2) implies that beer and wine are both permitted. But in standard
deontic logic, $\Pe(A \lor B)$ does not entail $\Pe A \land \Pe B$.

We could add the missing principles.
%
\begin{principles}
  \pri{R}{\Ob(A \lor B) \to (\Pe A \land \Pe B)}\\
  \pri{FC}{\Pe(A \lor B) \to (\Pe A \land \Pe B)}
\end{principles}
%
But both of these have unacceptable consequences when added to the minimal modal
logic K. With the help of \pr{R}, we could show that $\Ob A$ entails $\Pe B$:
$\Ob A$ entails $\Ob (A \lor B)$, which by \pr{R} entails $\Pe B$. But clearly
`you ought to mail the letter' does not entail `you may burn the letter'.
Similarly for \pr{FC}. In K, $\Pe A$ entails $\Pe(A \lor B)$; by \pr{FC},
$\Pe(A \lor B)$ entails $\Pe B$. But `you may have beer' does not entail `you
may have wine'.

% \pr{R} is not valid in any class of Kripke frames. But \pr{FC} is valid e.g.
% in frames with only dead ends, where every diamond sentence is false.

% There are also independent reasons to doubt the validity of \pr{R} and
% \pr{FC}. For example, suppose you are \emph{not} permitted to have beer or
% wine. Intuitively, this means that you are not permitted to have beer and you
% are not permitted to have wine. So the following principle also looks
% plausible.
% %
% \principle{FC$'$}{\neg \Pe(A \lor B) \to (\neg \Pe A \land \neg \Pe B)} 
% %
% But \pr{FC} and \pr{FC$'$} can't both be valid. Otherwise we could show that
% if anything is forbidden, then everything whatsoever is forbidden:
% %
% \begin{alignat*}{2}
%   1.\quad& \text{Suppose  $\neg \Pe A$.} &\quad& \\ 
%   2.\quad& \text{Then $\neg(\Pe A \land \Pe B)$.} &\quad& \text{(By propositional logic)}\\
%   3.\quad& \text{Then $\neg\Pe (A \lor B)$.}  &\quad& \text{(By \pr{FC} and modus tollens)}\\
%   4.\quad& \text{Then $\neg\Pe A \land \neg \Pe B$.}  &\quad& \text{(By \pr{FC$'$})}\\
%   5.\quad& \text{So $\neg\Pe B$.}  &\quad& \text{(By propositional logic)}
% \end{alignat*}

\begin{exercise}
  Analogous puzzles to those raised by Ross's Paradox and the Paradox of Free
  Choice arise for epistemic `must' and `might'. Can you give examples?
\end{exercise}
\begin{solution}
  Ross's Paradox: `Alice must be in the office or in the library'
  seems to imply that Alice might be in the office and that she might
  be in the library.

  The Paradox of Free Choice: `Alice might be in the office or in the
  library' seems to imply that Alice might be in the office and that
  she might be in the library.
\end{solution}


\section{Neighbourhood semantics}\label{sec:neighbourhood}

In reaction to apparent problems for standard deontic logic, some have argued
that we should not interpret obligation and permission in terms of
quantification over possible worlds. If we give up this core tenet of Kripke
semantics, we can define ``non-normal'' logics weaker than K. (A \textbf{normal}
modal logic is a modal logic that can be defined in terms of classes of Kripke
frames.)

% In C.I.\ Lewis's 1932 list of modal logics, S4 and S5 were normal, but S1--S3
% were non-normal.

A popular alternative to Kripke semantic is \textbf{neighbourhood semantics},
also known as Scott-Montague semantics, after its inventors Dana Scott and
Richard Montague.

Models in neighbourhood semantics still involve possible worlds. Validity is
still defined as truth at all worlds in all (suitable) models. But the box and
the diamond are no longer interpreted as quantifiers over accessible
worlds. Instead, we simply assume that at every world, some propositions are
``necessary'' and others are not.  $\Box A$ is true at a world if $A$ expresses
one of the necessary propositions at that world.

Formally, the accessibility relation in Kripke models is replaced by a
\textbf{neighbourhood function} $N$ that associates each world in a model with
the propositions that are necessary relative to $w$. Propositions are identified
with sets of possible worlds. Thus $N(w)$ is a set of sets of worlds. Each set
of world in $N(w)$ is necessary at $w$.

% We could equivalently think of N as a relation relating each world to the
% propositions in its neighbourhood.

\begin{definition}{}{neighbourhoodmodel}
  A \textbf{neighbourhood model} consists of
  \vspace{-3mm}
  \begin{itemize*}
  \item a non-empty set $W$,
  \item a function $N$ that assigns to each member of $W$ a set of subsets of
  $W$, and
  \item a function $V$ that assigns to each sentence letter of $\L_M$
    a subset of $W$.
  \end{itemize*}
\end{definition}

The interpretation of non-modal sentences at neighbourhood models works just as
in Kripke semantics (definition \ref{def:kripkesemantics}). To state the
semantics for modal sentences, let $[A]^M$ be the set of worlds in model $M$ at
which $A$ is true. This is our proxy for the proposition expressed by $A$. Then:
%
\begin{align*}
  M,w \models \Box A &\text{ \;iff\; $[A]^M$ is in $N(w)$}.\\
  M,w \models \Diamond A &\text{ \;iff\; $[\neg A]^M$ is not in $N(w)$.}
\end{align*}
%
Intuitively, the clause for the box says that $\Box A$ is true at $w$ iff the
proposition expressed by $A$ is one of those that are necessary at $w$. The
clause for the diamond ensures that the box and the diamond are duals.

In neighbourhood semantics, the modal operators are not closed under logical
consequence. The neighbourhood function $N$ can easily make $p$ necessary at a
world without making $p\lor q$ necessary, even thought $p$ entails $p \lor q$.
If we interpret $\Ob$ and $\Pe$ as the box and the diamond in neighbourhood
semantics, we can therefore say that Jones ought to tend to the victim's
injuries even thought it is not the case that someone ought to be injured.

% \begin{exercise}
%   What formal condition on the neighbourhood function would ensure that $\Box$
%   is closed under logical consequence?
% \end{exercise}
% \begin{solution}
%   Whenever $X \in N(w)$ then all sets that have $X$ as a subset are in
%   $N(w)$.
% \end{solution}

We can also allow for conflicting obligations. If the laws at $w$ require both
$p$ and $\neg p$, we simply have $[p]^M \in N(w)$ and $[\neg p]^M \in N(w)$. It
longer follows that any proposition whatsoever is obligatory.

We may further hope to escape the problems from section \ref{sec:oblig-circ}
that led us to introduce a primitive conditional obligation operator. I argued
that the wide-scope translation $\Ob(A \to B)$ of conditional obligation
sentences is problematic because $\Ob(A \to B)$ is entailed by $\Ob(\neg A)$. In
neighbourhood semantics, this entailment fails. 

Bare neighbourhood semantics determines a very weak logic called \textbf{E}. It
is axiomatized by \pr{Dual}, \pr{CPL}, and a rule (called ``RN'') that allows
inferring $\Box A \leftrightarrow \Box B$ from $A \leftrightarrow B$. We can get
stronger logics, with more validities, by imposing conditions on the
neighbourhood function $N$.

For example, suppose we want to maintain that if something is logically
guaranteed to be true, then it can't be forbidden. Equivalently, any logically
necessary truth should be permitted. By the neighbourhood semantics for $\Pe$,
$A$ is permitted at a world $w$ in a model $M$ iff $[\neg A]^M$ is not in
$N(w)$. If $A$ is a logical truth, then $A$ is true at all worlds; in that case,
$\neg A$ is true at no worlds, and $[\neg A]^M$ is the empty set. If we want
logical truths to be permissible, we therefore have to stipulate that $N(w)$
never contains the empty set.

In Kripke semantics, the assumption that logically necessary truths are
permitted is equivalent to the assumption that (every instance of) the
\pr{D}-schema $\Ob A \to \Pe A$ is valid. Both assumptions correspond to
seriality of the accessibility relation. In neighbourhood semantics, we can
distinguish between the two assumptions. While the permissibility of logical
truths requires that $N(w)$ doesn't contain the empty set, the validity of
$\Ob A \to \Pe A$ requires that $N(w)$ doesn't contains contradictory
propositions $[A]^M$ and $[\neg A]^M$.

If we assume that the neighbourhood function is closed under intersection, in
the sense that whenever two sets $X$ and $Y$ are in $N(w)$ then so is their
intersection $X\cap Y$, then $(\Box A \land \Box B) \to \Box (A \land B)$
becomes valid. If we also require the converse, that whenever $X\cap Y \in N(w)$
then $X \in N(w)$ and $Y\in N(w)$, and in addition that $W \in N(w)$, we get
back the minimal normal logic K.

% Humberstone 160: [](A & B) -> []A requries that whenever X in N(w) and Y is a
% superset of X then Y is in N(w).

\begin{exercise}
  Can you find a condition on the neighbourhood function that renders the
  \pr{T}-schema valid?
\end{exercise}
\begin{solution}
  For every world $w$, every member of $N(w)$ contains $w$.
\end{solution}

For some purposes, even the minimal logic of neighbourhood semantics is too
strong. Return to the intuitive ``Free Choice'' principle from the previous
section:
%
\principle{FC}{\Pe(A \lor B) \to (\Pe A \land \Pe B)}
%
We have seen that this principle is untenable in Kripke semantics. It is
still untenable in neighbourhood semantics.

To see why, note first that whenever two sentences $A$ and $B$ are logically
equivalent, then in neighbourhood semantics $\Pe A$ and $\Pe B$ are also
equivalent. The reason is that the modal operators in neighbourhood semantics
operate on the set of worlds at which the embedded sentence is true. If $A$ and
$B$ are logically equivalent, then in any model $M$, the set $[A]^M$ is the same
set as $[B]^M$, and so $[A]^M$ is in $N(w)$ iff $[B]^M$ is in $N(w)$. Likewise,
$[\neg A]^M$ is in $N(w)$ iff $[\neg B]^M$ is in $N(w)$. (That's why the RN rule preserves validity.)

Now any sentence $A$ is logically equivalent to
$(A \land B) \lor (A \land \neg B)$, for any $B$. In the logic E, $\Pe A$
therefore entails $\Pe ((A \land B) \lor (A \land \neg B))$. By \pr{FC},
$\Pe ((A \land B) \lor (A \land \neg B))$ entails $\Pe (A \land B)$. We could
still reason from `you may have a cookie' to `you may have a cookie and burn
down the house'.

\begin{exercise}
  Rational beliefs come in degrees, which are often assumed to satisfy the
  formal rules of probability. Suppose we say that someone believes $A$ iff
  their degree of belief in $A$ is above a certain threshold -- say, 0.9.
  Explain why one can't give a Kripke semantics for this concept of belief.
  (Although one can give a neighbourhood semantics.) \emph{Hint}: One rule of
  probability says that if $p$ and $q$ are independent propositions, then the
  probability of their conjunction $p \land q$ is the product of their
  individual probabilities.
\end{exercise}
\begin{solution}
  In Kripke semantics, $\Box p$ and $\Box q$ together entail $\Box(p \land
  q)$. But if the probability of $p$ is above the threshold and the probability
  of $q$ is above the threshold, it does not follow that the probability of
  $p\land q$ is above the threshold. For example, we could have Pr$(p)=0.95$,
  Pr$(q)=0.94$, and Pr$(p \land q) = 0.95 \times 0.94 = 0.893$.
\end{solution}

\begin{exercise}
  Some have argued that the logic of ability is weaker than K, on the grounds
  that there are cases in which an agent is able to make $p \lor q$ true, but
  not able to make $p$ true and not able to make $q$ true -- which would provide a counterexample to the K-valid schema $\Diamond(A \lor B) \to (\Diamond A \lor \Diamond B)$. Can you describe a case of this type?
\end{exercise}
\begin{solution}
  A bad dart player may have the ability to hit the dart board but lack the
  ability to hit the left half of the board and also the ability to hit the
  right half of the board.  
\end{solution}



%%% Local Variables: 
%%% mode: latex
%%% TeX-master: "logic2.tex"
%%% End:
