\chapter{Accessibility}\label{ch:accessibility}

\section{Variable modality}

In the previous chapter, we read $\Box A$ as saying that $A$ is true at every
possible world. We might hope to allow for different flavours of modality by
associating each flavour with a different space of worlds. If the box
represents epistemic necessity, a possible world would be a world that is
compatible with the available information. If the box represents historical
necessity, a possible world would be one that can be brought about. If the box
represents obligation, a possible world would be a world in which all relevant
norms are respected. (These worlds are more commonly called \emph{ideal}.)

But there is a problem. The semantics from the previous chapter determines a
particular logic: S5. This logic is not appropriate for every application of
modal logic. In deontic logic, for example, we don't want the schema
%
\begin{principles}
  \pri{T}{\Box A \to A}
\end{principles}
%
to be valid. We can easily conceive of scenarios in which $\Box p$ is true (on
some interpretation of $p$) even though $p$ is false.

The semantics from the previous chapter renders the \pr{T}-schema valid.
Whenever a sentence $\Box A$ is true at a world $w$ in a model then $A$ is true
at $w$ as well, because the box quantifies over all worlds, including $w$. To
make room for deontic logic, we need a semantics in which not all worlds in $W$
are among the ``possible'' worlds over which the modal operators quantify. Not
all worlds are ideal.

We might also want to allow that the worlds over which the modal operators
quantify depend on the world at which the relevant sentence is evaluated.
Perhaps you are obligated to do the dishes in worlds where you have promised to
do the dishes, but not in worlds where you haven't made the promise. Worlds in
which you don't do the dishes are then ideal relative to the second kind of
world, but not relative to the first.

This kind of variability is also needed for other flavours of modality. Suppose
the box quantifies over all worlds that are compatible with our knowledge. Which
worlds are compatible with our knowledge depends on what we know. But we don't
always know what we know. Sometimes we believe that we know something, but don't
actually know it because it is false. We don't know it, without knowing that we
don't know it. \label{par:notB}Among the worlds compatible with our knowledge
are then worlds in which we know more than we actually do. What's
compatible with our knowledge in \emph{these} worlds is different from what's
compatible with our knowledge in the actual world.

Let's assume, then, that for any world in any scenario there is a set of worlds
that are possible \emph{relative to $w$}. We assume that $\Box p$ as true at $w$
iff $p$ is true at all worlds that are possible relative to $w$. If a world $v$
is possible relative to $w$ we also say that $v$ is \textbf{accessible} from $w$,
or (informally) that $w$ \emph{can see} $v$.

Accessibility means different things in different applications. In epistemic
logic, a world $v$ is accessible from $w$ iff $v$ is compatible with what is
known at $w$. In the logic of historical necessity, $v$ is accessible from $w$
iff $v$ can be brought about at $w$. And so on. We can still allow for scenarios
in which every world is accessible from every world, so that the box and the
diamond are unrestricted quantifiers over all worlds in the scenario, as in the
previous chapter.

Since facts about accessibility matter to the truth-value of modal sentences,
they must be represented by our models. From now on, a model for $\L_M$ will
therefore specify which worlds in $W$ are accessible from which others (and from
themselves). This marks the difference between a ``basic model'' and a ``Kripke
model'' -- named after Saul Kripke, who popularised models of this kind.
%
% see Burgess, ``Kripke Models'', 2011, for more on the history of Kripke models.
% 
\begin{definition}{}{kripkemodel}
  A \textbf{Kripke model} of $\L_M$ is a triple $\t{W,R,V}$ consisting of
  \vspace{-3mm}
  \begin{itemize*}
  \item a non-empty set $W$,
  \item a binary relation $R$ on $W$, and
  \item a function $V$ that assigns to each sentence letter of $\L_M$
  a subset of $W$.
  \end{itemize*}
\end{definition}
%
\noindent%
$R$ is the accessibility relation. It is called a relation ``on $W$'' because it
holds between members of $W$. We write `$wRv$' to express that $R$ holds between
$w$ and $v$.

We also need to update definition \ref{def:basicsemantics}, which settles under
what conditions an $\L_M$-sentence is true at a world in a model. The old
definition had the following clauses for the box and the diamond:

\bigskip
\begin{tabular}{lll}
  (g) & $M,w \models \Box A$ &iff $M,v \models A$ for all $v$ in $ W$.\\
  (h) & $M,w \models \Diamond A$ &iff $M,v \models A$ for some $v$ in $ W$.
\end{tabular}
\bigskip

\noindent%
In the new semantics, the box and the diamond only quantify over
accessible worlds:

\bigskip
\begin{tabular}{lll}
  (g) & $M,w \models \Box A$ &iff $M,v \models A$ for all $v$ in $ W$ such that $wRv$.\\
  (h) & $M,w \models \Diamond A$ &iff $M,v \models A$ for some $v$ in $ W$ such that $wRv$.
\end{tabular}
\bigskip

Here is the full definition, for completeness.

\begin{definition}{Kripke Semantics}{kripkesemantics}
  If $\Mfr = \t{W,R,V}$ is a Kripke model, $w$ is a member of $W$, $P$ is
  any sentence letter, and $A,B$ are any $\L_M$-sentences, then

  \medskip
  \begin{tabular}{lll}
    (a) & $M,w \models P$ &iff $w$ is in $V(P)$.\\
    (b) & $M,w \models \neg A$ &iff $M,w \not\models A$.\\
    (c) & $M,w \models A \land B$ &iff $M,w \models A$ and $M,w \models B$.\\
    (d) & $M,w \models A \lor B$ &iff $M,w \models A$ or $M,w \models B$.\\
    (e) & $M,w \models A \to B$ &iff $M,w \not\models A$ or $M,w \models B$.\\
    (f) & $M,w \models A \leftrightarrow B$ &iff $M,w \models A\to B$ and $M,w \models B\to A$.\\
    (g) & $M,w \models \Box A$ &iff $M,v \models A$ for all $v$ in $ W$ such that $wRv$.\\
    (h) & $M,w \models \Diamond A$ &iff $M,v \models A$ for some $v$ in $ W$ such that $wRv$.
  \end{tabular}
\end{definition}
%
When I speak of truth at a world in a Kripke model, this should always be
understood in accordance with definition \ref{def:kripkesemantics}. Definition
\ref{def:basicsemantics} defines truth at a world in a basic model.

To see definition \ref{def:kripkesemantics} in action, consider a simple model
with two worlds, $w$ and $v$. World $v$ is accessible from world $w$, but $w$ is
not accessible from $v$. Neither world can access itself. The interpretation
function assigns $\{ v \}$ to $p$ and the empty set $\emptyset$ to all other
sentence letters. The model can be pictured as follows, with an arrow
representing accessibility:

\begin{center}
  \begin{tikzpicture}[modal, world/.append style={minimum size=0.8cm}, node distance = 15mm]
    \node[world] (w1) [label=above:{$w$}] {};
    \node[world] (w2) [label=above:{$v$}, right=of w1] {$p$};
    \path[->] (w1) edge (w2);
  \end{tikzpicture}
\end{center}
%
Using definition \ref{def:kripkesemantics}, we can figure out which $\L_M$-sentences
are true at which worlds in the model. For example:

\begin{itemize*}
  \item By clause (a) of definition \ref{def:kripkesemantics}, $p$ is true at
  $v$ and false at $w$.
  \item By clause (h), $\Diamond p$ is true at $w$ because $p$ is true at $v$
  and $v$ is accessible from $w$. $\Diamond p$ is false at $v$ because there is
  no world accessible from $v$ at which $p$ is true.
  \item By clause (g), $\Box\Diamond p$ is false at $w$ because $\Diamond p$ is
  false at $v$ and $v$ is accessible from $w$. $\Box\Diamond p$ is true at $v$
  because there is no world accessible from $v$ at which $\Diamond p$ is false.
\end{itemize*}
%
Note that $\Diamond p$ and $\Box \Diamond p$ have different truth-values at $w$
(and at $v$). In the new semantics, we can no longer ignore all but the last in a
string of modal operators. Note also that $\Box p$ is true at $w$ even though
$p$ is false; $\Box p \to p$ is no longer valid.

\begin{exercise}
  Explain why every sentence of the form $\Box A$ is true at world $v$
  in the above model.
\end{exercise}
\begin{solution}
  $v$ has access to no world. So any sentence $A$ is true at
  \emph{all} (zero) worlds accessible from $v$.

  If this seems strange, remember that $\Box A$ is equivalent to
  $\neg \Diamond \neg A$. And $\Diamond \neg A$ means that there's an
  accessible world where $\neg A$ is true. If there are no accessible
  worlds, then this is false. So $\neg \Diamond \neg A$ is true.
\end{solution}

The next three exercises refer to the following model:
%
\begin{center}
  \begin{tikzpicture}[modal, world/.append style={minimum size=0.8cm}, node distance = 15mm]
    \node[world] (w1) [label=above:{$w_1$}] {$p$};
    \node[world] (w2) [label=above:{$w_2$}, right=of w1] {};
    \node[world] (w3) [label=below:{$w_3$}, below=of w1] {$p$};
    \node[world] (w4) [label=below:{$w_4$}, right=of w3] {$q$};
    \path[->] (w1) edge (w4);
    \path[<-] (w1) edge (w2);
    \path[<->] (w1) edge (w3);
    \path[->] (w2) edge (w4);
    \path[->] (w4) edge [reflexive right] (w4);
  \end{tikzpicture}
\end{center}

\begin{exercise}
  At which worlds in the  model are the following sentences true?
  \begin{exlist}
  \item $p \lor \neg q$
  \item $\Box(p \lor \neg q)$
  \item $\Diamond(\neg p \land \neg q)$
  \item $\Diamond\Box q$
  \item $\Diamond\Diamond\Box q$
  \end{exlist}
\end{exercise}
\begin{solution}
  (a) $w_{1}, w_{2}$, and $w_{3}$; (b) $w_{3}$; (c) --; (d) $w_{1}, w_{2}$ and $w_{4}$; (e) all.
\end{solution}

\begin{exercise}
  For each world in the model, find an $\L_M$-sentence that is true only at
  that world.
\end{exercise}
\begin{solution}
  There are infinitely many correct answers for each world. For
  example: $w_1: \Diamond\Box p$, $w_2: \neg p \land \neg q$,
  $w_3: \Box p$, $w_4: \Box q$.
\end{solution}

\begin{exercise}
  Can you draw a diagram of a smaller model (with fewer worlds) in
  which the exact same $\L_M$-sentences are true at $w_1$?
\end{exercise}
\begin{solution}
  \begin{tikzpicture}[modal, world/.append style={minimum size=0.8cm}, node distance = 15mm]
    \node[world] (w1) [label=above:{$w_1$}] {$p$};
    \node[world] (w3) [label=below:{$w_3$}, below=of w1] {$p$};
    \node[world] (w4) [label=below:{$w_4$}, right=of w3] {$q$};
    \path[->] (w1) edge (w4);
    \path[<->] (w1) edge (w3);
    \path[->] (w4) edge [reflexive right] (w4);
  \end{tikzpicture}
\end{solution}

% \begin{exercise}
%   This exercise is due to Johan van Benthem. Consider the following structure.

%   \begin{center}
%     \begin{tikzpicture}[modal, world/.append style={minimum size=0.8cm}, node distance = 10mm]
%       \node[world] (w1) [label=above:{$w_1$}] {};
%       \node[world] (w2) [label=above:{$w_2$}, right=of w1] {};
%       \node[world] (w3) [label=above:{$w_3$}, right=of w2] {$p$};
%       \node[world] (w4) [label=left:{$w_4$}, below=of w1] {};
%       \node[world] (w5) [label={[label distance=-1mm]above left:{$w_5$}}, below=of w2] {};
%       \node[world] (w6) [label=right:{$w_6$}, below=of w3] {$p$};
%       \node[world] (w7) [label=below:{$w_7$}, below=of w4] {$p$};
%       \node[world] (w8) [label=below:{$w_8$}, below=of w5] {};
%       \node[world] (w9) [label=below:{$w_9$}, below=of w6] {$t$};
%       \path[->] (w1) edge (w2);
%       \path[->] (w2) edge (w3);
%       \path[->] (w1) edge (w4);
%       \path[->] (w2) edge (w5);
%       \path[->] (w3) edge (w6);
%       \path[->] (w4) edge (w5);
%       \path[->] (w5) edge (w6);
%       \path[->] (w4) edge (w7);
%       \path[->] (w5) edge (w8);
%       \path[->] (w6) edge (w9);
%       \path[->] (w7) edge (w8);
%       \path[->] (w8) edge (w9);
%     \end{tikzpicture}
%   \end{center}
%   % 
%   There's a treasure at $w_9$, marked by the sentence letter $t$;
%   pirates are standing at $w_3, w_6,$ and $w_7$. At which worlds are
%   the following sentences true?
%   \begin{exlist}
%   \item $\Diamond t$
%   \item $\Diamond \Box t$
%   \item $\Diamond p$
%   \item $\Box \Diamond p$
%   \end{exlist}

% \end{exercise}

% \begin{exercise}
%   For each world in the previous exercise, find a sentence that is
%   true only there.
% \end{exercise}


\section{The systems K and S5}\label{sec:systems-k-s5}

As in the previous chapter, we call a sentence \emph{valid} if it is true at all
worlds in all models. But we now use a different conception of models, and a
different definition of truth at a world in a model. To avoid confusion, it is
best to use different expressions for different kinds of validity. Let's call
the new kind of validity \emph{K-validity}. (`K' for Kripke.) The old kind will
henceforth be called \emph{S5-validity}, because the sentences that are valid by
the definition from the previous chapter are precisely the sentences in C.I.\
Lewis's system S5.

\begin{definition}{}{kvalid}
  A sentence $A$ is \textbf{K-valid} (for short, $\models_K A$) iff $A$
  is true at every world in every Kripke model.
\end{definition}

% Many properties of S5-consequence and S5-validity carry over to
% K-consequence and K-validity. In particular, observation \ref{obs:semantic-deduction-theorem} (p.\ \pageref{obs:semantic-deduction-theorem}), the
% propositional extension theorem (p.\
% \pageref{thm:propositional-extension-theorem}), and the replacement
% theorem (p.\ \pageref{thm:replacement-theorem}) still hold, and for
% the same reasons as before. I won't go through the arguments again.

The same distinction applies to the concept of entailment. Entailment in the old
sense (definition \ref{def:basicconsequence}) will henceforth be called
\emph{S5-entailment}. Our new definition of models and truth lead to the
concept of \emph{K-entailment}.

\begin{definition}{}{kconsequence}
  Some sentences $\Gamma$ \textbf{K-entail} a sentence $A$ (for short:
  $\Gamma \models_{K} A$) iff there is no world in any Kripke model at which all
  sentences in $\Gamma$ are true while $A$ is false.
\end{definition}

The set of K-valid sentences is a system of modal logic. This system did
not figure in C.I.\ Lewis's list of systems. It is known as \textbf{system K}.

K is \textbf{weaker} than S5, by which we mean that not all S5-valid
sentences are K-valid. $\Box p \to p$, for example, is S5-valid but not K-valid.
Conversely, however, every K-valid sentence is S5-valid. Let's prove this.

\begin{observation}{kins5}
  Every K-valid sentence is S5-valid.
\end{observation}
%
\begin{proof}
  \emph{Proof:} In essence, observation \ref{obs:kins5} holds because the basic
  models from the previous chapter can be simulated by Kripke models in which
  all worlds have access to all worlds. If a sentence $A$ is K-valid, meaning
  that $A$ is true throughout every Kripke model, then $A$ is true throughout
  every Kripke model of this kind, and so $A$ is also true in every basic model.

  It is worth going through this more carefully. For any basic model
  $M = \langle W,V \rangle$, let $M^*$ be the Kripke model
  $\langle W,R,V \rangle$ with the same worlds $W$ and the same interpretation
  function $V$, and with an accessibility relation $R$ that holds between all
  worlds in $W$. That is, every world in $M^*$ can see every other world as well
  as itself. If every world can see every world, then it makes no difference
  whether we use definition \ref{def:basicsemantics} or definition
  \ref{def:kripkesemantics} to evaluate the truth of sentences at a world.
  That's because the two definitions only differ for the case of the modal
  operators, which definition \ref{def:basicsemantics} interprets as quantifiers
  over all worlds, while definition \ref{def:kripkesemantics} interprets them as
  quantifiers over the accessible worlds. So we have:
  \begin{quote}
  \begin{itemize}
  \item[(*)] A sentence is true at a world $w$ in a basic model $M$ iff
    it is true at $w$ in the corresponding Kripke model $M^*$.
  \end{itemize}
  \end{quote}
  (A full proof of (*) would proceed by induction on complexity of the
  sentence.)

  Now suppose a sentence $A$ is \emph{not} S5-valid, meaning that it is false at
  some world $w$ in some basic model $M$. By (*), it follows that $A$ is also
  false at some world in some Kripke model -- namely, at the same world $w$ in
  $M^*$. And if $A$ is false at some world in some Kripke model, then $A$ is not
  K-valid. By contraposition, it follows that if $A$ is K-valid, then $A$ is
  S5-valid. \qed
\end{proof}

You may remember from section \ref{sec:systems} that S5 can be axiomatized by
five axiom schemas and two rules:
%
\begin{principles}
  \pri{Dual}{\neg\Diamond A \leftrightarrow \Box\neg A}\\
  \pri{K}{\Box(A\to B) \to (\Box A \to \Box B)}\\
  \pri{T}{\Box A \to A}\\
  \pri{4}{\Box A \to \Box \Box A}\\
  \pri{5}{\Diamond A \to \Box \Diamond A}\\
  \pri{Nec}{\text{If $A$ is in the system, then so is }\Box A.}\\
  \pri{CPL}{\text{If }\Gamma \models_{P} A\text{ and all members of }\Gamma\text{ are in the system, then so is }A.}
\end{principles}
%
All instances of \pr{Dual}, \pr{K}, \pr{T}, \pr{4}, and \pr{5} are S5-valid, and
all and only the S5-valid sentences can be derived from instances of these axioms
by \pr{Nec} and \pr{CPL}.

The system K can be axiomatized by dropping three of the axiom schemas: \pr{T},
\pr{4}, and \pr{5}, leaving only \pr{Dual} and \pr{K}. All and only the K-valid
sentences can be derived from instances of \pr{Dual} and \pr{K} by \pr{Nec}
and \pr{CPL}.

(Many authors define $\Box$ as $\neg\Diamond\neg$ or $\Diamond$ as
$\neg\Box\neg$, in which case \pr{Dual} is true by definition. The only
remaining axiom schema is then \pr{K}. Don't confuse the schema \pr{K} with the
system K!)

\begin{exercise}
  \begin{exlist}
    \item Describe a Kripke model in which some instance of \pr{4} is false at
    some world.
    \item Describe a Kripke model in which some instance of \pr{5} is false at
    some world.
  \end{exlist}
\end{exercise}
\begin{solution}
  \begin{sollist}
    \item For example: $W = \{ w,v \}$, $R = \{ (w,v), (v,w) \}$,
    $V(p) = \{ v \}$. $\Box p \to \Box\Box p$ is false at $w$.
    (`$R = \{ (w,v), (v,w) \}$' means that $R$ relates $w$ to $v$ and $v$ to $w$
    and nothing else to anything else.)
    \item For example: $W = \{ w,v \}$, $R = \{ (w,w), (w,v) \}$,
    $V(p) = \{ w \}$. $\Diamond p \to \Box\Diamond p$ is false at $w$.
  \end{sollist}
\end{solution}

\begin{exercise}
  Can you find an instance of the \pr{T}-schema that is K-valid?
\end{exercise}
\begin{solution}
  For example: $\Box(p \lor \neg p) \to (p \lor \neg p)$.
\end{solution}

\begin{exercise}
  Show that $\Box(p \lor \neg p)$ is K-valid, using definition
  \ref{def:kripkesemantics}.
\end{exercise}
\begin{solution}
  By clause (g) of definition \ref{def:kripkesemantics}, $\Box(p \lor \neg p)$
  is false at a world $w$ in a Kripke model only if $p \lor \neg p$ is false at
  some world accessible from $w$. By clause (d) of definition
  \ref{def:kripkesemantics}, $p \lor \neg p$ is false at a world only if both
  $p$ and $\neg p$ are false at the world, which by clause (a) means that $p$ is
  both true and false at the world. This is impossible. So $\Box(p \lor \neg p)$
  is not false at any world in any Kripke model.
\end{solution}

% Other valid principles: $\Box$ distributes over $\land$ and vice versa.
% $\Box (A \to B) \to (\Diamond A \to \Diamond B)$.
% $(\Box A \land \Diamond B) \to \Diamond (A \land B)$.
  
% Also: If $A \to B$ is valid, then so are $\Box A \to \Box A$ and
% $\Diamond A \to \Diamond B$.

% \begin{exercise}
%   The sentences that are true at a world $w$ in a model $M$ contain
%   information not just about $w$ but also about other worlds in $M$. For
%   example, if $\Box p$ is true at $w$, we know that $w$ is true at all worlds,
%   and if $\neg p$ and $\Diamond p$ are both true at $w$, we know that $p$ is
%   true at some other world. Question: do the sentences true at $w$ completely
%   determine the model $M$? If not, give an example of two worlds $w_1$ and
%   $w_2$ in two models $M1$, $M2$ that verify the same sentences even though
%   the models are not isomorphic.
% \end{exercise}

\section{Some other normal systems}\label{sec:normalsystems}

For many applications of modal logic, we need a concept of validity that lies in
between K-validity and S5-validity. Suppose, for example, we read the box as
physical necessity and the diamond as physical possibility, understood as
compatibility with the laws of nature. On a popular conception of what it means
to be a law of nature, nothing that happens is ever incompatible with the laws
of nature. Equivalently, anything that is physically necessary is actually the
case. We therefore want $\Box A$ to entail $A$. On the other hand, it is not
clear if $\Box A$ should entail $\Box\Box A$: if $A$ is physically necessary,
can we infer that it is physically necessary that $A$ is physically necessary?
Below I will argue that we can't. If that is right, then the logic of physical
necessity is neither K nor S5. We want a logic with \pr{T} ($\Box A \to A$)
but without \pr{4} ($\Box A \to \Box\Box A$). S5 gives us both, K gives us
neither.

Our current semantics makes it easy to define systems in between K and S5 by
putting restrictions on the accessibility relation in Kripke models.

Let's say that an $\L_M$-sentence is \textbf{valid in a class of Kripke models}
iff the sentence is true at every world in every model that belongs to the
class. K-validity is validity in the class of all Kripke models. S5-validity is
validity in the class of Kripke models in which every world has access to every
world (as mentioned earlier, in the proof of observation \ref{obs:kins5}).

% Humberstone says valid "over" a class of frames or models

If you inspect countermodels to the K-validity of $\Box p \to p$, you may notice
that all of them involve worlds that don't have access to themselves. If we
require that every world can see itself then all instances of the \pr{T}-schema
become valid.

\begin{observation}{treflexive}
  All instances of \pr{T} are valid in the class of Kripke models in which every
  world is accessible from itself.
\end{observation}
%
\begin{proof}
  \emph{Proof:} According to clause (e) of definition \ref{def:kripkesemantics},
  an instance of $\Box A \to A$ is false at a world $w$ only if $\Box A$ is true
  at $w$ and $A$ is false; but if $\Box A$ is true at $w$ and $w$ has access to
  itself, then by clause (g) of definition \ref{def:kripkesemantics}, $A$ is
  true at $w$. So if $\Box A \to A$ is false at $w$, and $w$ is accessible from
  itself, then $A$ is both true and false at $w$, which is impossible. Hence
  $\Box A \to A$ is true at every world in every model in which every world is
  accessible from itself. \qed
\end{proof}

A relation $R$ on a set $W$ is called \textbf{reflexive} if each member of $W$
is $R$-related to itself. If the accessibility relation in a Kripke model is
reflexive, we also call the model itself reflexive. Observation
\ref{obs:treflexive} therefore states that all instances of \pr{T}
are valid in the class of reflexive Kripke models.

The set of all sentences that are valid in the class of reflexive Kripke models
is known as \textbf{system T}. Accordingly, any sentence that is valid in this
class of Kripke models (every member of system T) is called \textbf{T-valid}.

% The system T was first discussed in Feys 1937, who dropped one of the axioms
% in Godel 1933.

System T is stronger than K, but weaker than S5. The system can be axiomatized
by adding the axiom schema \pr{T} to the axioms and rules of K. We don't have
\pr{4} or \pr{5}. $\Box p \to \Box \Box p$ is S5-valid but not T-valid.

Systems of modal logic sometimes share their name with a schema. For
disambiguation, I always put schema names in parentheses. \pr{T} is a schema, T
is a system. \pr{K} is a schema, K is a system. All instances of \pr{T} are in
T, but many sentences in T -- for example, all instances of \pr{K} -- are not
instances of \pr{T}.

\begin{exercise}
  Show that $\Box p \to \Diamond p$ is T-valid.
\end{exercise}
\begin{solution}
  By definition \ref{def:kripkesemantics}, $\Box p \to \Diamond p$ is false at a
  world $w$ in a Kripke model only if $\Box p$ is true at $w$ and $\Diamond p$
  is false at $w$. But if $w$ has access to itself then the truth of $\Box p$ at $w$ implies that $p$ is true at $w$, and then $\Diamond p$ is false at $w$. So $\Box p \to \Diamond p$ can't be false at any world in any Kripke model in which each world has access to itself.
\end{solution}

In chapter \ref{ch:time}, we will study a temporal application of modal
logic in which the box is read as `it is always going to be the case that'.
The ``worlds'' in a Kripke model here represent times. $\Box p$ is
understood to be true at a time $t$ iff $p$ is true at all times after $t$. The
accessibility relation is the earlier-later relation: $t_1Rt_2$ iff $t_1$ is
earlier than $t_2$. In this application, we don't want to assume that $R$ is
reflexive, which would mean that every point in time is earlier than itself.
% On the contrary, we may want to assume that $R$ is
% \emph{irreflexive}, meaning that no member of $W$ is $R$-related to
% itself.
But we'll want something else. Suppose $t_1$ is earlier than $t_2$, and $t_2$ is
earlier than $t_3$. Then surely $t_1$ is earlier than $t_3$. 

A relation $R$ is called \textbf{transitive} if whenever $xRy$ and $yRz$ then
$xRz$. As before, we call a Kripke model transitive if its accessibility
relation is transitive. When we do temporal logic, we will restrict the relevant
models to transitive models.

The set of sentences that are valid in the class of transitive Kripke models is
known as \textbf{system K4}. The name alludes to the fact that this system can
be axiomatized by adding schema \pr{4} to the axioms and rules of K.

\begin{observation}{4trans}
  All instances of \pr{4} are valid in the class of transitive Kripke models.
\end{observation}
%
\begin{proof}
  \emph{Proof:} Suppose for reductio that there is some transitive Kripke model
  in which some instance of $\Box A \to \Box \Box A$ is false at some world $w$.
  By clause (e) of definition \ref{def:kripkesemantics}, it follows that (i)
  $\Box A$ is true at $w$ and (ii) $\Box\Box A$ is false at $w$. By clause (g)
  of definition \ref{def:kripkesemantics}, (ii) implies that there is some world
  $v$ accessible from $w$ where $\Box A$ is false. And that, in turn implies
  that there is some world $u$ accessible from $v$ at which $A$ is false. Since
  $R$ is transitive, $u$ is accessible from $w$. By (i), $A$ is true at $u$. So
  $A$ is both true and false at $u$. Contradiction. \qed
\end{proof}

We can combine the systems T and K4 by requiring both reflexivity and
transitivity. The set of sentences valid in the class of reflexive and
transitive Kripke models is C.I.\ Lewis's \textbf{system S4}. It is stronger
than K, T, and K4, but weaker than S5.

There are many other conditions we could impose on the accessibility relation,
and many combinations of these conditions. Each of them defines a system of
modal logic. The following table lists some well-known model classes with the
conventional names for the corresponding systems, repeating (for future
reference) the ones we already know. We will have a closer look at some of these
systems in later chapters, when we turn to applications of modal logic.

\bigskip
\begin{tabular}{ll}
  \toprule
  \emph{System} & \emph{Constraint on $R$}\\
  \midrule
  K & --\\
  T & $R$ is \textbf{reflexive}: every world in $W$ can access itself\\
  D & $R$ is \textbf{serial}: every world in $W$ can access some world\\
  K4 & $R$ is \textbf{transitive}: whenever $wRv$ and $vRu$, then $wRu$\\
  K5 & $R$ is \textbf{euclidean}: whenever $wRv$ and $wRu$, then $vRu$\\
  KD45 & $R$ is serial, transitive, and euclidean\\
  B & $R$ is reflexive and \textbf{symmetric}: whenever $wRv$ then $vRw$\\
  S4 & $R$ is reflexive and transitive\\ 
  S4.2 & $R$ is reflexive, transitive, and \textbf{convergent}: whenever $wRv$ and $wRu$,\\[-0.5mm]
      & then there is some $t$ such that $vRt$ and $uRt$ \\ 
  S5 & $R$ is reflexive, transitive, and symmetric\\ 
  S5 & $R$ is \textbf{universal}: every world has access to every world\\
  \bottomrule
\end{tabular}\label{table:systems}

\bigskip

% Any system that can be defined by putting constraints on the accessibility
% relation in Kripke models is called \textbf{normal}. So K, T, D, K4, K5, B,
% S4, and S5 are examples of normal systems, or normal logics. There are also
% non-normal systems/logics. These require a different kind of semantics. We
% will look at one alternative in section~\ref{sec:neighbourhood}, but mostly we
% will stay within the realm of the normal.

S5 occurs twice in the list. We already know S5 as the system for
universal models, in which the box and the diamond quantify unrestrictedly over
the whole space $W$. But we also get S5 if we merely require the accessibility
relation to be reflexive, transitive, and symmetric.

Relations that are reflexive, transitive, and symmetric are called
\textbf{equivalence relations}. An equivalence relation on a set divides the
members of the set into classes within which everything stands in the relation
to everything. (These classes are called \textbf{equivalence classes}.)

For example, let $S$ be the relation that holds between two people iff they have
the same birthday. This is an equivalence relation. It is reflexive: everyone
has the same birthday as themselves. It is transitive: if $aSb$ and $bSc$ then
$aSc$. And it is symmetric: if $aSb$ then $bSa$. For any person $a$, consider
the class $[a]_S$ of everyone who has the same birthday as $a$. (A ``class'' is
essentially the same thing as a set.) Everyone in $[a]_S$ has the same birthday
as everyone else in $[a]_S$. So within $[a]_S$, the same-birthday relation $S$
is universal.

Now let me explain why the above two characterisations of S5 are equivalent.

\begin{observation}{equivalence-universal}
  A sentence is valid in the class of Kripke models whose accessibility relation
  is universal iff it is valid in the class of Kripke models whose accessibility
  relation is an equivalence relation.
\end{observation}
%
\begin{proof}
  \emph{Proof sketch:} The right-to-left direction is easy. If $R$ is the universal
  relation on $W$, then $R$ is reflexive, transitive, and symmetric. So the
  universal relation on $W$ is a special kind of equivalence relation on $W$. If
  a sentence is valid in every model in which $R$ is an equivalence relation, it
  must therefore be valid in every model in which $R$ is universal.

  The other direction is more interesting. We argue by contraposition, showing
  that if a sentence $A$ is not valid in the class of models in which $R$ is an
  equivalence relation, then $R$ is also not valid in the class of universal
  models. So assume $A$ is not valid in the class of models in which $R$ is an
  equivalence relation. Then there is some world $w$ in some such model
  $M = \t{W,R,V}$ such that $M,w \not\models A$. Define a new model
  $M' = \t{W',R',V'}$ as follows:
  \begin{quote}
    $W'$ is the class of worlds accessible in $M$ from $w$ (i.e., the
    equivalence class $[w]_R$).

    $R'$ is the universal relation on $W'$.

    $V'$ is the restriction of $V$ to $W'$, so that for any sentence letter $B$,
    $V'(B) = V'(B) \cap W'$.
  \end{quote}
  (If $X$ and $Y$ are sets, then $X \cap Y$ -- the \emph{intersection} of $X$ and
  $Y$ -- is the set of all things that are both in $X$ and in $Y$.)

  $M'$ has a universal accessibility relation. But from the perspective of $w$,
  $M$ and $M'$ are indistinguishable. \emph{Any sentence is true at $w$ in $M$ iff
    it is true at $w$ in $M'$.} This could be shown by induction, but I hope you
  see intuitively why it is the case.

  Granting the italicized sentence, the assumption that $A$ is false at some
  world in some model whose accessibility relation is an equivalence relation
  entails that $A$ is false at some world in some model whose accessibility
  relation is universal. \qed
\end{proof}

\begin{exercise}
  Let $R$ be the relation on the set of all people such that $aRb$ iff $b$ is a
  sibling of $a$. Is $R$ reflexive? serial? transitive? euclidean? symmetric?
  universal?
\end{exercise}
\begin{solution}
  Reflexive no, serial no, transitive no, euclidean no, symmetric yes,
  universal no.
\end{solution}

\begin{exercise}\label{ex:relations}
  Explain these facts:
  \begin{exlist}
  \item If $R$ is symmetric and transitive, then $R$ is euclidean.
  \item If $R$ is symmetric and euclidean, then $R$ is transitive.
  \item If $R$ is reflexive and euclidean, then $R$ is symmetric.
  \end{exlist}
\end{exercise}
\begin{solution}
  \begin{sollist}
    \item Suppose $R$ is symmetric and transitive, and that $xRy$ and $xRz$. By
    symmetry, $yRx$. By transitivity, $yRz$.
    \item Suppose $R$ is symmetric and euclidean, and that $xRy$ and $yRz$. By
    symmetry, $yRx$. By euclidity, $xRz$.
    \item Suppose $R$ is reflexive and euclidean, and that $xRy$. By
    reflexivity, $xRx$. By euclidity, $yRx$.
  \end{sollist}
\end{solution}

\begin{exercise}
  What is wrong with the following argument? ``If $R$ is symmetric,
  then $wRv$ implies $vRw$; if $R$ is transitive, it follows that
  $wRw$. So symmetry and transitivity together imply reflexivity.''
\end{exercise}
\begin{solution}
  It's true that if $R$ is symmetric and transitive then $wRv$ implies $vRw$
  which implies $wRw$. But this only shows that every world $w$ \emph{that can
    see some world $v$} can see itself. Symmetry, transitivity, \emph{and
    seriality} together imply reflexivity. Symmetry and transitivity alone do
  not.
\end{solution}

%\begin{exercise}
%  Show that seriality and transitivity and symmetry implies reflexivity.
%\end{exercise}


\section{Frames}\label{sec:frames}

There is a close connection between conditions on the accessibility relation in
Kripke models and modal schemas -- between reflexivity and the \pr{T}-schema,
between transitivity and the \pr{4}-schema, and so on. What exactly is the
connection?

You might think the connection between \pr{T} and reflexivity is this:
\begin{quote}
  \begin{itemize}
    \item[(?)] All instances of \pr{T} are valid in a model iff the model is reflexive.
  \end{itemize}
\end{quote}
%
But that's false. We know (observation \ref{obs:treflexive}) that all \pr{T}
instances are valid in the class of reflexive models. It follows that all \pr{T}
instances are valid in every reflexive model. But the other direction fails.
There are non-reflexive models in which all \pr{T} instances are valid. The
following model is an example.
\begin{center}
  \begin{tikzpicture}[modal, world/.append style={minimum size=0.8cm}, node distance = 15mm]
    \node[world] (w1) [label=above:{$w$}] {$p$};
    \node[world] (w2) [label=above:{$v$}, right=of w1] {$p$};
    \path[<->] (w1) edge (w2);
  \end{tikzpicture}
\end{center}
There are two worlds, both of which can see each other; neither can see itself.
$p$ is true at both worlds, all other sentence letters are false at both worlds.
This model is not reflexive, but no instance of the \pr{T}-schema $\Box A \to A$
is false at any world in the model. (Try to find a false instance!) The fact
that the \pr{T}-schema is valid in a class of models therefore does not entail
that all models in the class are reflexive. The class might contain models like
the one just described.

To understand the connection between modal schemas and conditions on the
accessibility relation, we need to talk about \emph{frames}. A frame is what you
get if take a model and remove the interpretation function.

\begin{definition}{}{Kripke frame}
  A \textbf{Kripke frame} is a pair of a non-empty set $W$ and a relation $R$ on $W$.
\end{definition}

Roughly speaking, if we think of a model as representing a scenario and an
interpretation, then a frame is the part of the model that represents the
scenario.

% In modal logic, we only care about the abstract structure of a
% scenario: how many worlds there are, and how they are accessible from one
% another.

Frames can be pictured just like Kripke models, but without any sentence letters
in the nodes. The frame of the model displayed above looks like this:
\begin{center}
  \begin{tikzpicture}[modal, world/.append style={minimum size=0.8cm}, node distance = 15mm]
    \node[world] (w1) [label=above:{$w$}] {$$};
    \node[world] (w2) [label=above:{$v$}, right=of w1] {$$};
    \path[<->] (w1) edge (w2);
  \end{tikzpicture}
\end{center}

Now remember that validity is truth in virtue of the meaning of the logical
expressions. Whether a sentence is valid should not depend on the
meaning of the non-logical expressions. So if we define a particular kind of
validity by reference to a class of Kripke models, the constraints we impose on
the models in the class should be constraints on the frame of the models, not on
the interpretation function.

To see the point, suppose I suggested that a sentence is ``$X$-valid'' iff it is true
at all worlds in all Kripke model whose interpretation function assigns the
empty set to the sentence letter $p$. So $\Box \neg p$ is $X$-valid, while
$\Box \neg q$ is $X$-invalid. But $\Box \neg p$ and $\Box \neg q$ have the same
logical form. If $\Box \neg p$ is true in virtue of its logical form, then
$\Box \neg q$ should also be true in virtue of its logical form. $X$-validity is
not a sensible concept of logical validity. The systems from the previous
section were all defined sensibly, by putting constraints on the frame of a
Kripke model, not on the interpretation function.

Let's say that a sentence is \textbf{valid on a frame} if it is true at all
worlds in all models with that frame. A sentence is \textbf{valid in a class of
  frames} if it valid on all frames in the class.

If a sentence is valid in the class of all models whose accessibility relation
satisfies a certain condition, then it is also valid in the class of all frames
whose accessibility relation satisfies that condition, and vice versa. We could
have defined the systems from the previous section in terms of frame classes
rather than model classes: K is the set of sentences valid in the class of all
frames, T is the set of sentences valid in the class of reflexive frames, and so
on. (A reflexive/transitive/etc.\ frame is a frame with a
reflexive/transitive/etc.\ accessibility relation.)

Now here is the connection between \pr{T} and reflexivity: All \pr{T} instances
are valid in a class of frames iff every frame in the class is reflexive. More
simply:

\begin{observation}{tcorrespondence}
  All instances of \pr{T} are valid on a frame iff the frame is reflexive.
\end{observation}
%
\begin{proof}
  \emph{Proof:} The right-to-left direction follows from observation
  \ref{obs:treflexive}, according to which all \pr{T} instances are valid in the
  class of reflexive models, and therefore in the class of reflexive frames, and
  therefore on any frame in that class. For the other direction, we have to show
  that if all instances of \pr{T} are valid on a frame $\t{W,R}$, then $R$ is
  reflexive. We do this by showing that if $R$ is not reflexive, then we can
  find an interpretation function $V$ that makes $\Box p \to p$ false at some
  world $w$. $w$ will be an arbitrary world in $W$ that can't see itself. (There
  must be some such world if $R$ is not reflexive.) Let $V(p)$ comprise all
  worlds in $W$ except $w$. Then $\Box p$ is true at $w$ and $p$ false. So
  $\Box p \to p$ is false at $w$. \qed
\end{proof}

If all instances of a schema are valid on all and only the frames whose
accessibility relation satisfies a certain property, the schema is said to
\textbf{correspond} to that property (and to \emph{define} the relevant class of
frames). Observation \ref{obs:tcorrespondence} says that the \pr{T} schema
corresponds to reflexivity.

% More generally, a class of frames is /modally defined/ by a set of formulas
% iff it is the class of frames for the set of formulas, i.e. the frames on
% which all the formulas are valid.
 
Instead of proving more facts about the correspondence between modal schemas and
frame conditions, I will simply give you a list of some important results.

\bigskip
\begin{tabular}{rll}
  \toprule
  \multicolumn{2}{l}{\emph{Schema}} & \emph{Corresponding Frame Condition}\\\midrule
  \pr{T} & $\Box A \to A$ & $R$ is reflexive: every world in $W$ is accessible from itself\\
  \pr{D} & $\Box A \to \Diamond A$ & $R$ is serial: every world in $W$ can access some world in $W$\\
  \pr{B} & $A \to \Box\Diamond A$ & $R$ is symmetric: whenever $wRv$ then $vRw$\\
  \pr{4} & $\Box A \to \Box\Box A$ & $R$ is transitive: whenever $wRv$ and $vRu$, then $wRu$\\
  \pr{5} & $\Diamond A \to \Box\Diamond A$ & $R$ is euclidean: whenever $wRv$ and $wRu$, then $vRu$\\
  \pr{G} & $\Diamond \Box A \to \Box\Diamond A$ & $R$ is convergent:
                                                      whenever $wRv$ and $wRu$, then there is\\[-0.5mm]
      && some $t$ such that $vRt$ and $uRt$ \\ 
  \bottomrule
\end{tabular}
\bigskip

Correspondence facts are often useful when trying to figure out which schemas
should be valid on a given interpretation of the modal operators. Return to the
case of physical possibility and necessity from the start of section
\ref{sec:normalsystems}. I claimed that on this interpretation of the box and
the diamond, we should not regard all instances of the \pr{4}-schema
$\Box A \to \Box\Box A$ as valid. My claim is not based on a direct intuition
that something could be physically necessary without it being physically
necessary that it is physically necessary. My claim is rather based on a
judgement about the non-transitivity of physical accessibility. My reasoning
goes like this. I assume that a world $v$ is physically possible relative to a
world $w$ if nothing that happens at $v$ contradicts the laws of nature at $w$.
This does not imply that $v$ has the same laws as $w$. For example, suppose the
only law at $w$ is that ravens are black; at $v$, there is no such law but there
happen to be no non-black ravens. Then what happens at $v$ does not contradict
the laws at $w$, even though $v$ has different laws. Relative to the laws of
$v$, worlds with white ravens are physically possible. So a world accessible
from a world that is accessible from $w$ need not itself be accessible from $w$.
Since \pr{4} corresponds to transitivity, I can infer that the logic of physical
necessity does not render all instances of that schema valid.

\begin{exercise}
  Can you find frame conditions that correspond to these schemas?
  \begin{exlist}
  \item $\Box A \leftrightarrow A$
  \item $\Box A$
  \end{exlist}
\end{exercise}
\begin{solution}
  \begin{sollist}
  \item Every world has access only to itself.
  \item No world has access to any world.
  \end{sollist}
\end{solution}

\section{More trees}
\label{sec:more-trees}

In section \ref{sec:trees}, I described the tree method for checking whether a
sentence is valid, and for constructing countermodels. These were the rules for
the box and the diamond:

\bigskip

\begin{minipage}{0.24\textwidth} \centering
\tree{
  \dotbelownode{8}{}{$\Box A$}{\omega}{}\\
  \\
  \nnode{8}{}{$A$}{\nu}{}\\
  \Kk[8]{0}{\color{red}$\uparrow$}\\
  \Kk[8]{0}{\color{red}\small old}
}
\end{minipage}
\begin{minipage}{0.24\textwidth}\centering
\tree{
  \dotbelownode{8}{}{$\Diamond A$}{\omega}{}\\
  \\
  \nnode{8}{}{$A$}{\nu}{}\\
  \Kk[8]{0}{\color{red}$\uparrow$}\\
  \Kk[8]{0}{\color{red}\small new}
}
\end{minipage}
\begin{minipage}{0.24\textwidth}\centering
\tree{
  \dotbelownode{10}{}{$\neg \Box A$}{\omega}{}\\
  \\
  \nnode{10}{}{$\neg A$}{\nu}{}\\
  \Kk[10]{0}{\color{red}$\uparrow$}\\
  \Kk[10]{0}{\color{red}\small new}
}
\end{minipage}
\begin{minipage}{0.24\textwidth}\centering
\tree{
  \dotbelownode{10}{}{$\neg \Diamond A$}{\omega}{}\\
  \\
  \nnode{10}{}{$\neg A$}{\nu}{}\\
  \Kk[10]{0}{\color{red}$\uparrow$}\\
  \Kk[10]{0}{\color{red}\small old}
}
\end{minipage}
%

\bigskip
\noindent
The rule for $\Box A$ allows us to infer, from the hypothesis that $\Box A$ is
true at some world, that $A$ is true at any world that occurs on a tree branch.
This made sense given the semantics of the previous chapter, where the box
quantified unrestrictedly over all worlds. With the new semantics of the present
chapter, we need to change the tree rules.

If $\Box A$ is true at a world $w$, and there's some other world $v$ on the
branch, we can only infer that $A$ is true at $v$ if $v$ is accessible from $w$.
So we need to keep track of which worlds are accessible from any world on a
tree. We do this by adding meta-linguistic statements about accessibility to the
tree.

For example, suppose we want to expand the following node.
%
\begin{center}
  \tree{%
    \nnode{18}{n.}{$\Diamond p$}{w}{} 
  }
\end{center}
%
The node represents the hypothesis that $\Diamond p$ is true at $w$.
It follows that $p$ is true at some world $v$. Moreover, that world
$v$ must be accessible from $w$. So we add two new nodes:

\begin{center}
  \tree{%
    \nnode{18}{m.}{$wRv$}{}{} \\
    \nnode{18}{m+1.}{$p$}{v}{} 
  }
\end{center}
%
Node $m+1$ is what we would have added by the old rules. Node $m$ is a
meta-linguistic statement reminding us that $v$ is accessible from
$w$. `$wRv$' is not a sentence of $\L_M$; it isn't true or false
relative to a world, which is why node $m$ has no world label.

What if we want to expand a box node?
%
\begin{center}
  \tree{%
    \nnode{18}{n.}{$\Box p$}{$w$}{} 
  }
\end{center}
%
By itself, this doesn't tell us anything about the truth-value of $p$
at any world. We can't infer that $p$ is true at $w$, because $w$
might not be accessible from itself. Indeed, if no world is accessible
from $w$, then $\Box p$ can be true even if $p$ is false at every
world. So we can't even infer that there is some world or other at
which $p$ is true.

However, suppose a branch that contains node $n$ also contains the
following node.
\begin{center}
  \tree{%
    \nnode{18}{m.}{$w R v$}{}{} 
  }
\end{center}
Now we can infer that $p$ is true at $v$. So to expand a box node on a
branch, there must be another node on the branch telling us that
the world $w$ at which the box sentence is true has access to some
world $v$.

Here are diagrams of the new rules for the box and the diamond.

\bigskip

\begin{minipage}{0.24\textwidth} \centering
\tree{
  \nnode{12}{}{$\Box A$}{\omega}{}\\
  \dotbelownode{12}{}{$\omega R\nu$}{}{}\\
  \\
  \nnode{12}{}{$A$}{\nu}{}\\
  \Kk[12]{0}{}\\
  \Kk[12]{0}{}
}
\end{minipage}
\begin{minipage}{0.24\textwidth}\centering
\tree{
  \dotbelownode{12}{}{$\Diamond A$}{\omega}{}\\
  \\
  \nnode{12}{}{$\omega R \nu$}{}{}\\
  \nnode{12}{}{$A$}{\nu}{}\\
  \Kk[12]{0}{\color{red}$\uparrow$}\\
  \Kk[12]{0}{\color{red}\small new}
}
\end{minipage}
\begin{minipage}{0.24\textwidth}\centering
\tree{
  \dotbelownode{12}{}{$\neg \Box  A$}{\omega}{}\\
  \\
  \nnode{12}{}{$\omega R \nu$}{}{}\\
  \nnode{12}{}{$\neg A$}{\nu}{}\\
  \Kk[12]{0}{\color{red}$\uparrow$}\\
  \Kk[12]{0}{\color{red}\small new}
}
\end{minipage}
\begin{minipage}{0.24\textwidth} \centering
\tree{
  \nnode{12}{}{$\neg \Diamond  A$}{\omega}{}\\
  \dotbelownode{12}{}{$\omega R \nu$}{}{}\\
  \\
  \nnode{12}{}{$\neg A$}{\nu}{}\\
  \Kk[12]{0}{}\\
  \Kk[12]{0}{}
}
\end{minipage}

\bigskip\noindent%
If two nodes occur above the dotted line in a rule, as in the rule for $\Box A$,
this means that the rule can only be applied if both nodes already occur on the
relevant branch (in any order, and not necessarily adjacent to each other).

The rules for negated boxes and diamonds are what you would expect from
the duality of the box and the diamond. Note that only nodes of type
$\Diamond A$ and $\neg \Box A$ allow us to introduce hypotheses about
accessibility into a tree.

The rule for the classical connectives all stay the same. Together,
all these rules are known as the \textbf{K-rules}; the tree rules from
section \ref{sec:trees} are the \textbf{S5-rules}.

Here is a schematic tree proof to show that
$\models_K \Box (A \land B) \to (\Box A \land \Box B)$.

\begin{center}
  \tree[4]{%
    & \nnode{32}{1.}{$\neg(\Box (A \land B) \to (\Box A \land \Box B))$}{w}{(Ass.)} &\\
    & \nnode{32}{2.}{$\Box(A \land B)$}{w}{(1)} &\\
    & \bnode{32}{3.}{$\neg(\Box A \land \Box B)$}{w}{(1)} & \\
    &&\\
    \nnode{12}{4.}{$\neg\Box A$}{w}{(3)} && \nnode{12}{5.}{$\neg\Box B$}{w}{(3)} \\
    \nnode{12}{6.}{$wRv$}{}{(4)} && \nnode{12}{11.}{$wRu$}{}{(5)} \\
    \nnode{12}{7.}{$\neg A$}{v}{(4)} && \nnode{12}{12.}{$\neg B$}{u}{(5)} \\
    \nnode{12}{8.}{$A\land B$}{v}{(2,6)} && \nnode{12}{13.}{$A \land B$}{u}{(2,11)} \\
    \nnode{12}{9.}{$A$}{v}{(8)} && \nnode{12}{14.}{$A$}{u}{(13)} \\
    \nnodeclosed{12}{10.}{$B$}{v}{(8)} && \nnodeclosed{12}{15.}{$B$}{u}{(13)} \\
  }
\end{center}
%
The annotation `(2,6)' for node 8 indicates that this node is based on two
assumptions from earlier in the branch: the assumption on node 2 that
$\Box (A \land B)$ is true at $w$, and the assumption on node 6 that $wRv$. Only
these two assumptions together allow us to infer that $A \land B$ is true at
$v$.

What happens if we try to prove $\Box p \to p$?

\begin{center}
  \tree{%
    & \nnode{15}{1.}{$\neg(\Box p \to p)$}{w}{(Ass.)} &\\
    & \nnode{15}{2.}{$\Box p$}{w}{(1)} &\\
    & \nnode{15}{3.}{$\neg p$}{w}{(1)} &
  }
\end{center}
%
At this point, no more rules can be applied. We can read off a countermodel from
the open branch:
\begin{center}
  $W = \{ w \}$\\
  $R = \emptyset$\\
  $V(p) = \emptyset$
\end{center}
This is the smallest possible Kripke model. It consists of a single world that
can't see itself. `$R = \emptyset$' is a way of saying that no world can see any
world. If you want to say that $R$ holds between $w$ and $v$ and between $v$ and
$u$, you might write `$R = \{ (w,v), (v,u) \}$' or simply `$wRv, vRu$'.

\begin{exercise}
  Use the K-rules to check which of the following sentences are K-valid. If a sentence is invalid, describe a countermodel.
  \begin{exlist}
  \item $(\Box p \land \Box q) \to \Box (p \land q)$
  \item $\Diamond (p \land q) \to (\Diamond p \land \Diamond q)$
  \item $(\Diamond p \land \Diamond q) \to \Diamond (p \land q)$
  \item $\Diamond(p \lor q) \leftrightarrow (\Diamond p \lor \Diamond q)$
  \item $\Box(p \lor q) \leftrightarrow (\Box p \lor \Box q)$
  \item $\Box (p \to q) \to (\Diamond p \to \Diamond q)$. % yes
  \item $(\Box p \land \Diamond q) \to \Diamond (p \land q)$. % yes
  % <>A->[]B entails []A->[]B.
  % $\Diamond A \to (\Box B \to \Diamond B)$
  % $\Diamond (A \to B) \leftrightarrow (\Box A \to \Diamond B)$
  \end{exlist}
\end{exercise}
\begin{solution}
  You can enter the sentences at
  \href{https://www.umsu.de/trees/}{umsu.de/trees}. To check for K-validity, leave all the checkboxes (for `universal' etc.) empty.
\end{solution}

For systems in between K and S5 that are characterised by certain constraints on
the accessibility relation, we add new rules for manipulating accessibility
nodes. For example, if we want to check whether a sentence is T-valid, we use a
\emph{reflexivity rule} in addition to the K-rules. The reflexivity rule says
that if a world variable $\omega$ occurs on a branch, then we may always add
$\omega R\omega$ to the branch.

Here is a proof of $\Box p \to p$, using the reflexivity rule.

\begin{center}
  \tree[4]{%
     \nnode{19}{1.}{$\neg(\Box p \to p)$}{w}{(Ass.)} \\
     \nnode{19}{2.}{$\Box p$}{w}{(1)} \\
     \nnode{19}{3.}{$\neg p$}{w}{(1)}  \\
     \nnode{19}{4.}{$wRw$}{}{(Ref.)}  \\
     \nnodeclosed{19}{5.}{$p$}{w}{(2,4)}  \\
  }
\end{center}

To test for validity in the class of transitive frames (or models), we need a
\emph{transitivity rule}, which allows us to infer $\omega R\upsilon$ from
$\omega R\nu$ and $\nu R\upsilon$. Here is a proof of $\Box p \to \Box\Box p$
that uses this rule.

\begin{center}
  \tree[4]{%
     \nnode{25}{1.}{$\neg(\Box p \to \Box\Box p)$}{w}{(Ass.)} \\
     \nnode{25}{2.}{$\Box p$}{w}{(1)} \\
     \nnode{25}{3.}{$\neg \Box\Box p$}{w}{(1)}  \\
     \nnode{25}{4.}{$wRv$}{}{(3)}  \\
     \nnode{25}{5.}{$\neg\Box p$}{v}{(3)}  \\
     \nnode{25}{6.}{$vRu$}{}{(5)}  \\
     \nnode{25}{7.}{$\neg p$}{u}{(5)}  \\
     \nnode{25}{8.}{$wRu$}{}{\quad(4,6,Tr.)}  \\
     \nnodeclosed{25}{9.}{$p$}{u}{(2,8)}  \\
  }
\end{center}

The following diagrams summarize the tree rules for the frame conditions we have
so far considered.
%
\begin{center}

  \begin{minipage}[t]{0.3\textwidth} \centering

    Reflexivity
    
    \vspace{-3mm}

    \tree{
      \dotbelowbarenode{}\\
      \\
      \barenode{$\omega R \omega$}\\
      \Kk[0]{0}{\color{red}$\uparrow$\hspace{6mm}}\\
      \Kk[0]{0}{\color{red}\small old\hspace{6mm}}
    }
  \end{minipage}
  \begin{minipage}[t]{0.3\textwidth} \centering
    Seriality

    \vspace{-3mm}

    \tree{
      \dotbelowbarenode{}\\
      \\
      \barenode{$\omega R \nu$}\\
      \Kk[0]{0}{\color{red}$\uparrow$\hspace{3mm}$\uparrow$}\\
      \Kk[0]{0}{\color{red}\small old \; new}
    }
  \end{minipage}
  \begin{minipage}[t]{0.3\textwidth} \centering
    Transitivity

    \medskip
    
    \tree{
      \barenode{$\omega R \nu$}\\
      \dotbelowbarenode{$\nu R \upsilon$}\\
      \\
      \barenode{$\omega R \upsilon$}\\
    }
  \end{minipage}
  \bigskip
  
  \begin{minipage}[t]{0.3\textwidth} \centering
    Symmetry

    \medskip
    
    \tree{
      \dotbelowbarenode{$\omega R \nu$}\\
      \\
      \barenode{$\nu R \omega$}\\
    }
  \end{minipage}
  \begin{minipage}[t]{0.3\textwidth} \centering
    Euclidity
    
    \medskip

    \tree{
      \barenode{$\omega R \nu$}\\
      \dotbelowbarenode{$\omega R \upsilon$}\\
      \\
      \barenode{$\nu R \upsilon$}\\
    }
  \end{minipage}
  \begin{minipage}[t]{0.3\textwidth} \centering
    Convergence
    
    \medskip

    \tree{
      \barenode{$\omega R \nu$}\\
      \dotbelowbarenode{$\omega R \upsilon$}\\
      \\
      \barenode{$\nu R \tau$}\\
      \barenode{$\upsilon R \tau$}\\
      \Kk[0]{0}{\hspace{6mm}\color{red}$\uparrow$}\\
      \Kk[0]{0}{\hspace{6mm}\color{red}\small new}
    }
  \end{minipage}

\end{center}

By selectively adding some of these rules to the K-rules, we get tree rules for
a variety of modal logics.  (Compare the table on p.\
\pageref{table:systems}.)

\bigskip
\begin{tabular}{ll}
  \toprule
  \emph{System} & \emph{Tree Rules}\\
  \midrule
  K & K-rules\\
  T & K-rules and reflexivity rule\\
  D & K-rules and seriality rule\\
  K4 & K-rules and transitivity rule\\
  K5 & K-rules and euclidity rule\\
  KD45 & K-rules, seriality rule, transitivity rule, and euclidity rule\\
  B & K-rules, reflexivity rule, and symmetry rule\\
  S4 & K-rules, reflexivity rule, and transitivity rule\\ 
  S4.2 & K-rules, reflexivity rule, transitivity rule, and convergence rule\\
  \bottomrule
\end{tabular}
\bigskip

% seriality tree rule should only be allowed for w if there's not already a node
% wRv on the tree. (Priest.)

\begin{exercise}
  Use the tree method to check the following claims.
  \begin{exlist}
    \item $\models_{K4} \Diamond p \to \Diamond\Diamond p$.
    \item $\models_{D} (\Box p \land \Box q) \to \Diamond (p \lor q)$.
    \item $\models_{B} \Diamond p \to \Box\Diamond p$.
    \item $\models_{T} (\Diamond\Box(p \to q) \land \Box p) \to \Diamond q$.
    \item $\models_{T} \Diamond (p \to \Box \Diamond p)$.
    \item $\models_{S4} \Diamond\Box(\Diamond p \to \Box\Diamond p)$. 
  \end{exlist}
\end{exercise}
\begin{solution}
  You can enter the sentences at
  \href{https://www.umsu.de/trees/}{umsu.de/trees}. To
  test for K4-validity, check the `transitive' box. To test for D-validity, check `serial'. To test for B-validity, check `symmetric'. To test for T-validity, check `reflexive'.
\end{solution}


% Should I introduce some jargon? A logic S is /determined by/ a class of frames
% if S is sound and complete wrt the class (i.e. iff its theorems are exactly
% the formulas valid over the class).

% Also point out: a logic S is determined by (sound and complete wrt) any class
% of frames iff it is determined by the class of frames for S.



%%% Local Variables: 
%%% mode: latex
%%% TeX-master: "logic2.tex"
%%% End:
