\chapter{Semantics for Modal Predicate Logic}\label{ch:qml2}

\section{Constant domain semantics}\label{sec:constantdomainsemantics}

We have met the language $\L_{M\!P}$ of (first-order) modal predicate logic. It
is time to think about how this language should be interpreted. This will tell
us which sentences and inferences in the language are valid.

As in modal propositional logic, we will assume that the box and the diamond are
quantifiers over accessible worlds, where ``accessibility'' is a placeholder
whose meaning depends on the application. If we want to reason about knowledge,
a world $v$ might be accessible from a world $w$ iff $v$ is compatible with what
is known at $w$. If we're interested in metaphysical modality then a world $v$
might be accessible from a world $w$ iff it is compatible with the nature of
things at $w$. Here we might, for example, read $\Diamond Fa$ as saying that
Aristotle could have been a sailor, assuming that $a$ picks out Aristotle and
$F$ the property of being a sailor.

Our topic in logic is not whether a particular claim about Aristotle is true. We
want to know which statements are \emph{logically true} or \emph{valid}, meaning
that they are true in any conceivable scenario, under any interpretation of the
non-logical expressions (but holding fixed the meaning of the modal operators).

As always, we use models to represent a scenario together with an interpretation
of the non-logical vocabulary. A model for $\L_{M\!P}$ contains just enough
information about a scenario and an interpretation to determine, for every
$\L_{M\!P}$-sentence and every world, whether the sentence is true at that
world.

The non-logical vocabulary of $\L_{M\!P}$ are the names and the predicates (with
the exception of the identity predicate `='). Let's assume, for now, that the
purpose of a name is simply to pick out an individual. Intuitively, a predicate
picks out a property or relation. In non-modal predicate logic, we could
represent these properties or relations by their extension -- by the sets of
individuals (or tuples of individuals) to which they apply. In modal predicate
logic, however, we typically want to allow for scenarios in which an individual
has different properties at different worlds. In one world, Aristotle might be a
sailor, in another he might be a shoemaker. If $F$ expresses the property of
being a sailor, then the set of individuals to whom $F$ applies will differ from
world to world. To determine the truth-value of $Fa$ at a world, we need to know
to which individuals $F$ applies \emph{at that world}. A model's interpretation
function will therefore assign a set of (tuples of) individuals to each
predicate \emph{relative to each world}.

Consider a model with two worlds $w$ and $v$. Both worlds, let's assume, are
accessible from $w$ and neither is accessible from $v$. The model's
interpretation function tells us that the name $a$ picks out, say, Aristotle. It
also tells us that the predicate $F$ applies to Aristotle and Boethius at $w$
and only to Boethius at $v$. We can write this as follows:
%
\begin{quote}
  $V(a) = \text{Aristotle}$\\
  $V(F,w) = \{ \text{Aristotle, Boethius} \}$\\
  $V(F,v) = \{ \text{Boethius} \}$
\end{quote}
%
\noindent%
We don't know what property is expressed by $F$, nor which properties Aristotle
and Boethius have at $w$ and $v$. Nonetheless, we can figure out that $Fa$ is
true at $w$, because the predicate $F$ applies to Aristotle at $w$. We can also
figure out that $Fa$ is false at $v$, and that $\Box Fa$ is false at $w$.

To determine the truth-value of arbitrary $\L_{M\!P}$-sentences, we need some
more information. As it stands, we can't tell whether (say) $\forall x Fx$ is
true at $w$. Informally, $\forall x Fx$ says that every individual is $F$. We
know that Aristotle and Boethius are $F$ at $w$. But we don't know if there are
other individuals besides Aristotle and Boethius. If yes, then $\forall x Fx$ is
false at $w$. If no, the sentence is true. We therefore assume that a model for
$\L_{M\!P}$ also specifies a domain of individuals.

\begin{definition}{}{VDM}
  A \textbf{constant-domain Kripke model} for $\L_{M\!P}$ is a structure $M$
  consisting of%
  \medskip
  \begin{compactenum}
    \item a non-empty set $W$ (the ``worlds''),
    \item a binary (``accessibility'') relation $R$ on $W$,
    \item a non-empty set $D$ (of ``individuals''), and
    \item an interpretation function $V$ that assigns%
    \vspace{-1mm}
    \begin{itemize}
      \itemsep-1mm
      \item to each $\L_{M\!P}$-name a member of $D$, and
      \item to each $n$-place predicate of $\L_{M\!P}$ and world $w \in W$ a set of
            $n$-tuples from $D$.
    \end{itemize}
  \end{compactenum}
\end{definition}

Models of this type are called ``constant-domain models'' because the domain of
individuals is the same for each world. This may seem questionable -- and we are
soon going to question it -- but it simplifies the semantics. Let’s stick
with it for the moment.

Having defined a concept of a model, we can lay down the rules that determine
whether any given $\L_{M\!P}$-sentence is true at a world in a model.

In fact, truth will be defined relative to three parameters: a model, a world,
and an assignment function. The assignment function plays the same role as in
non-modal predicate logic. $\forall x \Diamond Fx$, for example, is true at a
world $w$ in a model iff there is some assignment of an individual to $x$ that
renders $\Diamond Fx$ true at $w$. We continue to use $[\tau]^{M,g}$ for the
individual picked out by a term (name or variable) $\tau$ relative to a model
$M = \t{D,W,R,V}$ and an assignment function $g$:
\[
  [\tau]^{M,g} =_\text{def} \begin{cases} \;V(\tau) & \text{ if $\tau$ is a name}\\
    \;g(\tau) & \text{ if $\tau$ is a variable}.
  \end{cases}
\]

\begin{definition}{Constant-domain Kripke semantics}{constantdomainsemantics}
  If $\Mfr = \t{W,R,D,V}$ is a constant-domain Kripke model, $w$ is a member of
  $W$, $\phi$ is an $n$-place predicate (for $n\geq 0$),
  $\tau_1,\tau_{2},\ldots,\tau_{n}$ are terms, $\chi$ is a variable, and $g$ is a
  variable assignment, then
  
  \medskip\hspace{-4mm}
  \begin{tabular}{lll}
    (a) & $M,w,g \models \phi \tau_1\ldots \tau_n$ &iff $\t{[\tau_1]^{M,g},\ldots,[\tau_n]^{M,g}} \in V(\phi,w)$.\\
    (b) & $M,w,g \models \tau_1=\tau_2$ &iff $[\tau_1]^{M,g} = [\tau_2]^{M,g}$.\\
    (c) & $M,w,g \models \neg A$ &iff $M,w,g \not\models A$.\\
    (d) & $M,w,g \models A \land B$ &iff $M,w,g \models A$ and $M,w,g \models B$.\\
    (e) & $M,w,g \models A \lor B$ &iff $M,w,g \models A$ or $M,w,g \models B$.\\
    (f) & $M,w,g \models A \to B$ &iff $M,w,g \not\models A$ or $M,w,g \models B$.\\
    (g) & $M,w,g \models A \leftrightarrow B$ &iff $M,w,g \models (A\to B)$ and $M,w,g \models (B\to A)$.\\
    (h) & $M,w,g \models \forall \chi A$ &iff $M,w,g' \models A$ for all $\chi$-variants $g'$ of $g$.\\
    (i) & $M,w,g \models \exists \chi A$ &iff $M,w,g' \models A$ for some $\chi$-variant $g'$ of $g$.\\
    (j) & $M,w,g \models \Box A$ &iff $M,v,g \models A$ for all $v\in W$ such that $wRv$.\\
    (k) & $M,w,g \models \Diamond A$ &iff $M,v,g \models A$ for some $v\in W$ such that $wRv$.
  \end{tabular}

  $A$ is \textbf{true at $w$ in $M$} iff $M,w,g \models A$ for
  every assignment function $g$ for $M$.

\end{definition}

Let's return to the model from above, and let's add the information that the
domain of individuals consists of just Aristotle and Boethius. That is, let $M$
be the following model:
%
\begin{quote}
  $W = \{ w,v \}$\\
  $R = \{ \t{w,w}, \t{w,v} \}$\\
  $D = \{ \text{Aristotle, Boethius} \}$\\
  $V(a) = \text{Aristotle}$\\
  $V(F,w) = \{ \text{Aristotle, Boethius} \}$\\
  $V(F,v) = \{ \text{Boethius} \}$
\end{quote}
%
This isn't a complete specification of a model because I haven't assigned a
meaning to names and predicates other than $a$ and $F$, but we have enough
information to determine the truth-value of any $\L_{M\!P}$-sentence whose only
non-logical vocabulary are $a$ and $F$.

We can, for example, verify that $Fa$ is true at $w$ in $M$. A sentence is true
at $w$ in $M$ iff it is true at $w$ in $M$ relative to every assignment function
$g$. By clause (a) of definition \ref{def:constantdomainsemantics}, $Fa$ is true
at $w$ in $M$ relative to $g$ iff $[a]^{M,g}$ is a member of $V(F,w)$. Since $a$
is a name, $[a]^{M,g}$ is $V(a)$. And $V(a)$ is Aristotle. So $Fa$ is true at
$w$ relative to $g$ iff Aristotle is a member of $V(F,w)$. We know that $V(F,w)$
is $\{ \text{Aristotle, Boethius} \}$. Aristotle evidently is a member of
$\{ \text{Aristotle, Boethius} \}$. So $Fa$ is true at $w$ in $M$, relative to
any assignment $g$.

We can also verify that $\Box Fa$ is false at $w$. By clause (j) of definition
\ref{def:constantdomainsemantics}, $\Box Fa$ is true at $w$ (in $M$ relative to
$g$) iff $Fa$ is true (in $M$ relative to $g$) at all worlds accessible from
$w$. And $Fa$ is false at $v$ because Aristotle is not a member of $\{ \text{Boethius} \}$.


% Here is a (partial) picture of a constant domain model, with three worlds and
% three individuals.

% \begin{center}
%   \begin{tikzpicture}[modal, world/.append style={minimum size=22mm}, node distance=15mm]
%     \node[world] (w) [label=above:{$w$}] {};
%     \draw[gray,looseness=1] (w.south west) -- (w.north east);
%     \node[gray] at ([xshift=3mm,yshift=-2mm]w.north west){\small $F$};
%     \node[gray] at ([xshift=-3mm,yshift=2mm]w.south east){\small $\neg F$};
%     \node at ([yshift=15mm]w.south){\small $a$};
%     \node at ([yshift=7mm]w.south){\small $b$};
%     \node at ([yshift=10mm]w.south east){\small $c$};
%     %
%     \node[world] (v) [label=above:{$v$}, right=of w] {};
%     \draw[gray,looseness=1] (v.south west) -- (v.north east);
%     \node[gray] at ([xshift=3mm,yshift=-2mm]v.north west){\small $F$};
%     \node[gray] at ([xshift=-3mm,yshift=2mm]v.south east){\small $\neg F$};
%     \node at ([yshift=15mm]v.south){\small $a$};
%     \node at ([xshift=2mm,yshift=8mm]v.south west){\small $b$};
%     \node at ([yshift=6mm]v.south){\small $c$};
%     %
%     \node[world] (u) [label=above:{$u$}, right=of v] {};
%     \draw[gray,looseness=1] (u.south west) -- (u.north east);
%     \node[gray] at ([xshift=3mm,yshift=-2mm]u.north west){\small $F$};
%     \node[gray] at ([xshift=-3mm,yshift=2mm]u.south east){\small $\neg F$};
%     \node at ([yshift=17mm,xshift=3mm]u.south){\small $a$};
%     \node at ([xshift=-2mm,yshift=12mm]u.south){\small $b$};
%     \node at ([xshift=-7mm,yshift=8mm]u.south){\small $c$};
%     \path[->] (w) edge (v);
%     \path[->] (v) edge (u);
%   \end{tikzpicture}
% \end{center}
% %
% Each world is inhabited by $a,b$, and $c$. At world $w$, only $a$ is $F$; at
% $v$, $a$ and $b$ are $F$; at $u$, all three individuals are $F$. So $Fa$ is true
% at $w$. $Fx$ is true at $w$ relative to an assignment $g$ that maps $x$
% to $a$. By definition \ref{def:constantdomainsemantics}, this means that
% $\exists x Fx$ is true at $w$. Along the same lines, we can figure out that
% $\exists x Fx$ is true at $v$. Since $w$ can see $v$, it follows that
% $\Diamond \exists x Fx$ is true at $w$. The \emph{de re} sentence
% $\exists x \Diamond Fx$ is also true at $w$, because $\Diamond Fx$ is true at
% $w$ relative to an assignment $g$ that maps $g$ to (say) $b$.

\begin{exercise}
  Which of the following sentences are true at $w$ in $M$? 
  \begin{exlist}
  \item $\neg Fa \to Fa$ % true
  \item $\Box \exists x Fx$ % true
  \item $\Box \forall x Fx$ % false
  \item $\exists x \Box Fx$ % true
  \item $\forall x \Box Fx$ % false
  \item $\forall x (\Box Fx \to \Box\Box Fx)$ % true
  \end{exlist}
\end{exercise}
\begin{solution}
  (a), (b), (d), and (f) are true; (c) and (e) are false.
\end{solution}

Validity is truth at all worlds in all models of a certain kind. A sentence is
\textbf{CK-valid} iff it is true at all worlds in all constant-domain Kripke
models. `C' comes from `constant domains'; `K' indicates that we have put no
constraints on the accessibility relation. We get stronger concepts of validity
-- stronger logics -- if we require the accessibility relation to be reflexive,
or transitive, or euclidean, etc.

It is not hard to see that every sentence that is valid in classical predicate
logic is CK-valid. Similarly, every K-valid sentence is CK-valid. We also get
some new interaction principles between modal operators and quantifiers. For
example, consider the following schema, known as the \emph{Barcan Formula},
after Ruth Barcan Marcus.
%
\principle{BF}{\forall x \Box A \to \Box \forall x A}

\begin{observation}{BFinCDM}
  All instances of \pr{BF} are CK-valid.
\end{observation}
\begin{proof}
  \emph{Proof.} Suppose a sentence $\forall x \Box A$ is true at some world $w$
  in some constant-domain model $M$ relative to some assignment $g$. By clause
  (h) of definition \ref{def:constantdomainsemantics}, it follows that $\Box A$
  is true at $w$ relative to every $x$-variant $g'$ of $g$. By clause (j) of
  definition \ref{def:constantdomainsemantics}, it follows that $A$ is true at
  every world $v$ accessibility from $w$ relative to every $x$-variant $g'$ of
  $g$. By clause (h), this means that $\forall x A$ is true relative to $g$ at
  every world $v$ accessible from $w$. So by clause (j), $\Box \forall x A$ is
  true at $w$ relative to $g$.

  We've shown that whenever $\forall x \Box A$ is true at some world $w$ in some
  model $M$ relative some assignment $g$, then $\Box A \forall x A$ is also true
  at $w$ in $M$ relative to $g$. By clause (f) of definition
  \ref{def:constantdomainsemantics}, it follows that
  $\forall x \Box A \to\Box A \forall x A$ is true at every world in every model
  relative to every assignment. \qed
\end{proof}

Instead of working through definition \ref{def:constantdomainsemantics}, we can
use trees to test if a sentence is CK-valid. The tree rules for CK are all the
rules for K (from chapter 3) together with all the rules for standard predicate
logic, with an added world parameter on each node that is held fixed when
applying a rule from standard predicate logic. (In the predicate logic rules, a
name counts as `old' if it already occurs on the relevant branch, no matter at which world.)

To get a complete proof system, we need one further identity rule, reflecting
the fact that the reference of a name does not vary from world to world:

\medskip
\begin{center}
  \begin{minipage}[t]{0.3\textwidth} \centering

    Identity Invariance
    
    \bigskip
    \tree{
      \dotbelownode{15}{}{$\eta_1 = \eta_2$}{\omega}{}\\
      \\
      \nnode{15}{}{$\eta_1 = \eta_2$}{\nu}{}\\
      \Kk[15]{0}{\color{red}$\uparrow$}\\
      \Kk[15]{0}{\color{red}\small old}
    }
  \end{minipage}
\end{center}
\bigskip

% Can we prove the Necessity of Identity? Girle p.111 suggests we can't. So does
% https://softoption.us/content/node/647. But here's a proof (which I can run
% even on https://softoption.us/content/node/647, where it says the tree won't
% close):
%
% 1. a=b    (w)
% 2. -[]a=b (w)
% 3. -[]a=a (w)  (1,2,LL)
% 4. wRv         (3)
% 5. -a=a   (v)  (3)
%
% Girle says LL is restricted to literals, which is probably enough for
% predicate logic, but blocks this proof.
%
% Without Identity Invariance, we can't prove the Necessity of Distinctness, even with the liberal form of LL, and even if we have the S5 rules:
%
% 1. -a=b    (w)
% 2. -[]-a=b (w)
% 3. wRv         (2)
% 4. a=b     (v) (2) 
% 5. vRw         (3,Symmetry)
%
% At this point, we'd like to say that <>-a=b is true at v, because -a=b is true
% at the accessible w; then the tree could easily be closed. But there's no such
% rule.
%
% If we have a global LL rule that allows substituting corefering names at
% arbitrary other worlds then we can infer -a=a from lines 4 and 1, so we're
% done, even without Symmetry. That's equivalent to my version of LL and
% Identity Invariance. To mimick the axiomatic situation perhaps we should use a
% version of LL which says that we can substitute only at accessible worlds. Equivalently, we could say that

Here is a tree proof for a simple instance of the Barcan Formula,
$\forall x \Box Fx \to \Box \forall x Fx$.

\medskip
\begin{center}
  \tree{
    \nnode{30}{1.}{$\neg(\forall x \Box Fx \to \Box\forall x Fx)$}{w}{(Ass.)}\\
    \nnode{30}{2.}{$\forall x \Box Fx$}{w}{(1)}\\
    \nnode{30}{3.}{$\neg \Box\forall x Fx$}{w}{(1)}\\
    \nnode{30}{4.}{$wRv$}{}{(3)}\\
    \nnode{30}{5.}{$\neg \forall x Fx$}{v}{(3)}\\
    \nnode{30}{6.}{$\neg Fa$}{v}{(5)}\\
    \nnode{30}{7.}{$\Box Fa$}{w}{(2)}\\
    \nnodeclosed{30}{8.}{$Fa$}{v}{(7,4)}\\
  }
\end{center}
\medskip

And here is a proof of $\forall x \forall y(x\!=\!y \to \Box\, x\!=\!y)$, the
``necessity of identity'':
%
\medskip
\begin{center}
  \tree{
    \nnode{30}{1.}{$\neg\forall x\forall y(x\!=\!y \to \Box\, x\!=\!y)$}{w}{(Ass.)}\\
    \nnode{30}{2.}{$\neg\forall y(a\!=\!y \to \Box \,a\!=\!y)$}{w}{(1)}\\
    \nnode{30}{3.}{$\neg(a\!=\!b \to \Box \,a\!=\!b)$}{w}{(2)}\\
    \nnode{30}{4.}{$a=b$}{w}{(3)}\\
    \nnode{30}{5.}{$\neg \Box \,a\!=\!b$}{w}{(3)}\\
    \nnode{30}{6.}{$\neg \Box \,b\!=\!b)$}{w}{\qquad(4, 5, LL)}\\
    \nnode{30}{7.}{$wRv$}{}{(6)}\\
    \nnode{30}{8.}{$b\not=b$}{v}{(6)}\\
    \nnodeclosed{30}{9.}{$b=b$}{v}{(SI)}
  }
\end{center}

\begin{exercise}\label{ex:CKexamples}
  Use the tree method to show that the following sentences are
  CK-valid. 
  \begin{exlist}
  \item $ \Box \forall x Fx \to \forall x \Box Fx$
  \item $ \exists x \Box Fx \to \Box \exists x Fx$
  \item $ \forall x \Box (Fx \land Gx) \to \Box \forall x Fx$
  \item $ \Box\Diamond \exists x Fx \to \Box \exists x \Diamond(Fx \lor Gx)$ % Priest p.327
  \item $\forall x \Box \exists y\, y\!=\!x$
  \item $ \forall x \forall y(x\!\not=\!y \to \Box x\!\not=\!y)$
  \end{exlist}
\end{exercise}
\begin{solution}
  Use \href{https://www.umsu.de/trees/}{umsu.de/trees/}.
  Note that the website uses slightly different identity rules: instead of the
  Self-Identity rule, it has a rule for closing any branch that contains a
  statement of the form $\tau \not= \tau$.
\end{solution}

\begin{exercise}
  The following sentences are CK-invalid. Can you describe a countermodel for
  each? (It may help to construct a tree and inspect its open branches.)
  \begin{exlist}
    \item $\Diamond \exists x Fx \to \Diamond\exists x(Fx \land Gx)$
    \item $\Box \exists x Fx \to \exists x \Box Fx$
    \item
    $\forall x \forall y ((\Diamond Fx \land \Diamond \neg Fy) \to x\!\not=\!y)$
    \item $\forall x \Box (Px \to Qx) \to \forall x (Px \to \Box Qx)$
  \end{exlist}
\end{exercise}
\begin{solution}
  \begin{sollist}
    \item $W=\{ w \}$, $wRw$, $D = \{ \text{Alice} \}$,
    $V(F,w) = \{ \text{Alice} \}$, $V(G,w) = \emptyset$
    \item $W=\{ w,v \}$, $wRw$ and $wRv$, $D = \{ \text{Alice}, \text{Bob} \}$,
    $V(F,w) = \{ \text{Alice} \}$, $V(F,v) = \{ \text{Bob} \}$
    \item $W=\{ w,v \}$, $wRw$ and $wRv$, $D = \{ \text{Alice}, \text{Bob} \}$,
    $V(F,w) = \{ \text{Alice} \}$, $V(F,v) = \emptyset$
    \item $W=\{ w,v \}$, $wRw$ and $wRv$, $D = \{ \text{Alice}, \text{Bob} \}$,
    $V(P,w) = \{ \text{Alice} \}$, $V(P,v) = \emptyset$,
    $V(Q,w) = \{ \text{Alice} \}$, $V(Q,v) = \emptyset$
  \end{sollist}
\end{solution}

There are also axiomatic calculi for CK. We can, for example, combine the axiom
schemas and rules of classical predicate logic with those of K, and add two new
schemas: the Barcan Formula \pr{BF} and the ``necessity of distinctness'',
%
\principle{ND}{\forall x \forall y(x\!\not=\!y \to \Box x\!\not=\!y).}

% (BF) is provable if we have the B-schema. The proof (due to Prior) is
% surprisingly difficult.
% \begin{align*}
%    (1) \quad & \forall x \Box A \to \Box A &&\text{ by first-order logic}\\
%    (2) \quad & \Diamond\forall x \Box A \to \Diamond\Box A &&\text{ from (1) by K\ }\\
%    (3) \quad & \neg A \to \Box\Diamond \neg A &&\text{ (B)\ }\\ 
%    (4) \quad & \Diamond\Box A \to A &&\text{ from (3)\ }\\ 
%    (5) \quad & \Diamond\forall x \Box A \to A &&\text{ from (2) and (4)\ }\\ 
%    (6) \quad & \Diamond\forall x \Box A \to \forall x A &&\text{ from (5) by first-order logic\ }\\
%    (7) \quad & \Box\Diamond\forall x \Box A \to \Box\forall x A &&\text{ from (6) by K\ }\\  
%    (8) \quad & \forall x \Box A \to \Box\Diamond\forall x \Box A &&\text{ (B)}\\
%    (9) \quad & \forall x \Box A \to \Box \forall x A &&\text{ from (7) and (8)}
% \end{align*}

% Like the Barcan formula, \pr{ND} is provable if we add the axioms or rules for
% the modal logic B or S5, but it is not provable in the minimal combination of
% predicate logic with K. Proof in B:
%
% 1. <>-(x=y) -> -(x=y)      (NI)
% 2. []<>-(x=y) -> []-(x=y)  (1,K,M\!P)
% 3. -(x=y) -> []<>-(x=y)    (B)
% 4. -(x=y) -> []-(x=y)      (2,3)

% \begin{exercise}\label{ex:nni}
%   Show that \pr{ND} is CK-valid. 
% \end{exercise}
% \begin{solution}
%   Suppose for reductio that some instance of \pr{ND} is false at
%   some world $w$ in some constant domain model $M$. Then there is some
%   assignment $g$ such that $M,w,g \models \neg(x=y)$ and
%   $M,w,g \not\models \Box\neg(x=y)$. The latter means that
%   $M,v,g \not\models \neg(x=y)$ for some world $v$. But
%   $M,w,g \models \neg(x=y)$ holds only if $g(x)\not=g(y)$, and 
%   $M,v,g \not\models \neg(x=y)$ holds only if $g(x)=g(y)$. Contradiction. 
% \end{solution}

As I mentioned above, stronger logics can be defined by putting constraints on
the accessibility relation. For example, the system \textbf{CT} is the set of
$\L_{M\!P}$-sentences that are valid in the class of constant-domain Kripke
models with a reflexive accessibility relation. \textbf{CS4} is the set of
$\L_{M\!P}$-sentences that are valid in the class of constant-domain Kripke
models with a reflexive and transitive accessibility relation. And so on.

Properties of the accessibility relation still correspond to modal schemas, just
as in chapter \ref{ch:accessibility}: \pr{T} corresponds to reflexivity, \pr{4}
to transitivity, \pr{G} to convergence, etc. Recall that a schema
\emph{corresponds} to a property of the accessibility relation if the schema is
valid in all and only the frames in which the accessibility relation has that
property. A \emph{frame} is a model without an interpretation function. In the
present context, a frame therefore consists of two non-empty sets $W$ and $D$ and a
relation $R$ on $W$.

We can still use the tree method or the axiomatic method to test for validity in
logics stronger than CK. To test for CT-validity, for example, we would add the
Reflexivity rule to the tree rules for CK. To test for CS4-validity, we would
add the Reflexivity and Transitivity rules. We can get an axiomatic calculus for
CT by adding the \pr{T}-schema to the calculus for CK; for CS4, we can add
\pr{T} and \pr{4}. And so on for other systems.

But there are exceptions. Remember S4.2 -- the set of $\L_M$-sentences valid in
the class of reflexive, transitive, and convergent Kripke models. Reflexivity
corresponds to \pr{T}, transitivity to \pr{4}, and convergence to \pr{G}. If we
add these schemas to the axiomatic calculus for system K, we get a sound and
complete calculus for S4.2. But if we add the schemas to the calculus for CK,
the resulting calculus is \emph{not} complete for CS4.2. There are
$\L_{MP}$-sentences that are valid in the class of reflexive, transitive, and
convergent constant-domain models that can't be derived.

% In particular, we can't prove
% $\neg(\Diamond(\exists x Ax \land \forall x(Ax \to \Box Bx) \land \Box \neg\forall x Bx) \,\land\, \Diamond\forall x(Ax \lor \Box Bx)\,\land\, \forall x (\Diamond Ax \to \Box (\exists x Ax \to Ax)))$.)
%
% Similarly for S4M: QS4M+(BF) cannot prove
% $\Box \exists x Ax \to \Diamond \exists x \Box Ax$, which is valid in the
% class of all its frames.
%
% For the proofs, see \cite{cresswell95incompleteness}.
%
% In either case, the canonicity proof does not carry over to the first-order
% case because first-order canonicity requires that certain formula sets have
% not only a maximally consistent extension, but a maximally consistent
% \emph{and witnessed} extension.
%
% Does a problem like this also arise for the tree rules?

% The calculus we get if we add T+4+G is not complete with respect to any class
% of constant-domain frames.

\section{Quantification and existence}

We have assumed that the domain of individuals is the same for every world. This
may seem problematic.

Earlier today I was baking bread. Let's call the loaf of bread that I made
Loafy. Intuitively, Loafy could have failed to exist. I could have decided not
to bake bread. Even if determinism is true, we can consider worlds at which the
laws of nature or the origin of the universe are different. In many of these
worlds, there are no humans, and no loafs of bread. So we should allow for
worlds at which Loafy doesn't exist.

If we use $b$ as a name for Loafy, we can arguably express Loafy's existence as
\[
  \exists x \,x\!=\!b.
\]
Why might this express that Loafy exists? Consider a scenario in which Loafy does
exist. In that scenario, there is some thing $x$ which is identical to Loafy
(namely, Loafy). Conversely, consider a scenario in which Loafy does not exist.
In that scenario, there is no thing $x$ which is identical to Loafy. So
$\exists x\, x\!=\!b$ is true in all and only the scenarios in which Loafy exists.

Now we can sharpen the above worry. Intuitively, it could have been the case
that Loafy doesn't exist. So $\Diamond \neg \exists x\, x\!=\!b$ is true, on a
suitable understanding of the diamond. But in constant-domain semantics, that
sentence is a contradiction: it is false at every world in every model.

A converse problem arises if we think that something could have existed that
doesn't actually exist. For example, let's assume that there could have been
unicorns. If we interpret the predicate $U$ as `-- is a unicorn' and the box as
a suitable kind of circumstantial necessity, $\Box \forall x \neg Ux$ should
then be false. But let's also assume that no individual in our world could have
been a unicorn. So $\forall x \Box \neg Ux$ is true. We then have a
counterexample to the Barcan Formula $\forall x \Box A \to \Box \forall x A$.
And all instances of the Barcan Formula are valid in constant-domain semantics.

% This problem is a little harder to bring out because there is no direct way to
% say, in $\L_{M\!P}$, that something could have existed that doesn't actually
% exist. We could express it if we add an `actually' operator:
% $\neg \Diamond\exists x \neg \Always \exists y(y=x)$ comes out valid.

\begin{exercise}
  The \textbf{Converse Barcan Formula} is the schema
  $\Box \forall x A \to \forall x \Box A$. All instances of the Converse Barcan
  Formula are CK-valid. Explain why Loafy's possible non-existence seems to
  provide a counterexample to the Converse Barcan Formula.
\end{exercise}
\begin{solution}
  $\Box \forall x \exists y\, x\!=\!y \to \forall x \Box\exists y\, x\!=\!y$ is an
  instance of the Converse Barcan Formula. If we read the box as a relevant kind
  of circumstantial necessity, and Loafy could have failed to exist, then the
  consequent of this conditional is false. But the antecedent is true.
\end{solution}

% Here is a (schematic) proof of \pr{CBF} in the combined axiomatic
% calculus for predicate logic and the modal logic K.
% \begin{align*}
%    1. \quad & \forall x A \to A[c/x] &&\text{ (UI)}\\
%    2. \quad & \Box(\forall x A \to A[c/x]) &&\text{ (from 1 by Nec)}\\
%    3. \quad & \Box(\forall x A \to A[c/x]) \to (\Box\forall x A \to \Box A[c/x]) &&\text{ (K)}\\
%    4. \quad & \Box\forall x A \to \Box A[c/x] &&\text{ (from 2 and 3 by M\!P)}\\
%    5. \quad & \Box \forall x A \to \forall x \Box A &&\text{ (from 4 by UG)}.
% \end{align*}

\begin{exercise}
  Consider the following four schemas.
  \begin{enumerate}[leftmargin=14mm]
    \itemsep-1mm
  \item[(1)] $\Diamond \exists x A \to \exists x \Diamond A$
  \item[(2)] $\Box \exists x A \to \exists x \Box A$
  \item[(3)] $\exists x \Box A \to \Box \exists x A$
  \item[(4)] $\exists x \Diamond A \to \Diamond \exists x A$
  \end{enumerate}
  \vspace{-3mm}
  \begin{exlist}
    \item Are any of (1)--(4) equivalent to the Barcan Formula or the Converse
    Barcan Formula (given the duality of $\Box$ and $\Diamond$, of $\forall x$
    and $\exists x$, and the standard truth-tables for propositional
    connectives)?
    \item Which of these schemas do you think are intuitively valid on a
    metaphysical interpretation of the box and the diamond?
  \end{exlist}
\end{exercise}
\begin{solution}
  (1) is equivalent to the Barcan Formula, (4) to the Converse Barcan Formula.
  (2) is highly implausible. (1) and (4) are often regarded as implausible, for
  the reasons I discuss in the text. Like the Converse Barcan Formula, the
  validity of (3) rules out scenarios in which individuals at one world may fail
  to exist at an accessible world.
  % (3) isn't equivalent to CBF though: (3) can be valid even without increasing domains, provided that everything at w exists at /some/ accessible world.
\end{solution}  
  
% \begin{exercise}
%   The \pr{K}-like principle
%   $\forall x (A \to B) \to (\forall x A \to \forall x B)$ is easily provable
%   in classical predicate logic. Can you see how the strict version
%   $\forall x (A \strictif B) \to (\forall x A \strictif \forall x B)$ is
%   related to the Barcan Formula?
% \end{exercise}

An obvious response to these problems is to replace constant-domain semantics
with a semantics in which the domain of individuals can vary from world to
world. We will explore this option in the following section. First I want to
mention two other lines of response.

Some philosophers have argued that we should bite the bullet: we are simply
mistaken when we judge that Loafy could have failed to exist, or that anything
could have existed that doesn't actually exist. In temporal logic, biting the
bullet means to accept that anything that has ever existed still exists today,
and that anything that exists today has always existed and is always going to
exist. In epistemic logic, biting the bullet means to accept that nobody can be
unsure or ignorant about which individuals exists: if something exists, nobody
can fail to know that it exists, nor can anyone believe that an individual
exists that doesn't really exist.

% It seems odd to say that if I'm unsure whether there are dragons then I'm
% actually sure the relevant objects exist, just not whether they are dragons.

A different response is to break the link between quantification and
existence. $\exists x$ is traditionally called an ``existential'' quantifier,
and pronounced `there is an $x$' or `there exists an $x$'. But $\L_{M\!P}$ is a
made-up language. We can make its symbols mean whatever we want. We can give a
different interpretation of $\exists x$ so that `Loafy exists' can't be
translated as $\exists x\, x\!=\!b$.

One alternative to the standard interpretation of quantifiers is associated with
the Austrian philosopher Alexius Meinong. Meinong observed that when we describe
beliefs, plans, hopes, or fears, we often seem to refer to non-existent objects.
We might say that someone is afraid of \emph{a ghost}, or that they are
searching for \emph{a golden mountain} -- even though there are no ghosts or
golden mountains. According to Meinong, people who are searching for a golden
mountain are really searching for \emph{something}. That something is a golden
mountain. But it is not an existent golden mountain. Meinong concluded that
besides existent mountains, there are also non-existent mountains.

Quantifiers that range over both existent and non-existent individuals are
called \emph{Meinongian}. If the $\L_{M\!P}$-quantifiers are Meinongian, then
clearly $\exists x\, x\!=\!b$ does not translate `Loafy exists'.

Meinong's postulation of non-existent individuals is widely rejected as
incoherent. It certainly raises difficult questions. Suppose you are
searching for a golden mountain. You probably don't have any firm views about
the mountain's height. You are not looking for a mountain that is exactly 2000
meters tall, nor are you looking for a mountain that is exactly 2100 meters
tall. On the Meinongian account, there is a genuine mountain that you
are looking for. It is a mountain that is not 2000 meters tall, not 2100 meters
tall, and doesn't have any other particular height either. But how could there
be a mountain without any particular height? Besides, it also doesn't seem right
to say that you are looking for a peculiar ``mountain'' that doesn't have any
height and doesn't exist. Intuitively, you are looking for an \emph{existent}
mountain that \emph{does} have a height.

% Even if we accept Meinongian quantification as coherent, it is not clear whether
% it fully avoids the problem of constant domains. For example, couldn't I be
% unsure about how many things there are, even in the Meinongian, extended sense
% of `there are'? To model my uncertainty in terms of accessible worlds, the
% domain of the Meinongian quantifier would have to vary from world to world.

% One motivation for this move is that in natural language, we can apparently
% quantify over non-existent objects: We seem to have names for them, like
% `Pegasus', and we can quantify over them, as when I say that there's a strange
% house I often see in my dreams, made of chocolate. The question is whether we
% want to say such things in our formal language, and if so, how we want to say
% them. The purpose of our language is to avoid confusion and aid clear
% reasoning. And for that purpose, many think it advisable to not quantify over
% things that don't exist.

% Consider that house in my dream. If someone made an inventory of all houses,
% should that house really be included? If someone says that \emph{there are no}
% houses made of chocolate, are they wrong? What other properties does that house
% have? When was it built? Does someone live in it? My dreams don't give an
% answer. It is hard to believe that there is a fact of the matter. So should we
% accept that there are houses that are neither inhabited nor uninhabited?

% Also, as Lycan 1994, p.5 points out, if `the city 100 km South of Edinburgh
% and 100 km North of London' refers, doesn't it follow that Edinburgh is 200 km
% North of London?

% Another obvious problem that arises if we allow for non-existent objects is
% that we must now explain both the quantifier $\exists$ and the predicate
% ``existing'', in a way that doesn't trivialize the issue. (For example, it
% won't do to say that `existing' means being concrete.) Lewis has an answer:
% ``existing'' means `being located in our world (and at the present)'.

% Barcan Marcus proposed that Meinongian quantification is substitutional. That
% looks plausible for `there are things that don't exist'.

A more straightforward alternative to the standard interpretation of quantifiers
is the \emph{possibilist} interpretation. Here we assume that $\forall x$ and
$\exists x$ range not only over things that exist at the world at which the
quantifiers are interpreted, but over everything that exists at any possible
world. On this interpretation, too, $\exists x\, x\!=\!b$ no longer states that Loafy
exists. It merely states that Loafy could have existed, in an unrestricted sense
of `could'. Constant-domain semantics then only assumes that the set of
individuals that exist at some world or other does not vary from world to world.

% Now, it seems rather odd to say that something at $w$ is red at $w$ and in a
% box at $w$ even though no existing object is red at $w$ and even though the
% box is empty at $w$. So most predicates are existence-entailing. The things
% that don't exist aren't red or in a box. They are possibly red and possibly in
% a box.

% I must not conflate the view that actuality contains non-concrete individuals
% that are dragons in other worlds with the view that quantifiers directly range
% over non-actual individuals. Rini and Cresswell say that `there could have
% been a unicorn' means that there is a world $w'$ at which something $u$ is a
% unicorn; ``If \emph{unicorn} is a natural kind term then presumably nothing
% actual \emph{could} be a unicorn, and so $u$ is a non-actual possible.''
% (p.72)

One  downside of the possibilist interpretation is that it goes against the
``internalist'' spirit of modal logic. As we saw in section \ref{sec:fragment},
one of the key features of modal logic is that it looks at the structure of
worlds from the inside, from the perspective of a particular world, with only
the modal operators providing (incomplete) access to other worlds.  Possibilist
quantifiers would provide unrestricted access to the inhabitants of other
worlds.

Let's set aside these alternatives and see how constant-domain semantics could
be changed to allow for variable domains.

\section{Variable-domain semantics}\label{sec:variabledomains}

In variable-domain models, every world $w$ is associated with its own individual
domain $D_w$. Loafy the bread may be a member of $D_w$ but not of $D_v$.
Quantifiers range over the individuals in the local domain of the world at which
they are interpreted: $\exists x Fx$ is true at $w$ iff $Fx$ is true (at $w$) of
some individual in $D_w$.

Here is our revised definition of an $\L_{M\!P}$-model.

\begin{definition}{}{variabledomainmodel}
  A \textbf{variable-domain Kripke model} for $\L_{M\!P}$ is a structure $M$
  consisting of%
  \medskip
  \begin{compactenum}
  \item a non-empty set $W$ (the ``worlds''),
  \item a binary (``accessibility'') relation $R$ on $W$,
  \item for each world $w$, a non-empty set $D_w$ (of ``individuals''), and
  \item an interpretation function $V$ that assigns
    \vspace{-1mm}
    \begin{itemize}
      \itemsep-1mm
      \item to each name a member of some domain $D_w$, and
      \item to each $n$-place predicate and world $w$ a set of $n$-tuples from
            $D_w$.
    \end{itemize}
  \end{compactenum}
\end{definition}

To complete the semantics, we need to explain how $\L_{M\!P}$-sentences are
interpreted relative to any given world in a variable-domain model. This raises
a problem.

Since Loafy could have failed to exist, we want to have models in which
$\Diamond \neg \exists x\, {x\!=\!b}$ is true at some world $w$. It follows that
$\neg \exists x\, x\!=\!b$ is true at some world $v$ accessible from $w$.
Intuitively, $v$ is a world at which Loafy doesn't exist. The problem is that we
need to explain how a sentence that contains a name (here, $b$) should be
interpreted at a world (here, $v$) where the thing that's picked out by the name
doesn't exist.

In the case of $\neg \exists x\, x\!=\!b$, the sentence should come out true.
Other cases are less clear. What about $b\!=\!b$? Is Loafy identical to Loafy at
$v$, where Loafy doesn't exist? What about $Fb$, $\neg Fb$, or
$Fb \lor \neg Fb$? Is Loafy delicious at $v$? Is Loafy not delicious at $v$? Is
Loafy either delicious or not delicious at $v$?

These questions are discussed not just in modal logic, but also in a branch of
non-modal logic called \textbf{free logic}. Free logic differs from classical
predicate logic by dropping the assumption that every name has a referent. The
assumption is, after all, not true for names in natural language.

Consider the story of `Vulcan'. In the 19th century, it was observed that
Mercury's path around the Sun conforms to Newton's laws only if there is
another, smaller planet between Mercury and the Sun. With the help of Newton's
laws, astronomers calculated the size and position of that planet, and called it
Vulcan. But Vulcan was never discovered. Eventually, Mercury's path was
explained by Einstein's theory of relativity, without assuming any new planets.
The name `Vulcan' turned out to be \emph{empty}: it doesn't refer to anything.

% Before applying classical logic one would have to determine which names refer
% and which don't, which often requires substantial empirical information. Free
% logic avoids these problems. It can be applied even if the emptiness of names
% is unknown, and it can capture valid inference with empty names.

How should we formalize reasoning with empty names? The orthodox answer is that
we shouldn't: the function of a name is to pick out an individual; if there is
no individual to be picked out, we shouldn't use a name. Proponents of free
logic disagree. They hold that we can perfectly well reason with empty names. We
then need to answer the same questions that I posed above: if $b$ is an empty
name, how should we interpret $b=b$, $Fb$, $\neg Fb$, and $Fb \lor \neg Fb$?

Within free logic, there are broadly three approaches.

The first is Meinongian. It assumes that apparently empty names are not really
empty after all; they merely pick out a non-existent individual. Statements with
such names are then interpreted as usual: $Fb$ may be true or false, depending
on whether the (non-existent) individual picked out by $b$ has the property
expressed by $F$.

Non-Meinongian versions of free logic usually assume that \emph{atomic}
sentences with empty names are never true: if $b$ is empty, then $Fb$ can't be
true. The idea is that predicates express properties, and if something doesn't
exist then it doesn't have any properties. For example, it is not true that
Vulcan is a planet -- as you can see from the fact that Vulcan would not occur
on a list of all planets. Nor is it true that Vulcan orbits the sun, or that
Vulcan has any particular mass.

% To be sure, Vulcan was \emph{believed} to be a planet and to orbit the Sun.
% But one could argue that this should be formalized as something like
% $\Bel(Pa \land Oas)$, not $Fa$.

What shall we say about $\neg Fb$ then, if $b$ is an empty name? In some
versions of free logic, the standard semantic rules for complex sentences are
applied: since $Fb$ is not true, $\neg Fb$ is true, and so is $Fb \lor \neg Fb$.
Other versions of free logic assume that if $b$ doesn't refer then neither $Fb$
nor $\neg Fb$ is true. Since a sentence is called false iff its negation is
true, this means that $Fb$ and $\neg Fb$ are neither true nor false. We get a
three-valued semantics that can be spelled out in different ways, with different
verdicts on sentences like $Fb \lor \neg Fb$.

Each version of free logic can be used to give a semantics for modal predicate
logic with variable domains. I am going to use the two-valued non-Meinongian
approach, mainly because it is the simplest. We will assume that at worlds where
Loafy doesn't exist, every atomic sentence involving a name for Loafy is false:
$b=b$ is false, $Fb$ is also false, but $\neg Fb$ and $Fb \lor \neg Fb$ are
true.

% (see \cite{lambert03philosophical} or \cite{nolt07free} for a more in-depth
% overview of all this) and e.g. \cite[para 2.3]{sainsbury05reference} for a
% defense of my choice.

\begin{definition}{Variable-domain Kripke semantics}{variabledomainsemantics}
  If $\Mfr = \t{W,R,D,V}$ is a variable-domain Kripke model, $w$ is a member of
  $W$, $\phi$ is an $n$-place predicate (for $n\geq 0$), $\tau_1,\ldots,\tau_n$
  are terms, $\chi$ is a variable, and $g$ is a variable assignment, then
  
  % Note that [\tau]^{M,g} is never empty. We've assumed that names have a
  % referent.
  
  \medskip\hspace{-4mm}
  \begin{tabular}{lll}
    (a) & $M,w,g \models \phi \tau_1\ldots \tau_n$ &iff $\t{[\tau_1]^{M,g},\ldots,[\tau_n]^{M,g}} \in V(\phi,w)$.\\
    (b) & $M,w,g \models \tau_1=\tau_2$ &iff $[\tau_1]^{M,g} = [\tau_2]^{M,g}$ and $[\tau_1]^{M,g} \in D_w$.\\
    (c) & $M,w,g \models \neg A$ &iff $M,w,g \not\models A$.\\
    (d) & $M,w,g \models A \land B$ &iff $M,w,g \models A$ and $M,w,g \models B$.\\
    (e) & $M,w,g \models A \lor B$ &iff $M,w,g \models A$ or $M,w,g \models B$.\\
    (f) & $M,w,g \models A \to B$ &iff $M,w,g \not\models A$ or $M,w,g \models B$.\\
    (g) & $M,w,g \models A \leftrightarrow B$ &iff $M,w,g \models (A\to B)$ and $M,w,g \models (B\to A)$.\\
    (h) & $M,w,g \models \forall \chi A$ &iff $M,w,g' \models A$ for all $\chi$-variants $g'$ of $g$ for\\[-1mm]
        && which $g'(\chi)\in D_w$.\\
    (i) & $M,w,g \models \exists \chi A$ &iff $M,w,g' \models A$ for some $\chi$-variant $g'$ of $g$ for\\[-1mm]
        && which $g'(\chi)\in D_w$.\\
    (j) & $M,w,g \models \Box A$ &iff $M,v,g \models A$ for all $v\in W$ such that $wRv$.\\
    (k) & $M,w,g \models \Diamond A$ &iff $M,v,g \models A$ for some $v\in W$ such that $wRv$.
  \end{tabular}
  $A$ is \textbf{true at $w$ in $M$} iff $M,w,g \models A$ for all assignments
  $g$ for $M$.
\end{definition}

A sentence is \textbf{VK-valid} (`V' for `variable-domain') iff it is true at all
worlds in all variable-domain models.

The system VK is weaker than classical predicate logic. Not everything
that is valid in classical predicate logic is CK-valid. For example, both $b=b$
and $\exists x\, x\!=\!b$ are valid in classical predicate logic, but they are
not true at every world in every variable-domain model. If $V(b)$ is not a
member of $D_w$, then $b=b$ and $\exists x \, x\!=\!b$ are false at $w$.

On the other hand, you can check that $\forall x\, x\!=\!x$ is VK-valid. So we
don't just have to revise the rules for identity. We also need to revise the
rule of ``universal instantiation'': from the fact that a universal
generalisation like $\forall x\, x\!=\!x$ is true (at a world, or at all worlds),
we can't infer that all its instances are true: $b=b$ may be false. For another
example, consider a world $w$ where everything is made of chocolate. Let $F$
express the property of being made of chocolate. $\forall x Fx$ is true at $w$.
But we can't infer that Loafy the bread is made of chocolate ($Fb$) at $w$, for
Loafy may not exist at $w$.

In the type of free logic we have adopted, the rule of universal instantiation
requires another premise: from $\forall x A$ we can infer $A[b/x]$ only if we
also know that $b$ exists -- which can be expressed as $\exists x \, x\!=\!b$, or
even simpler as $b\!=\!b$, given our assumption that atomic sentences with empty
names are always false.

Here are the revised tree rules for VK. I only give the quantifier rules for
$\forall \chi A$ and $\exists \chi A$. You can find the rules for
$\neg \forall \chi A$ and $\neg\exists \chi A$ by converting these into
$\exists \chi \neg A$ and $\forall \chi \neg A$, respectively.

\bigskip

\begin{minipage}{0.6\textwidth} \centering
\tree[2]{
  & \dotbelowbnode{12}{}{$\forall \chi A$}{\omega}{} &\\
  && \\
  && \\
  \nnode{10}{}{$\eta\!\not=\!\eta$}{\omega}{} && \nnode{12}{}{$A[\eta/\chi]$}{\omega}{}\\
  \Kk[3]{0}{\color{red}$\uparrow$} &&\\
  \Kk[3]{0}{\color{red}\small old} &&
}
\end{minipage}
\begin{minipage}{0.4\textwidth}\centering
\tree{
  \dotbelownode{12}{}{$\exists \chi A$}{\omega}{}\\
  \\
  \nnode{12}{}{$\eta=\eta$}{\omega}{}\\
  \nnode{12}{}{$A[\eta/\chi]$}{\omega}{}\\
  \Kk[-1]{0}{\color{red}$\uparrow$}\\
  \Kk[0]{0}{\color{red}\small new}
}
\end{minipage}

% \begin{minipage}{0.24\textwidth}\centering
% \tree{
%   \dotbelownode{12}{}{$\neg\forall \chi A$}{\omega}{}\\
%   \\
%   \nnode{12}{}{$\eta=\eta$}{\omega}{}\\
%   \nnode{12}{}{$\neg A[\eta/\chi]$}{\omega}{}\\
%   \Kk[-1]{0}{\color{red}$\uparrow$}\\
%   \Kk[0]{0}{\color{red}\small new}
% }
% \end{minipage}
% \begin{minipage}{0.24\textwidth} \centering
% \tree[2]{
%   & \dotbelowbnode{12}{}{$\neg\exists \chi A$}{\omega}{} &\\
%   && \\
%   && \\
%   \nnode{15}{}{$\eta\!\not=\!\eta$}{\omega}{} && \nnode{15}{}{$\neg A[\eta/\chi]$}{\omega}{}\\
%   \Kk[4]{0}{\color{red}$\uparrow$} &&\\
%   \Kk[4]{0}{\color{red}\small old or first} &&
% }
% \end{minipage}


\bigskip

We keep the rule for Leibniz's Law. But we replace the Self-Identity and Identity Invariance rules by the following three rules.

\bigskip

\begin{minipage}{0.32\textwidth}\centering
    Existence
    
    \tree{
      \dotbelowbarenode{}\\
      \\
      \nnode{10}{}{$\eta=\eta$}{\omega}{}\\
      \Kk[0]{0}{\color{red}$\uparrow$\hspace{6mm}}\\
      \Kk[0]{0}{\color{red}\small new\hspace{6mm}}
    }
\end{minipage}
\begin{minipage}{0.32\textwidth}\centering
    Identity Invariance
    
    \bigskip
    \tree{
      \nnode{15}{}{$\eta_1 = \eta_2$}{\omega}{}\\
      \dotbelownode{15}{}{$\eta_1 = \eta_1$}{\nu}{}\\
      \\
      \nnode{15}{}{$\eta_1 = \eta_2$}{\nu}{}
    }
\end{minipage}
\begin{minipage}{0.32\textwidth}\centering
\tree{
  \dotbelownode{16}{}{$\Phi\eta_1\ldots\eta_n$}{\omega}{}\\
  \\
  \nnode{16}{}{$\eta_1=\eta_1$}{\omega}{}\\
  \nnode{16}{}{$\eta_2=\eta_2$}{\omega}{}\\
  \nnode{16}{}{$\vdots$}{}{}\\
  \nnode{16}{}{$\eta_n=\eta_n$}{\omega}{}
}
\end{minipage}

% Shouldn't I turn Identity Invariance into a branching rule, like FUI?

\bigskip

The Existence rule reflects our assumption that the domain of individuals is
never empty. The unnamed last rule is a rule for expanding atomic nodes. From the
assumption that $Fb$ is true at a world, for example, the rule allows us to
infer that $b$ exists at that world, which can be expressed as $b\!=\!b$. We
then don't need a separate rule of Self-Identity.

% These rules are adapted from Priest, pp. 331, 337, and 353. The main
% differences are these:
%
% (1) in NFL, we can use a=a instead of Ea. This makes Priest's Self-Identity
% rule trivial.
%
% (2) Priest allows all inner domains to be empty, I require them all to be
% non-empty. This makes a difference. From $\forall x Fx$, I can infer that $Fx$
% is true of something. Priest's rules don't allow that. I assume this can be
% fixed by adding a new rule that allows us to introduce a new name and say that
% it exists. This is the new Self-Identity rule. When expanding universal
% quantifiers, one then never needs to introduce a new (first) name.

\begin{exercise}
  Use the tree method to show that the following sentences are VK-valid.
  \begin{exlist}
  \item $\exists x \Box Fx \to \Box \exists x Fx$
  \item $\Box\forall x(Fx \to Gx) \to (\Box\forall x Fx \to \Box \forall x Gx)$ % Priest p.332
  \item $\Box \exists x \, x\!=\!x$
  \item $\Diamond Fa \to \Diamond \exists x Fx$
  \item $a\!=\!b \to \Box(a\!=\!a \to a\!=\!b)$
  \end{exlist}
\end{exercise}
\begin{solution}
  \begin{sollist}
    \item \tree[4]{%
        & \nnode{22}{1.}{$\exists x \Box Fx \to \Box \exists x Fx$}{w}{(Ass.)} & \\
        & \nnode{22}{2.}{$\exists x \Box Fx$}{w}{(1)} & \\
        & \nnode{22}{3.}{$\neg \Box \exists x Fx$}{w}{(1)} & \\
        & \nnode{22}{4.}{$\Box Fa$}{w}{(2)} & \\
        & \nnode{22}{5.}{$wRv$}{}{(3)} & \\
        & \nnode{22}{6.}{$\neg \exists x Fx$}{v}{(3)} & \\
        & \nnode{22}{7.}{$Fa$}{v}{(4,5)} & \\
        & \bnode{22}{8.}{$a\!=\!a$}{v}{(7)} & \\
        && \\
        \nnodeclosed{12}{9.}{$a\!\not=\!a$}{v}{(6)} &&  \nnodeclosed{12}{9.}{$\neg Fa$}{v}{(6)}\\
    }
    \medskip

    \item DIY. The tree has four branches. I can't typeset it.
    \medskip
    
    \item \tree[4]{%
        & \nnode{25}{1.}{$\neg\Box \exists x\, x\!=\!x$}{w}{(Ass.)} &\\
        & \nnode{25}{2.}{$wRv$}{}{(1)} &\\
        & \nnode{25}{3.}{$\neg \exists x\, x\!=\!x$}{v}{(1)} &\\
        & \bnode{25}{4.}{$a\!=\!a$}{v}{(Ex.)} &\\
        && \\
        \nnodeclosed{12}{9.}{$a\!\not=\!a$}{v}{(3)} &&  \nnodeclosed{12}{9.}{$a\!\not=\!a$}{v}{(3)}\\
    }
    \medskip

    \item \tree[4]{%
        & \nnode{25}{1.}{$\neg(\Diamond Fa \to \Diamond \exists x Fx)$}{w}{(Ass.)} &\\
        & \nnode{25}{2.}{$\Diamond Fa$}{w}{(1)} &\\
        & \nnode{25}{3.}{$\neg \Diamond \exists x\, Fx$}{w}{(1)} &\\
        & \nnode{25}{4.}{$wRv$}{}{(2)} &\\
        & \nnode{25}{5.}{$Fa$}{v}{(2)} &\\
        & \nnode{25}{6.}{$a\!=\!a$}{v}{(5)} &\\
        & \bnode{25}{7.}{$\neg \exists x\, Fx$}{v}{(3,4)} &\\
        && \\
        \nnodeclosed{12}{9.}{$a\!\not=\!a$}{v}{(3)} &&  \nnodeclosed{12}{10.}{$\neg Fa$}{v}{(3)}\\
    }
    \medskip

    \item \tree[4]{%
        & \nnode{38}{1.}{$\neg(a\!=\!b \to \Box(a\!=\!a \to a\!=\!b))$}{w}{(Ass.)} &\\
        & \nnode{38}{2.}{$a\!=\!b$}{w}{(1)} &\\
        & \nnode{38}{3.}{$\neg\Box(a\!=\!a \to a\!=\!b)$}{w}{(1)} &\\
        & \nnode{38}{4.}{$wRv$}{}{(3)} &\\
        & \nnode{38}{5.}{$\neg(a\!=\!a \to a\!=\!b)$}{v}{(3)} &\\
        & \nnode{38}{6.}{$a\!=\!a$}{v}{(5)} &\\
        & \nnode{38}{7.}{$\neg a\!=\!b$}{v}{(5)} &\\
        & \nnodeclosed{38}{8.}{$a\!=\!b$}{v}{(2,6)} &\\
    }
    \medskip
  \end{sollist}
\end{solution}

% What about the axiomatic approach?
%
% The \pr{UI} principle $\forall x A \to A[a/x]$ has become invalid. It is
% replaced by a weaker principle of ``free universal instantiation'', \pr{FUI}:
% 
% \principle{FUI}{\forall \chi A \to (\eta\!=\!\eta \to A[\eta/\chi])}
% 
% What is a complete axiomatisation?
%
% For soundness and completeness of typical systems on variable domains see HC
% 294--302. The completeness technique for constant domain QML generally fails
% because we don't have the BF which allows us to construct maximally consistent
% and witnessed sets.

It is easy to check that the Barcan Formula,
$\forall x \Box A \to \Box \forall x A$, and its converse,
$\Box \forall x A \to \forall x \Box A$, are invalid in variable-domain
semantics. (By this I mean that not all their instances are valid.) In fact, we
can now prove that the Barcan formula corresponds to the assumption that
whatever exists at an accessible world also exists at the original world, while
its converse corresponds to the assumption that whatever exists at a world also
exists at all accessible worlds.

\begin{observation}{barcancorr}
  \vspace{-1mm}
  \begin{enumerate}[leftmargin=10mm]
    \itemsep0mm
  \item[(i)] \pr{CBF} is valid on a variable-domain frame iff the frame has
    \emph{increasing domains}, meaning that whenever $wRv$, then
    $D_w \subseteq D_v$.
    
  \item[(ii)] \pr{BF} is valid on a variable-domain frame iff the frame has
    \emph{decreasing domains}, meaning that whenever $wRv$ then
    $D_v \subseteq D_w$.
  \end{enumerate}
  \vspace{-2mm}
\end{observation}
%
\begin{proof}
  \emph{Proof of (i).} Suppose some variable-domain frame $F$ does not have
  increasing domains. Then $F$ has a world $w$ whose domain $D_w$ contains an
  individual $d$ that does not exist at some $w$-accessible world $v$. Let $V$
  be an interpretation function on $F$ so that $V(F,w) = D_w$ and
  $V(F,v) = D_v$. In the model composed of $F$ and $V$, $\Box \forall x Fx$ is
  true at $w$, but $\forall x \Box Fx$ is false, since $d$ is not in $V(F,v)$.
  So \pr{CBF} is not true at all worlds in all models based on $F$.

  In the other direction, suppose \pr{CBF} is not valid on a frame $F$. This
  means that there is a world $w$ in some model $M$ based on $F$ at which some
  instance of $\Box \forall x A$ is true while $\forall x \Box A$ is false. If
  $\forall x \Box A$ is false at $w$, then there is some $w$-accessible world
  $v$ at which $A$ is false of some individual $d$ in $D_w$. But since
  $\Box \forall x A$ is true at $w$, $A$ is true of all members of $D_v$. So $d$
  is not in $D_v$. And so $F$ does not have increasing domains.

  The proof of (ii) is similar.\qed
  
\end{proof}

% In modal propositional logic, we saw that many interesting modal schemas
% corresponded to properties of the accessibility relation. In variable-domain
% semantics, interaction principles like \pr{BF} and \pr{CBF} correspond
% to frame conditions that link the accessibility relation with the domain of
% individuals.

% It is also easy to prove that (CBF) is valid in a VD frame iff (NE) is valid
% there. For (BF), the equivalent principle stating the necessity of
% non-existence cannot easily put in a single formula. But one can put it in
% terms of a schema: $\neg E!x \to \Box \neg E!x$, or contrapositively,
% $\Diamond E!x \to E!x$.

% In models with identity, the same can be proved for $\forall x \Box
% \exists y (y=x)$ in place of (CBF) and $\Diamond \exists x (x=y) \to
% \exists x (x=y)$ in place of (BF). This can be useful because unlike
% (BF) and (CBF) these are single formulas rather than schemas (see
% \cite[181f.]{fitting98first}).

% We've remarked that (CBF) is provable if we merge the axiomatic approach to
% first-order logic with K. $\Box (\forall x A \to A[c/x])$ is derivable from UI
% and Nec. If we want to hold on to classical rules, the most conservative
% variable domain systems are therefore the \emph{increasing domain systems}
% studied in \citey[ch.10]{hughes68introduction} (and \citey[15]{hughes96new} or
% e.g. \cite[44ff.]{gabbay76investigations}). (CBF) is valid in these systems,
% but not (BF). Recall that proving the Barcan Formula,
% $\forall x\Box A \to \Box\forall x A$, requires the Brouwerian axiom (B). So
% if we are willing to give up (B), we might consider relaxing the requirement
% of fixed domains. An \emph{expanding domain model} is a variable-domain model
% with the Increasing Domains requirement.

% \begin{exercise}
%   Suppose we interpret the domain of a constant domain model to
%   include all possible individuals (that is, everything that exists
%   relative to some possible world). It is then useful to introduce an
%   existence predicate $E$ that applies to an individual at a world
%   only if the individual exists at that world.
%   \begin{exlist}
%   \item Can you define an expression that functions like $E$ xxx
%   \item Show that it's free...?
%   \end{exlist}
% \end{exercise}

\begin{exercise}
  Definition \ref{def:variabledomainmodel} requires that every name in every
  model picks out a possible individual. In that sense, the definition does not
  allow for genuinely empty names. How could we change definitions
  \ref{def:variabledomainmodel} and \ref{def:variabledomainsemantics} if we
  wanted to allow for names that don't pick out anything?
\end{exercise}
\begin{solution}
  In the definition of a model, we could allow the interpretation function to be
  undefined for some names. We might also allow the sets $D_{w}$ to be empty. In
  the truth definition \ref{def:variabledomainsemantics}, we only need to
  clarify that $M,w,g \not\models A$ for every atomic sentence $A$ that contains a
  term $\tau$ for which $[\tau]^{M,g}$ is undefined.
\end{solution}

\section{Trans-world identity}\label{sec:twi}

In section \ref{sec:identity} I mentioned an apparent problem with Leibniz' Law.
The Law allows us to reason from $\Box Fa$ and $a\!=\!b$ to $\Box Fb$. On some
interpretations of the box, however, the inference looks problematic. In the
Superman stories, Lois Lane knows that Superman can fly, and Superman is
identical to Clark Kent. Can we infer that Lois knows that Clark Kent can fly?

If we can, we would have to conclude that Lois Lane has inconsistent beliefs,
since she also believes that Clark Kent \emph{cannot} fly. She would believe
that Clark Kent can't fly, but also that he can fly. Intuitively, however,
Lois's beliefs are perfectly consistent. What she lacks is information, not
logical acumen. Her belief worlds are not worlds at which someone can both fly
and not fly. Rather, they are worlds at which one person plays the Superman role
and a different person plays the Clark Kent role.

Consider also the case of Julius. When we introduce the name `Julius' for
whoever invented the zip, we can be sure that Julius invented the zip. But it
would be absurd to think that we have found out who invented the zip merely by
making a linguistic stipulation. If before introducing the name `Julius', we
were unsure whether the zip was invented by Benjamin Franklin or Whitcomb L.\
Judson, the introduction of the new name does nothing to remove our ignorance.
There are still epistemically accessible worlds at which the zip was invented by
Franklin and others at which it was invented by Judson. Knowing that Julius
invented the zip is not the same thing as knowing that Judson invented the zip,
even if in fact Julius = Judson.

Similar problems have been argued to arise in the logic of metaphysical
modality. Imagine a clay statue, standing on a shelf. Let's call it
Goliath. Since Goliath is made of clay, there is also a piece of clay on the
shelf, at the exact same spot as the statue. Let's call that piece of clay
Lumpl. How is Lumpl related to Goliath? We might want to say that they are one
and the same thing: Lumpl = Goliath. After all, there is only \emph{one}
statue-shaped object on the shelf, not two. But we might also want to say that
Lumpl could have had the shape of a bowl, while Goliath could not: if the clay
had been formed into a bowl rather than a statue, then Lumpl would have been a
bowl, but Goliath, the statue, would not have existed. Goliath is necessarily
not a bowl, but Lumpl is not necessarily not a bowl. We have $\Box \neg Bg$ but
not $\Box \neg Bl$, even though $l\!=\!g$.

\begin{exercise}
  Explain why the three examples I just presented also cast doubt on the
  ``necessity of identity'', $\forall x\forall y(x\!=\!y \to \Box\, x\!=\!y)$.
\end{exercise}
\begin{solution}
  In the Superman case, Clark Kent and Superman are the same person, but Lois
  Lane doesn't know that they are. So we appear to have $s\!=\!c$ but not
  $\Box\, s\!=\!c$. Similarly, in the Julius case, Julius and Whitcomb L.\ Judson
  are the same person, but one may well not know that they are. In the Goliath
  case, we have Lumpl = Goliath without it being metaphysically necessary that
  Lumpl = Goliath, as there are worlds in which Lumpl is a bowl and Goliath is
  not.
\end{solution}

Semantically, Leibniz' Law corresponds to the assumption that names are
\textbf{directly referential}, meaning that the only contribution a name makes
to the truth-value of a sentence is its referent. If names are directly
referential, and two names have the same referent, then it makes no difference
which of them we use: replacing one by the other never affects the truth-value
of a sentence.

So far, we have assumed direct reference in both constant-domain and
variable-domain semantics. On either account, names are interpreted as simply
picking out an individual. It is a matter of debate whether names in ordinary
language are directly referential. Some hold that Lois Lane really has
inconsistent beliefs. Others hold that Lois neither believes that Superman can
fly nor that Clark Kent cannot fly, because the objects of belief or knowledge
are never adequately represented by statements involving ordinary names. (This
also gets around the Julius problem.) With respect to Lumpl and Goliath, some
simply deny that Lumpl is identical to Goliath.

We will not descend into these debates. Instead, let's explore how we could
change our semantics for $\L_{M\!P}$ to block the relevant applications of
Leibniz' Law. There are several ways to achieve this. We will only look at one.

The approach we will explore drops the assumption that names are rigid. A name
is \textbf{rigid} if it picks out the same individual relative to any possible
world. Earlier, we assumed that no matter at which world the sentence $Fa$ is
interpreted, the name $a$ always picks out the same individual, $V(a)$. A name
like `Julius', however, seems to be non-rigid. It picks out different
individuals relative to different (epistemically) possible worlds. Relative to a
world where Benjamin Franklin invented the zip, `Julius' picks out Benjamin
Franklin. Relative to a world where Whitcomb L.\ Judson invented the zip, the
name picks out Whitcomb L.\ Judson.

Let's assume, then, that a model's interpretation function assigns an individual
to each name \emph{relative to each world}. This is equivalent to assuming that
each name is interpreted as expressing a \emph{function from worlds to
  individuals}, telling us which individual the name picks out relative to any
given world. Functions from worlds to individuals are known as
\textbf{individual concepts}, which is why the present approach is often called
\textbf{individual concept semantics}.

To motivate this label, return to Lois Lane. When Lois is thinking about
Superman, she is thinking about the audacious hero whose superhuman powers she
has witnessed on several occasions. When she is thinking about Clark Kent, she
is thinking about her shy and awkward colleague. Lois has distinct ``concepts''
for Superman and Clark Kent, one associated with the Superman role, the other
with the Clark Kent role. The two concepts actually pick out the same person
because one and the same person plays both the Superman role and the Clark Kent
role. We can model each of these roles as a function from worlds to individuals.
The Superman role is represented by a function that maps every world to whoever
plays the Superman role at that world. The Clark Kent role is represented by a
function that maps every world to whoever plays the Clark Kent role at that
world. For the world of the Superman stories, both functions return the same
individual. For Lois Lane's belief worlds, they return different individuals.

\begin{exercise}
  What individual concepts might be associated with the names `Lumpl' and
  `Goliath'?
\end{exercise}
\begin{solution}
  `Lumpl' might express a concept that maps every world $w$ to a certain piece
  of clay at $w$, where that piece is perhaps individuated by its matter or
  origin. The piece's shape doesn't matter. `Goliath' might instead express a concept that maps every world $w$ to a certain statue at $w$, where the statue is perhaps individuated by its shape and origin.
\end{solution}

We can easily convert our earlier constant-domain and variable-domain semantics
into an individual concept semantics. We first need  to change the definition of a
model, so that $V$ assigns individual concepts to names. In variable-domain
semantics, we might stipulate that an individual concept never maps a world to
an individual that doesn't exist at the world. We might also want to allow for
``partial concepts'': individual concepts that don't return any value for
certain worlds.

It is advisable to give a parallel treatment for names and variables. So we'll
also assume that an assignment function $g$ interprets each variable as
expressing an individual concept. In the truth definition, we replace
$[\tau]^{M,g}$, by $[\tau]^{M,w,g}$, which is defined as the referent of $\tau$
in $M$ \emph{at $w$}, relative to $g$. (That is, if $\tau$ is a name, then
$[\tau]^{M,w,g} = V(\tau)(w)$; if $\tau$ is a variable, then
$[\tau]^{M,w,g} = g(\tau)(w)$.) Finally, we adjust the definition of an
$x$-variant so that $g'$ is an $x$-variant of $g$ iff $g'$ differs from $g$ at
most in the individual concept it assigns to $x$.

The resulting logic of individual concepts has some unexpected features. For
example, all instances of the following schema become valid:
\[
  \Box \exists x A \to \exists x \Box A
\]
To see why, consider the instance $\Box \exists x Fx \to \exists x \Box Fx$.
Suppose the antecedent is true at some world in some model. This means that at
every accessible world $v$, there is at least one individual that is $F$. In
this case, there are functions that map every accessible world to some
individual that is $F$. Let $g'(x)$ be some such function. Relative to $g'$,
$\Box Fx$ is true at $w$. So $\exists x \Box Fx$ is true at $w$.

This is widely regarded as problematic. It would suggest that the two readings
of `something necessarily exists' are actually equivalent: it is necessary that
something or other exists just in case there is something that necessarily
exists.

% Also conceptually unsatisfactory that names effectively pick out trans-world
% individuals; ``intensional objects''. Note that when we're interpreting
% predicates, variables contribute regular "extensional" objects, members of the
% domain: g(x)(w) = I(w) = o. But quantifiers range over intensional objects.
% That seems odd. If we allow for intensional predicates (as one might think we
% should), we have gained nothing.

Another problematic feature of individual concept semantics is that the
resulting logic has no sound and complete proof procedure. There are no tree
rules, or natural deduction rules, or axioms and inference rules that would
allow proving all and only the sentences that are true at all worlds in all
models of individual concept semantics (no matter if we assume constant or
variable domains). It's not just that no-one has yet found a suitable proof
method. One can prove that no such method exists.

Both of these problems can be avoided by putting further constraints on models.
We have assumed that any function from worlds to individuals is a candidate
interpretation for a name or a variable. Relative to a given assignment
function, a variable may pick out Donald Trump in one world, the Eiffel tower in
another, a fried egg in a third, and so on. Ordinary concepts are not that
gerrymandered. We might therefore identify a certain subset of all individual
concepts as ``eligible'' for being expressed by names or variables. If this is
done sensibly, $\Box \exists x A \to \exists x \Box A$ becomes invalid, and
complete proof methods become available.

\begin{exercise}
  The following line of thought may be attributed to Descartes. ``I am certain
  that I exist, but not that my body exists. [After all, it could turn out that
  I am a disembodied soul.] Therefore: I am not my body.'' Translate the
  argument into $\L_{M\!P}$. Is it CK-valid? Is it VK-valid? Do you find it
  convincing?
\end{exercise}
\begin{solution}
  The premises are $\Box \exists x\, x\!=\!i$ and
  $\neg\Box \exists x\, x\!=\!b$. The conclusion is $i\!\not=\!b$. The argument
  is CK-valid and VK-valid. (It is not valid in individual concept semantics.)
\end{solution}

\begin{exercise}
  The following sentence sounds contradictory.
  \begin{quote}
    Some ticket will win, but I don't know if it will win.
  \end{quote}
  Translate the sentence into $\L_{M\!P}$. Explain why its apparent
  contradictoriness poses a problem for accounts on which variables are treated
  as directly referential.
\end{exercise}
\begin{solution}
  Translation: $\exists x (Tx \land Wx \land \neg K Wx \land \neg K \neg Wx)$,
  where $T$ translates `-- is a ticket' and `-- will win'.

  If variables are directly referential, then this sentence is true in any
  scenario in which I don't know which ticket will win.
\end{solution}

\begin{exercise}
  In individual concept semantics, both the necessity of identity and the
  necessity of distinctness are invalid. How could we change the semantics to
  make the necessity of identity valid, but not the necessity of distinctness?
  (Assume constant domains.)
\end{exercise}
\begin{solution}
  To render $\forall x \forall y (x\!=\!y \to \Box x\!=\!y)$ valid, we can
  restrict the eligible individual concepts in a model as follows. For any
  individual concepts $f$ and $g$ and worlds $w$ and $v$, if $wRv$ and
  $f(w)=g(w)$ then $f(v)=g(w)$. (We do not stipulate that if $wRv$ and
  $f(v)=g(v)$ then $f(w)=g(w)$, which would render the necessity of distinctness
  valid.)
\end{solution}

\iffalse

An alternative to individual concept semantics is \textbf{counterpart
  semantics}. Here the guiding idea is that when a name occurs in the scope of a
modal operator, then at the relevant accessible worlds it picks out whichever
individual resembles its actual referent in certain respects. Consider Lois
Lane's belief worlds -- the worlds in which things are as Lois believes they
are. In these worlds, Lois has a shy colleague called `Clark Kent' who can't
fly; there is also a superhero called `Superman' who can fly; the two are
different people. The shy colleague in Lois's belief worlds resembles the actual
Clark Kent (by which I mean the Clark Kent of the Superman stories) in certain
respects, which is why he is picked out by the name `Clark Kent' when that name
is interpreted relative to Lois's belief worlds. The superhero in Lois's belief
worlds resembles the actual Clark Kent in other respects, respects that are
associated with the name `Superman'. The difference between the two names is
that they invoke different criteria for trans-world resemblance.

% If we think of the object on the shelf as Lumpl, and we consider a world where a
% copper hat has been added, we judge that Lumpl is only the clay part of the
% resulting statue; if we think of the same object on the shelf as Goliath, we
% judge that it includes the copper hat in the counterfactual scenario. In
% counterpart semantics, different ways of tracking objects across worlds are
% represented by different \emph{counterpart relations}. The object on the shelf
% at our world stands in different counterpart relations to different objects at
% the world where the copper hat has been added.

If an individual at some world $v$ sufficiently resembles an individual at $w$
in relevant respects, then the first individual is called a \textbf{counterpart}
of the second (relative to $v$ and $w$, and relative to the given resemblance
criteria). The shy colleague in each of Lois Lane's belief worlds is a
counterpart of Clark Kent relative to some resemblance criteria; the superhero
in her belief worlds is a counterpart of Clark Kent relative to other criteria.

In counterpart semantics, we now assume that names are associated with an
individual, but also with a counterpart relation which determines how the name
should be interpreted when modal operators shift the point of evaluation to
other worlds. $\Box Fa$ is true at a world $w$ iff at every accessible world
$v$, every individual that stands in the $a$-relevant counterpart relation to
$a$ at $w$ is $F$.

Counterpart relations play a similar role as individual concept, but they are
more liberal. For example, we can allow for cases in which an individual has
multiple counterparts at an accessible world, relative to the same counterpart
relation. Suppose Bob lives next door to Alice, and has sometimes met her on the
stairwell. For some reason, Bob believes that Alice's flat is inhabited by
identical twins, so when he sees Alice, he thinks he sees one of the
twins. Alice has two counterparts in Bob's belief worlds, but the two
counterparts don't correspond to different roles or different similarity
standards.

Counterpart relations also needn't be transitive: a counterpart of a counterpart
of an individual need not itself be a counterpart of the individual. This might
help with the following puzzle about metaphysical modality.

Intuitively, my bicycle could have been composed of somewhat different parts. If
you exchange the lights or the seatposts on a bike, you aren't destroying the
old bike and creating a new one. So my bike could have had different lights, or
a different seatpost. On the other hand, arguably my bike could not have been
composed of \emph{entirely} different parts. A bike composed of entirely
different parts would be a different bike. Now let $b$ denote my actual bike;
let $F_1$ be a predicate that gives a detailed description of my bike as it
actually is. Let $F_2$ give a detailed description of a bike that is just like
mine except for the seatpost. We want to say that my bike could have fit that
description. So $\Diamond F_2b$ should be true. But as we make more and more
changes to $F_1$, we reach a point -- say $F_{32}$ -- where the description could
no longer have applied to my bike. So $\Diamond F_{32}b$ is false. But now
consider what would have been the case if $F_{31}b$ had been true. In that case,
it seems that $F_{32}b$ could have been true. After all, an $F_{32}$ bike differs
from an $F_{31}$ bike only by a single part. It would be strange if the bike in a
world where $F_{31}b$ is true could not possibly have had any different parts. So
while $\Diamond F_{32}b$ is false at the actual world, it seems to be true at any
world where $F_{31}b$ is true. Since $\Diamond F_{31}b$ is true at the
actual world, it follows that $\Diamond\Diamond F_{32}b$ is true as well. So we
have $\Diamond\Diamond F_{32}b$ but not $\Diamond F_{32}b$.

This is puzzling, because metaphysical possibility is often assumed to be an
absolute kind of possibility, with a universal accessibility relation. One would
then expect the logic of metaphysical possibility to be S5. Yet in S5,
$\Diamond\Diamond A$ entails $\Diamond A$.

In counterpart semantics, modal schemas like
$\Diamond \Diamond A \to \Diamond A$ correspond not just to properties of the
accessibility relation but to combined properties of the accessibility relation
and the counterpart relations. So one can explain what's going on in the puzzle
of the bike without giving up the assumption that the accessibility relation for
metaphysical modality is universal. $\Diamond \Diamond A \to \Diamond A$ is
invalid because a counterpart of a counterpart of my bike need not be a
counterpart of my bike.

\fi


%%% Local Variables: 
%%% mode: latex
%%% TeX-master: "logic2.tex"
%%% End:
