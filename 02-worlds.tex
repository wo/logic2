\chapter{Possible Worlds}\label{ch:worlds}

\section{The possible-worlds analysis of possibility and necessity}

An important breakthrough in the history of modal logic was the development of
``possible-worlds semantics'' in the 1940s-60s.
%
% The discovery was made by a number of authors, including Carnap 1946, 1947,
% Kanger 1957, Hintikka 1957, 1961, Bayart 1958, 1959, Montague 1960, Kripke
% 1959, 1963a, 1963b.
%
The core idea of possible-worlds semantics is to analyze modal notions in
terms of truth at possible worlds. In its simplest form, the analysis goes like
this:
%
\begin{quote}
  A proposition is possible iff it is true at some possible world.\\
  A proposition is necessary iff it is true at all possible worlds.
\end{quote}

In philosophy jargon, a \textbf{possible world} is a maximally specific
possibility. An example of a possible world is the \textbf{actual world} -- the
totality of everything that is the case. In the actual world, light travels
faster than sound and the Conservatives are in government. In other possible
worlds, sound travels faster than light and Labour is in government.

The possible-worlds analysis translates modal statements into quantificational
statements about possible worlds. You may feel uneasy about this. Talking about
merely possible worlds may strike you as fanciful and unscientific. Besides, you
may wonder if anything is really gained by the translation, since we now face
the question what sorts of worlds should be classified as ``possible''.

Remember that there are different flavours of modality. A proposition might be
epistemically possible, historically possible, metaphysically possible, and so
on. If we want to analyse all these kinds of possibility in terms of possible
worlds, we need different flavours of worlds. There must be epistemically
possible worlds, historically possible worlds, metaphysically possible worlds,
etc. And if we ask how these types of worlds are defined it looks like we have
to turn back to relevant features of the actual world. The ultimate reason why
you can't go from Auckland to Sydney by train is surely that there is no
suitable train line here in our world, not that you don't make the journey in
some non-actual worlds.

These objections cast doubt on the possible-worlds analysis as a piece of
reductive metaphysics. But the metaphysics of modality is not our topic. When we
use the possible-world analysis, we don't assume that the translation in terms
of possible worlds reveals the metaphysical grounds of the original modal
statements. We merely assume that the original statements can be paraphrased in
the fanciful language of possible worlds.

% The reductivist use of possible-worlds semantics is often associated with
% Lewis 1986, who does indeed give a non-modal analysis of `possible world'.

% If we take seriously the idea that a possible world is a complete way things
% might have been, then the hypothesis that something might have been the case
% iff it is the case at some possible world does not amount to much more than
% the hypothesis that any incomplete way things might have been (like, me having
% coffee with breakfast) can be extended to a complete way things might have
% been: If $p$ might have been the case, then there is a way the entire world
% might have been that would have included $p$.

% In practice, we don't even take the completeness of possible worlds all that
% seriously. We will often work with toy worlds that merely settle all questions
% in which we're currently interested, leaving lots of other questions open.

% When we go beyond modality in the application of modal logic, the ``possible
% worlds'' will look even less like genuine worlds. In temporal logic, the role of
% worlds is played by times; in the logic of relativistic spacetime, it is played
% by spacetime point, and in so-called dynamic logics by states of a computer
% program.

For a first glimpse of why this might be useful, consider the following
hypothesis.
%
\begin{equation*}
  \Box\Diamond\Box p \models \Box p
\end{equation*}
% 
Is this true? If something is necessarily possibly necessary, does it follow
that it is necessary? Hard to say. We know that $A$ logically entails $B$ iff
there is no conceivable scenario in which $A$ is true and $B$ false, under any
interpretation of the non-logical expressions. The problem is that it is not
obvious what a scenario would have to look like for $\Box \Diamond \Box p$ to be
true, under a given interpretation of $p$.

The possible-worlds analysis can help clear things up. By the possible-worlds
analysis, $\Box \Diamond \Box p$ says that $\Diamond \Box p$ is true at every
possible world. The hypothesis that $\Box \Diamond \Box p$ is true in a scenario
therefore reduces to the hypothesis that $\Diamond \Box p$ is true at every
world in the scenario. $\Diamond\Box p$ says that $\Box p$ is true at some
world. So if $\Diamond\Box p$ is true at every world in a scenario then $\Box p$
is true at some world in the scenario. And if $\Box p$ is true at some world in
a scenario then $p$ is true at every world in the scenario. This is just what
$\Box p$ says. So whenever $\Box \Diamond\Box p$ is true in a scenario (under
some interpretation of $p$), then $\Box p$ is true in that scenario (under that
interpretation). We've shown that $\Box\Diamond\Box p$ entails $\Box p$.

\begin{exercise}
  Explain, in the same informal manner, why $\Diamond p$ does not entail
  $\Box p$, assuming the possible-worlds analysis of the box and the diamond.
\end{exercise}
\begin{solution}
  Consider a scenario in which (say) it is raining at some worlds and not
  raining at others. Let $p$ express that it is raining. In this scenario, under
  this interpretation, $\Diamond p$ is true, because $p$ is true at some world.
  But $\Box p$ is false, because $p$ is not true at all worlds. So there are conceivable scenarios and interpretations that render $\Diamond p$ true and $\Box p$ false.
\end{solution}

% \begin{exercise}
%   By the possible-worlds analysis, every modal statement about
%   necessity and possibility can be translated into a non-modal
%   statement that quantifies over possible worlds. The translation also
%   works backwards: many statements that quantify over possible worlds
%   can be translated into modal statements with boxes and diamonds. But
%   not all. Can you find a statement that quantifies over possible
%   worlds but cannot be translated into the language of boxes and
%   diamonds?
% \end{exercise}

\section{Models}
\label{sec:basicmodels}

In section \ref{sec:turnstile}, I defined validity and entailment in terms of
scenarios and interpretations. A sentence is valid, I said, iff it is true in
every conceivable scenario under every interpretation of the non-logical
expressions. This is a little vague. What, exactly, is a conceivable scenario,
and what counts as a relevant interpretation? Also, scenarios and
interpretations are unwieldy objects. It is difficult to give a full description
of a scenario and an interpretation. Fortunately, most of the details are
irrelevant if all we care about is which $\L_{M}$-sentences are true and which
are false in a scenario under a particular interpretation. This observation will
lead us to a more precise definition of validity and entailment.

Suppose I tell you the following about a scenario $S$ and an interpretation $I$
of the sentence letters.

\begin{quote}
  There are three worlds in $S$, $w_{1}, w_{2}$, and $w_{3}$. Under the
  interpretation $I$, the sentence $p$ expresses a proposition that is true at
  $w_{1}$, false at $w_{2}$, and true at $w_{3}$. All other sentence letters
  express propositions that are false at all three worlds.
\end{quote}

This tells you almost nothing about what the scenario looks like. You don't know
if $w_{1}$ is a world at which it is currently raining. You don't know who is in
government at $w_{2}$. You also don't know what the sentence letters mean under
my interpretation. Does $p$ mean that it is raining? That Labour is in
government? I haven't told you. Yet the sparse information I have given is
enough to determine the truth-value of every $\L_{M}$-sentence at every world.

\begin{exercise}
  Which of the following sentences are true at $w_{1}$ in my scenario $S$ under
  my interpretation $I$?
  \begin{exlist}
  \item $\neg p$ % false
  \item $\neg p \to \Box p$ % true
  \item $\Box p$ % false
  \item $\Diamond\Box p$ % false
  \item $\Diamond \Diamond p \lor \Diamond \Box p$ % true
  \item $\Box (\Box p \to p)$ % true
  \end{exlist}
\end{exercise}
\begin{solution}
  (b), (e), and (f) are true at $w_{1}$, the others false.
\end{solution}

A joint representation of a scenario and an interpretation (of non-logical
expressions) that contains just enough information to determine the truth-value
of every sentence is called a \textbf{model}. Just as a model airplane often
leaves out important aspects of a real airplane -- the motor, the seats, etc. --
models in logic leave out many important aspects of the scenarios and
interpretations they represent.

Adopting the simple possible-worlds analysis of the box and the diamond, we can
define a model for $\L_{M}$ as consisting of two parts. First, a model must
specify a set of things we call ``worlds''. They don't need to be genuine
worlds. They can be arbitrary (usually not further specified) objects whose job
is to represent genuine worlds. Second, a model must specify an ``interpretation
function'' that tells us for each sentence letter at which of the worlds it is
true.

\begin{definition}{}{basicmodel}
  A \textbf{basic model} of $\L_M$ is a pair $\t{W,V}$ of%
  \vspace{-3mm}
  \begin{itemize*}
  \item a non-empty set $W$, and
  \item a function $V$ that assigns to each sentence letter of $\L_M$
  a subset of $W$.
  \end{itemize*}
\end{definition}
\noindent
In the next chapter, we will replace this definition by a slightly more
complicated definition. That's why I've called models of the present kind
`basic'.

You should be familiar with elementary concepts of set theory. A \emph{set} is
a collection of objects, called the \emph{members} or \emph{elements} of the
set. Sets can be defined by listing their members enclosed in curly braces:
`$\{ a, b, c \}$'. The \emph{empty set}, with no members, is denoted by
`$\emptyset$'. A \emph{subset} of a set $X$ is a set all whose members are
members of $X$. A \emph{function} is a mapping -- a kind of abstract machine
that takes objects of a certain kind as input and outputs objects of a possibly
different kind.

The interpretation function $V$ in a model maps each sentence letter to the set
of worlds at which the sentence is true. For example, if $W$ contains three
worlds $w_{1}, w_{2}$, and $w_{3}$, and $V(p) = \{ w_{1}, w_{3} \}$ -- meaning
that $V$ maps $p$ to the set $\{ w_{1}, w_{3} \}$ --, then $p$ is true at
$w_{1}$ and $w_{3}$ but not at $w_{2}$.

Notice that an interpretation function only specifies at which worlds the
\emph{sentence letters} are true. $V$ is defined for $p$, $q$, and $r$, but not
for $p \to q$ or $\Box p$ or $\Diamond \Box q$. This is the key idea behind the
possible-worlds analysis. Once we know at which worlds each sentence letter is
true, we have all we need to determine the truth-value of every sentence at
every world.

To formally define how the truth-value of complex sentences is determined, I
will use (meta-linguistic) statements of the form
\[
  M,w \models A
\]
%
as shorthand for
%
\[
  \text{$A$ is true at world $w$ in model $M$}.
\]
I use `$M,w \not\models A$' for the negation of `$M,w \models A$'.

Yes, it's the same turnstile that we use for entailment and validity. This
should cause no confusion because it is usually clear if the things to the left
of the turnstile are $\L_M$-sentences or meta-linguistic expressions for a model
and a world. (In its present use, the turnstile is often pronounced `makes true'
or `satisfies'.)

The relation $\models$ between a model, a world and an $\L_M$-sentence is
defined as follows.

\begin{definition}{Basic Possible-Worlds Semantics}{basicsemantics}
  If $M = \t{W,V}$ is a basic model, $w$ is a member of $W$, $P$ is
  any sentence letter, and $A,B$ are any $\L_M$-sentences, then

  \medskip
  \begin{tabular}{lll}
    (a) & $M,w \models P$ &iff $w$ is in $V(P)$.\\
    (b) & $M,w \models \neg A$ &iff $M,w \not\models A$.\\
    (c) & $M,w \models A \land B$ &iff $M,w \models A$ and $M,w \models B$.\\
    (d) & $M,w \models A \lor B$ &iff $M,w \models A$ or $M,w \models B$.\\
    (e) & $M,w \models A \to B$ &iff $M,w \not\models A$ or $M,w \models B$.\\
    (f) & $M,w \models A \leftrightarrow B$ &iff $M,w \models A\to B$ and $M,w \models B\to A$.\\
    (g) & $M,w \models \Box A$ &iff $M,v \models A$ for all $v$ in $W$.\\
    (h) & $M,w \models \Diamond A$ &iff $M,v \models A$ for some $v$ in $W$.
  \end{tabular}
\end{definition}

Let's go through the clauses in this definition.

Clause (a) says that a sentence letter is true at a world in a model iff the
world is an element of the set of worlds which the model's interpretation
function assigns to the sentence letter. This is just what I explained above.

Clause (b) says that the negation $\neg A$ of an $\L_M$-sentence $A$ is true at
a world in a model iff $A$ is not true at that world in that model. In other
words, the truth-table for negation applies locally at every world: at any
world, $\neg A$ is true iff $A$ is not true.
Clauses (c)--(f) similarly tell us that the truth-tables for the other
truth-functional connectives apply locally at each world.

Clauses (g) and (h) spell out the possible-worlds analysis of the box and the
diamond. According to (g), a sentence $\Box A$ is true at a world in a model iff
$A$ is true at all worlds in the model. According to (h), $\Diamond A$ is true
at a world in a model iff $A$ is true at some world in the same model.

The whole definition is called a \emph{semantics} because a semantics for a
language is an account of what the expressions in the language mean, and
definition \ref{def:basicsemantics} can be seen as giving the meaning of the
logical expressions in $\L_M$. (The non-logical expressions in $\L_{M}$ don't
have a fixed meaning.)

Since every $\L_M$-sentence is built up from sentence letters with the operators
covered in definition \ref{def:basicsemantics}, the definition settles the
truth-value of every sentence at every world in every model.

Consider, for example, the following model $M$:
%
\begin{gather*}
  W = \{ w_{1},w_{2} \}\\
  V(p) = \{ w_{1},w_{2} \}\\
  V(q) = \{ w_{1} \}\\
  V(P) = \emptyset \text{ for all other sentence letters $P$ }
\end{gather*}
This model contains only two worlds, $w_{1}$ and $w_{2}$. The interpretation
function $V$ says that $p$ is true at both worlds, $q$ is true at $w_{1}$,
and all other sentence letters are true nowhere. With the help of definition
\ref{def:basicsemantics}, we can figure out at which of the two worlds, say,
$\Box\Diamond(\Box q \to \Diamond\Box p)$ is true. We start with the smallest
parts of the sentence.

\begin{enumerate*}
  \item $p$ is true at $w_{1}$ and $w_{2}$ (by clause (a) of definition
  \ref{def:basicsemantics}).
  \item $q$ is true at $w_{1}$ and not true at $w_{2}$ (by clause (a) of definition
  \ref{def:basicsemantics}).
  \item $\Box p$ is true at $w_{1}$ and $w_{2}$ (by 1 and clause (g) of definition
  \ref{def:basicsemantics}).
  \item $\Box q$ is true at no world (by 2 and clause (g) of definition
  \ref{def:basicsemantics}).
  \item $\Diamond\Box p$ is true at $w_{1}$ and $w_{2}$ (by 3 and clause (h) of
  definition \ref{def:basicsemantics}).
  \item $(\Box q \to \Diamond\Box p)$ is true at $w_{1}$ and $w_{2}$ (by 4, 5,
  and clause (e) of definition \ref{def:basicsemantics}).
  \item $\Diamond(\Box q \to \Diamond\Box p)$ is true at $w_{1}$ and $w_{2}$ (by
  6 and clause (h) of definition \ref{def:basicsemantics}).
  \item $\Box\Diamond(\Box q \to \Diamond\Box q)$ is true at $w_{1}$ and $w_{2}$ (by 7
  and clause (g) of definition \ref{def:basicsemantics}).
\end{enumerate*}

\begin{exercise}
  At which worlds in the model just described is
  $\Diamond p \to (q \lor \Diamond\Box p)$ true?
\end{exercise}
\begin{solution}
  $\Diamond p \to (q \lor \Diamond\Box p)$ is true at both worlds.
\end{solution}

% I should perhaps clarify that models and worlds often represent the same kind
% of things: conceivable scenarios. But they represent different parts of them.
% E.g., a world represents only non-modal aspects of a scenario, whereas a model
% also represents what's accessible.

% In the early days of possible-worlds semantics, philosophers thought that a
% possible world is more or less the same thing as a conceivable scenario, and
% they often identified the space of possible worlds with the class of all
% models of predicate logic. However, it has proved useful to treat the space of
% possible worlds as a non-logical matter, so that different models may involve
% different possible worlds, just as different models of predicate logic may
% involve different sets of individuals.

\section{Basic entailment and validity}%
\label{sec:redefining}

Using the concept of a model, we can sharpen the hand-wavy definitions of
entailment and validity from section \ref{sec:turnstile}.

Imagine a list of all conceivable scenarios and all possible interpretations of
the sentence letters. By definition \ref{def:valid-informal}, a sentence is
valid iff it is true in all of these scenarios under each of these
interpretations. Every combination of a scenario $S$ and an interpretation $I$
is represented by a model. The model contains enough information to figure out
whether any given sentence is true or false in $S$ under $I$. Assuming that,
conversely, every model represents some combination of a scenario and an
interpretation, it follows that a sentence is valid iff it is true in every
model. In the same way, some sentences $\Gamma$ entail a sentence $A$ iff $A$ is
true in every model in which all members of $\Gamma$ are true.

That's the idea. There is, however, a small problem. Take a model with two
worlds, $W = \{ w_{1}, w_{2} \}$, and assume that $V(p) = \{ w_{1} \}$. Is $p$
true in this model? We can't say. Definition \ref{def:basicsemantics} only
specifies under what conditions a sentence is true \emph{at a world in a model}.
We have not defined what it means for a sentence to be true in a model. So we
can't say that a sentence is valid iff it is true in all models.

There are two ways to fix this. The conceptually cleaner response is to change
the definition of a model. Intuitively, the worlds in a scenario are not all on
a par. Think of a scenario in which it is raining although it might have been
snowing. This scenario has worlds at which it is raining and others at which it
is snowing. One of these worlds -- a rain world -- is special: it represents the
actual world in the scenario. `It is raining' is true in the scenario because it
is raining in the actual world of the scenario. Following this line of thought,
we could define a model to consist of \emph{three} elements: a set of worlds
$W$, an interpretation function $V$, and a ``designated element of $W$'' that
indicates which world in $W$ represents the actual world of the scenario. We
could then say that a sentence is \emph{true in a model} iff it is true at the
actual world of the model. Models of this type -- with a designated element of
$W$ -- are called \emph{pointed models}.

We will adopt the more popular second response. Here we change the definition of
entailment and validity. Instead of saying that a sentence is valid iff it is
true in every model, we say that a sentence is valid iff it is true \emph{at
  every world in every model}. Similarly, we say that some sentences $\Gamma$
entail a sentence $A$ iff $A$ is true at every world in every model at which all
members of $\Gamma$ are true.

The two responses amount to the same thing. Since every world in every basic
(un-pointed) model could be chosen as the designated world, a sentence is true at all worlds in all basic models iff
it is true in all pointed models. The response we adopt has the minor advantage
of keeping models slightly simpler, and logicians want their models to be as
simple as possible.

\begin{definition}{}{valid}
  A sentence $A$ is \textbf{valid} (for short: $\models A$) iff
  it is true at every world in every basic model.
\end{definition}
%
\begin{definition}{}{basicconsequence}
  Some sentences $\Gamma$ \textbf{(logically) entail} a sentence $A$ (for short:
  $\Gamma \models A$) iff there is no world in any basic model at which all sentences in $\Gamma$ are true while $A$ is false.
\end{definition}

\begin{exercise}
  Call a sentence true \emph{throughout} a model iff it is true at every world
  in the model. What do you think of the following definition?
  `$\Gamma \models A$ iff there is no model throughout which all sentences in 
  $\Gamma$ are true and throughout which $A$ is false.' Is this equivalent to
  definition \ref{def:basicconsequence}? (Hint: consider the hypothesis that
  $p \models \Box p$.)
\end{exercise}
\begin{solution}
  % The corresponding notion of validity is equivalent to ours, so the question
  % is whether Observation 1.1 holds. |= A->B says that there are no worlds in
  % any model at which A,~B are true. A |= B allows for such worlds, as long as
  % any model in which A is true at /all/ worlds is also a model in which B is
  % true at all worlds. 
  The two definitions are not equivalent, as can be seen from the fact that the
  definition proposed in the exercise would render $p \models \Box p$ true.
  Whenever $p$ is true at every world in a model then (by definition
  \ref{def:basicsemantics}) $\Box p$ is also true at every world in the model.
  Definition \ref{def:basicconsequence} renders $p \models \Box p$ false, since
  there are models in which $p$ is true at some worlds and not at others.
\end{solution}

Above I mentioned an assumption implicit in our new definitions: that every
model represents a pair of a conceivable scenario and interpretation. This isn't
obvious. For example, if our topic is metaphysical possibility and necessity, it
may be hard to conceive of a scenario with exactly two possible worlds. Is it
really conceivable that there are only two ways a world might have been,
compatible with the nature of things? We could stipulate that a model, at least
for this application, must contain at least (say) a million worlds, or
infinitely many. It turns out, however, that this would make no difference to
the logic. The very same sentences are valid whether we impose the restriction
or not. So we'll allow for models with very few worlds. Such models are often
useful as toy models to illustrate facts about entailment and validity.

\section{Explorations in S5}%
\label{sec:basiclogic}

By definition \ref{def:valid}, a sentence is valid iff it is true at all worlds
in all (basic) models. Definition \ref{def:basicmodel} explains what a (basic)
model is; definition \ref{def:basicsemantics} specifies the truth-value of any
sentence at any world in any model. Together, these definitions settle which
sentences are valid.

Take, for instance, $\Box p \to p$. This is valid on our definitions. To see
why, let $w$ be an arbitrary world in an arbitrary model $M$. Either $p$ is true
at $w$ or not. If $p$ is true at $w$, then by clause (e) of definition
\ref{def:basicsemantics}, $\Box p \to p$ is also true at $w$. If $p$ is not true
at $w$, then by clause (g) of definition \ref{def:basicsemantics}, $\Box p$ is
not true at $w$ in $M$, and then $\Box p \to p$ is true at $w$ by clause
(e). Either way, $\Box p \to p$ is true at $w$. Since $w$ and $M$ were chosen
arbitrarily, this shows that $\Box p \to p$ is true at every
world in every model.

(In the previous chapter, I mentioned that for some applications of modal logic,
we don't want $\Box p \to p$ to be valid. In the next chapter, we will see how
this can be achieved, by adding a slight tweak to the definitions of the present
chapter.)

How about, say, $\Box p \to \Box \Box p$? If something is necessary, is it
necessarily necessary? Our semantics says yes. Let $w$ be an arbitrary world in
an arbitrary model. If $\Box p$ is false at $w$, then $\Box p \to \Box\Box p$ is
true at $w$, by clause (e) of definition \ref{def:basicsemantics}. Suppose then
that $\Box p$ is true at $w$. In that case, $p$ is true at all worlds, by clause
(g) of definition \ref{def:basicsemantics}. And then $\Box p$ is true at all
worlds, again by clause (g). And so $\Box\Box p$ is also true at all worlds, by
clause (g). So whenever $\Box p$ is true at a world in a model, then so is
$\Box\Box p$. By clause (e) of definition \ref{def:basicsemantics}, it follows
that $\Box p \to \Box\Box p$ is true at every world in every model.

\begin{exercise}
  Show that $\Box p \to \Diamond p$ is valid. 
\end{exercise}
\begin{solution}
  By definition \ref{def:valid}, a sentence is valid iff it is true at every
  world in every model. Suppose for reductio that $\Box p \to \Diamond p$ is
  false at some world $w$ in some model. By definition \ref{def:basicsemantics},
  $\Box p$ is then true at $w$ and $\Diamond p$ false. But if $\Diamond p$ is
  false at $w$ then (by definition \ref{def:basicsemantics}) $p$ is false at
  every world in the model. And then $\Box p$ isn't true at $w$ (by definition
  \ref{def:basicsemantics}). Contradiction.
\end{solution}

There is a shorter way to show that $\Box p \to \Box\Box p$ is valid. Definition
\ref{def:basicsemantics} entails that if a sentence starts with a modal
operator, then its truth-value never varies from world to world. For example, if
$\Diamond p$ is true at some world $w$ in some model, then $\Diamond p$ is true
at all worlds in the model. It follows that if a sentence starts with a modal
operator, then its truth-value doesn't change if you stack further modal
operators in front. If $\Diamond p$ is true at a world in a model, then so
are $\Box\Diamond p$ and $\Diamond \Diamond p$.

This means that any sentence that begins with a sequence of modal operators is
equivalent to the same sentence with all but the last operator removed.
$\Diamond\Box\Box\Diamond\Diamond p$ is equivalent to $\Diamond p$. $\Box\Box p$
is equivalent to $\Box p$. Since replacing logically equivalent sentences inside
a larger sentence never affects the larger sentence's truth-value at any world,
$\Box\Box p \to \Box p$ is equivalent to $\Box p \to \Box p$. And this is
obviously valid.

Do not conflate the concepts of necessity and validity. Necessity means truth at
all worlds (or so we currently assume). Validity means truth at all worlds
\emph{in all models}. Whether an $\L_M$-sentence is necessary generally varies
from model to model. In a model whose interpretation function makes $p$ true at
all worlds, $p$ is necessary insofar as $\Box p$ is true at all worlds. In a
model whose interpretation function makes $p$ false at some world, $\Box p$ is
false at all worlds. Validity, by contrast, is not relative to a model. The
sentence $p$ is definitely not valid. The sentence $\Box p \to p$ is.

\begin{exercise}
  Show that if a sentence $A$ is valid, then so is $\Box A$.
\end{exercise}
\begin{solution}
  Suppose $A$ is valid -- true at all worlds in all models (definition
  \ref{def:valid}). It follows that in any given model, $A$ is true at every
  world. By definition \ref{def:basicsemantics}, it follows that $\Box A$ is
  true at every world in any model.
\end{solution}

Here is an example of an invalid sentence:
\[
  \Box(p \lor q) \to (\Box p \lor \Box q)
\]
How could we show that this is invalid? By definition \ref{def:valid}, a
sentence is valid iff it is true at all worlds in all models. So we have to find
some model in which there is some world at which the sentence is false. Such a
model is called a \textbf{countermodel} for the sentence. The following model is
a countermodel for the sentence above, as you should verify with the help of
definition \ref{def:basicsemantics}.
%
\begin{gather*}
  W = \{ w,v \}\\
  V(p) = \{ w \}\\
  V(q) = \{ v \}
\end{gather*}
%
I haven't explained at which worlds sentence letters other than $p$ and $q$ are
true, because it doesn't matter.

% In general, to specify a countermodel for a sentence $A$, you have to specify
% two things: a set $W$ of worlds, and an interpretation function $V$ that assigns
% a subset of $W$ to all the sentence letters in $A$.

\begin{exercise}
  Show that $p \to \Box p$ is invalid (and thus $p \not\models \Box p$), by
  giving a countermodel. Explain why this doesn't contradict the previous
  exercise.
\end{exercise}
\begin{solution}
  $p\to \Box p$ is false at world $w$ in the model(s) given by
  $W = \{ w, v \}, V(p) = \{ w \}$.

  This shows that the \emph{truth} of $p$ (at a world in a model) does not
  entail the truth of $\Box p$ (at the world in the model), even though the
  \emph{validity} of $p$ entails the validity of $\Box p$, as per the previous
  exercise.
\end{solution}

% \begin{exercise}
%   The sentences that are true at a world $w$ in a model $M$ contain
%   information not just about $w$ but also about other worlds in
%   $M$. For example, if $\Box p$ is true at $w$, we know that $w$ is
%   true at all worlds, and if $\neg p$ and $\Diamond p$ are both true
%   at $w$, we know that $p$ is true at some other world. Question: do
%   the sentences true at $w$ completely determine the model $M$? If
%   not, give an example of two worlds $w_1$ and $w_2$ in two models
%   $M1$, $M2$ that verify the same sentences even though the models are
%   not isomorphic.
% \end{exercise}

\begin{exercise}
  Show that for any sentences $A$, $B$, if $\models A \to B$, then also
  $\models \Box A \to \Box B$.
\end{exercise}
\begin{solution}
  Assume $\models A \to B$. Then there is no world in any model at which $A$ is
  true and $B$ is false. So if $A$ is true at every world in a model, then $B$
  is also true at every world in the model. It follows that $\Box A \to \Box B$
  is true at every world in every model.
\end{solution}

Earlier in this section, I showed that $\Box p \to p$ and
$\Box p \to \Box\Box p$ are valid. The arguments I gave easily generalise to
other sentences in place of $p$. So all instances of the following schemas are
valid: $\Box A \to A$ and $\Box A \to \Box\Box A$.

You may remember these schemas as the schemas (T) and (4) from section
\ref{sec:systems}. You may also remember that I defined the system S5 by
stipulating that it contains all instances of the following schemas:
%
\begin{principles}
  \pri{Dual}{\neg\Diamond A \leftrightarrow \Box\neg A}\\
  \pri{T}{\Box A \to A}\\
  \pri{K}{\Box(A\to B) \to (\Box A \to \Box B)}\\
  \pri{4}{\Box A \to \Box \Box A}\\
  \pri{5}{\Diamond A \to \Box \Diamond A}
\end{principles}
%
You can check that all instances of these schemas are valid by the definitions of
the present chapter.

% \begin{exercise}
%   Show that all instances of the \textbf{D}-schema $\Box A \to \Diamond A$ are
%   valid by the definitions from sections \ref{sec:basicmodels} and
%   \ref{sec:redefining}.
% \end{exercise}
% \begin{solution}
%   By definition \ref{def:valid}, a sentence is valid iff it is true at every
%   world in every model. Suppose for reductio that some instance of
%   $\Box A \to \Diamond A$ is false at some world $w$ in some model. By
%   definition \ref{def:basicsemantics}, $\Box A$ is then true at $w$ and
%   $\Diamond A$ false. But if $\Diamond A$ is false at $w$ then (by definition
%   \ref{def:basicsemantics}) $A$ is false at every world in the model, including
%   $w$. And then $\Box A$ isn't true at $w$ (by definition
%   \ref{def:basicsemantics}). Contradiction.
% \end{solution}

% Exercise: $\Box(A \lor B) \leftrightarrow (\Box A \lor \Box B)$ is
% invalid, but one direction is valid; which one. Dito for the
% dual. [MLOM 18]

I also specified two rules for S5. The first says that any truth-functional
consequence of any sentences in S5 is itself in S5. The second says that
whenever a sentence $A$ is in S5, then so is $\Box A$. As we will show chapter
\ref{ch:proofs}, these rules preserve validity (as defined in the previous
section). Indeed, you will learn how to show that the sentences that are
valid by our present definitions are precisely the sentences in S5.

In the meantime, let's prove a simpler fact to which I have appealed above (as
well as on page \pageref{claim:replacement} in the previous chapter): that
replacing logically equivalent sentences inside a larger sentence never affects
the larger sentence's truth-value at any world.

To show this, I am going to use a technique called \textbf{induction on
  complexity}. It works like this. Suppose we want to show that every sentence
of a language has a certain property. To do so, we first show that all simple,
atomic sentences of the language have the property. The atomic sentences of
$\L_{M}$ are the sentence letters. In a second step, we then show that the
logical operators preserve the property, meaning that if an operator (like
`$\land$' or `$\Box$') is applied to one or more sentences, and these sentences
have the property, then the resulting sentence (that we get by applying the
operator) still has the property. In this second step, we therefore
\emph{assume} that the sentences to which the operator is applied have the
property. This is called the \emph{induction hypothesis}. Based on this
assumption, we show that the more complex resulting sentence still has the
property.

\begin{observation}{replacement-theorem}
  If $A$ is an $\L_{M}$-sentence and $A'$ results from $A$ by replacing a
  subsentence of $A$ with a logically equivalent sentence, then $A$ and $A'$ are
  logically equivalent.
\end{observation}
\begin{proof}
  \emph{Proof.} Remember that two sentences are logically equivalent if each
  entails the other. By definition \ref{def:basicconsequence}, this means that
  the two sentences are true at the same worlds in every model.

  Now let $A$ be an arbitrary $\L_{M}$-sentence and assume that $A'$ results
  from $A$ by replacing a subsentence of $A$ with a logically equivalent
  sentence. To show that $A$ and $A'$ are equivalent, we first consider the case
  where $A$ is a sentence letter. In this case, $A$ has no sentences as proper
  parts and the target hypothesis is vacuously true: there is no way of turning
  $p$ into a non-equivalent sentence by replacing a subsentence within $p$.

  Next we consider the case where $A$ is a complex sentence that results by
  applying some logical operator to one or more simpler sentences. We assume (as
  our induction hypothesis) that the target hypothesis holds for the simpler
  sentences.

  Assume that $A$ is the negation of another sentence $B$. So $A$ is $\neg B$
  and $A'$ is $\neg B'$ for some sentence $B'$ that is either equivalent to $B$
  (if $B$ is the subsentence of $A$ that has been replaced to yield $A'$) or
  that results from $B$ by replacing a subsentence within $B$ by an equivalent
  sentence (if the subsentence of $A$ that has been replaced to yield $A'$ isn't
  $B$). In the latter case, our assumption that the observation holds for
  sentences simpler than $A$ implies that $B$ and $B'$ are equivalent. Either
  way, then, $B$ and $B'$ are logically equivalent: they are true at the same
  worlds in every model. By clause (b) of definition \ref{def:basicsemantics},
  it follows that $A$ and $A'$ are also true at the same worlds in every model.

  Essentially the same reasoning applies in the case where $A$ is a conjunction
  $B \land C$, a disjunction $B \lor C$, a conditional $B \to C$, a
  biconditional $B \leftrightarrow C$, a box sentence $\Box B$, and a diamond
  sentence $\Diamond B$. I won't bore you by going through all of them. Here is
  the case for $\Box B$.
  
  Assume that $A$ has the form $\Box B$. So $A$ is $\Box B$ and $A'$ is
  $\Box B'$ for some sentence $B'$ that is equivalent to $B$ (by the same
  reasoning as before). By clause (g) of definition \ref{def:basicsemantics} it
  follows that $A$ and $A'$ are also equivalent. \qed

\end{proof}


\section{Trees}\label{sec:trees}

I will now introduce a streamlined method for working through definition
\ref{def:basicsemantics} to check whether a sentence is valid: the method of
\textbf{analytic tableau} or \textbf{tree proofs}. (You may be familiar with
this method for non-modal logic. If so, good. If not, no problem.) It is best
introduced by example.

Let's check if $\Diamond p \to \Box p$ is valid. We do this by trying to
construct a countermodel. A countermodel for $\Diamond p \to \Box p$ is a model
in which there is some world $w$ at which $\Diamond p\to \Box p$ is false. We
start our construction by assuming that the \emph{negation} of
$\Diamond p \to \Box p$ is \emph{true} at $w$. We write this down as follows.

\begin{center}
  \Tree{%
    \nnode{18}{1.}{$\neg(\Diamond p \to \Box p)$}{w}{(Ass.)}%
  }
\end{center}
%
`1.' and `(Ass.)' are for book-keeping; `Ass.'\ is short for `Assumption', since
we're assuming that $\neg(\Diamond p \to \Box p)$ is true at $w$. Now we
unfold this assumption in accordance with definition \ref{def:basicsemantics}.
The definition tells us that a conditional $A\to B$ is false at a world $w$ iff
the antecedent $A$ is true at $w$ and the consequent $B$ is false at $w$. So the
assumption on line 1 implies that $\Diamond p$ is true at $w$ and that $\Box p$
is false at $w$. We expand our ``tree'' (or ``tableau'') by adding these consequences.

\begin{center}
  \tree{%
    \nnodeticked{18}{1.}{$\neg(\Diamond p \to \Box p)$}{w}{(Ass.)}\\
    \nnode{18}{2.}{$\Diamond p$}{w}{(1)}\\
    \nnode{18}{3.}{$\neg\Box p$}{w}{(1)}%
  }
\end{center}
%
I have ticked off line 1 (with `$\checkmark$') to mark that we won't need to
look at it again. All the information in line 1 is contained in lines 2 and 3.
The parenthetical `(1)' at lines 2 and 3 reminds us that these lines are
derived from line 1.

We continue drawing out further consequences. What does the truth of
$\Diamond p$ at $w$ imply for the subsentence $p$? By definition
\ref{def:basicsemantics}, there must be some world -- let's call it $v$ -- at
which $p$ is true.

\begin{center}
  \tree{%
    \nnodeticked{18}{1.}{$\neg(\Diamond p \to \Box p)$}{w}{(Ass.)}\\
    \nnodeticked{18}{2.}{$\Diamond p$}{w}{(1)}\\
    \nnode{18}{3.}{$\neg\Box p$}{w}{(1)}\\
    \nnode{18}{4.}{$p$}{v}{(2)}%
  }
\end{center}

Line 3 claims that $\Box p$ is false at $w$. By definition
\ref{def:basicsemantics}, $\Box p$ is true at $w$ iff $p$ is true at all worlds.
So if $\Box p$ is false at $w$, there must be some world at which $p$ is false.
Let's introduce such a world, naming it $u$. Our tree looks as follows.

\begin{center}
  \tree{%
    \nnodeticked{18}{1.}{$\neg(\Diamond p \to \Box p)$}{w}{(Ass.)}\\
    \nnodeticked{18}{2.}{$\Diamond p$}{w}{(1)}\\
    \nnodeticked{18}{3.}{$\neg\Box p$}{w}{(1)}\\
    \nnode{18}{4.}{$p$}{v}{(2)}\\
    \nnode{18}{5.}{$\neg p$}{u}{(3)}%
  }
\end{center}

Now the only unprocessed lines are hypotheses about sentence letters and
negations of sentence letters. Sentence letters don't have (non-trivial)
subsentences, so we can't use definition \ref{def:basicsemantics} to further
break down 4 or 5. The tree is complete. We have found a countermodel for
$\Diamond p \to \Box p$.

Let's read off the countermodel. There are three worlds in our tree: $w$, $v$,
and $u$. So $W = \{ w, u, v \}$. By line 4, $p$ is true at $v$. By line 5, $p$
is false at $u$. We don't know whether $p$ is true or false at $w$, and it
doesn't matter -- otherwise the tree would say. Let's assume that
$V(p) = \{ v \}$. As you can verify, $\Diamond p\to \Box p$ is indeed false at
world $w$ in this model.

One more example, before I state the general rules. Let's try to find a
countermodel for $\Box(p\to q) \to (p \to \Box q)$. That's another conditional,
so we begin as before.
\begin{center}
  \tree[3]{%
    & \nnodeticked{32}{1.}{$\neg(\Box(p\to q) \to (p \to \Box q))$}{w}{(Ass.)} &&\\
    & \nnode{32}{2.}{$\Box(p\to q)$}{w}{(1)} &&\\
    & \nnode{32}{3.}{$\neg(p \to \Box q)$}{w}{(1)} && \\
  }
\end{center}

Line 1 assumes that the negation of the conditional is true at some world $w$.
Lines 2 and 3 break down this assumption, using the fact that $\neg (A \to B)$
is true (at a world) iff $A$ is true and $B$ false. We could deal with line 2
next, but it's better to ignore it for the moment and process 3 first, which is
yet another negated conditional.

\begin{center}
  \tree[3]{%
    & \nnode{32}{4.}{$p$}{w}{(3)} && \\
    & \nnode{32}{5.}{$\neg \Box q$}{w}{(3)} && \\
  }
\end{center}

Line 5 tells us that $\Box q$ is false at $w$. We can infer that there is a
world -- call it $v$ -- at which $q$ is false.

\begin{center}
  \tree[3]{%
    & \nnode{32}{6.}{$\neg q$}{v}{(5)} && \\
  }
\end{center}

Now we need to return to line 2. What can we infer from the hypothesis that
$\Box(p\to q)$ is true at $w$ about the subsentence $p \to q$? By definition
\ref{def:basicsemantics}, $p \to q$ must be true at \emph{every} world. So, in
particular, $p\to q$ must be true at $w$. Let's write that down. We'll add
another line for $v$ later, so we don't check off node 2.

\begin{center}
  \tree[3]{%
    & \nnode{32}{7.}{$p\to q$}{w}{(2)} && \\
  }
\end{center}

If you are used to proofs in the natural deduction style, you may now be tempted
to apply \emph{modus ponens} and infer that $q$ is true at $w$, from lines 4 and
7. In the tree method, however, we try not to draw inferences from multiple
premises. We simply look at any lines that can still be processed and check what
definition \ref{def:basicsemantics} tells us about the immediate subsentences of
the sentence on that line. So we process line 7 without looking at line 4.

What can we infer from the truth of $p\to q$ at $w$ about the subsentences $p$
and $q$? By definition \ref{def:basicsemantics}, $p \to q$ is true at $w$ if $p$
is false at $w$ \emph{or} $q$ is true at $w$. We have to keep track of both
possibilities. Our (upside down) tree will branch. Here is the full tree at its
present stage.

\begin{center}
  \tree[3]{%
    & \nnodeticked{32}{1.}{$\neg(\Box(p\to q) \to (p \to \Box q))$}{w}{(Ass.)} &&\\
    & \nnode{32}{2.}{$\Box(p\to q)$}{w}{(1)} &&\\
    & \nnodeticked{32}{3.}{$\neg(p \to \Box q)$}{w}{(1)} && \\
    & \nnode{32}{4.}{$p$}{w}{(3)} && \\
    & \nnodeticked{32}{5.}{$\neg \Box q$}{w}{(3)} && \\
    & \nnode{32}{6.}{$\neg q$}{v}{(5)} && \\
    & \bnodeticked{32}{7.}{$p\to q$}{w}{(2)} && \\
    &&& \\
    \nnodeclosed{8}{8.}{$\neg p$}{w}{(7)} && \nnode{8}{9.}{$q$}{w}{(7)} & \\
  }
\end{center}

So far, I have called the numbered items on a tree `lines'. The proper term is
\textbf{nodes}. Since nodes 8 and 9 are visually on the same line, it would be
confusing to call them lines. While we're at it, a \textbf{branch} of a tree is
series of nodes that extends from the top (or ``root'') node all the way down to
a node below which there is no other node. The present tree has two branches,
both of which contain 8 nodes.

What does this tree tell us? Remember that our aim is to construct a model in
which the sentence at node 1 is true at world $w$. At this stage, the tree tells
us that this model contains two worlds $w$ and $v$; nodes 4 and 6 tell us
something about the model's interpretation function: $p$ is true at $w$, $q$ is
false at $v$. After node 7, the tree branches. This means that there are two
ways of extending the model we have construed so far. On the left branch, we
explore an extension of the model in which $p$ is false at $w$. On the right
branch, we explore an extension in which $q$ is true at $w$. But hold on. We
already know that $p$ is true at $w$ (from node 4). There's no model in which
$p$ is both true and false at $w$. So the possibility explored on the left
branch is a dead-end. it doesn't lead to a countermodel. That's why I've
\emph{closed} the left branch by drawing a cross below node 8.

We continue on the right-hand branch. Here we expand node 2 again, this time for
world $v$, which leads to another branching.

\begin{center}
  \tree[3]{%
    & \nnodeticked{32}{1.}{$\neg(\Box(p\to q) \to (p \to \Box q))$}{w}{(Ass.)} &&\\
    & \nnode{32}{2.}{$\Box(p\to q)$}{w}{(1)} &&\\
    & \nnodeticked{32}{3.}{$\neg(p \to \Box q)$}{w}{(1)} && \\
    & \nnode{32}{4.}{$p$}{w}{(3)} && \\
    & \nnodeticked{32}{5.}{$\neg \Box q$}{w}{(3)} && \\
    & \nnode{32}{6.}{$\neg q$}{v}{(5)} && \\
    & \bnodeticked{32}{7.}{$p\to q$}{w}{(2)} && \\
    &&& \\
    \nnodeclosed{8}{8.}{$\neg p$}{w}{(7)} && \nnode{13}{9.}{$q$}{w}{(7)} & \\
    && \bnode{13}{10.}{$p\to q$}{v}{(2)} & \\
    &&& \\
    & \nnode{8}{11.}{$\neg p$}{v}{(10)} & & \nnodeclosed{8}{12.}{$q$}{v}{(10)} 
  }
\end{center}
%
On the right-most branch, $q$ is true at $v$ (by node 12) but also false at $v$
(by node 6), so that branch is closed. But the middle possibility is still open,
and there are no more nodes to unfold. We have found a countermodel.

The countermodel is given by all the nodes \emph{on the middle branch},
the one that remained open. (The other branches were dead-ends and can be
ignored.) We have two worlds, $W = \{ w,v \}$. The interpretation function $V$
makes $p$ true at $w$ (node 4) and false at $v$ (node 11); $q$ is also true at
$w$ (node 9) and false at $v$ (node 6). Again, you may verify that the sentence
on node 1 is true at world $w$ in this model.

Now for the general rules.

In order to find a countermodel for a sentence $A$ with the help of the tree
method, you always begin by assuming that the \emph{negation} of $A$ is true at
world $w$:
%
\begin{center}
  \tree{%
    \nnode{10}{1.}{$\neg A$}{w}{(Ass.)}
    }
\end{center}
%
You then expand this node, and you continue expanding new nodes that appear on
the tree, until no more nodes can be expanded.

To expand a node with a \emph{non-negated sentence}, you consider what the truth of that
sentence at the node's world implies for the truth-value of the sentence's
immediate parts. The result may be added to the end of any open branch
containing the node.

(The immediate parts of a sentence of the form $A\land B$, $A \lor B$,
$A \to B$, or $A \leftrightarrow B$ are the corresponding sentences $A$ and $B$;
the only immediate part of $\Box A$, $\Diamond A$, and $\neg A$ is $A$.)

To expand a node with a negation $\neg A$, you consider what the falsity of the
relevant sentence $A$ at the node's world implies for the immediate parts of
$A$. The result may again be added to the end of any open branch containing the
node.

The following diagrams summarize how the different kinds of nodes are expanded.
I use `$\omega$' and `$\nu$' as placeholders for arbitrary world variables.


\bigskip\noindent%
% Perhaps I shouldn't use 'nu' and 'omega' in the schematic tree rules? students
% just thought they mean w and v, and asked if the box rule can be applied with
% an old world other than v. Also might need to clarify that v can be the same
% world as w.
\begin{minipage}{0.33\textwidth}\centering
\tree{
  \dotbelownode{12}{}{$A \land B$}{\omega}{}\\
  \\
  \nnode{12}{}{$A$}{\omega}{}\\
  \nnode{12}{}{$B$}{\omega}{}
}
\end{minipage}
\begin{minipage}{0.33\textwidth}\centering
\tree{
  & \dotbelowbnode{12}{}{$A \lor B$}{\omega}{} &\\
  && \\
  && \\
  \nnode{8}{}{$A$}{\omega}{} && \nnode{8}{}{$B$}{\omega}{}
}
\end{minipage}
\begin{minipage}{0.33\textwidth}\centering
\tree{
  & \dotbelowbnode{12}{}{$A \to B$}{\omega}{} &\\
  && \\
  && \\
  \nnode{8}{}{$\neg A$}{\omega}{} && \nnode{8}{}{$B$}{\omega}{}
}
\end{minipage}

\vspace{10mm}\noindent%
\begin{minipage}{0.33\textwidth}\centering
\tree{
  & \dotbelowbnode{12}{}{$A \leftrightarrow B$}{\omega}{} & \\
  & \\
  & \\
  \nnode{8}{}{$A$}{\omega}{} & & \nnode{8}{}{$\neg A$}{\omega}{} & & \\
  \nnode{8}{}{$B$}{\omega}{} & & \nnode{8}{}{$\neg B$}{\omega}{} & & \\
}
\end{minipage}
\begin{minipage}{0.33\textwidth} \centering
\tree{
  \dotbelownode{8}{}{$\Box A$}{\omega}{}\\
  \\
  \nnode{8}{}{$A$}{\nu}{}\\
  \Kk[8]{0}{\color{red}$\uparrow$}\\
  \Kk[8]{0}{\color{red}\small old}
}
\end{minipage}
\begin{minipage}{0.33\textwidth}\centering
\tree{
  \dotbelownode{8}{}{$\Diamond A$}{\omega}{$\!\!\!\!\!\!\checkmark$}\\
  \\
  \nnode{8}{}{$A$}{\nu}{}\\
  \Kk[8]{0}{\color{red}$\uparrow$}\\
  \Kk[8]{0}{\color{red}\small new}
}
\end{minipage}

\vspace{10mm}\noindent%
\begin{minipage}{0.33\textwidth}\centering
\tree{
  & \dotbelowbnode{15}{}{$\neg(A \land B)$}{\omega}{} &\\
  && \\
  && \\
  \nnode{8}{}{$\neg A$}{\omega}{} && \nnode{8}{}{$\neg B$}{\omega}{}
}
\end{minipage}
\begin{minipage}{0.33\textwidth}\centering
\tree{
  \dotbelownode{15}{}{$\neg(A \lor B)$}{\omega}{}\\
  \\
  \nnode{15}{}{$\neg A$}{\omega}{}\\
  \nnode{15}{}{$\neg B$}{\omega}{}
}
\end{minipage}
\begin{minipage}{0.33\textwidth}\centering
\tree{
  \dotbelownode{15}{}{$\neg(A \to B)$}{\omega}{}\\
  \\
  \nnode{15}{}{$A$}{\omega}{}\\
  \nnode{15}{}{$\neg B$}{\omega}{}
}
\end{minipage}

\vspace{10mm}\noindent%
\begin{minipage}{0.33\textwidth}\centering
\tree{
  & \dotbelowbnode{15}{}{$\neg(A \leftrightarrow B)$}{\omega}{} & \\
  & \\
  & \\
  \nnode{8}{}{$A$}{\omega}{} & & \nnode{8}{}{$\neg A$}{\omega}{} & & \\
  \nnode{8}{}{$\neg B$}{\omega}{} & & \nnode{8}{}{$B$}{\omega}{} & & \\
}
\end{minipage}
\begin{minipage}{0.33\textwidth}\centering
\tree{
  \dotbelownode{10}{}{$\neg \Box A$}{\omega}{$\!\!\!\!\!\!\checkmark$}\\
  \\
  \nnode{10}{}{$\neg A$}{\nu}{}\\
  \Kk[10]{0}{\color{red}$\uparrow$}\\
  \Kk[10]{0}{\color{red}\small new}
}
\end{minipage}
\begin{minipage}{0.33\textwidth}\centering
\tree{
  \dotbelownode{10}{}{$\neg \Diamond A$}{\omega}{}\\
  \\
  \nnode{10}{}{$\neg A$}{\nu}{}\\
  \Kk[10]{0}{\color{red}$\uparrow$}\\
  \Kk[10]{0}{\color{red}\small old}
}
\end{minipage}

\vspace{10mm}\noindent%
\begin{minipage}{0.33\textwidth}\centering
\tree{%
  \dotbelownode{12}{}{$\neg\neg A$}{\omega}{}\\
  \\
  \nnode{12}{}{$A$}{\omega}{}
}
\end{minipage}

\vspace{5mm}

If a branch of a tree contains a sentence $A$ as well as its negation $\neg A$,
for the same world $\omega$, then the branch is \emph{closed} with an {\sffamily
  x} at the bottom.

% NB: By this I mean that the branch /must/ be marked as closed. You're not
% allowed to keep expanding -- otherwise there could be a fully expanded open
% branch that does not yield a countermodel.

The rule for $\Box A$ says that from the assumption that $\Box A$ is true at a
world $\omega$ you may infer that $A$ is true at any ``old'' world $\nu$, by
which I mean any world \emph{that already occurs on the branch to which you want to add a node}. You're not allowed to introduce a new world variable (`$v$',
`$u$', etc.) when expanding $\Box A$ nodes. The same is true for
$\neg \Diamond A$ nodes (which by duality means the same as $\Box \neg A$). When
you expand a $\Diamond A$ node (or a $\neg \Box A$ node), by contrast, you must
introduce a new world variable.

Nodes of type $\Box A$ and $\neg \Diamond A$ can be expanded several times, once
for every world variable on any branch containing the node.

If you have expanded a node that is not of type $\Box A$ or $\neg \Diamond A$,
and you have added the new nodes to every open branch containing the node, then
you can tick off the node. You don't need to look at it again. Nodes of type
$\Box A$ and $\neg \Diamond A$ are never ticked off.

I have added a checkmark next to the rules for $\Diamond A$ and $\neg \Box A$ as
a reminder that these rules can only applied once on each open branch. If you've
already introduced a new world by expanding a $\Diamond A$ or $\neg \Box A$
node, you're not allowed to introduce further new worlds on the same branch by
expanding the same node again.

If no more rules can be applied, the tree is complete. Any open branch on a
complete tree defines a countermodel for the target sentence.
%
\begin{exercise}
  Use the tree method to find countermodels for the following
  sentences. (Spell out the countermodel, in addition to drawing the tree.)
  \begin{exlist}
  \item $p\to q$ 
  \item $p \to \Box(p \lor q)$
  \item $\Box p \lor \Box \neg p$
  \item $\Diamond(p \to q) \to (\Diamond p \to \Diamond q)$
  % This is invalid, but one can only read off a counterexample once all rules
  % have been applied. Also emphasize that countermodels must come from /one/
  % branch only. 
  \item $\Box \Diamond p \to p$ % infinite
  \end{exlist}
\end{exercise}
\begin{solution}
  \medskip
    
  \begin{sollist}
  \item Target: $p\to q$
  
    \tree{
      &\nnode{15}{1.}{$\neg(p \to q)$}{w}{(Ass.)}&\\
      &\nnode{15}{2.}{$p$}{w}{(1)}&\\
      &\nnode{15}{3.}{$\neg q$}{w}{(1)}&
    }

    Countermodel: $W = \{ w \}, V(p) = \{ w \}, V(q) = \emptyset$.
    
    \medskip
  \item Target: $p \to \Box(p \lor q)$

    \tree{
      \nnode{25}{1.}{$\neg (p \to \Box(p \lor q))$}{w}{(Ass.)}\\
      \nnode{25}{2.}{$p$}{w}{(1)}\\
      \nnode{25}{3.}{$\neg \Box(p \lor q)$}{w}{(1)}\\
      \nnode{25}{4.}{$\neg(p \lor q)$}{v}{(3)}\\
      \nnode{25}{5.}{$\neg p$}{v}{(4)}\\
      \nnode{25}{5.}{$\neg q$}{v}{(4)}
    }

    Countermodel: $W = \{ w, v \}, V(p) = \{ w \}, V(q) = \emptyset$.

  \medskip
  \item Target: $\Box p \lor \Box \neg p$
    
    \tree{
      \nnode{25}{1.}{$\neg (\Box p \lor \Box \neg p)$}{w}{(Ass.)}\\
      \nnode{25}{2.}{$\neg \Box p$}{w}{(1)}\\
      \nnode{25}{3.}{$\neg \Box \neg p$}{w}{(1)}\\
      \nnode{25}{4.}{$\neg p$}{v}{(2)}\\
      \nnode{25}{5.}{$\neg \neg p$}{u}{(3)}\\
      \nnode{25}{6.}{$p$}{u}{(5)}
    }

    Countermodel: $W = \{ w,v,u \}, V(p) = \{ u \}$.
    
    \medskip
  \item Target: $\Diamond(p \to q) \to (\Diamond p \to \Diamond q)$

    \tree[3]{
      &\nnode{35}{1.}{$\neg(\Diamond(p \to q) \to (\Diamond p \to \Diamond q))$}{w}{(Ass.)}&\\
      &\nnode{35}{2.}{$\Diamond(p \to q)$}{w}{(1)}&\\
      &\nnode{35}{3.}{$\neg (\Diamond p \to \Diamond q)$}{w}{(1)}&\\
      &\nnode{35}{4.}{$\Diamond p$}{w}{(3)}&\\
      &\nnode{35}{5.}{$\neg \Diamond q$}{w}{(3)}&\\
      &\nnode{35}{6.}{$p\to q$}{v}{(2)}&\\
      &\nnode{35}{7.}{$p$}{u}{(4)}&\\
      &\nnode{35}{8.}{$\neg q$}{w}{(5)}&\\
      &\nnode{35}{9.}{$\neg q$}{v}{(5)}&\\
      &\bnode{35}{10.}{$\neg q$}{u}{(5)}&\\
      &&\\
      \nnode{10}{11.}{$\neg p$}{v}{(6)} && \nnodeclosed{10}{12.}{$q$}{v}{(6)}
    }

    Countermodel: $W = \{ w,v,u \}, V(p) = \{ u \}, V(q) = \emptyset$.
    
    \medskip
  \item $\Box \Diamond p \to p$
  
    \tree{
      &\nnode{20}{1.}{$\neg(\Box\Diamond p \to p))$}{w}{(Ass.)}&\\
      &\nnode{20}{2.}{$\Box \Diamond p$}{w}{(1)}&\\
      &\nnode{20}{3.}{$\neg p$}{w}{(1)}&\\
      &\nnode{20}{4.}{$\Diamond p$}{w}{(2)}&\\
      &\nnode{20}{5.}{$p$}{v}{(4)}&\\
      &\nnode{20}{6.}{$\Diamond p$}{v}{(2)}&\\
      &\nnode{20}{7.}{$p$}{u}{(6)}&\\
      &\nnode{20}{8.}{$\Diamond p$}{u}{(2)}&\\
      &\nnode{20}{9.}{$p$}{t}{(8)}&\\
      &\nnode{20}{}{\vdots}{}{}&
  }

  \medskip\noindent The tree grows forever. The target sentence isn't valid, but
  the tree method only gives us an infinite countermodel. In such a case, it may
  be useful to read off a model from an incomplete version of the tree and
  manually check whether it is a genuine countermodel. The model determined by
  the first five nodes of the present tree is $W = \{ w,v \}, V(p)=\{ v \}$,
  and you can confirm that it is a countermodel to the target sentence.

  If you read off a model from an \emph{incomplete} tree, you can't be sure that
  it is a countermodel for the target sentence. You must always double-check!
   
  \end{sollist}
  
\end{solution}

What if all branches on a tree close? Then there is no countermodel for the
target sentence. If there is no countermodel for a sentence, then the sentence
is valid. This is how the tree method is used to show that a sentence is valid.

The following tree shows that
$\Diamond \neg p \leftrightarrow \neg \Box p$ is valid. Make sure you understand
each step. (I've omitted the check marks since these are only useful during the
construction phase.)

\begin{center}
  \tree[3]{
    & \bnode{22}{1.}{$\neg(\Diamond \neg p \leftrightarrow \neg \Box p)$}{w}{(Ass.)} & \\
  && \\
  \nnode{12}{2.}{$\Diamond \neg p$}{w}{(1)} && \nnode{12}{4.}{$\neg \Diamond\neg p$}{w}{(1)}  \\
  \nnode{12}{3.}{$\neg \neg \Box p$}{w}{(1)} && \nnode{12}{5.}{$\neg \Box p$}{w}{(1)}  \\
  \nnode{12}{6.}{$\Box p$}{w}{(3)} && \nnode{12}{9.}{$\neg p$}{v}{(5)}  \\
  \nnode{12}{7.}{$\neg p$}{v}{(2)} && \nnodeclosed{12}{10.}{$\neg\neg p$}{v}{(4)}  \\
  \nnodeclosed{12}{8.}{$p$}{v}{(6)} &&  \\
}
\end{center}

A similar tree could obviously be drawn for
$\Diamond \neg q \leftrightarrow \neg \Box q$, and for any other formula of the
form $\Diamond \neg A \leftrightarrow \neg \Box A$: we would simply replace each
occurrence of $p$ on the tree with $A$.

To show that all instances of a schema are valid, we can also directly draw
\textbf{schematic trees} in which we use schematic variables `$A$', `$B$', `$C$'
instead of sentence letters.

\begin{exercise}
  Use the tree method to show that all instances of the following schemas are valid.
   \begin{exlist}
   \item[\pr{K}] $\Box (A \to B) \to (\Box A \to \Box B)$
   \item[\pr{T}] $\Box A \to A$
   % \item[\pr{D}] $\Box A \to \Diamond A$ 
   \item[\pr{4}] $\Box A \to \Box \Box A$
   \item[\pr{5}] $\Diamond A \to \Box \Diamond A$
   % \item[\pr{G}] $\Diamond\Box A \to \Box \Diamond A$ 
   \end{exlist}
\end{exercise}
\begin{solution}
  You can enter the schemas at
  \href{https://www.umsu.de/trees/}{umsu.de/trees}.
  After entering a formula, tick the checkbox for `universal (S5)'.
  Alternatively, follow these links:
  \href{https://www.umsu.de/trees/\#~8(A~5B)~5(~8A~5~8B)||universality}{\pr{K}}, \href{https://www.umsu.de/trees/\#~8A~5A||universality}{\pr{T}},
  % \href{https://www.umsu.de/trees/\#~8A~5~9A||universality}{\pr{D}},
  \href{https://www.umsu.de/trees/\#~8A~5~8~8A||universality}{\pr{4}},
  \href{https://www.umsu.de/trees/\#~9A~5~8~9A||universality}{\pr{5}},
  % \href{https://www.umsu.de/trees/\#~9~8A~5~8~9A||universality}{\pr{G}}.
\end{solution}

\begin{exercise}
  For each of the following sentences, either show that it is valid or
  give a countermodel to show that it is invalid, using the tree method.
  \begin{exlist}
  \item $p \to \Box\Diamond p$
  \item $\Diamond\Diamond p \to \Diamond p$
  %\item $\Box\Box p \to \Box p$
  \item $\Diamond(p \land q) \to (\Diamond p \land \Diamond q)$
  \item $(\Diamond p \land \Diamond q) \to \Diamond(p \land q)$
  %\item $\Diamond p \lor \Box\neg p$
  \item $\Diamond(p \lor q) \leftrightarrow (\Diamond p \lor \Diamond q)$
  \item $\Box\Diamond p \to \Diamond\Box p$
  % \item $(\Diamond p \to \Box q) \to (\Box p \to \Box q)$
  \end{exlist}
\end{exercise}
\begin{solution}
  (a), (b), (c) and (e) are valid. You can find the trees at
  \href{https://www.umsu.de/trees/}{umsu.de/trees} (Remember to tick the checkbox for `universal (S5)') or by following these links:
  \href{https://www.umsu.de/trees/\#p~5~8~9p||universality}{(a)},
\href{https://www.umsu.de/trees/\#~9~9p~5~9p||universality}{(b)},
\href{https://www.umsu.de/trees/\#~9(p~1q)~5(~9p~1~9q)||universality}{(c)},
\href{https://www.umsu.de/trees/\#~9(p~2q)~4(~9p~2~9q)||universality}{(e)}.
% \href{https://www.umsu.de/trees/\#(~9p~5~8q)~5(~8p~5~8q)||universality}{(g)}.
  
  (d) and (f) are invalid. Here is a tree for (d):

  \medskip
  \tree[3]{
    &\nnode{35}{1.}{$\neg ((\Diamond p \land \Diamond q) \to \Diamond(p \land q))$}{w}{(Ass.)}&&\\
    &\nnode{35}{2.}{$\Diamond p \land \Diamond q$}{w}{(1)}&&\\
    &\nnode{35}{3.}{$\neg \Diamond(p \land q)$}{w}{(1)}&&\\
    &\nnode{35}{4.}{$\Diamond p$}{w}{(2)}&&\\
    &\nnode{35}{5.}{$\Diamond q$}{w}{(2)}&&\\
    &\nnode{35}{6.}{$p$}{v}{(4)}&&\\
    &\nnode{35}{7.}{$q$}{u}{(5)}&&\\
    &\bnode{35}{8.}{$\neg (p \land q)$}{v}{(3)}&&\\
    &&&\\
    \nnodeclosed{9}{9.}{$\neg p$}{v}{(8)}&&\nnode{15}{10.}{$\neg q$}{v}{(8)}&\\
    &&\bnode{15}{11.}{$\neg(p \land q)$}{u}{(3)}&\\
    &&&\\
    &\nnode{15}{12.}{$\neg p$}{u}{(11)}&&\nnodeclosed{9}{13.}{$\neg q$}{u}{(11)}&\\
    &\bnode{15}{14.}{$\neg(p \land q)$}{w}{(3)}&&\\
    &&&\\
    \nnode{9}{15.}{$\neg p$}{w}{(14)}&&\nnode{9}{16.}{$\neg q$}{w}{(14)}&
  }

  \medskip\noindent
  We can choose either of the open branches to read off a countermodel. In fact, here we get the same countermodel no matter which open branch we choose:
$W = \{ w,v,u \}, V(p)=\{ v\}, V(q)=\{u\}$.
  \medskip

  A tree for (e) might begin like this:

  \tree[3]{
    &\nnode{25}{1.}{$\neg (\Box\Diamond p \to \Diamond\Box p)$}{w}{(Ass.)}&&\\
    &\nnode{25}{2.}{$\Box\Diamond p$}{w}{(1)}&&\\
    &\nnode{25}{3.}{$\neg \Diamond\Box p$}{w}{(1)}&&\\
    &\nnode{25}{4.}{$\Diamond p$}{w}{(2)}&&\\
    &\nnode{25}{5.}{$p$}{v}{(4)}&&\\
    &\nnode{25}{6.}{$\neg \Box p$}{w}{(3)}&&\\
    &\nnode{25}{7.}{$\neg p$}{u}{(6)}&&\\
    &\nnode{25}{8.}{$\Diamond p$}{v}{(2)}&&\\
    &\nnode{25}{9.}{$p$}{s}{(8)}&&\\
    &\nnode{25}{10.}{$\neg \Box p$}{v}{(3)}&&\\
    &\nnode{25}{11.}{$\neg p$}{t}{(10)}&&\\
    &\nnode{25}{}{\vdots}{}{}&&
  }

  \medskip\noindent The tree grows forever. The model determined by
  the first seven nodes of the present tree is $W = \{ w,v,u \}, V(p)=\{ v \}$.
  It is a countermodel to the target sentence.
    
\end{solution}

When constructing a tree, you often have a choice of which node to expand next.
In that case, a good idea is to start with any $\Diamond A$ or $\neg \Box A$
nodes. If there are none, choose a node of type $A \land B$, $\neg (A \lor B)$
or $\neg(A \to B)$. Choose a node of another type only if none of the above are
available. This heuristic often helps to keep trees small, but it is not part of
the official tree rules.

\begin{exercise}\label{ex:trees-with-premises}
  Can we use the tree method to show that some premises $A_1,\ldots,A_n$ entail
  a conclusion $B$? Can we use it to show that two sentences $A$ and $B$ are
  equivalent?
\end{exercise}
\begin{solution}
  By observation \ref{obs:semantic-deduction-theorem}, $A_{1},\ldots,A_{n}$
  entail $B$ iff $(A_1\land\ldots\land A_n) \to B$ is valid. To show that
  $A_{1},\ldots,A_{n}$ entail $B$ we could therefore draw a tree for
  $(A_1\land\ldots\land A_n) \to B$. In practice, we can save a few steps by
  starting the tree with multiple assumptions: one for each of the premises
  $A_{1},\ldots, A_{n}$, and one for the negated conclusion $\neg B$. (All of
  these are assumed to be true at world $w$.) If the tree closes,
  $A_1,\ldots,A_n$ entail $B$.

  To show that $A$ and $B$ are equivalent, we can draw a tree for
  $A \leftrightarrow B$.
\end{solution}


%%% Local Variables: 
%%% mode: latex
%%% TeX-master: "logic2.tex"
%%% End:
