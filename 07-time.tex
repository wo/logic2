\chapter{Temporal Logic}\label{ch:time}

\section{Reasoning about time}\label{sec:time-intro}

It is currently raining in Edinburgh. But it wasn't raining yesterday, and
perhaps it won't rain tomorrow. Let's introduce some operators to formalize
reasoning about the unfolding of events through time.

If we read $r$ as `it is raining', we will use $\tF\! r$ to express that is will be
raining at some point in the future. We will use $\tP\! r$ to express that it has been
raining at some point in the past. In general:
%
\begin{quote}
  $\tF A$ is true at a time $t$ iff $A$ is true at some time after $t$.\\
  $\tP A$ is true at a time $t$ iff $A$ is true at some time before $t$.
\end{quote}

The operators $\tF$ and $\tP$ can be nested. We can use $\tF \tP r$ to express
that at some point it will have rained, $\tP\tF r$ to say that it was once going
to rain, $\tP\tP r$ to say that there was a time before which it rained, and
$\tF\tF r$ to say that there will come a time after which it will rain.

Unlike $\Box$ and $\Diamond$, $\tF$ and $\tP$ are not duals of each other:
$\neg \!\tP\! A$ is not equivalent to $\tF \!\neg A$, and $\neg \!\tF\! A$ is not
equivalent to $\tP \!\neg A$. But it is useful to have duals of $\tF$ and $\tP$.
We therefore introduce two more operators. $\tG$ will be the dual of $\tF$, and
$\tH$ the dual of $\tP$.

Intuitively, $\tG A$ means that $A$ is always \emph{going} to be the case.
(Hence the symbol `G'.) If it is not the case that at some point in the future
it will not rain ($\neg \!\tF\! \neg r$), then it is always going to be the case
that it will rain ($\tG r$). Similarly, $\tH A$ means that $A$ \emph{has} always
been the case. If it is not the case that at some point in the past it was not
raining ($\neg \!\tP\! \neg r$), then it has always been raining ($\tH r$).

We can state the truth-conditions of $\tG A$ and $\tH A$ in parallel to the
above truth-conditions for $\tF A$ and $\tP A$:
\begin{quote}
  $\tG A$ is true at a time $t$ iff $A$ is true at all times after $t$.\\
  $\tH A$ is true at a time $t$ iff $A$ is true at all times before $t$.
\end{quote}

The language of standard propositional logic, extended by the four operators
$\tF,\tP,\tG,\tH$ is known as the \textbf{language of basic temporal logic}. We
will sometimes call it $\L_t$.

% $\L_t$ is only a crude approximation to the means by which we talk about time in
% natural language. In English, for example, there is a past tense -- as in `it
% rained' -- but no future tense; to express that something happens in the future
% we typically use the modal verb `will' applied to an infinitive: `it will rain'.
% In many languages, there is a complex system of tenses and aspects to indicate
% an event's position in time and its status as finished or ongoing. $\L_t$ lacks
% most of these subtleties, but it proves sufficient to formalise many interesting
% hypotheses and arguments about time. We will look at some extensions of the
% language in section \ref{sec:2d}.

\begin{exercise}
  Translate the following sentences into the language of basic temporal logic.
  \begin{exlist}
  \item It has never been warm. %H-w, -Pw
  \item There will be a sea battle. %Fc (note English may not be existential)
  \item There will not have been a sea battle. % F-Pc or -FPc (clarify difference)
  \item At some point, it will be warm or it will have been warm. % F(w v Pw)
  \item If you haven't studied, you won't pass the exam. % -Ps -> -Fp
  \item I was having tea when the door bell rang. % P(t & r)
  \end{exlist}
\end{exercise}
\begin{solution}
  \begin{sollist}
  \item $\tH \neg p$\\
    $p$: It is warm\\[-2mm]
  \item $\tF p$\\
    $p$: There is a sea battle\\[-2mm]
  \item $\neg \tF \tP p$ or, perhaps, $\tF\neg \tP p$\\
    $p$: There is a sea battle\\[-2mm]
  \item  $\tF(p \lor \tP q)$ or $\tF(\tF p \lor \tF\tP q)$\\
    $p$: It is warm\\[-2mm]
  \item $\neg \tP p \to \neg \tF q$ or $\tG(\neg\tP p \to \neg q)$\\
    $p$: You study, $q$: you pass the exam\\[-2mm]
  \item $\tP(p \land q)$\\
    $p$: I am having tea, $q$: the door bell rings
  \end{sollist}
\end{solution}


\section{Temporal models}

A complete scenario for temporal logic needs to tell us what times there are,
how they are ordered, and what is going on at each of them. We can represent
such a scenario, together with an interpretation of $\L_{t}$'s non-logical
vocabulary, by a structure that settles (a) what times there are, (b) which
times come before or after which others, and (c) which sentence letters are true
at which times. This is enough to determine, for every $\L_{t}$-sentence and
every time, whether the sentence is true at that time.

\begin{definition}{Temporal Model}{temporalmodel}
  A \textbf{temporal model} consists of
  \vspace{-3mm}
  \begin{itemize*}
  \item a non-empty set $T$ (of ``times''),
  \item a binary relation $<$ on $T$ (the \textbf{precedence relation}),
  \item a function $V$ that assigns to each sentence letter of $\L_T$
    a subset of $T$.
  \end{itemize*}
\end{definition}

% In a sense, a time is a world. A world is a maximally specific way things
% might be. Now here's a way things might be: it might be Monday. Ignorance of
% time plays an important role in our life, so the worlds of epistemic logic,
% say, must not just specify what is the case from an eternal, God's eye
% perspective. If worlds have a time coordinate, then past times are worlds that
% stand in a certain relationship to our world. They share the same untensed
% truths.

We use `$M, t \models A$' as a short-hand notation to express that sentence $A$
is true at time $t$ in model $M$. The following definition formally specifies
the truth-value of every $\L_T$-sentence at every time in every model.

\begin{definition}{Standard Temporal Semantics}{temporalsemantics}
  If $\Mfr = \t{T,<,V}$ is a temporal model, $t$ is a member of $T$, $P$ is
  any sentence letter, and $A,B$ are any $\L_T$-sentences, then

  \medskip
  \begin{tabular}{lll}
    (a) & $M,t \models P$ &iff $t$ is in $V(P)$.\\
    (b) & $M,t \models \neg A$ &iff $M,t \not\models A$.\\
    (c) & $M,t \models A \land B$ &iff $M,t \models A$ and $M,t \models B$.\\
    (d) & $M,t \models A \lor B$ &iff $M,t \models A$ or $M,t \models B$.\\
    (e) & $M,t \models A \to B$ &iff $M,t \not\models A$ or $M,t \models B$.\\
    (f) & $M,t \models A \leftrightarrow B$ &iff $M,t \models (A\to B)$ and $M,t \models (B\to A)$.\\
    (g) & $M,t \models \tF A$ &iff $M,s \models A$ for some $s\in T$ such that $t<s$.\\
    (h) & $M,t \models \tG A$ &iff $M,s \models A$ for all $s\in T$ such that $t<s$.\\
    (i) & $M,t \models \tP A$ &iff $M,s \models A$ for some $s\in T$ such that $s<t$.\\
    (j) & $M,t \models \tH A$ &iff $M,s \models A$ for all $s\in T$ such that $s<t$.
  \end{tabular}
\end{definition}
Clause (a) says that a sentence letter is true at a time in a model iff the
model's interpretation function specifies that the sentence letter is true at
that time. Clauses (b)--(f) say that the truth-functional connectives have their
normal truth-table meaning at each time. Clauses (g)--(j) formalize the
truth-conditions for temporal sentences from the previous section.

% We'll say that an $\L_T$-sentence is \textbf{valid} iff it is true at every time
% in every (suitable) temporal model; an $\L_T$-sentence $B$ is a \textbf{logical
%   consequence} of sentences $A_1,\ldots,A_n$ iff $B$ is true at every time in
% every (suitable) model at which $A_1,\ldots,A_n$ are all true. I say
% ``suitable'' because we will want to put some constraints on the precedence
% relation $<$. More on that in a moment.

All this should remind you of our Kripke semantics for $\L_M$ in chapter
\ref{ch:accessibility}. In fact, temporal models \emph{are} Kripke models, as
defined on page \pageref{def:kripkemodel}. I have merely relabelled the set
`$W$' as `$T$', and the relation `$R$' as `$<$'. Definition
\ref{def:temporalsemantics} resembles definition \ref{def:kripkesemantics} from
page \pageref{def:kripkesemantics}, except that we have two box-like operators
$\tG$ and $\tH$, and two diamond-like operators $\tF$ and $\tP$. The language of
basic temporal logic is bi-modal, with forward-looking operators ($\tF$ and
$\tG$) and backward-looking operators ($\tP$ and $\tH$). Unlike ordinary models
for multi-modal languages (definition \ref{def:multikripkemodel}), temporal
models have only a single accessibility relation. That's because
the accessibility relation for $\tP$ and $\tH$ is definable from the
accessibility relation for $\tF$ and $\tG$: a time $s$ is earlier than a time
$t$ iff $t$ is later than $s$.

% Exercise: define operators that look back on the epistemic or doxastic
% accessibility relation. ...

Let's look at an example of a temporal model. For the set of times $T$, we use
the set of natural numbers 0,1,2, etc. Let's say that the precedence relation
$<$ holds between $t$ and $s$ iff $t$ is smaller than $s$. So $0<1$ and
$1 < 25$. (We could just as well have stipulated that $<$ holds between $t$ and
$s$ iff $t$ is greater than $s$; we would then have $1<0$ and $25<1$. In
temporal logic, the symbol `$<$' means `earlier than', not `smaller than'.)
Finally, let's say that the interpretation function assigns to $p$ the set of
all even numbers.

Let's call this model $M$. By definition \ref{def:temporalsemantics}, we can
figure out the following facts, among others.
\begin{itemize}[leftmargin=10mm]
\itemsep-1mm
\item $M,0 \models p$ (because 0 is even);
\item $M,0 \models \tF p$ (because there are even numbers greater than 0);
\item $M,0 \models \tG\tF p$ (because for every number there is a greater number that is even);
\item $M,0 \models \neg\tF\tG p$ (because there is no number for which all greater numbers are even).
\end{itemize}

\begin{exercise}
  Now let $M$ be the following model. As before, $T$ is the set of natural
  numbers $\{ 0,1,2,\ldots \}$, and $t < s$ iff $t$ is smaller than $s$. This
  time, $V(p)$ is the set of numbers smaller than 10. Which of the following
  statements are true?
  \begin{exlist}
  %\item $M,0 \models \tF (p \land \neg p)$
  \item $M,0 \models \tF p \land \tF \neg p$ 
  \item $M,0 \models \tG \neg p$
  \item $M,0 \models \tF\tG \neg p$
  \item $M,0 \models \tG\tF p$
  \item $M,0 \models \tG(\tF p \to \tF\tF p)$
  \item $M,0 \models \tF\tH p$
  \item $M,0 \models \neg \tP(p \lor \neg p)$
  \item $M,0 \models \tH p$
  \end{exlist}
\end{exercise}
\begin{solution}
  (a), (c), (f), (g), and (h) are true, (b), (d), and (e) are false. 
\end{solution}

Real times are, of course, not numbers. When I say that `it is raining' is true
now, I don't mean that the sentence is true at a number. It isn't
obvious what kinds of things times are. Fortunately, this doesn't matter for us,
just as the nature of possible worlds doesn't matter for the logic of
possibility and necessity. As long as the formal structure of the times in a
scenario matches the structure of the natural numbers, it does no harm to use
numbers as times in a model of the scenario.

The formal structure of time in a temporal model is captured by the relevant
frame: the pair $\t{T,<\!}$ of the set of times and the precedence relation.
Frames in temporal logic are also called \textbf{flows of time}. Different
applications of temporal logic often come with different assumptions about the
flow of time.

In computer science, for example, the ``times'' $T$ are often understood as
possible states of a computational process; the precedence relation holds
between states $t$ and $s$ iff the computation can lead from $t$ to $s$. If
the computation is indeterministic, so that a given state can have different
successors, the relevant flow of time will involve forks towards the future: we
can have different ``times'' $s$ and $r$ such that $t<s$ and $t<r$ but neither
$s<r$ nor $r<s$. Here the precedence relation cannot be modelled by the
less-than relation on the natural numbers, because the structure of the
less-than relation does not include forks.

In other applications, we may be interested in how the weather changes from day
to day. Here we might identify the relevant times with days and the precedence
relation with the earlier-relation between days -- even though intuitively a day
is not a single time, but an interval comprising many times. For this
application, the natural numbers might have the right formal structure.

For yet other applications, we may want to assume that time is \textbf{dense},
meaning that whenever $t < s$ then there is another point of time lying in
between $t$ and $s$. This assumption is common in physics. The natural numbers,
by contrast, have a \textbf{discrete} structure. There is no natural number in
between 2 and 3. For dense models, we could use real or rational numbers
(fractions) instead of natural numbers.

% Humberstone 199f. distinguishes two notions of discreteness that come apart in models of branching time. The easier one of them corresponds to $(A \land H A) \to FH A$.

If we want to take seriously what physics tells us about time, it is not enough
to assume that time is dense. We also need to reconceptualize the set $T$.
According to the theory of special relativity, whether a point in time is
earlier or later than another is relative to a spatial frame of reference. An
adequate model of relativistic time must therefore include a representation of
space. In these \textbf{spacetime models} (or \emph{Minkowski models}), the set
$T$ consists of spacetime points $\t{x_1,x_2,x_3,t}$ with three spatial and one
temporal coordinate; $(x_1,x_2,x_3,t) < (y_1,y_2,y_3,s)$ holds iff the second
point can be reached from the first without travelling faster than the speed of
light.

\section{Logics of time}

Let's define the minimal temporal logic K$_t$ as the set of $\L_{t}$-sentences
that are true at all times in all temporal models. Since temporal models are
just Kripke models, proof methods for the minimal modal logic K are easily
adapted to K$_{t}$. The main novelty is that the rules for the box and the
diamond can be used twice over, once for the forward-looking operators $\tF$ and
$\tG$, and once for the backward-looking $\tP$ and $\tH$.

In the tree method for K$_{t}$, we have all the K-rules, with $\tG$ as the box
and $\tF$ as the diamond. In addition, we have rules for $\tH$ as the box and
$\tP$ as the diamond with a reversed perspective on the accessibility (or
precedence) relation:

\bigskip\hspace{-3mm}
\begin{minipage}{0.26\textwidth} \centering
  \tree{
    \nnode{12}{}{$\tG A$}{\omega}{}\\
    \dotbelownode{12}{}{$\omega<\nu$}{}{}\\
    \\
    \nnode{12}{}{$A$}{\nu}{}\\
    \Kk[12]{0}{}\\
    \Kk[12]{0}{}
}
\vspace{8mm}


  \tree{
    \dotbelownode{12}{}{$\neg\!\tG A$}{\omega}{}\\
    \\
    \nnode{12}{}{$\omega<\nu$}{}{}\\
    \nnode{12}{}{$\neg A$}{\nu}{}\\
    \Kk[12]{0}{$\uparrow$}\\
    \Kk[12]{0}{\small new}
}
\end{minipage}
\begin{minipage}{0.26\textwidth}\centering
\tree{
    \dotbelownode{12}{}{$\tF A$}{\omega}{}\\
    \\
    \nnode{12}{}{$\omega<\nu$}{}{}\\
    \nnode{12}{}{$A$}{\nu}{}\\
    \Kk[12]{0}{$\uparrow$}\\
    \Kk[12]{0}{\small new}
}

\vspace{8mm}

  \tree{
    \nnode{12}{}{$\neg\!\tF A$}{\omega}{}\\
    \dotbelownode{12}{}{$\omega<\nu$}{}{}\\
    \\
    \nnode{12}{}{$\neg A$}{\nu}{}\\
    \Kk[12]{0}{}\\
    \Kk[12]{0}{}
}
\end{minipage}
\begin{minipage}{0.26\textwidth} \centering
\tree{
    \nnode{12}{}{$\tH A$}{\omega}{}\\
    \dotbelownode{12}{}{\color{red}{$\nu<\omega$}}{}{}\\
    \\
    \nnode{12}{}{$A$}{\nu}{}\\
    \Kk[12]{0}{}\\
    \Kk[12]{0}{}
}

\vspace{8mm}


\tree{
    \dotbelownode{12}{}{$\neg\! \tH A$}{\omega}{}\\
    \\
    \nnode{12}{}{\color{red}{$\nu<\omega$}}{}{}\\
    \nnode{12}{}{$\neg A$}{\nu}{}\\
    \Kk[12]{0}{$\uparrow$}\\
    \Kk[12]{0}{\small new}
  }
  
\end{minipage}
\begin{minipage}{0.26\textwidth}\centering
\tree{
    \dotbelownode{12}{}{$\tP A$}{\omega}{}\\
    \\
    \nnode{12}{}{\color{red}{$\nu<\omega$}}{}{}\\
    \nnode{12}{}{$A$}{\nu}{}\\
    \Kk[12]{0}{$\uparrow$}\\
    \Kk[12]{0}{\small new}
  }
  
\vspace{8mm}

\tree{
    \nnode{12}{}{$\neg\!\tP A$}{\omega}{}\\
    \dotbelownode{12}{}{\color{red}{$\nu<\omega$}}{}{}\\
    \\
    \nnode{12}{}{$\neg A$}{\nu}{}\\
    \Kk[12]{0}{}\\
    \Kk[12]{0}{}
}
\end{minipage}
\bigskip

In the axiomatic approach, we have two versions of the \pr{K} schema, one for
the forward-looking box $\tG$ and one for the backward-looking box $\tH$:%
%
\begin{principles}
   \pri{GK}{\tG(A\to B) \to (\tG A \to \tG B)}\\
   \pri{HK}{\tH(A\to B) \to (\tH A \to \tH B)}
\end{principles}
%
We also have two versions of Necessitation, and two versions of \pr{Dual}:
%
\begin{principles}
  \pri{GDl}{\neg\tF A \leftrightarrow \tG\neg A}\\
  \pri{HDl}{\neg\tP A \leftrightarrow \tH\neg A}\\
  \pri{GNec}{\text{If $A$ occurs in a proof, $\tG A$ may be appended.}}\\
  \pri{HNec}{\text{If $A$ occurs in a proof, $\tH A$ may be appended.}}
\end{principles}
%
In addition, we need two interaction principles, reflecting the fact that the
accessibility relation for $\tF$ and $\tG$ is the inverse of the accessibility
relation for $\tP$ and $\tH$:
%
\begin{principles}
\pri{Con1}{A \to \tG\tP A}\\
\pri{Con2}{A \to \tH\tF A}
\end{principles}

These axioms and rules, added to those of classical propositional logic, define
an axiomatic calculus that is sound and complete for K$_{t}$. (Completeness is
easily proved with the canonical model technique.)

\begin{exercise}
  Show with the help of definition \ref{def:temporalsemantics} that all
  instances of \pr{Con1} and \pr{Con2} are true at all times in all temporal
  models. % delete?
\end{exercise}
\begin{solution}
  (Con1): Suppose some sentence of the form $A \to \tG\tP A$ is false at some time $t$ in some temporal model. By clause (e) of definition \ref{def:temporalsemantics}, this means that $A$ is true at $t$ and $\tG\tP A$ is false at $t$. By clause (h), the latter means that there is a time $s$ with $t<s$ such that $\tP A$ is not true at $s$. By clause (i), it follows that $A$ is not true at $t$. Contradiction.

  The argument for (Con2) is analogous.
\end{solution}

% \begin{exercise}
%   Explain why \pr{Con1} and \pr{Con2} could be equivalently expressed as
%   $\tF\tH A \to A$ and $\tP\tG A \to A$.
% \end{exercise}

\begin{exercise}
  Give K$_{t}$-tree proofs for the following schemas.
  \begin{exlist}
    \item $A \to \tG \tP A$
    \item $A \to \tH \tF A$
    \item $\tF A \to \tH\tF\tF A$ 
    \item $\tP\tG A \to \tP\tF A$ 
    \item $\tH A \leftrightarrow \tH\tF\tH A$ 
  \end{exlist}
\end{exercise}
\begin{solution}
  \begin{sollist}
    
    \item \tree[4]{%
        \nnode{19}{1.}{$\neg(A \to \tG\tP A)$}{t}{(Ass.)} \\
        \nnode{19}{2.}{$A$}{t}{(1)} \\
        \nnode{19}{3.}{$\neg \tG\tP A$}{t}{(1)}  \\
        \nnode{19}{4.}{$t<s$}{}{(3)}  \\
        \nnode{19}{5.}{$\neg \tP A$}{s}{(3)}  \\
        \nnodeclosed{19}{6.}{$\neg A$}{t}{(4,5)}  \\
    }
    \medskip

    \item \tree[4]{%
        \nnode{19}{1.}{$\neg(A \to \tH\tF A)$}{t}{(Ass.)} \\
        \nnode{19}{2.}{$A$}{t}{(1)} \\
        \nnode{19}{3.}{$\neg \tH\tF A$}{t}{(1)}  \\
        \nnode{19}{4.}{$s<t$}{}{(3)}  \\
        \nnode{19}{5.}{$\neg \tF A$}{s}{(3)}  \\
        \nnodeclosed{19}{6.}{$\neg A$}{t}{(4,5)}  \\
   }
    \medskip

   \item \tree[4]{%
        \nnode{24}{1.}{$\neg(\tF A \to \tH\tF\tF A)$}{t}{(Ass.)} \\
        \nnode{24}{2.}{$\tF A$}{t}{(1)} \\
        \nnode{24}{3.}{$\neg \tH\tF\tF A$}{t}{(1)}  \\
        \nnode{24}{4.}{$s<t$}{}{(3)}  \\
        \nnode{24}{5.}{$\neg \tF\tF A$}{s}{(3)}  \\
        \nnodeclosed{24}{6.}{$\neg \tF A$}{t}{(4,5)}  \\
      }
    \medskip

    \item \tree[4]{%
        \nnode{22}{1.}{$\neg(\tP\tG A \to \tP\tF A)$}{t}{(Ass.)} \\
        \nnode{22}{2.}{$\tP\tG A$}{t}{(1)} \\
        \nnode{22}{3.}{$\neg \tP\tF A$}{t}{(1)}  \\
        \nnode{22}{4.}{$s<t$}{}{(2)}  \\
        \nnode{22}{5.}{$\tG A$}{s}{(2)}  \\
        \nnode{22}{6.}{$A$}{t}{(4,5)}  \\
        \nnode{22}{7.}{$\neg \tF A$}{s}{(3,4)}  \\
        \nnodeclosed{22}{8.}{$\neg A$}{t}{(4,7)}  \\
    }
    \medskip
      
   \item \tree[4]{%
        & \bnode{24}{1.}{$\neg(\tH A \leftrightarrow \tH\tF\tH A)$}{t}{(Ass.)} & \\
        &&\\
        \nnode{15}{2.}{$\tH A$}{t}{(1)} && \nnode{15}{4.}{$\neg \tH A$}{t}{(1)} \\
        \nnode{15}{3.}{$\neg\tH\tF\tH A$}{t}{(1)} && \nnode{15}{5.}{$\tH\tF\tH A$}{t}{(1)} \\
        \nnode{15}{6.}{$s<t$}{}{(3)} && \nnode{15}{9.}{$s<t$}{}{(4)} \\
        \nnode{15}{7.}{$\neg\tF\tH A$}{s}{(3)} && \nnode{15}{15.}{$\neg A$}{s}{(4)} \\
        \nnodeclosed{15}{8.}{$\neg\tH A$}{t}{(6,7)} && \nnode{15}{11.}{$\tF\tH A$}{s}{(5,9)} \\
        && \nnode{15}{12.}{$s<r$}{}{(11)} \\
        && \nnode{15}{13.}{$\tH A$}{r}{(11)} \\
        && \nnodeclosed{15}{14.}{$A$}{s}{(12,13)} \\
   }
   
  \end{sollist}
  
\end{solution}

For most applications, K$_{t}$ is too weak. We will want to impose further
restrictions on the relevant temporal models. For example, definition
\ref{def:temporalmodel} allows for cases in which $t < s$ and $s < r$ without
$t < r$. But if a time $t$ is earlier than $s$, and $s$ is earlier than $r$,
then surely $t$ must be earlier than $r$. For almost every application of
temporal logic, we assume that the precedence relation is transitive. This
corresponds to the \pr{4}-schema for $\tG$. It also corresponds to the
\pr{4}-schema for $\tH$.
\begin{principles}
  \pri{4G}{\tG A \to \tG\tG A}\\
  \pri{4H}{\tH A \to \tH\tH A}
\end{principles}

\begin{exercise}
  Explain why, if a relation $<$ is transitive, then so is its converse. The
  converse $>$ of $<$ is the relation that holds between $x$ and $y$ iff $y<x$.
\end{exercise}
\begin{solution}
  Suppose < is transitive, and $x > y$ and $y > z$. Equivalently, $y < x$ and
  $z < y$.By transitivity of <, we have $z < x$. So $x > z$.
\end{solution}

Another plausible condition is that no time is earlier than itself. Formally,
$<$ should be \emph{irreflexive}, so that no element of $T$ is $<$-related to
itself. We know that reflexivity corresponds to the \pr{T}-schema, whose
(forward-looking) temporal analogue would be $\tG A \to A$. What corresponds to
irreflexivity? The following observation reveals the answer: nothing.

\begin{observation}{noirrefl}
  A sentence is valid in the class of irreflexive frames iff it is valid in the
  class of all frames.
\end{observation}
%
\begin{proof}
  \emph{Proof sketch:} The right-to-left direction is obvious. The left-to-right direction is implied by the answer to exercise \ref{ex:acyclical}.
  % In that exercise we showed that every sentence that is satisfiable in a
  % cyclical model is satisfiable in an acyclical model. Suppose X is true at
  % all worlds in all irreflexive models, but that it is not true at all worlds
  % in all reflexive models. So ~X is true at some world in some reflexive
  % model, but not at any world in any irreflexive model. This contradicts the
  % exercise (because every acyclical model is irreflexive).
  But we can give a more direct argument.

  Suppose that some sentence $A$ is not valid in the class of all frames. We
  show that $A$ is not valid in the class of irreflexive frames. That $A$ is not
  valid in the class of all frames means that there is some world $w$ in some
  model $M = \t{W,R,V}$ at which $A$ is false. We will show that there is some
  world in some irreflexive model at which $A$ is false.

  To this end, we will construct an irreflexive model $M^i = \t{W',R',V'}$ from
  $M$ in which the same sentences are true at $w$ as in $M$. Since $A$ is true
  at $w$ in $M$, it follows that $A$ is true at $w$ in $M^i$.

  Initially, $M^i$ has the same worlds, the same accessibility relation, and the
  same interpretation function as $M$. Now for any world $w$ in $M$ that can see
  itself, we add a new world $w'$ to $M^i$ so that
  % 
  \begin{itemize}[leftmargin=10mm]
    \itemsep-1mm
    \item $w'$ verifies the same sentence letters as $w$: if $w \in V(P)$ then $w' \in V(P)$;
    \item $w'$ can see the same worlds as $w$: whenever $wR'v$ then $w'R'v$; and
    \item $w'$ can be seen from the same worlds as $w$: whenever $vR'w$ then
          $vR'w'$.
  \end{itemize}
  % Since $w$ can see itself, this means that $wR'w'$ and $w'R'w$.
  Finally, we make $w$ inaccessible from itself in $M^i$. A simple proof
  by induction on complexity shows that if a sentence is true at a
  world $w$ in $M$ then it is also true at $w$ in $M^i$.
  \qed

  % Can we adjust this proof for an arbitrary class of frames? I.e., $A$ is
  % valid in the class of irreflexive C-frames iff it is valid in all
  % C-frames. No: If $A$ is false at some world in some C-model $M$, we can
  % construct an irreflexive model $M^i = \t{W',R',V'}$ from $M$ in which the
  % same sentences are true at $w$ as in $M$.  But we also need to show that
  % $M^i$ still satisfies the condition C. That won't be the case if C is, for
  % example, the condition that there is exactly one world, or that now world
  % can see more than one world.
\end{proof}

% Observation \ref{obs:noirrefl} tells us that there is no modal principle that is
% valid in all and only the irreflexive frames. So the logic of irreflexive frames
% is the same as logic of all frames. The proof carries over to many
% other classes of frames. For example, the logic of irreflexive and transitive
% frames is the same as the logic of transitive frames (namely, K4).

% We can easily add a tree rule for irreflexivity: simply allow any branch to be
% closed that contains $\omega < \omega$. But while that rule may help to find
% irreflexive countermodels, it won't allow us to prove anything we couldn't
% prove without the rule.

% From Humberstone 84: Why is the class of universal frames not modally
% definable? Take any two universal frames. Their disjoint union is not
% universal. Similar arguments show that the class of finite frames is not
% definable, nor is the class of frames in which each world is accessible from
% some world. We could here mention that we've implicitly seen something like
% this in ch.3, where we saw that a sentence is valid in universal frames iff it
% is valid in equivalence frames. It follows that there's no sentence that
% corresponds to universal frames.

Given transitivity, irreflexivity is closely related to asymmetry. Recall from
the previous chapter that $<$ is asymmetric if whenever $t<s$ then not $s<t$.
There is no modal schema that corresponds to asymmetry.

\begin{exercise}\label{ex:partialorder}
  Show that a transitive relation is irreflexive iff it is asymmetric.
\end{exercise}
\begin{solution}
  Suppose $R$ is transitive. If there are points $x$ and $y$ for which $xRy$ and
  $yRx$ then $xRx$ by transitivity. So if $R$ isn't asymmetric then it isn't
  irreflexive. If $R$ isn't irreflexive then there is a point $x$ with $xRx$.
  This violates asymmetry, because asymmetry demands that if $xRx$ then not
  $xRx$.
\end{solution}

\begin{exercise}
  A popular idea in many cultures is that time is circular. Does this
  cast doubt on asymmetry? What about irreflexivity? 
\end{exercise}
\begin{solution}
  If time is transitive and circular, then it is neither asymmetric nor
  irreflexive.
\end{solution}

\begin{wrapfigure}{r}{3cm}
  \quad
  \begin{tikzpicture}[modal, world/.append style={minimum size=0.5cm}]
    \node[world] (w1) [label=above:{$t$}] {};
    \node[world] (w2) [label=above:{$s$}, above right=5mm and 15mm of w1] {};
    \node[world] (w3) [label=below:{$r$}, below right=5mm and 15mm of w1] {};
    \draw[->] (w1) -- (w2);
    \draw[->] (w1) -- (w3);
  \end{tikzpicture}
\end{wrapfigure}
In the previous chapter, I mentioned that transitive and irreflexive relations
are called (strict) partial orders. The name reflects the fact that such orders
need not order everything. In a model of branching time, for example, we can
have $t<s$ and $t<r$ but neither $s<r$ nor $r<s$; in that case, $r$ and $s$ are
not ordered by the precedence relation.

We can rule out such cases by imposing the requirement of
\textbf{connectedness}, also known as \emph{completeness} or \emph{totality}.
This demands that for any points $t$ and $s$, either $t < s$ or $t=s$ or
$s < t$.  An irreflexive, transitive, and connected relation is called a
\textbf{(strict) linear order} (or a \emph{strict total order}).

For some applications, we may want linearity in only one direction. Many
philosophers have been attracted to a branching-future conception of time,
where a point in time may have more than one future, but only one past. In
such models, we would only require \textbf{left-linearity}: that \emph{if
  $s < t$ and $r < t$}, then either $s < r$ or $s=r$ or $r < s$.

An axiom schema corresponding to left-linearity is \pr{LL}:
%
\principle{LL}{\tF\tP A \to (\tF A \lor A \lor \tP A)}
%
Right-linearity -- the assumption that if $t < s$ and $t < r$, then either
$s < r$ or $s=r$ or $r < s$ -- corresponds to \pr{RL}:
%
\principle{RL}{\tP\tF A \to (\tP A \lor A \lor \tF A)}
%
% These are from Mueller, p.333. HC (143) use
% $\tG((A \land \tG A) \to B) \lor \tG((B \land \tG B) \to A)$, which looks like
% a redundant disjunction; but apparently this really is the axiom Lem_0.
%
The conjunction of \pr{LL} and \pr{RL} is valid on a frame iff the frame's
precedence relation does not branch in either direction. This is not quite the
same as connectedness, because it allows for frames with parallel time lines.
There is no schema that corresponds to connectedness.
% Gore mentions this. It seems obvious.

% Humberstone 77: K4.3 is determined by the class of transitive, irreflexive,
% and connected frames. .3 is []([]A & A) -> B) v []([]B->A). Humberstone 195
% says this disallows branching.

The tree rules for left-linearity and right-linearity directly reflect the
definition of the two properties.
% These rules are from Priest. Girle's rules are wrong.
\bigskip
\begin{center}

  \begin{minipage}[t]{0.4\textwidth} \centering
    Left-Linearity
    
    \tree[2]{
      & \barenode{$\nu < \omega$} & \\
      & \dotbelowbaretribnode{$\upsilon < \omega$} & \\
      && \\
      && \\
      \barenode{$\nu < \upsilon$} & \barenode{$\nu = \upsilon$} & \barenode{$\upsilon < \nu $}
    }
  \end{minipage}
  \begin{minipage}[t]{0.4\textwidth} \centering
    Right-Linearity

    \tree[2]{
      & \barenode{$\omega < \nu$} & \\
      & \dotbelowbaretribnode{$\omega < \upsilon$} & \\
      && \\
      && \\
      \barenode{$\nu < \upsilon$} & \barenode{$\nu = \upsilon$} & \barenode{$\upsilon < \nu $}
    }
  \end{minipage}
\end{center}
\bigskip%
These rules create \emph{three} branches. They also create ``identity nodes'' of
the form $\nu = \upsilon$, stating that two world/time labels refer to the same
thing. (This must be taken into account when we read off a countermodel from an
open branch.) We need two further rules to deal with identity nodes. Both of
these rules are called `Identity'.%
\bigskip
\begin{center}
  \begin{minipage}[t]{0.3\textwidth} \centering
    
    \tree[2]{
      \nnode{10}{}{$A$}{\omega}{} \\
      \dotbelowbarenode{$\omega = \nu$} \\
      && \\
      \nnode{10}{}{$A$}{\nu}{}
    }
  \end{minipage}
  \begin{minipage}[t]{0.3\textwidth} \centering

    \tree[2]{
      \nnode{10}{}{$A$}{\omega}{} \\
      \dotbelowbarenode{$\nu = \omega$} \\
      && \\
      \nnode{10}{}{$A$}{\nu}{}
    }
  \end{minipage}
\end{center}

\begin{exercise}
  Use the tree method to check which of the following sentences are valid,
  assuming time is linear (i.e., using the Transitivity, Left-Linearity,
  Right-Linearity, and Identity rules).
  \begin{exlist}
  \item $(\tF p \land \tF q) \to \tF(p \land q)$
  \item $\tP\tG\tG p \to \tG\tG p$
  \item $\tP\tF p \to (\tP p \lor (p \lor \tF p))$
  \item $\tP\tH p \to \tH p$
  \item $\tF\tG p \to \tG\tF p$
  \item $\tF (\tG q \land \neg p) \to \tG(p \to (\tG p \to q))$ 
  % \item $(\tF p \land \tF q) \to (\tF(p \land q) \lor (\tF (p \land \tF q) \lor \tF(\tF p \land q)))$ % add
  \end{exlist}
\end{exercise}
\begin{solution}
  (a), (d), and (e) are invalid. Here are trees for (b), (c), and (f):

  \bigskip
  \begin{sollist}

    \item[(b)] \tree[4]{%
        \nnode{25}{1.}{$\neg(\tP\tG\tG p \to \tG\tG p)$}{t}{(Ass.)} \\
        \nnode{25}{2.}{$\tP\tG\tG p$}{t}{(1)} \\
        \nnode{25}{3.}{$\neg \tG\tG p$}{t}{(1)}  \\
        \nnode{25}{4.}{$s<t$}{}{(2)}  \\
        \nnode{25}{5.}{$\tG\tG p$}{s}{(2)}  \\
        \nnode{25}{6.}{$t<r$}{}{(3)}  \\
        \nnode{25}{7.}{$\neg \tG p$}{r}{(3)}  \\
        \nnode{25}{8.}{$s<r$}{}{(3,6)}  \\
        \nnodeclosed{25}{9.}{$\tG p$}{r}{(5,8)}  \\
    }
    \medskip

    \item[(c)] \tree[8]{%
        & \nnode{35}{1.}{$\neg(\tP\tF p \to (\tP p \lor (p \lor \tF p)))$}{t}{(Ass.)} & \\
        & \nnode{35}{2.}{$\tP\tF p$}{t}{(1)} & \\
        & \nnode{35}{3.}{$\neg(\tP p \lor (p \lor \tF p))$}{t}{(1)} & \\
        & \nnode{35}{4.}{$\neg \tP p$}{t}{(3)} & \\
        & \nnode{35}{5.}{$\neg (p \lor \tF p)$}{t}{(3)} & \\
        & \nnode{35}{6.}{$\neg p$}{t}{(5)} & \\
        & \nnode{35}{7.}{$\neg \tF p$}{t}{(5)} & \\
        & \nnode{35}{8.}{$s < t$}{}{(2)} & \\
        & \nnode{35}{9.}{$\tF p$}{s}{(2)} & \\
        & \nnode{35}{10.}{$s < r$}{}{(9)} & \\
        & \tribnode{35}{11.}{$p$}{r}{(9)} & \\
        && \\
        \nnode{10}{12.}{$t<r$}{}{} &    \nnode{10}{13.}{$t=r$}{}{} & \nnode{10}{14.}{$r<t$}{}{} \\
        \nnodeclosed{10}{15.}{$\neg p$}{r}{(7,12)} &  \nnodeclosed{10}{16.}{$\neg p$}{r}{(6,13)} & \nnodeclosed{10}{17.}{$\neg p$}{r}{(4,16)}  \\
    }
    \medskip

    \item[(f)] \tree[8]{%
      & \nnode{38}{1.}{$\neg(\tF (\tG q \land \neg p) \to \tG(p \to (\tG p \to q)))$}{t}{(Ass.)} & \\
      & \nnode{38}{2.}{$\tF (\tG q \land \neg p)$}{t}{(1)} & \\
      & \nnode{38}{3.}{$\neg\tG(p \to (\tG p \to q))$}{t}{(1)} & \\
      & \nnode{38}{4.}{$t < s$}{}{(2)} & \\
      & \nnode{38}{5.}{$\tG q \land \neg p$}{s}{(2)} & \\
      & \nnode{38}{6.}{$\tG q$}{s}{(5)} & \\
      & \nnode{38}{7.}{$\neg p$}{s}{(5)} & \\
      & \nnode{38}{8.}{$t < r$}{}{(3)} & \\
      & \nnode{38}{9.}{$\neg(p \to (\tG p \to q))$}{r}{(3)} & \\
      & \nnode{38}{10.}{$p$}{r}{(9)} & \\
      & \nnode{38}{11.}{$\neg(\tG p \to q)$}{r}{(9)} & \\
      & \nnode{38}{12.}{$\tG p$}{r}{(11)} & \\
      & \tribnode{38}{13.}{$\neg q$}{r}{(11)} & \\
      && \\
      \nnode{10}{14.}{$s<r$}{}{} &    \nnode{10}{15.}{$s=r$}{}{} & \nnode{10}{16.}{$r<s$}{}{} \\
      \nnodeclosed{10}{17.}{$q$}{r}{(6,14)} &  \nnodeclosed{10}{18.}{$p$}{s}{(10,15)} & \nnodeclosed{10}{19.}{$p$}{s}{(12,16)}  \\
    }
    \medskip
  \end{sollist}
\end{solution}

% While we can define non-branchingness, we can't define branchingness.

The precedence relation in relativistic spacetime is neither left-linear nor
right-linear. But it has a weaker property: convergence.
%
% Haven't really talked about convergence so far. Draw picture like in HC 134?
%
A spacetime point $p_1$ can precede two points $p_2$ and $p_3$ neither of which
precedes the other, but these two points will always precede a common later
point $p_4$. Convergence corresponds to the \pr{G}-schema. In temporal
logic, we have one \pr{G}-schema for future convergence and one for past
convergence:
%
\begin{principles}
\pri{FG}{\tF\tG A \to \tG\tF A}\\
\pri{PG}{\tP\tH A \to \tH\tP A}
\end{principles}

% Goldblatt proved that S4.2 is the temporal logic of diodorean necessity in
% Minkowski spacetime. Diodorean has F mean 'true from now on'. Is 0.2 implied by
% no branching?

% The logic of Minkowski spacetime with an irreflexive relation is D4.2+(<>p \land
% <>q \to <>(<>p \land <>q)), see \cite[955]{kracht07logically}.

\begin{exercise}
  Can you find schemas that correspond to the following frame properties?
  \begin{exlist}
  \item There is no last time. (That is, every time precedes some time.)
  \item There is no first time. 
  \item There is a last time.
  \item There is a first time.
  \end{exlist}
% HC 131 say that []<>p & []<>q -> <>(p v q) characterises the idea that time
% has an end, or rather: that every point can see a final point. In the presence
% of S4, they say this simplifies to the McKinsey axiom []<>p -> <>[]p.
\end{exercise}
\begin{solution}
  \begin{sollist}
    \item For example, $\tG A \to \tF A$.
    % Suppose there's a last time $t$. Let $p$ be true there. Then $\tG p \to \tF p$ is false at $g$. Conversely, suppose there's no last time and $\tG A$ is true at some time. Then $\tF A$ is true there as well.
  \item For example, $\tH A \to \tP A$.
  \item No schema corresponds to the class of frames with a last time. If we
  also assume transitivity and quasi-connected (see page
  \pageref{quasiconnected}), then
  $\tG(A \land \neg A) \lor \tF\tG(A \land \neg A)$ works.
  % Mueller and Garanko and Humberstone say this. (But why does Garanko say
  % "assuming irreflexivity"?) Girle says FA -> F-FA = GA v FGA. But that
  % doesn't seem right. Without transitivity, there can be a last time, but
  % $G\bot v FG\bot$ is false at any world that's at least two steps away from a
  % last time. Transitivity gets around this. But it's not enough: consider a
  % frame with one eternal time and a disconnected lonely further time t. Here
  % we have a last time, but $G\bot v FG\bot$ is false at all times on the line.
  \item No schema corresponds to the class of frames with a first time. If we
    also assume transitivity and quasi-connectedness, then
    $\tH(A \land \neg A) \lor \tP\tH(A \land \neg A)$ works.
  \end{sollist}
\end{solution}
\vspace{-2mm}

\begin{exercise}
  Show that the schema $\tF A \to \tF\tF A$ corresponds to density. (You
  have to show that (a) whenever a frame is dense then $\tF A \to \tF\tF A$ is
  valid on the frame, and (b) whenever $\tF A \to \tF\tF A$ is valid on a frame
  then the frame is dense.)
\end{exercise}
\begin{solution}
  Assume a frame is dense. Suppose for reductio that some instance of
  $\tF A \to \tF\tF A$ is false at some point $t$ in some model $M$ based on
  that frame. Then $\tF A$ is true at $t$ and $\tF\tF A$ is false. Since $\tF A$
  is true at $t$, it follows by definition \ref{def:temporalsemantics} that $A$
  is true at some point $s$ such that $t<s$. By density, there is a point $r$
  such that $t<r<s$. But since $A$ is true at $s$, $\tF A$ is true at $r$, and
  so $\tF\tF A$ is true at $t$; contradiction.

  In the other direction, we have to show that if a frame isn't dense then some
  instance of $\tF A \to \tF\tF A$ is false at some point $t$ in some model $M$
  based on that frame. We take the simplest instance $\tF p \to \tF\tF p$. If a
  frame isn't dense then there are points $t,s$ such that $t<s$ and no point
  lies in between $t$ and $s$. Let $V$ be an interpretation function that makes
  $p$ true at $s$ and false everywhere else. Then $\tF p$ is true at $t$ but
  $\tF\tF p$ is false. So $\tF p \to \tF\tF p$ is false at $t$.
\end{solution}
\vspace{-2mm}

\begin{exercise}
  Can you find an $\L_T$-expression stating that $p$ is true at all times? Can
  you do so if you make assumptions about the precedence relation?
\end{exercise}
\begin{solution}
  Without assumptions about the flow of time there is no way to express in
  $\L_T$ that $p$ is true at all times (or at some time). In linear flows,
  $p \land \tH p \land \tG p$ does the job. 
  % Does $GA\land HGA$ work with negatively transitive time (no parallel lines)?
\end{solution}

% Adding \pr{LL} and \pr{RL} yields a sound and complete logic of linear time.
% The completeness proof is non-trivial because the canonical model is not
% linear. We need techniques such as bulldozing. See Venema p.10f.

% \begin{exercise}
%   Can you find a frame condition that corresponds to the \pr{GL}-schema
%   $\Box(\Box A \to A) \to \Box A$? (Hints: (a) In Kripke semantics, \pr{GL}
%   entails \pr{4}; (b) The dual of \pr{GL} is
%   $\Diamond A \to \Diamond(A \land \neg\Diamond A$.)
% \end{exercise}
% \begin{solution}
%   Informally, $\Diamond A \to \Diamond(A \land \neg\Diamond A$ says that if it
%   will ever be the case that $A$ then it will at some time be the case that $A$
%   for the last time. So time will come to an end.
% \end{solution}



\section{Branching time}\label{sec:branching}

In section \ref{sec:systems} we looked at the idea that the future is ``open''
while the past is ``settled'', insofar as we can still influence (say)
whether we will exercise tomorrow, but not whether we have exercised
yesterday. Some have argued that this calls for a non-linear model of time, with
multiple branches into the future. On one branch, we would exercise tomorrow, on
another we would not.

This line of thought appears to conflate temporal and modal considerations.
The precedence relation in models of time is normally understood as a purely
temporal relation -- as the earlier-later relation. The fact that we can bring
about a world in which we exercise tomorrow and a world in which we don't
exercise does not entail that both kinds of tomorrow take place here in the
actual world.

If we want to make explicit the connections between settledness and time, it is
better to use a multi-modal language with circumstantial operators for
settledness and openness in addition to the purely temporal operators
$\tF, \tG, \tP, \tH$. We could then say things like $\tP p \to \Box \tP p$ to
formalize the claim that if $p$ has happened then it is settled that $p$ has
happened.

% The accessibility relation for the box would have to hold fixed the past, so
% that a world $v$ is accessible from a world $w$ only if the past of $v$
% coincides with the past of $w$.

\begin{exercise}
  Suppose we endorse all instances of the schema (S1) $\tP A \to \Box \tP A$.
  Suppose we also endorse all instances of (S2)
  $\neg \!\tP A \to \Box \neg \!\tP A$, on the grounds that if something has
  failed to happen then there is nothing we can do that would make it have
  happened. Let's also assume that the present time is not the first, and that
  the box is closed under logical consequence, meaning that if $\Box A$ and
  $\Box B$ are true at a time, and $C$ is entailed by $A$ and $B$, then $\Box C$
  is true (at the time) as well. Show that we can then derive the fatalist
  conclusion that anything that never actually happen is settled to never 
  happen: all instances of $(\neg A \land \neg \tP A \land \neg \\to \Box \neg \tF A$ are true. (Hint:
  use instances of (S1) and (S2) in which $A$ is a statement about the future.)
\end{exercise}
\begin{solution}
  Suppose $\neg \tP A \land \neg A \land \neg \tF A$ is true at the present time
  $t$. Then $\neg \tP\tF A$ is true (at $t$). By (S2), we can infer
  $\Box \neg \tP\tF A$. But $\neg \tP\tF A$ $K_{t}$-entails
  $\neg(\tF A \land \tP(A \lor \neg A))$. Since the box is closed under logical
  consequence, this means that $\Box \neg(\tF A \land \tP(A \lor \neg A))$ is
  true at $t$. Since $t$ is not the first time, $\tP(A \lor \neg A)$ is true at
  $t$, and so $\Box \tP(A \lor \neg A)$ is true at $t$ as well, by (S1). have
  $\neg(\tF A \land \tP(A \lor \neg A))$ and $\tP(A \lor \neg A)$ together
  entail have $\neg \tF A$. Since the box is closed under logical consequence,
  it follows that $\Box \tF A$ is true at $t$.
\end{solution}

% Exercise: multi-modal time travel -- can we change the past? Can you kill your
% Grandfather? We'll see that in the worlds where you do, the person you kill is
% not your grandfather. Calling the person your grandfather is smuggling in facts
% about the future.

There are nonetheless good reasons to consider branching models of time. I already
mentioned that such models are widely used in computer science, where
the ``times'' represent states of a computational process and the precedence
relation has a semi-modal interpretation, holding between two states iff the
first can lead to the second.
% (The relevant logics of branching time are called ``computational tree
% logics'').
I also mentioned that the precedence relation in relativistic spacetime allows
for branching, although diverging spacetime branches ultimately reconverge. A
more classical form of branching (without reconvergence) has been argued to
follow from the so-called ``Everett interpretation'' of quantum physics. On this
interpretation, what are normally understood to be chance events are really
branching events in which all possible outcomes actually take place.

Another way to motivate a branching conception of time arises from a
metaphysical view called \emph{presentism}. According to presentism, only the
present is real; all truths that seem to concern other times are reducible to
more fundamental truths about the present. If, for example, it is true that
there was a sea battle yesterday, then according to presentism this must
ultimately be explained by what is true \emph{now}; there must be facts about
the present state of the world that entail (and explain) yesterday's sea battle.
Different forms of presentism disagree over what the relevant facts about the
present might be. On one view, they are particular facts about the distribution
of physical particles and fields etc.\ together with the general laws of nature.
If the laws of nature are deterministic, then the complete truth about the
present distribution of particles and fields etc.\ together with the laws fixes
all truths about the past and about the future. But suppose the laws are
indeterministic towards the future: they merely settle that if the present
physical state of the world is so-and-so, then the future is \emph{either like
  this or like that}. In that case, the presentist will regard both of these
futures as equally actual.

Let's assume, then, that we want to reason about branching time. This is less
straightforward than it might at first appear.

The models we are interested in are not right-linear. I will, however, assume
that they satisfy the following weaker property of
\textbf{quasi-connectedness}:\label{quasiconnected}
\[
  \text{if $t < s$ then for any $r$, either $t < r$ or $r<s$}.
\]
% I've made up that label.
Quasi-connectedness is more often called \emph{negative transitivity}, because
it is equivalent to the assumption that if $t \not< s$ and $s\not<r$ then
$t\not<r$. It slightly simplifies our models, for example by ruling out entirely
disconnected parallel time lines.

Two pieces of terminology will be useful. First, let's define a \textbf{history}
in a model $\t{T,<,V}$ as a maximal linearly ordered subset of $T$. That is, a
history is a collection of times $H$ such that
\begin{itemize}[leftmargin=10mm]
  \itemsep0mm
\item[(i)] for all $t$ and $s$ in $H$, either $t<s$ or $t=s$ or $s<t$, and
\item[(ii)] no further member of $T$ could be added to $H$ without
  making (i) false.
\end{itemize}
%
\noindent
\begin{wrapfigure}{r}{4cm}
  \vspace{-8mm}
  \quad
  \begin{tikzpicture}[modal, world/.append style={minimum size=0.5cm}]
    \node[world] (w1) [label=above:{$t_1$}] {};
    \node[world] (w2) [label=above:{$t_2$}, right=10mm of w1] {};
    \node[world] (w3) [label=above:{$t_3$}, above right=5mm and 10mm of w2] {};
    \node[world] (w4) [label=below:{$t_4$}, below right=5mm and 10mm of w2] {};
    \draw[->] (w1) -- (w2);
    \draw[->] (w2) -- (w3);
    \draw[->] (w2) -- (w4);
  \end{tikzpicture}
  \vspace{-10mm}
\end{wrapfigure}

\noindent
The model (or rather, frame) depicted on the right contains two
histories: $\{ t_1, t_2, t_3 \}$ and $\{ t_1, t_2, t_4 \}$.

For the second piece of terminology, let $t$ be any time in any model. Any
maximal linearly ordered set of times \emph{later than $t$} will be called a
\textbf{future of $t$}. In the model on the right, $t_1$ has two futures:
$\{ t_2, t_3 \}$ and $\{ t_2, t_4 \}$.

If you look back at definition \ref{def:temporalsemantics}, you can see that in
the standard semantics for temporal logic, $\tG p$ is true at $t$ iff $p$ is
true at all times \emph{in all futures of $t$}; $\tF p$, on the other hand, is
true at $t$ iff $p$ is true at some time \emph{in at least one future of $t$}.
This ensures that $\tG$ and $\tF$ are duals, but it is often thought to be
problematic if we want $\tF p$ to translate `it will be the case that $p$'.

To illustrate, suppose I'm about to toss a coin. In one future (let's assume),
the coin will land heads, in another it will land tails. By definition
\ref{def:temporalsemantics}, both $\tF h$ and $\tF t$ are true. But should we
say that the coin will land heads and also that it will land tails?

%Which of Con1 and Con2 should we reject?

We could adopt an alternative semantics for $\tF$ according to which $\tF p$
is true at $t$ iff $p$ is true at some time in \emph{all} futures of $t$:
%
\begin{quote}
  $M,t \models \tF A \;\text{iff every future of $t$ contains some $s$ such that $M,s \models A$}.$
\end{quote}
This is known as the \textbf{Peircean interpretation} of $\tF$ (after
Charles S.\ Peirce; the name is due to Arthur Prior).

On the Peircean account, $\tF p$ is false whenever $p$ only takes place in one
of several futures. If we keep the classical interpretation of $\tG$, both
$\tF p$ and $\tG \neg p$ can be false; the two operators are no longer duals.
The dual of $\tF$ is a strange operator that applies to a sentence $A$ iff there
is \emph{some} future in which $A$ is always true.

\begin{exercise}
  Explain why the Peircean interpretation renders $p \to \tH\tF p$, an instance
  of \pr{Con2}, invalid.
\end{exercise}
\begin{solution}
  Consider a model with three times ordered by $s<t$ and $s<r$. Assume $p$ is true at $t$ and not at $r$. Then $p \to \tH\tF p$ is false on the Peircean interpretation.
\end{solution}

A rather different approach is taken by (what Prior called) the
\textbf{Ockhamist} approach. According to Ockhamism, if there are several
futures then it doesn't make sense to say -- without qualification -- that $p$
will be the case, or that $p$ won't be case. To talk about what will or won't be
the case we must specify which future we have in mind.

Formally, in Ockhamist semantics, the truth-value of every sentence is evaluated
at a pair consisting at a time and a history. Histories are linear by
definition, so the problems raised by multiple futures disappear. To say that
$p$ is the case in \emph{some} history, or in \emph{all} histories, Ockhamists
add new operators $\Diamond$ and $\Box$ that quantify over histories. The
Peircean $\tF$ operator is equivalent to $\Box \tF$ in Ockhamism. $\Box \tF p$
says that every future contains a time at which $p$ is true; $\Diamond Fp$, by
contrast, would say that some future contains a time which $p$ is true.

Here is the full Ockhamist semantics.

\begin{definition}{Ockhamist Semantics}{ockhamistsemantics}
  If $M = \t{T,<,V}$ is a temporal model, $H$ is a history in $M$, $t$
  is a member of $H$, $P$ is any sentence letter, and $A,B$ are any
  sentences in the Ockhamist language, then

  \medskip
  \begin{tabular}{lll}
    (a) & $M,H,t \models P$ &iff $t$ is in $V(P)$.\\
    (b) & $M,H,t \models \neg A$ &iff $M,H,t \not\models A$.\\
    (c) & $M,H,t \models A \land B$ &iff $M,H,t \models A$ and $M,H,t \models B$.\\
    (d) & $M,H,t \models A \lor B$ &iff $M,H,t \models A$ or $M,H,t \models B$.\\
    (e) & $M,H,t \models A \to B$ &iff $M,H,t \not\models A$ or $M,H,t \models B$.\\
    (f) & $M,H,t \models A \leftrightarrow B$ &iff $M,H,t \models (A\to B)$ and $M,H,t \models (B\to A)$.\\
    (g) & $M,H,t \models \tF A$ &iff $M,H,s \models A$ for some $s$ in $ H$ such that $t<s$.\\
    (h) & $M,H,t \models \tG A$ &iff $M,H,s \models A$ for all $s$ in $ H$ such that $t<s$.\\
    (i) & $M,H,t \models \tP A$ &iff $M,H,s \models A$ for some $s$ in $ H$ such that $s<t$.\\
    (j) & $M,H,t \models \tH A$ &iff $M,H,s \models A$ for all $s$ in $ H$ such that $s<t$.\\
    (k) & $M,H,t \models \Box A$ &iff $M,J,t \models A$ for all histories $J$ that contain $t$.\\
    (l) & $M,H,t \models \Diamond A$ &iff $M,J,t \models A$ for some history $J$ that contains $t$.
  \end{tabular}
\end{definition}

% ``The Ockhamist semantics for temporal logic was intuitively conceived by Prior
% but formally developed later, in Burgess (1979)''

A sentence is \emph{valid} in Ockhamist semantics if it is true at all times $t$
on all histories $H$ (containing $t$) in all models. As always, we can get
stronger conceptions of validity -- stronger logics -- by adding further
constraints on the precedence relation.

% Axiomatic systems for many such concepts of validity have been found,
% but I am not aware of any provably complete tree method.
% for ax.systems see \cite{kracht07logically}, p.958, who cite Zanardo 1985 and
% Reynolds 2003.

\begin{exercise}
  Which of the following schemas are valid in Ockhamist semantics?
  \begin{exlist}
  \item $\Box A \to A$
  \item $\Box A \to \Box\Box A$
  \item $\Diamond A \to \Box\Diamond A$
  \item $\Box \tF A \to \tF\Box A$
  \item $\tP A \to \Box \tP \Diamond A$%
  \end{exlist}
\end{exercise}
\begin{solution}
  (a)--(d) are valid, (e) is invalid.

  To show that a schema is valid, assume for reductio that there is some time
  $t$ on some history $H$ in some model $M$ at which the schema is false. Then
  (repeatedly) use definition \ref{def:ockhamistsemantics} to derive a
  contradiction.

  For (e), consider a model with three times $t,s,r$ such that $s\prec t$,
  $r\prec t$, and neither $s \prec r$ nor $r\prec s$. Let $q$ be true at $s$ and
  false at the other two times. $\tP q \to \Box \tP \Diamond q$ is false at $t$
  on the history $\t{s,t}$.
\end{solution}

There is something odd about the Ockhamist approach. Consider a scenario in
which there are multiple futures; one future holds a sea battle, another holds
no sea battle. Let $p$ translate `there is a sea battle'. Is $\tF p$ is true in
this scenario (under the given interpretation of $p$)? What about
$\tF(p \lor \neg p)$? Or $Gp \to GGp$?

Ockhamism refuses to give an answer. In Ockhamism, sentences are only true or
false relative to a model and a time \emph{and a history}. A branching-time
scenario, however, does not fix a particular history. We'd like to know which
sentences are true today if there are multiple futures. Ockhamism only tells us
which sentences are true relative to each of the different futures. Relative to
a history that contains a sea battle, $\tF p$ is true. Relative to other
histories, $\tF p$ is false.

% In almost every approach to modal semantics, sentences are evaluated as true
% or false relative to a model and a certain \textbf{evaluation point} taken
% from the model. In standard Kripke semantics, for example, sentences are
% defined as true or false relative to a model and a world (see definition
% \ref{def:kripkesemantics}). Here the world is the evaluation point. In
% Ockhamist semantics, an evaluation point consists of a time and a history:
% definition \ref{def:ockhamistsemantics} settles which sentences are true
% relative to a model $M$, a history $H$, and a time $t$. The problem is that an
% intuitive brnaching-time scenario determines only a model and a time, but not
% a history.

If we insist that logical validity should formalize the idea of truth in all
scenarios under all interpretations of non-logical vocabulary then we can't
accept the official definition of validity in Ockhamist semantics. We have to
extend the Ockhamist semantics to specify under what conditions a sentence is
true \emph{in a model at a time}, without fixing a history. Then we can say that
a sentence is valid iff it is true at all times in all models.

A simple way to do this is to stipulate that a sentence is true at time in an
(Ockhamist) model iff it is true relative to \emph{all} histories that contain
the time:
\begin{align*}
  M,t \models A & \;\text{iff $M,H,t \models A$ for all histories $H$ that contain $t$}.
\end{align*}
This is known as a \textbf{supervaluationist} semantics.

Supervaluationism is often used when a formal semantics defines truth relative
to an ``extra'' parameter that doesn't correspond to any feature of a
conceivable scenario. In Ockhamist semantics, that parameter is $H$. For a
different application, consider vagueness. If $p$ translates `it is warm', and
the temperature is borderline warm, it is not clear what we should say about the
truth-value of $p$, and about various complex sentences containing $p$. One
popular approach to vagueness is to first define truth relative to a
\emph{sharpening} of vague expressions. Relative to a sharpening on which
temperatures above 15.0 degrees Celsius are warm, $p$ has a clear truth-value in
any conceivable scenario, as do complex sentences containing $p$. Since an
actual scenario does not fix a particular sharpening, this semantics contains an
extra parameter. We can define a notion of truth without that parameter by
saying that a sentence is true in a scenario iff it is true in that scenario
relative to every eligible sharpening.

Supervaluationist accounts tend to have some non-classical features. Suppose we
live in a branching world in which one future contains a sea battle and another
doesn't. Let $p$ express that a sea battle takes place. According to
supervaluationist Ockhamism, neither $\tF p$ nor $\neg \tF p$ is true in that
scenario. Both are true relative to some but not relative to all histories.
So neither is simply true. Assuming that a sentence is \emph{false}
if its negation is true, $\tF p$ is neither true nor false!

Logics in which a sentence can have a third status besides (mere) truth and
(mere) falsity are called \textbf{three-valued}. Three-valued approaches to
branching time are sometimes defended by the intuition that if a sea battle
occurs on some but not all branches of the future, then one can't truly assert
that a battle \emph{will} occur nor that it \emph{won't} occur.

The Polish logician Jan \polishL{}ukasiewicz argued that statements about the
future are either true, false, or ``indeterminate''. To accommodate this third
truth-value, he proposed three-valued truth-tables specifying how the
truth-value of complex sentences are determined by the truth-value of their
parts. For example, he suggested that if two sentences $A$ and $B$ are
indeterminate, then their conjunction $A \land B$, disjunction $A \lor B$, and
negations $\neg A, \neg B$ are also indeterminate.

In the sea battle scenario, \polishL{}ukasiewicz's account renders
$\tF s \,\lor \neg\! \tF s$ indeterminate, assuming $\tF s$ is indeterminate. This
is often regarded as problematic: even if we shouldn't assert that there will be
a sea battle, it is argued that we are justified to assert that there either
will or there won't be a sea battle. The supervaluationist form of Ockhamism,
while also three-valued, avoids this problem. On the supervaluationist
interpretation, $\tF s$ and $\neg \!\tF s$ are neither true nor false in the sea
battle scenario, but $\tF s \,\lor \neg \!\tF s$ is true.

\begin{exercise}
  Let's say that a sentence is \emph{super-valid} if it is true at all times in
  all models, where truth at a time in a model is understood in accordance with
  supervaluationist Ockhamism. Explain why the super-valid sentences are
  precisely the sentences that are valid by the original Ockhamist definition of
  validity (just below definition \ref{def:ockhamistsemantics}).
\end{exercise}
\begin{solution}
  A sentence $A$ is super-valid iff $M,t \models A$ for all temporal models $M$
  and times $t$ in $M$. By supervaluationism, this holds iff $M,H,t \models A$
  for all $M,t$, and histories $H$ containing $t$. That's how Ockhamist validity was originally defined.
\end{solution}
\vspace{-2mm}
\begin{exercise}
  Things are more complicated for entailment. Let's say that $A$
  \emph{Ockham-entails} $B$ iff there is no time on any history in any temporal
  model at which $A$ is true and $B$ false. Let's say that $A$
  \emph{super-entails} $B$ iff there is no time in any temporal model at which
  $A$ is true and $B$ false, where truth at a time in a model is defined in
  accordance with supervaluationism. Is Ockham-entailment equivalent to
  super-entailment? Explain.
\end{exercise}
\begin{solution}
  Ockham-entailment is stronger than super-entailment: whenever $A$
  Ockham-entails $B$, then $A$ super-entails $B$, but not the other way around.

  Suppose $A$ Ockham-entails $B$. Let $t$ be any time in any temporal model at
  which $A$ is true, i.e.: true relative to all histories through $t$. Since $A$
  Ockham-entails $B$, $B$ is true at $t$ relative to all histories through
  $t$. So $A$ super-entails $B$.

  But suppose $A$ super-entails $B$. Let $t$ be any time on any history $h$ in
  any temporal model at which $A$ is true. We can't infer that $B$ is true at
  $t$ on $h$, for $A$ may be false at $t$ relative to other histories $h'$. So
  we can't infer that $A$ Ockham-entails $B$. Indeed, $\tF p$ super-entails
  $\Box \tF p$, but $\tF p$ does not Ockham-entail $\Box \tF p$.
\end{solution}

% So is the true logic of time three-valued? I think this is not a good
% question. The way I see it, it's a matter of choice. We can easily speak about
% branching time in a classical, two-valued logic. And we could do this in
% different ways. If it really turns out that the world has a branching time
% structure, it might be reasonable to never say things like 'so-and-so will
% happen tomorrow', and instead say 'it will happen on one branch', or 'it will
% happen on all branches'. But we could also adopt a convention to use 'it will
% happen' as shorthand for 'it will happen on all branches', and then, if the
% convention is spelled out in a certain way, we'll have chosen to speak in a
% three-valued language.


\section{Extending the language}\label{sec:2d}

The expressive resources of standard modal and temporal logic are weak. There
are many things we might want to say about the unfolding of events in time that
can't be said with $\tF, \tG, \tP$, and $\tH$. The Ockhamist history quantifiers
are one way of adding expressive power to the basic language of temporal logic.
In this section, we will look at some others.

A useful operator for logics of discrete and linear time is the ``next''
operator $\tX$ (also written `$\bigcirc$'). Informally, $\tX A$ means that $A$
is true at the next point in time. Formally:
%
\begin{align*}
\hspace{-17mm}
  M,t \models \tX A \;\text{iff} & \;\text{$M,s \models A$ for some $s$ such that (i) $t<s$ and (ii) $s<r$ for all $r$}\\[-0.5mm]
  & \;\text{such that $r\not=s$ and $t<r$.}
\end{align*}

With the help of $\tX$, we can also say that $A$ is true in two units of time
($\tX\tX A$), in three units of time ($\tX\tX\tX A$), and so on. The
corresponding operator for talking about the \emph{previous} point in time is
usually written $\mathsf{Y}$.

A more powerful extension of $\L_T$ adds binary operators for ``since'' and
``until'', which can be used to translate sentences like (1) and (2).
%
\begin{itemize}[leftmargin=10mm]
  \itemsep-1mm
\item[(1)] Ever since we left the house it has been raining.
\item[(2)] It will be raining until we go back inside.
\end{itemize}
%
Informally, $A\tS B$ is true iff $B$ was true at some time in the past and $A$
has always been true since then; $A \tU B$ is true iff $B$ will be true at some
time in the future and $A$ will always be true until then. Formally:

\vspace{-5mm}
\begin{align*}
  \hspace{-17mm}
  M,t \models A \tS B \;\text{iff} &\; \text{there is some $s$ with $s < t$ for which $M,s \models B$, and for all $r$}\\[-0.5mm]
                                    &\;\text{with $s < r < t$, we have $M, r \models A$.}
\end{align*}
\vspace{-12mm}
\begin{align*}
  \hspace{-17mm}
M,t \models A \tU B \;\text{iff} &\; \text{there is some $s$ with $t < s$ for which $M,s \models B$, and for all $r$}\\[-0.5mm]
  &\;\text{with $t < r < s$, we have $M, r \models A$.}
\end{align*}

The operators $\tF, \tG,\tP,$ and $\tH$ can all be defined in terms of $\tS$ and
$\tU$. For example, $\tP A$ is equivalent to $(p\lor \neg p)\tS A$. And $\tF A$
is equivalent to $(p \lor \neg p)\tU A$.

\begin{exercise}
  Define $\tX A$ in terms of $\tU$.
\end{exercise}
\begin{solution}
  $(A \land \neg A) \tU A$.
\end{solution}

Another noteworthy addition to temporal logic is the ``Now'' operator $\tN$.
To see the point of this operator, consider the following multi-modal statement.

\begin{itemize}[leftmargin=10mm]
  \itemsep-1mm
\item[(3)] We already knew yesterday that there would be a test today.
\end{itemize}

Using $\tY$ for `yesterday', we might try to translate (3) as $\tY \Kn p$, where
$p$ translates `there is a test'. But that's wrong. By the semantics for $\tY$,
$\tY \Kn p$ is true today iff $\Kn p$ is true yesterday (using days as temporal
units). Since $\Kn p$ entails $p$, it follows that $\tY \Kn p$ is true today
only if $p$ is true \emph{yesterday}. But the test takes place today, not
yesterday.

Intuitively, the problem is that `today' in (3) refers to the present day, even
though it occurs in the scope of the `yesterday' operator. The same thing
happens in the quantified statement (4).
%
\begin{itemize}[leftmargin=10mm]
  \itemsep-1mm
\item[(4)] One day everyone who is now rich will be poor.
\end{itemize}
% 
Here, `now' refers to the present time, even though it is in the scope
of the $\tF$ operator `one day'.

% Kamp 1971 proved that every $\L_T$ sentence containing Now is
% equivalent to one without Now. Things change in quantified logic.

With the ``Now'' operator $\tN$, we can translate (3) as $\tY \Kn \tN p$, and
(4) as $\tF \forall x (\tN Rx \to Px)$. (We will have a closer look at
quantified modal logic in later chapters.)

Intuitively, the $\tN$ operator allows us to look outside the scope of an
embedding operator. $\tP \tN p$, for example, is true if there is some time in
the past such that $p$ is true not at that time, but at the present. How
does this work formally?

By the semantics of $\tP$,%
\begin{align*}
  M,t \models \tP \tN p \;\text{iff} \;\text{$M,s \models \tN p$ for some time $s < t$}.
\end{align*}
Now we want $M,s\models \tN p$ to be true iff $p$ is true at the original time
$t$. So we need to keep track of the original time at which we evaluate a
sentence, even if a temporal operator shifts the time at which a subsentence is
evaluated.

The simplest way to achieve this is to define truth relative to pairs of
times. One of the times is shifted by the temporal operators, the other is held
fixed.

\begin{definition}{Two-Dimensional Temporal Semantics}{2dtemporalsemantics}
  If $\Mfr = \t{T,<,V}$ is a temporal model, $t,t_0$ are members of $T$, $P$ is
  any sentence letter, and $A,B$ are any $\L_T$-sentences, then

  \medskip
  \begin{tabular}{lll}
    (a) & $M,t_0,t \models P$ &iff $t$ is in $V(P)$.\\
    (b) & $M,t_0,t \models \neg A$ &iff $M,t_0,t \not\models A$.\\
    (c) & $M,t_0,t \models A \land B$ &iff $M,t_0,t \models A$ and $M,t_0,t \models B$.\\
    (d) & $M,t_0,t \models A \lor B$ &iff $M,t_0,t \models A$ or $M,t_0,t \models B$.\\
    (e) & $M,t_0,t \models A \to B$ &iff $M,t_0,t \not\models A$ or $M,t_0,t \models B$.\\
    (f) & $M,t_0,t \models A \leftrightarrow B$ &iff $M,t_0,t \models (A\to B)$ and $M,t_0,t \models (B\to A)$.\\
    (g) & $M,t_0,t \models \tF A$ &iff $M,t_0,s \models A$ for some $s$ in $ T$ such that $t<s$.\\
    (h) & $M,t_0,t \models \tG A$ &iff $M,t_0,s \models A$ for all $s$ in $ T$ such that $t<s$.\\
    (i) & $M,t_0,t \models \tP A$ &iff $M,t_0,s \models A$ for some $s$ in $ T$ such that $s<t$.\\
    (j) & $M,t_0,t \models \tH A$ &iff $M,t_0,s \models A$ for all $s$ in $ T$ such that $s<t$.\\
    (k) & $M,t_0,t \models \tN A$ &iff $M,t_0,t_0 \models A$.
  \end{tabular}
\end{definition}

Like the Ockhamist semantics from the previous section, this semantics has an
extra parameter. An ordinary scenario is represented by a single time in a
model, not by a pair of times. So we need to specify under what
conditions a sentence is true at a (single) time. Here, the standard approach is
not supervaluation but ``diagonalization'':
\begin{align*}
  M,t \models A & \;\text{iff $M,t,t \models A$}.
\end{align*}

This ``two-dimensional'' semantics correctly predicts that $\tP\tN p$ entails $p$.
\begin{enumerate}[leftmargin=10mm]
  \itemsep-1mm
\item Assume $M,t \models \tP\tN p$.
\item Then $M,t,t \models \tP\tN p$, by the definition of truth at a time in a model.
\item Then $M,t,s \models \tN p$ for some $s<t$, by clause (i) of definition \ref{def:2dtemporalsemantics}.
\item Then $M,t,t \models p$, by clause (k) of definition \ref{def:2dtemporalsemantics}.
\item Then $M,t \models p$, by the definition of truth at a time in a model.
\end{enumerate}

The presence of a ``Now'' operator has far-reaching consequences for the logic
of time. For example, $\tN p \to p$ is valid, in the sense that it is true at
all times in all models. But $\tG(\tN p \to p)$ is invalid. If $p$ is true at
$t$ and false at some time after $t$, then $\tG(\tN p \to p)$ is false at $t$.
So we must give up the forward and backward Necessitation rules. The fact that
something is logically true does not entail that it will always be true!

\begin{exercise}
  `It might have been that everyone who is actually rich is poor.' This says
  that there is a world $w$ such that everyone who is rich \emph{at the actual
    world} is poor \emph{at $w$}. To formalize statements like these, we need a
  modal operator analogous to $\tN$ that takes us back to the actual world, even
  in the scope of other modal operators. This operator is called the
  \emph{actually} operator. Let's write it as $\tA$ and add it to $\L_{M}$. Can
  you find a sentence $B$ in this language that is logically true but not
  necessarily true, in the sense that $B$ is true at all worlds in all models
  but $\Box B$ is not?
\end{exercise}
\begin{solution}
  $\tA p \to p$.
\end{solution}

% \begin{align}
%   1. & N p \to p & \text{theorem}\\
%   2. & \Kn(Np \to p) & \text{necessitation}\\
%   2. & \tG \Kn(Np \to p) & \text{necessitation}\\
%   3. & \tG (\Kn(Np \to p) \to \Kn(Np \to p))& \text{nec factivity}\\
%   4. & \tG (Np \to p)& \text{2,3}\\
%   5. & p\to \tG Np& \text{theorem}\\
%   6. & p\to \tG p& \text{4,5}
% \end{align}

% From Rini & Cresswell p.20: It is usually assumed that worlds and times can be
% shifted independently. Meyer 2006 "Worlds and Times" considers the possibility
% that what times there are varies from world to world, so that we might need to
% allow a sentence to be true/false at a triple <w,t,p> even if t doesn't exist
% at p. [p is a person index.]


%%% Local Variables: 
%%% mode: latex
%%% TeX-master: "logic2.tex"
%%% End:
