\chapter{Epistemic Logic}\label{ch:epistemic}

% ``I know you think you understand what you thought I said but I'm not sure you
% realize that what you heard is not what I meant.'' -- Alan Greenspan%

\section{Epistemic accessibility}

When we say that something is possible, we often mean that it is compatible with
our information. This ``epistemic'' flavour of possibility -- along with related
concepts such as knowledge, belief, information, and communication -- is studied
in epistemic logic.

Standard epistemic logic relies heavily on the possible-worlds semantics
introduced in chapters \ref{ch:worlds} and \ref{ch:accessibility}. The guiding
idea is that information rules out possibilities. Suppose, for example, we are
investigating a crime. There are three suspects: the gardener, the butler, and
the cook. A credible eye-witness tells us that the gardener was out of town at
the time of the crime. This allows us to rule out the previously open
possibility that the gardener is the culprit. When we gain information, the
space of open possibilities shrinks.

% This model of attitudes goes back to Hintikka [1962]. The representation as a
% binary relation is formally interchangeable with the “possibility
% correspondences” as introduced by Aumann [1976] (see also Aumann [1999]) and
% used throughout economic theory. For the interchangeability, see, e.g. Fagin
% et al. [1995].

Let's say that a world is \emph{epistemically accessible} for an agent if it is
compatible with the agent's knowledge. Recall that a world is a maximally
specific possibility. For any such possibility, we may ask whether it might be
the actual world. If your information allows you to give a negative answer then
the world is not epistemically possible for you -- it is epistemically
inaccessible. Before we learned that the gardener was out of town, our
epistemically accessible worlds included worlds at which the gardener committed
the crime. When we received the eye-witness report, these worlds became
inaccessible.

\begin{exercise}
  Which worlds are epistemically accessible for an agent who knows all truths?
  Which worlds are epistemically accessible for an agent who knows nothing?
\end{exercise}
\begin{solution}
  For an agent who knows all truths only the actual world is epistemically accessible. For an agent who knows nothing all worlds are epistemically accessible.
\end{solution}

We will interpret the box and the diamond in terms of epistemic accessibility.
In this context, the box is usually written `$\Kn$', which for once doesn't
stand for Kripke but for knowledge. I will use `$\Mi$' (`might') for the
diamond. So $\Kn A$ means that $A$ is true at all epistemically accessible
worlds, while $\Mi A$ means that $A$ is true at some epistemically accessible
world. If we want to clarify which agent we have in mind, we can add a
subscript: $\Mi_{\text{b}} A$ might say that $A$ is epistemically possible for
Bob.

We often informally read $\Kn$ as `the agent knows'. In at least one respect,
however, our $\Kn$ operator does not match the knowledge operator of ordinary
English.

To see why, note that if some propositions are true at a world, then anything
that logically follows from these propositions is also true at that world. For
example, if $p\to q$ and $p$ are both true at $w$, then so is $q$ (by definition
\ref{def:kripkesemantics}). As a consequence, if $p \to q$ and $p$ are true at
all epistemically accessible worlds (for some agent), then $q$ is also true at
all these worlds. $\Kn (p\to q)$ and $\Kn p$ together entail $\Kn q$. More
generally, the $\Kn$ operator is \textbf{closed under logical consequence},
meaning that if $B$ logically follows from $A_1,\ldots,A_n$, and
$\Kn A_1, \ldots,\Kn A_n$, then $\Kn B$.

Our ordinary conception of knowledge does not seem to be closed under logical
consequence. If you know the axioms of a mathematical theory, you don't
automatically know everything that logically follows from the axioms. Our $\Kn$
operator might be taken to formalise the concept of \emph{implicit knowledge},
where an agent implicitly knows a proposition if the proposition follows from
things the agent knows. An agent's implicit knowledge represents the information
the agent has about the world. If what you know entails $p$, then the
information you have settles that $p$, even though you may not realise that it
does.

% \begin{exercise}
%   One might think that ordinary knowledge is closed under \emph{obvious
%   logical consequence}: If we know some propositions, and these propositions
%   obviously entail another proposition (say, by modus ponens, or by a similar
%   elementary rule), then we also know that other proposition. Explain why any
%   operator that is closed under obvious logical consequence is closed under
%   logical
%   consequence. % this needs to be made more precise. What if logical consequence is 2nd-order?
% \end{exercise}
% \begin{solution}
%   If some proposition $B$ is logically entailed by $A_{1}, \ldots, A_{n}$, then
%   there is a derivation of $B$ from $A_{1}, \ldots, A_{n}$ in which each
%   individual step is obvious.
% \end{solution}

\begin{exercise}
  Translate the following sentences into the language of epistemic logic,
  ignoring my warnings about the mismatch between $\Kn$ and the ordinary concept
  of knowledge.
  \begin{exlist}
  \item Alice knows that it is either raining or snowing.
  \item Either Alice knows that it is raining or that it is snowing.
  \item Alice knows whether it is raining.
  \item You know that you're guilty if you don't know that you're innocent.
  \end{exlist}
\end{exercise}
\begin{solution}
  \begin{sollist}
  \item $\Kn(r \lor s)$\\
    $r$: It is raining; $s$: It is snowing\\[-2mm]
  \item $\Kn r \lor \Kn s$\\
    $r$: It is raining; $s$: It is snowing\\[-2mm]
  \item $\Kn r \lor \Kn \neg r$\\
    $r$: It is raining\\[-2mm]
  \item This sentence is ambiguous. On one reading, it could be translated as $\Mi g\to Kn g$, on the other as $\Kn(\Mi g \to g)$\\
    $g$: You are guilty
  \end{sollist}
\end{solution}

% \begin{exercise}
%   The ordinary concept of knowledge logically ill-behaved in more than one
%   respect. Let $\Kn^*$ be an operator that applies to a sentence $A$ iff we
%   would intuitively say that some fixed agent knows $A$. Assume the agent in
%   question knows the axioms of ZFC set theory. Define $\Kn^+$ as the logical
%   closure of $\Kn^*$; that is,

%   \bigskip
%   $\Kn^+ A \;\Leftrightarrow_{\text{def}}\; \text{$A$ is entailed by sentences $A_1,\ldots,A_n$ such that $\Kn^* A_1,\ldots,\Kn^* A_n$}.$
%   \smallskip%
  
%   Note the similarity between $\Kn^+$ and the mathematical provability operator
%   from section \ref{sec:provability}.  Indeed, with minimal further assumptions
%   one can prove that $\Kn^+$ validates the \textbf{GL} schema:
%   \[
%     \Kn^+(\Kn^+ A \to A) \to \Kn^+ A
%   \]
%   From the definition of $\Kn^+$, we can infer that the following is also valid:
%   \[
%     \Kn^*(\Kn^+ A \to A) \to \Kn^+ A.
%   \]
%   Explain why this is an intuitively unacceptable principle about
%   knowledge.
% \end{exercise}
% \begin{solution}
%   One possible answer: There are many propositions that I don't know, and that
%   don't logically follow from things I know. And for many of these propositions
%   I \emph{know} that that they don't follow from things I know. (For example, I
%   know that it doesn't follow from anything I know that Edinburgh is in Italy.)
%   Let $p$ be some such proposition. Since I know that $\neg \Kn^+ p$ is true, I
%   can easily figure out that $\Kn^+ p \to p$ is true as well (by the truth-table
%   for the arrow). So $\Kn^*(\Kn^+p \to p)$. If
%   $\Kn^*(\Kn^+ A \to A) \to \Kn^+ A$ were valid, it would follow that $p$ does
%   after all follow from what I know!
% \end{solution}

% \begin{exercise}
%   We could add ``impossible'' worlds to avoid closure under logical entailment
%   (Rantala 1982). Explain why definition \ref{def:possibleworldssemantics}
%   needs to be changed for impossible worlds $w$.
% \end{exercise}

% Mention centring?


\section{The logic of knowledge}

What is the logic of (implicit) knowledge? Which sentences in the language of
epistemic logic are valid? Which are logical consequences of which others?

The basic system K is arguably too weak. There are Kripke models in which
$\Box p$ is true at some world while $p$ is false. But knowledge entails truth.
If $p$ is genuinely known (or entailed by what is known) then $p$ is true. In
the logic of knowledge, all instance of the \pr{T}-schema are valid.
%
\principle{T}{\Kn A \to A}

We know from section \ref{sec:frames} that the \pr{T}-schema corresponds to
reflexivity, in the sense that all instances of the schema are valid on a frame
iff the frame is reflexive. To ensure that all \pr{T} instances are valid, we
will therefore assume that Kripke models for epistemic logic are always reflexive.
Every world is accessible from itself.

This makes sense if you remember what accessibility means in epistemic logic. We
said that a world $v$ is (epistemically) accessible from a world $w$ if $v$ is
compatible with what the agent knows at $w$. Whatever the agent knows at $w$
must be true at $w$. So any world in any conceivable scenario must be
accessible from itself.

% Reflexivity implies seriality, which corresponds to the schema
% \begin{equation}\tag{\pr{D}}
%   \Kn A \to \Mi A
% \end{equation}
% Intuitively, this means that the information available to an agent is never
% contradictory. If the information entails $A$ (as $\Kn A$ asserts), then it
% does not entail $\neg A$ (that is, then $\neg \Kn\neg A$).

Let's look at other properties of the epistemic accessibility relation. Is the
relation symmetric? If $v$ is compatible with what is known at $w$, is $w$
compatible with what is known at $v$? I will give two arguments for a negative
answer.

My first argument assumes that we have non-trivial knowledge about the external
world. Let's say we know that we have hands. Now consider a possible world in
which we are brains in a vat, falsely believing that we have hands. In that
world, we know very little. We don't know that we have hands, nor that we are
handless brains in a vat. Perhaps we know that we are conscious, and what kinds
of experiences we have. But since our experiences are the same in the vat world
and in the actual world (let's assume), the actual world is compatible with what
little we know in the vat world. So the actual world is accessible from the vat
world. But the vat world is not accessible from the actual world -- otherwise we
wouldn't know that we have hands. If the actual world is accessible from the vat
world and the vat world is inaccessible from the actual world then the
accessibility relation isn't symmetric.

My second argument starts with a scenario in which someone has misleading
evidence that some proposition $p$ is false. This is easily conceivable. In that
scenario, $p$ is true but the agent believes $\neg p$. Often, when we believe
something, we also believe that we know it. Let's assume that our agent believes
that they know $\neg p$. Let's also assume that their beliefs are consistent, so
they don't believe that they \emph{don't} know $\neg p$. Since they don't
believe this proposition (that they don't know $\neg p$) they don't know it
either: they don't know that they don't know $\neg p$. So we have a scenario in
which $p$ is true but $\Kn \neg \Kn \neg p$ false.

Can you see what this has to do with symmetry? In section \ref{sec:frames} I
mentioned that symmetry corresponds to the schema
%
\principle{B}{A \to \Kn \Mi A.}
%
This means that all instances of \pr{B} are valid on a frame iff the frame is
symmetric. If the epistemic accessibility relation were symmetric, then all
instances of \pr{B} would be valid. But I've just described a scenario in which
an instance of \pr{B} is false. So the epistemic accessibility relation isn't
symmetric.

What about transitivity, which corresponds to schema \pr{4}?
%
\principle{4}{\Kn A \to \Kn\Kn A}
%
In epistemic logic, \pr{4} is known as the \textbf{KK principle}, or
(misleadingly) as \textbf{positive introspection}. There is an ongoing debate
over whether the principle should be considered valid. I will review one
argument for either side.

A well-known argument against the KK principle draws on the idea that knowledge
requires ``safety'': you know $p$ only if you couldn't easily have been wrong
about $p$. To motivate this idea, consider a Gettier case. Suppose you are
looking at the only real barn in a valley which, unbeknownst to you, is full of
fake barns. Your belief that you're looking at a barn is true, and it seems to
be justified. But intuitively, it isn't knowledge. You don't know that what
you're looking at is a real barn. Why not? Advocates of the safety condition
suggest that you don't have knowledge because you could easily have been wrong.
You genuinely know $p$ only if there is no ``nearby'' possibility at which $p$
is false, where ``nearness'' is a matter of similarity in certain respects.

On the safety account, you know \emph{that you know $p$} only if there is no
nearby world at which you don't know $p$. That is, you know at world $w$ that
you know $p$ only if you know $p$ at all worlds $v$ that are relevantly similar
to $w$. And you know $p$ at $v$ only if $p$ is true at all worlds $u$ that are
relevantly similar to $v$. But similarity isn't transitive: the fact that $u$ is
similar to $v$ and $v$ is similar to $w$ does not entail that $u$ is similar to
$w$. So it can happen that $p$ holds at all nearby worlds, but not at all worlds
that are nearby a nearby world. In that case, you may know $p$ without knowing
that you know $p$.

Not everyone accepts the safety condition. Other accounts of knowledge vindicate
the KK principle. For example, some have argued that an agent knows $p$
(roughly) iff the agent's belief state \emph{indicates} $p$, in the sense that
%
\begin{quote}
\begin{enumerate*}
\item[(1)] under normal conditions, being in that state implies $p$, and
\item[(2)] conditions are normal.
\end{enumerate*}
\end{quote}
% 
We can formalize this concept in modal logic. Let $N$ mean that conditions are
normal (whatever exactly that means), and let $\Box$ be a non-epistemic operator
that formalizes `at all worlds'. $\Box(N \to A)$ then means that $A$ is true at
all world at which conditions are normal. According to the definition I just
gave, a belief state $s$ indicates $p$ iff
\begin{equation}\tag{*}
  \Box(N \to (s\to p)) \land N.
\end{equation}
The state $s$ indicates that $s$ indicates $p$ iff
\begin{equation}\tag{**}
  \Box(N \to (s \to (\Box(N \to (s \to p)) \land N))) \land N.
\end{equation}
A quick tree proof reveals that (*) entails (**). That is, whenever a state
indicates $p$ then it also indicates that it indicates $p$. On the indication
account of knowledge, a belief state that constitutes knowledge therefore
automatically constitutes knowledge of knowledge: the \pr{4} schema is valid.

\begin{exercise}
  Give an S5 tree proof to show that (*) entails (**). Why can we
  assume S5 here?
\end{exercise}
\begin{solution}
  You can use
  \href{https://www.umsu.de/trees/}{umsu.de/trees/} to
  see the tree proof. We can assume S5 for the box because quantifies
  unrestrictedly over all worlds (as in chapter \ref{ch:worlds}).
\end{solution}

The \pr{4}-schema says that people have knowledge of their knowledge. The
\pr{5}-schema says that people have knowledge of their ignorance: if you don't
know something, then you know that you don't know it. This hypothesis is
(misleadingly) known as \textbf{negative introspection}.
%
\principle{5}{\Mi A \to \Kn \Mi A.}
%
We know that the \pr{5}-schema corresponds to euclidity. This gives us a quick
argument against the schema. As you showed in exercise \ref{ex:relations},
reflexivity and euclidity together entail symmetry. The epistemic accessibility
relation is reflexive. If it were euclidean, it would be symmetric. But I've
argued that it isn't symmetric. So the logic of knowledge doesn't validate
\pr{5}.

We can also give a more direct argument against negative introspection. Consider
again a scenario in which someone has misleading evidence that some proposition
$p$ is false. Since $p$ is actually true, the agent doesn't know $\neg p$. But
the agent might not know that they don't know $\neg p$. (On the contrary, they
might believe that they do know $\neg p$.) In that scenario, $\neg\Kn \neg p$ is
true but $\Kn \neg \Kn \neg p$ is false.

Here it is important to not be misled by a curiosity of ordinary language. When
we say that someone doesn't know $p$, this seems to imply that $p$ is true. If I
told you that my neighbour doesn't know that I have a pet aardvark, you could
reasonably infer that I have a pet aardvark. You might therefore be tempted to
regard all instances of the following schema as valid:
%
\principle{NT}{\neg \Kn A \to A}
%
On reflection, however, \pr{NT} is unacceptable. If $\neg \Kn A$ entails $A$,
then by contraposition $\neg A$ entails $\Kn A$: everything that is false would
be known! Indeed, if I \emph{don't} have a pet aardvark then surely my neighbour
does not know that I have one. We shall therefore not regard the inference from
$\neg \Kn A$ to $A$ as valid.

\begin{exercise}
  Can you find a Kripke frame on which \pr{NT} is valid?
\end{exercise}
\begin{solution}
  \pr{NT} is valid on all and only the frames in which no world can see any world. 
\end{solution}

\begin{exercise}
  Let's say that an agent is \emph{ignorant of} a proposition if they don't know
  the proposition and the proposition is true. (In English, saying that someone
  doesn't know a proposition normally conveys that they are ignorant of the
  proposition, in this sense.) Show that if the logic of knowledge is at least as
  strong as K, then ignorance of $A$ entails ignorance of ignorance of $A$.
    % I(A) = A . -KA.
    % II(A) = A . -KA . -K(A . -KA).
\end{exercise}
\begin{solution}
  We assume that ignorance of $A$ can be formalized as $A \land \neg \Kn A$. Ignorance of ignorance of $A$ is therefore formalized as $(A \land \neg \Kn A) \land \neg \Kn(A \land \neg \Kn A)$. A tree proof shows that the former K-entails the latter. 
  % See Fine 2018:
  % \begin{enumerate}
  % \item $\vdash_{S4} \neg Ip \to K\neg Ip$
  % \item $\not\vdash_{S4} Ip \to K Ip$
  % \item $\vdash_{S4} IIp \to Ip$
  % \item $\not\vdash_{S4} Ip \to IIp$
  % \item $\vdash_{S4} IIp \to \neg KIIp$
  % \item $\vdash_{S4} IIp \leftrightarrow IIIp$
  % \item $\vdash_{S5} \neg IIp$
  % \item $\vdash_{S4M} Ip \to IIp$
  % \end{enumerate}
  % \end{exercise}
\end{solution}
  
% Rumsfeld suggested there are things of which we don't know that we don't know
% them. We might say that someone is Rumsfeld ignorant of $p$ iff
% $\neg K\neg K p$. But that's not quite what Rumsfeld had in mind. For note that
% if $Kp$ then $\neg K \neg Kp$, because knowledge is factive. But $Kp$ is not a
% case of Rumsfeld ignorance. (I.e., one reason why we don't know that we don't
% know $p$ is that we actually know $p$, but that's not the interesting case.) So
% let's say someone is Rumsfeld ignorant whether $p$ iff
% $\neg K p \land \neg K \neg Kp$. (Fine says it's $Ip \land \neg K Ip$. Suppose
% $K\neg p$. Then $\neg K p$ and $K \neg Kp$. So that doesn't look problematic.)

% Notice that this is a Fitchean truth. So we can't know that we're Rumsfeld
% ignorant. Which is why it's hard to give an \emph{example} of something of which
% we're Rumsfeld ignorant.


% Oddly, when computer scientists reason about knowledge and information, they
% often assume both positive and negative introspection, along with the \pr{T}
% schema. Since every transitive, euclidean, and reflexive relation is an
% equivalence relation, the logic of knowledge then becomes S5. The comparative
% simplicity of S5 -- think of the simple tree rules from chapter \ref{ch:worlds}
% -- may be one reason to make the philosophically dubious posit of negative
% introspection. However, the posit can also be justified by assumptions about the kinds of system we want to model.

% Imagine an artificial agent whose database can store the truth-value for a
% finite number of propositions $p_1,\ldots,p_n$. The agent receives information
% through a reliable channel, so that the database is guaranteed to never
% contain false information. Since the agent knows, say, $p_{1}$ iff their
% database says that $p_{1}$ is true, the agent can easily find out whether or
% not they know $p_{1}$ by scanning their own database.

We have looked at six schemas: \pr{T}, \pr{B}, \pr{4}, \pr{5}, and \pr{NT}.
Philosophers working in epistemic logic generally reject \pr{B}, \pr{5}, and
\pr{NT}, accept \pr{T}, and are divided over \pr{4}. Theorists in other
disciplines often assume that the logic of knowledge is S5, which would render
all instances of \pr{T}, \pr{4}, \pr{B}, and \pr{5} valid. If we drop \pr{B} and
\pr{5} but keep \pr{T} and \pr{4}, we get S4. If we also drop \pr{4}, we get
system T.

We might, however, look at further schemas, corresponding to further conditions
on the accessibility relation. For example, some have argued that we should
adopt a weakened form of negative introspection. The above counterexample to
negative introspection -- schema \pr{5} -- involved an agent who doesn't know
that they don't know a certain proposition because they don't know that the
proposition is false. This kind of counterexample can't arise if the relevant
proposition is true. One might therefore suggest that if an agent doesn't know a
proposition $p$ \emph{and $p$ is true}, then the agent always knows that they
don't know $p$. This would give us a schema known as 0.4:
%
\principle{0.4}{(\neg \Kn A\land A) \to \Kn \neg \Kn A}
%
All instances of \pr{0.4} are S5-valid, but not all of them are S4-valid. Adding
the schema to S4 leads to a system known as S4.4.

% .4 corresponds to xRy & xRz -> (yRz v x=z)

\begin{exercise}
  Explain why Gettier cases cast doubt on \pr{0.4}.
\end{exercise}
\begin{solution}
  In a Gettier case, the relevant proposition $p$ (say, that you're looking at a
  barn) is true but unknown. By \pr{0.4}, it would follow that the agent knows
  that they don't know $p$. But in a typically Gettier case the agent does not
  know that they don't know $p$.
\end{solution}
  
A more modest extension of S4 adds the schema \pr{G}, which corresponds to
convergence of the accessibility relation:
%
\principle{G}{\Mi\Kn A \to \Kn\Mi A}
%
The resulting logic is called S4.2; it is weaker than S4.4 but stronger than S4.
We will meet an argument in favour of \pr{G} in section \ref{sec:kb}.

% \begin{exercise}
%   Give an S4 tree proof to show that
%   $(A \land \neg \Kn A) \to \Kn \neg \Kn A$ and
% %  $(\neg A \land \neg \Kn \neg A) \to \Kn \neg \Kn \neg A$ (both of which
%   $(\neg A \land \Mi A) \to \Kn \Mi A$ (both of which
%   are covered by 0.4) together entail \pr{G}.
% \end{exercise}

\begin{exercise}
  Use the tree method to check the following claims. (See the table at the end
  of chapter \ref{ch:accessibility} for the tree rules that go with B, S4, and
  S4.2.)
  \begin{exlist}
  \item $\models_{T} \Mi\Kn p \to \Kn\Mi p$.
  \item $\models_{B} \Mi\Kn p \to \Kn\Mi p$.
  \item $\models_{S4} \Mi\Kn\Mi p \to \Mi p$.
  \item $\models_{S4} \Mi\Kn p \leftrightarrow \Kn\Kn p$.
  \item $\models_{S4} \Mi\Kn(p \to \Kn\Mi p)$.
  \item $\models_{S4.2} (\Mi\Kn p \land \Mi\Kn q) \to \Mi \Kn(p \land q)$.
  \end{exlist}
\end{exercise}
\begin{solution}
  All except (a) and (d) are correct. You can find trees or counterexamples for (a)-(e) on
  \href{https://www.umsu.de/trees/}{umsu.de/trees/} if
  you write $\Kn$ as a box and $\Mi$ as a diamond. Here is a tree for (f):
  \begin{center}
  \tree{
   & \nnode{35}{1.}{$\neg((\Mi\Kn p \land \Mi\Kn q) \to \Mi \Kn(p \land q))$}{w}{(Ass.)} & \\
    & \nnode{35}{2.}{$\Mi\Kn p \land \Mi\Kn q$}{w}{(1)} & \\
    & \nnode{35}{3.}{$\neg\Mi \Kn(p \land q)$}{w}{(1)} & \\
    & \nnode{35}{4.}{$\Mi\Kn p$}{w}{(2)} & \\
    & \nnode{35}{5.}{$\Mi\Kn q$}{w}{(2)} & \\
    & \nnode{35}{6.}{$wRv$}{}{(4)} & \\
    & \nnode{35}{7.}{$\Kn p$}{v}{(4)} & \\
    & \nnode{35}{8.}{$wRu$}{}{(5)} & \\
    & \nnode{35}{9.}{$\Kn q$}{u}{(5)} & \\
    & \nnode{35}{10.}{$vRt$}{}{(6,8,Con)} & \\
    & \nnode{35}{11.}{$uRt$}{}{(6.8,Con)} & \\
    & \nnode{35}{12.}{$wRt$}{}{(6.10,Tr)} & \\
    & \nnode{35}{13.}{$\neg\Kn(p \land q)$}{t}{(3,12)} & \\
    & \nnode{35}{14.}{$tRs$}{}{(13)} & \\
    & \bnode{35}{15.}{$\neg(p \land q)$}{s}{(13)} & \\
    &&\\
    \nnode{10}{16.}{$\neg p$}{s}{(15)} && \nnode{10}{17.}{$\neg q$}{s}{(15)}\\
    \nnode{10}{18.}{$vRs$}{}{(10.14,Tr)} && \nnode{10}{19.}{$uRs$}{}{(11.14,Tr)}\\
    \nnodeclosed{10}{20.}{$p$}{s}{(7,18)} && \nnodeclosed{10}{21.}{$q$}{s}{(9,19)}\\
  }
  \end{center}
\end{solution}


\section{Multiple Agents}
\label{sec:multi}

A world that is epistemically accessible for one agent may not be accessible for
another. If we want to reason about the information available to different
agents, we need separate $\Kn$ operators and accessibility relations for each
agent.

We can easily expand the language $\L_M$ to a \textbf{multi-modal language} by
introducing a whole series of box operators $\Kn_1, \Kn_2, \Kn_3, \ldots$ with
their duals $\Mi_1, \Mi_2, \Mi_3, \ldots$. This multi-modal language is
interpreted in multi-modal Kripke models.

\begin{definition}{}{multikripkemodel}
  A \textbf{multi-modal Kripke model} consists of
  \vspace{-3mm}
  \begin{itemize*}
  \item a non-empty set $W$,
  \item a set of binary relation $R_1,R_2,R_{3},\ldots$ on $W$, and
  \item a function $V$ that assigns to each sentence letter a subset of $W$.
  \end{itemize*}
\end{definition}
%
In our present application, every accessibility relation $R_i$ represents the
information available to a particular agent. A world $v$ is $R_i$-accessible
from $w$ iff $v$ is compatible with the information agent $i$ has at world $w$.

The definition of truth at a world in a Kripke model (definition
\ref{def:kripkesemantics}) is easily extended to multi-modal Kripke models.
Instead of clauses (g) and (h), we have the following conditions, for each pair
of a modal operator $\Kn_i$/$\Mi_i$ and the corresponding accessibility
relation $R_i$:

\bigskip
\begin{tabular}{lll}
  & $M,w \models \Kn_i A$ &iff $M,v \models A$ for all $v$ in $W$ such that $wR_iv$.\\
  & $M,w \models \Mi_i A$ &iff $M,v \models A$ for some $v$ in $W$ such that $wR_iv$.
\end{tabular}
\bigskip

As an application of this machinery, let's look at the \emph{Muddy Children}
puzzle.

\begin{quote}
  Three (intelligent) children have been playing outside. They can't see or feel
  if their own face is muddy, but they can see who of the others has mud on
  their face. Coming inside, mother tells them: `At least one of you has mud on
  their face'. She then asks, `Do you know if you have mud on your face?''.
  All three children say that they don't know. Mother asks again, `Do you know
  if you have mud on your face?'. This time, two children say that they know.
  How many children have mud on their face? And what happens when the mother
  asks her question a third time?
\end{quote}

We begin by drawing a model. I'll call the three children Alice, Bob, and Carol,
and I'll use $m_a, m_b, m_c$ as sentence letters expressing, respectively, that
Alice/Bob/Carol is muddy; $c_a, c_b, c_c$ mean that Alice/Bob/Carol is clean.
Before the mother's first announcement, there are eight relevant possibilities.

\begin{center}
  \resizebox{ 8.5cm}{!}{%
  \begin{tikzpicture}[modal, world/.append style={minimum size=1.7cm}]
    \node[world] (w1) {$c_am_b\newline m_c$};
    \node[world] (w2) [above right=of w1] {$c_am_bc_c$};
    \node[world] (w3) [below=3cm of w1] {$c_ac_bm_c$};
    \node[world] (w4) [above right=of w3] {$c_ac_bc_c$};
    \draw[<->, rblue] (w1) -- (w2) node[midway,above left]{$c$};
    \draw[<->, rblue] (w3) -- (w4) node[midway,above left]{$c$};
    \draw[<->, rgreen] (w1) -- (w3) node[midway,left]{$b$};
    \draw[<->, rgreen] (w2) -- (w4) node[near end,left]{$b$};
    \node[world] (w5) [right=3cm of w1]{$m_am_bm_c$};
    \node[world] (w6) [above right=of w5] {$m_am_bc_c$};
    \node[world] (w7) [below=3cm of w5] {$m_ac_bm_c$};
    \node[world] (w8) [above right=of w7] {$m_ac_bc_c$};
    \draw[<->, rblue] (w5) -- (w6) node[midway,above left]{$c$};
    \draw[<->, rblue] (w7) -- (w8) node[midway,above left]{$c$};
    \draw[<->, rgreen] (w5) -- (w7) node[near end,left]{$b$};
    \draw[<->, rgreen] (w6) -- (w8) node[midway,left]{$b$};
    \draw[<->, rred] (w1) -- (w5) node[near end,above]{$a$};
    \draw[<->, rred] (w2) -- (w6) node[midway,above]{$a$};
    \draw[<->, rred] (w3) -- (w7) node[midway,above]{$a$};
    \draw[<->, rred] (w4) -- (w8) node[near start,above]{$a$};
  \end{tikzpicture}
  }%
\end{center}
%
Since we have three epistemic agents, we have three accessibility relations, one
for Alice (drawn in red), one for Bob (green), and one for Carol (blue). To
remove clutter, I have left out the three times eight arrows leading from each
world to itself, but we should keep in mind that every world is also accessible
from itself, for each agent.

Don't confuse an arrow in the diagram of a model with an accessibility relation.
We have three accessiblity relations, but more than three arrows. All the red
arrows in the picture represent one and the same accessibility relation. The
accessibility relation for Alice holds between a world and another whenever a
red arrow leads from the first world to the second.

Notice how the fact that every child can see the others is reflected in the
diagram. For example, at the top left world ($c_am_bc_c$), Alice sees that Bob
is muddy and that Carol is clean; the only epistemic possibilities for Alice at
that world are the two worlds at the top: $c_am_bc_c$ itself and $m_am_bc_c$. In
general, the only accessible worlds for a given child at a given world $w$ are
worlds at which the other children's state of muddiness is the same as at $w$.

What changes through the mother's first announcement, `At least one child has
mud on their face'? The announcement tells \emph{us} that we're not in the
$c_ac_bc_c$ world. More importantly, it allows \emph{each child} to rule out the
$c_ac_bc_c$ world (since they all hear and accept the announcement).

\begin{center}
  \resizebox{ 8.5cm}{!}{%
  \begin{tikzpicture}[modal, world/.append style={minimum size=1.7cm}]
    \node[world] (w1) {$c_am_bm_c$};
    \node[world] (w2) [above right=of w1] {$c_am_bc_c$};
    \node[world] (w3) [below=3cm of w1] {$c_ac_bm_c$};
    \node[world] (w4) [above right=of w3, opacity=0.4] {$c_ac_bc_c$};
    \draw[<->, rblue] (w1) -- (w2) node[midway,above left]{$c$};
    %\draw[<->, rblue] (w3) -- (w4) node[midway,above left]{$c$};
    \draw[<->, rgreen] (w1) -- (w3) node[midway,left]{$b$};
    %\draw[<->, rgreen] (w2) -- (w4) node[near end,left]{$b$};
    \node[world] (w5) [right=3cm of w1]{$m_am_bm_c$};
    \node[world] (w6) [above right=of w5] {$m_am_bc_c$};
    \node[world] (w7) [below=3cm of w5] {$m_ac_bm_c$};
    \node[world] (w8) [above right=of w7] {$m_ac_bc_c$};
    \draw[<->, rblue] (w5) -- (w6) node[midway,above left]{$c$};
    \draw[<->, rblue] (w7) -- (w8) node[midway,above left]{$c$};
    \draw[<->, rgreen] (w5) -- (w7) node[near end,left]{$b$};
    \draw[<->, rgreen] (w6) -- (w8) node[midway,left]{$b$};
    \draw[<->, rred] (w1) -- (w5) node[near end,above]{$a$};
    \draw[<->, rred] (w2) -- (w6) node[midway,above]{$a$};
    \draw[<->, rred] (w3) -- (w7) node[midway,above]{$a$};
    %\draw[<->, rred] (w4) -- (w8) node[near start,above]{$a$};
  \end{tikzpicture}
  }%
\end{center}

Next, the mother asks if anyone knows whether they are muddy. No child says yes.
So no-one knows whether they are muddy. And everyone now knows that no-one knows
whether they are muddy. We can go through the above seven possibilities to see
if at any of them, anyone knows whether they are muddy. At the top left world
($c_am_bc_c$) Alice doesn't know whether she is muddy, because the $m_am_bc_c$
world (top right) is $a$-accessible; nor does Carol know whether she is muddy,
because $c_am_bm_c$ is $c$-accessible. But Bob knows that he is muddy: no other
world is $b$-accessible. Intuitively, at the $c_am_bc_c$ world, Bob sees two
clean children (Alice and Carol), and he has just been told that not all
children are clean. So he can infer that he is muddy. But we know that Bob
didn't say that he knows whether he is muddy. So we (and all the children) can
rule out the top left world as an open possibility.

By the same reasoning, every world connected with only two arrows to other
worlds can be eliminated at this stage.

\begin{center}
  \resizebox{ 8.5cm}{!}{%
  \begin{tikzpicture}[modal, world/.append style={minimum size=1.7cm}]
    \node[world] (w1) {$c_am_bm_c$};
    \node[world] (w2) [above right=of w1, opacity=0.4] {$c_am_bc_c$};
    \node[world] (w3) [below=3cm of w1, opacity=0.4] {$c_ac_bm_c$};
    \node[world] (w4) [above right=of w3, opacity=0.4] {$c_ac_bc_c$};
    %\draw[<->, rblue] (w1) -- (w2) node[midway,above left]{$c$};
    %\draw[<->, rblue] (w3) -- (w4) node[midway,above left]{$c$};
    %\draw[<->, rgreen] (w1) -- (w3) node[midway,left]{$b$};
    %\draw[<->, rgreen] (w2) -- (w4) node[near end,left]{$b$};
    \node[world] (w5) [right=3cm of w1]{$m_am_bm_c$};
    \node[world] (w6) [above right=of w5] {$m_am_bc_c$};
    \node[world] (w7) [below=3cm of w5] {$m_ac_bm_c$};
    \node[world] (w8) [above right=of w7, opacity=0.4] {$m_ac_bc_c$};
    \draw[<->, rblue] (w5) -- (w6) node[midway,above left]{$c$};
    %\draw[<->, rblue] (w7) -- (w8) node[midway,above left]{$c$};
    \draw[<->, rgreen] (w5) -- (w7) node[midway,left]{$b$};
    %\draw[<->, rgreen] (w6) -- (w8) node[midway,left]{$b$};
    \draw[<->, rred] (w1) -- (w5) node[midway,above]{$a$};
    %\draw[<->, rred] (w2) -- (w6) node[midway,above]{$a$};
    %\draw[<->, rred] (w3) -- (w7) node[midway,above]{$a$};
    %\draw[<->, rred] (w4) -- (w8) node[near start,above]{$a$};
  \end{tikzpicture}
  }%
\end{center}

When the mother asks again if anyone knows whether they are muddy, two children
say `yes'. So everyone comes to know that two children know whether they are
muddy. In the middle world of the above model ($m_am_bm_c$), however, no child
knows whether they are muddy. That world is not actual, and it is no longer
accessible for anyone. The remaining open possibilities are $c_am_bm_c$,
$m_ac_bm_c$, and $m_am_bc_c$, each of which is only accessible from itself.

Now we can answer the questions. In the three remaining worlds, every child
knows who is muddy and who is clean. If the mother asks her question for the
third time, everyone says yes. Also, exactly two children have mud on their
face.

\begin{exercise}
  Albert and Bernard just met Cheryl. `When is your birthday?', Albert asks.
  Cheryl answers, `I'll give you some clues'. She writes down a list of 10
  dates:
  % 
  \begin{quote}
    5 May, 6 May, 9 May\\
    7 June, 8 June\\
    4 July, 6 July\\
    4 August, 5 August, 7 August
  \end{quote}
  %
  `My birthday is one of these', she says. Then she announces that she will
  whisper the month of her birthday in Albert's ear and the day in Bernard's.
  After the whispering, she asks Albert if he knows her birthday. Albert says,
  `no, but I know that Bernard doesn't know either'. To which Bernard responds:
  `Right. I didn't know until now, but now I know'. Albert: `Now I know too!'
  Draw a multi-modal Kripke model for each stage of the conversation. When is
  Cheryl's birthday?
\end{exercise}
\begin{solution}
  see
  \href{https://plato.stanford.edu/entries/dynamic-epistemic/appendix-B-solutions.html}{https://plato.stanford.edu/entries/dynamic-epistemic/appendix-B-solutions.html} (where all the dates are 10 days later than they are in my version).
\end{solution}

What logic do we have for our multi-modal language? Each pair of a $\Kn_{i}$ and
$\Mi_{i}$ operator should obey whatever conditions we want to impose on the
logic of knowledge. Are there also new principles governing the interaction
between operators for different agents?

We plausibly want all instances of the following to come out valid:
\[
  \Kn_1 \Kn_2 A \to \Kn_1 A.
\]
If I know that you know that it's raining, then I (implicitly) also know that
it's raining. Principles like this, containing multiple modal operators that are
not definable in terms of each other, are called \textbf{interaction
  principles}.

A common assumption in epistemic logic is that there are no genuinely new
interaction principles for the knowledge of multiple agents -- no principles
that don't already follow from the logic of individual knowledge. The above
principle, for example, is entailed by the assumption that the \pr{T}-schema
holds for $\Kn_2$. Think of the relevant Kripke models. Suppose, as
$\Kn_1 \Kn_2 A$ asserts, that $A$ holds at each world that is $R_2$-accessible
from any $R_1$-accessible world. If the \pr{T}-schema holds for $\Kn_2$, then
every world is $R_{2}$-accessible from itself. In particular, then, any
$R_1$-accessible world is $R_2$-accessible from itself. It follows that $A$
holds at every $R_1$-accessible world. So $\Kn_1 A$ is true.

% Girle says Hintikka has a rule to the effect that $K_aK_b A \to K_a A$.
% ``Transmissability of Knowledge''. But that seems redundant.

We can use the tree rules to streamline arguments like this. When multiple
agents are in play, we need to keep track of which world is accessible for which
agent. When expanding a node of type $\Mi_{i} A\; (w)$, for example, we add a
node $wR_{i}v$, with subscript $i$, and another node $A\; (v)$.

Here is a tree proof of the schema $\Kn_{1}\Kn_{2} A \to \Kn_{1} A$, assuming
that $R_{2}$ is reflexive. \bigskip
  \begin{center}
  \tree{
    \nnode{25}{1.}{$\neg(\Kn_1 \Kn_2 A \to \Kn_1 A)$}{w}{(Ass.)}\\
    \nnode{25}{2.}{$\Kn_1 \Kn_2 A$}{w}{(1)}\\
    \nnode{25}{3.}{$\neg \Kn_1 A$}{w}{(1)}\\
    \nnode{25}{4.}{$wR_1 v$}{}{(3)}\\
    \nnode{25}{5.}{$\neg A$}{v}{(3)}\\
    \nnode{25}{6.}{$\Kn_{2} A$}{v}{(2,4)}\\
    \nnode{25}{7.}{$vR_{2}v$}{}{(Refl.)}\\
    \nnodeclosed{25}{8.}{$A$}{v}{(6,7)}
  }
  \end{center}

\begin{exercise}
  Use the tree method to check which of the following interaction principles are
  valid if the logic of individual knowledge is S4. If a principle is invalid,
  give a counterexample.
  \begin{exlist}
  \item $\Mi_1 \Kn_2 p \to \Mi_1 p$ 
  \item $\Mi_1 \Kn_2 p \to \Mi_2\Mi_1 p$
  \item $\Mi_1 \Kn_2 p \to \Mi_2\Kn_1 p$
  \item $\Kn_1\Kn_2 p \to \Kn_2\Kn_1 p$
  \end{exlist}
\end{exercise}
\begin{solution}
  (a) and (b) are valid, (c) and (d) are invalid. Here is a tree proof for (a).
  
  \bigskip
  \begin{center}
  \tree{
    \nnode{25}{1.}{$\neg(\Mi_1 \Kn_2 p \to \Mi_1 p)$}{w}{(Ass.)}\\
    \nnode{25}{2.}{$\Mi_1 \Kn_2 p$}{w}{(1)}\\
    \nnode{25}{3.}{$\neg \Mi_1 p$}{w}{(1)}\\
    \nnode{25}{4.}{$wR_1 v$}{}{(2)}\\
    \nnode{25}{5.}{$\Kn_2 p$}{v}{(2)}\\
    \nnode{25}{6.}{$\neg p$}{v}{(3,4)}\\
    \nnode{25}{7.}{$vR_2 v$}{}{\;\;(Refl.)}\\
    \nnodeclosed{25}{8.}{$p$}{v}{(5,7)}
  }
  \end{center}

  The tree for (c) doesn't close:
  
  \bigskip
  \begin{center}
  \tree{
    \nnode{25}{1.}{$\neg(\Mi_1 \Kn_2 p \to \Mi_2\Kn_1 p)$}{w}{(Ass.)}\\
    \nnode{25}{2.}{$\Mi_1 \Kn_2 p$}{w}{(1)}\\
    \nnode{25}{3.}{$\neg \Mi_2\Kn_1 p$}{w}{(1)}\\
    \nnode{25}{4.}{$wR_1 v$}{}{(2)}\\
    \nnode{25}{5.}{$\Kn_2 p$}{v}{(2)}\\
    \nnode{25}{6.}{$vR_2v$}{}{\;\;(Refl.)}\\
    \nnode{25}{7.}{$p$}{v}{(5,6)}\\
    \nnode{25}{8.}{$wR_2 w$}{}{\;\;(Refl.)}\\
    \nnode{25}{9.}{$\neg\Kn_1 p$}{w}{(3,8)}\\
    \nnode{25}{10.}{$wR_1u$}{}{(9)}\\
    \nnode{25}{11.}{$\neg p$}{u}{(9)}
  }
  \end{center}
  % 
  We could add a few more applications of Reflexivity, but the tree would remain
  open. It also gives us a countermodel: let $W$ = $\{ w,v,u \}$; $w$ has
  1-access to $v$ and $u$; each world has 1- and 2-access to itself;
  $V(p,v) = 1$ and $V(p,u) = 0$. In this model, at world $w$, $\Mi_1\Kn_2 p$ is
  true while $\Mi_2\Kn_1 p$ is false.

  Cases (b) and (d) are similar.
  
\end{solution}

We can also define new modal operators for groups of agents. A proposition is
said to be \textbf{mutually known} in a group $G$ if it is known by every member
of the group. Let $\EKn_G$ be an operator for mutual knowledge. Clearly,
$\EKn_G A$ can be defined as $\Kn_1 A \land \Kn_2 A \land \ldots \land \Kn_n A$,
where $\Kn_1, \Kn_2, \ldots, \Kn_n$ are the knowledge operators for the members
of the group. So we can't say anything new with the help of $\EKn_G$ (at least
for finite groups). But it can be instructive to see how $\EKn_G$ behaves
depending on the behaviour of the underlying operators $\Kn_1,\Kn_2,$ etc. For
example, if each individual knowledge operator validates the \pr{T}-schema, then
so does $\EKn_G$; but if each $\Kn_i$ validates \pr{4} (positive introspection),
it does not follow that $\EKn_G$ validates \pr{4}. As a counterexample, consider
a group of two agents; both know $p$, and both know of themselves that they know
$p$, but agent 1 does not know that agent 2 knows $p$. Then $\EKn_G p$ but
$\neg \EKn_G \EKn_G p$.

\begin{exercise}
  Give an example to show that if each $\Kn_i$ validates \pr{5},
  it does not follow that $\EKn_G$ validates \pr{5}.
\end{exercise}
\begin{solution}
  The \pr{5}-schema for $\EKn_G$ states that
  $\neg \EKn_G \neg A \to \EKn_G \neg \EKn_G \neg A$. To show that some instance
  of this is invalid, we need to find a case where some instance of
  $\neg \EKn_G \neg A$ is true while $\EKn_G \neg \EKn_G \neg A$ is false. We
  can take the simplest instance, with $A=p$. Assume the relevant group has two
  agents, and consider a world $w$ at which $\Kn_1 \neg p$ and
  $\neg \Kn_2 \neg p$ are true. By the assumption that \pr{5} is valid for
  $\Kn_i$, $\Kn_2\neg\Kn_2\neg p$ is also true at $w$. But
  $\Kn_1\neg \Kn_2\neg p$ can be false (at $w$). If it is, then
  $\neg \EKn_G \neg p$ is true at $w$ while $\EKn_G \neg \EKn_G \neg p$ is
  false.
\end{solution}

% \begin{exercise}
%   Define the accessibility relation $R_G$ for $\EKn_G$ in
%   terms of the accessibility relations for the members of $G$, so that
%   \[
%     M,w \models \EKn_G A \text{ iff } M,w' \models A \text{ for all $w'$ such that } wRw'.
%   \]
%   \cmnt{%
%     Suppose we define $R_E = \bigcup_i R_i$. Let $M,w \models E^*(A)$
%     iff $M,w \models A$ for all $v$ such that $wR_E v$. Show that
%     $\models E^*(A) \leftrightarrow E(A)$.
%   } %
% \end{exercise}

A more interesting concept that has proved useful in many areas is that of
common knowledge. A proposition is \textbf{commonly known} in a group if
everyone knows it, everyone knows that everyone knows it, everyone knows that
everyone knows that everyone knows it, and so on forever. Let's use $\CKn_{G}$
as an operator for common knowledge. $\CKn_G$ is not definable in terms of
$\Kn_1, \ldots,\Kn_n$. Still, we can define it semantically in terms of the
accessibility relations for the individual agents: $\CKn_{G} A$ is true at a
world $w$ iff $A$ is true at all worlds that are reachable from $w$ by some
finite sequence of steps following the agents' accessibility relations.

%  So the accessibility relation is the transitive closure of $R_E$.

It is easy to see that common knowledge validates (all instances of) \pr{4}. It
validates \pr{T} whenever individual knowledge validates \pr{T}. So the logic of
common knowledge is at least S4. The complete logic of common knowledge also
contain some non-trivial interaction principles, which are easiest to state in
terms of $\EKn_G$:
%
\begin{principles}
  \pri{CK1} \CKn_{G}A \leftrightarrow (A \land \EKn_{G}\CKn_{G} A)\\
  \pri{CK2} (A \land \CKn_{G}(A \to \EKn_{G} A)) \to \CKn_{G} A
\end{principles}
%
You may want to confirm that these are valid. (They also provide a complete
axiomatization of common knowledge when added to an axiomatic calculus for
individual knowledge, but that is much harder to see.)

% \begin{exercise}
%   Show that if the logic of knowledge is S4, then the logic of common
%   knowledge is also S4. And if the logic of K is S4.2, the logic of
%   common knowledge is still S4. (Stalnaker 175, 195f.)
% \end{exercise}

\section{Knowledge, belief, and other modalities}
\label{sec:kb}

Issues in the logic of knowledge can sometimes be clarified by looking at the
connections between knowledge and belief. To formalise these connections, let's
introduce a new operator $\Bel$ for belief -- or rather, for \emph{implicit
  belief}, since $\Bel$, like $\Kn$, will be closed under logical
consequence.

An agent's belief state represents the world as being a certain way. For every
possible world, we can ask whether it matches what the agent believes. If, for
example, your only non-trivial belief is that there are seventeen types of
parrot, then every world in which there are seventeen types of parrot matches
your beliefs. Every such world is \emph{doxastically accessible} for you. As you
acquire further beliefs, the space of doxastically accessible worlds becomes
smaller and smaller.

We interpret $\Bel p$ as saying that $p$ is true at all doxastically accessible
worlds (for whatever agent we have in mind). Since we won't spend a lot of time
with this operator, we will simply write its dual as $\neg\Bel\neg$.

The logic of $\Bel$ is different from the logic of $\Kn$, if only because
beliefs can be false. So we will not regard all instances of 
%
\principle{T}{\Bel A \to A}
%
as valid. We may, however, accept the weaker schema
%
\principle{D}{\Bel A \to \neg \Bel \neg A.}
%
This reflects the assumption that a belief state that represents the world as
being a certain way $A$ can't also represent the world as being the opposite way
$\neg A$.

In the previous section, I argued that (implicit) knowledge does not validate
the negative introspection principle \pr{5}, and I reviewed an argument against
the positive introspection principle \pr{4}. Neither argument carries over to
belief. Many epistemic logicians accept positive and negative introspection for
(implicit) belief:
%
\begin{principles}
  \pri{4}{\Bel A \to \Bel \Bel A}\\
  \pri{5}{\neg \Bel A \to \Bel \neg \Bel A}
\end{principles}

The logic that results by adding the schemas \pr{D}, \pr{4}, and \pr{5} to the
axiomatic basis for K is known as KD45.

\begin{exercise}
  Is a transitive, serial, and euclidean relation always symmetric? If yes,
  explain why. If no, give a counterexample. What does your result mean for
  schema \pr{B} in KD45?
\end{exercise}
\begin{solution}
  No, a transitive, serial, and euclidean relation is not always symmetric.
  Counterexample: wRv, vRv. This means that not all instances of \pr{B} (which
  corresponds to symmetry) are valid in KD45.
\end{solution}

\begin{exercise}
  Show (in any way you like) that $\Bel(\Bel A \to A)$ is valid if the logic of
  belief is KD45.
\end{exercise}
\begin{solution}
  You can e.g. do a tree proof, using $\Bel$ as the box.
\end{solution}

If we want to model the connection between knowledge and belief, we need a
multi-modal language with both the $\Kn$ operator and the $\Bel$ operator.
Models for this language will have two accessibility relations $R_{e}$ and
$R_{d}$. The first represents epistemic accessibility and is used for the
interpretation of $\Kn$, the second represents doxastic accessibility and is
used to interpret $\Bel$.

The power of combined logics for (implicit) knowledge and belief lies in the
interaction principles that might link the two concepts. Here is a list of
popular principles that don't follow from the individual logics of knowledge and
belief.
\begin{principles}
  \pri{KB} \Kn A \to \Bel A\\
  \pri{PI} \Bel A \to \Kn\Bel A\\
  \pri{NI} \neg \Bel A \to \Kn\neg \Bel A\\
  \pri{SB} \Bel A \to \Bel \Kn A
\end{principles}

\pr{KB} assumes that knowledge implies belief. \pr{PI} and \pr{NI} strengthen
the introspection principles for belief. They assume that a state of belief or
disbelief is always known to the agent. \pr{SB} assumes that if an agent
believes something then they also believe that they know it. This is sometimes
said to reflect a conception of ``strong belief'', on which belief is
incompatible with doubt. If you believe $p$ in the sense that you have no doubt
that $p$, then you plausibly believe that you know $p$.

These interaction principles, together with the \pr{D}-schema for belief, imply
that an agent believes a proposition just in case they don't know that they
don't know it:
%
\principle{BMK}{\Bel A \leftrightarrow \Mi\Kn A}
%
Somewhat surprisingly, then, we could define belief in terms of knowledge.

Here is how we can get from $\Bel A$ to $\Mi\Kn A$.
%
\begin{enumerate}[leftmargin=10mm]
  \itemsep-1mm
\item Suppose $\Bel A$.
\item By \pr{SB}, it follows that $\Bel \Kn A$.
\item By \pr{D}, it follows that $\neg \Bel \neg \Kn A$.
\item By \pr{KB}, it follows that $\neg \Kn \neg \Kn A$, and so that $\Mi\Kn A$.
\end{enumerate}

To show that $\Mi\Kn A$ entails $\Bel A$, I'll show that $\neg \Bel A$
entails $\neg \Mi\Kn A$.
%
\begin{enumerate}[leftmargin=10mm]
  \itemsep-1mm
\item By \pr{KB}, $\neg \Bel A \to \neg \Kn A$ is a logical truth.
\item Since logical truths are true at every world, we have $\Kn(\neg \Bel A \to \neg \Kn A)$.
\item By the \pr{K}-schema, it follows that $\Kn\neg \Bel A \to \Kn \neg \Kn A$.
\item Now suppose $\neg \Bel A$.
\item By \pr{NI}, it follows that $\Kn \neg \Bel A$.
\item By 3 above, it follows that $\Kn \neg \Kn A$, which is equivalent to $\neg \Mi\Kn A$.
\end{enumerate}

Given the equivalence between $\Bel A$ and $\Mi\Kn A$, the \pr{D}-schema for
belief
\[
  \Bel A \to \neg \Bel\neg A
\]
is equivalent to
\[
  \Mi \Kn A \to \neg\Mi\Kn \neg A
\]
which in turn is equivalent to
\[
  \Mi \Kn A \to \Kn\Mi A.
\]
This is the \pr{G}-schema for knowledge. So if we accept the above interaction
principles, and principle \pr{D} for belief, then the logic of knowledge must
validate \pr{G}.

(In fact, we don't need to assume that the interaction principles and \pr{D}
hold for our ordinary concept of belief. As long as one can coherently define a
concept $\Bel$ that validates these principles we can derive the \pr{G}-schema
for $\Kn$.)

\begin{exercise}
  Show that the interaction principles entail principles \pr{4} and \pr{5} for
  belief: $\Bel A \to \Bel\Bel A$ and $\neg\Bel\neg A \to \Bel\neg\Bel\neg A$.
\end{exercise}
\begin{solution}
  Let $A$ be an arbitrary proposition.

  By \pr{PI}, $\Bel A \to \Kn\Bel A$ is valid. By \pr{KB}, so is
  $\Kn\Bel A \to \Bel\Bel A$. By propositional logic, these entail
  $\Bel A \to \Bel\Bel A$.

  By \pr{NI}, $\neg \Bel \neg A \to \Kn\neg \Bel \neg A$ is valid. By \pr{KB},
  so is $\Kn\neg \Bel \neg A \to \Bel\neg \Bel \neg A$. By propositional logic,
  these entail $\neg \Bel \neg A \to \Bel\neg \Bel \neg A$.
\end{solution}

\begin{exercise}
  Suppose the logic of knowledge validates \pr{5}, the logic of belief validates
  \pr{D}, and we have the interaction principles \pr{KB} and \pr{SB}. Show that
  knowledge is then equivalent to belief: $KA \leftrightarrow BA$ comes out as
  valid. (Another reason to think that \pr{5} is not valid in the logic of
  knowledge.)
\end{exercise}
\begin{solution}
  The left-to-right direction is \pr{KB}. For the right-to-left direction, let
  $A$ be an arbitrary proposition. By \pr{SB}, $\Bel A \to \Bel\Kn A$ is valid.
  By \pr{D} for belief, $\Bel \Kn A \to \neg\Bel \neg \Kn A$ is valid. \pr{KB}
  gives us $\neg \Bel \neg \Kn A \to \neg\Kn\neg\Kn A$. Finally, \pr{5} for
  knowledge yields $\neg \Kn\neg A \to \Kn A$. The target proposition
  $\Bel A \to \Kn A$ is a truth-functional consequence of these four
  propositions.
  % BA -> BKA (SB)
  % BKA -> -B-KA (D)
  % -B-KA -> -K-KA (KB)
  % -K-KA -> KA (5)
\end{solution}

% In KD45 R is an equivalence relation ``once removed'': any accessible world sees
% itself, and there's symmetry among all accessible worlds.

\begin{exercise}
  There seems to be no natural expression in English for the dual of belief. A
  common way to express that someone does not believe not $p$ is to say that
  they believe that it might be that $p$, which has the surface form
  $\Box\Diamond p$. Can you explain why this might be an adequate way of expressing $\Diamond p$?
\end{exercise}
\begin{solution}
  If the logic of belief is KD45 then $\Box\Diamond p$ is equivalent to
  $\Diamond p$ (as you can show, for example, with a tree proof).
\end{solution}

It can also be instructive to combine epistemic with non-epistemic operators.
Philosophers have often been interested not just in what we \emph{do} know, but
also in what we \emph{can} know. Various skeptical arguments, for example,
suggest that we \emph{cannot know} that we have hands. For another example, the
``verificationist'' movement in the early 20th century assumed that a sentence
is meaningful only if its truth-value can in principle be settled by
mathematical proof or empirical investigation. This would imply that a sentence
is meaningful only if \emph{it is possible to know} that it is true.

We can formalize claims like these in a multi-modal language with a knowledge
operator $\Kn$ and a diamond $\Diamond$ for the relevant kind of circumstantial
possibility. The verificationist hypothesis that every truth is in principle
knowable is then expressed by the following interaction principle:
%
\begin{equation}\tag{Knowability}
  A \to \Diamond \Kn A
\end{equation}

The principle is refuted by the following argument, due to Alonzo Church.

\begin{enumerate}[leftmargin=10mm]
  \itemsep-1mm
  \item Let $p$ be any unknown truth. (Nobody thinks all truths are actually
        known.)
  \item So we have $p \land \neg \Kn p$.
  \item In any logic that extends the minimal system K,
        $\Kn(p \land \neg \Kn p)$ entails $\Kn p \land \Kn \neg \Kn p$.
  \item By the \pr{T}-schema for knowledge, $\Kn \neg \Kn p$ entails
        $\neg \Kn p$.
  \item So $\Kn(p \land \neg \Kn p)$ entails both $\Kn p$ and $\neg \Kn p$.
  \item So the hypothesis $\Kn(p \land \neg \Kn p)$ is inconsistent.
  \item So $\neg \Diamond \Kn(p \land \neg \Kn p)$.
  \item Lines 2 and 7 together provide a counterexample to the Knowability
        principle.
\end{enumerate}

\begin{exercise}
  Show that if the logic of belief is at least KD4, then there are
  \emph{unbelievable truths}: truths of which it is impossible that anyone
  believes them. (You can assume that there are truths which no-one in fact
  believes.)
\end{exercise}
\begin{solution}
  Suppose $\Bel(p \land \neg \Bel p)$. In any logic that extends K, it follows
  that $\Bel p$ and $\Bel \neg \Bel p$. By \pr{4}, $\Bel p$ entails
  $\Bel\Bel p$. Now we have $\Bel \neg \Bel p$ and $\Bel\Bel p$, which violates \pr{D}.
\end{solution}








%%% Local Variables: 
%%% mode: latex
%%% TeX-master: "ml.tex"
%%% End:
