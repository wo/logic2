\begingroup
\thispagestyle{empty}

\begin{center}

%\renewcommand{\bfdefault}{sb}%
%\fontfamily{put}\bfseries

  % {\Large A Philosophical Introduction to}
  
  % \vspace{2mm}

  {\huge Logic 2: Modal Logic\par}

  \vspace{5mm}
  
  % {\Large A Philosophical Introduction}
  

  \vspace{10mm}

  {\normalsize Wolfgang Schwarz}

  \vspace{2mm}
  {\normalsize \today}

%\renewcommand{\bfdefault}{bx}%
\end{center}

\vfill
\endgroup
{

  \footnotesize

 \noindent \copyright\ \the\year\ Wolfgang Schwarz

 \smallskip
 \noindent \href{https://github.com/wo/logic2}{github.com/wo/logic2}

 \vspace{-1.5mm}
 \doclicenseThis

}

  

{
\tableofcontents 
}

%\cleardoublepage % Forces the first chapter to start on an odd page so it's on the right

%\pagestyle{fancy} % Print headers again


\cmnt{%

  Math and logic can only be learned slowly. If you practise a little
  bit every day, you'll find it much easier than if you binge-study
  before the exams. 

} %

\cmnt{%

It's okay to jump around a bit, using as much probability theory as
required for one task, then move on, and later return to learn more
for another task, etc. Make sure you often zoom out to the big
picture.

Go on tangents. Don't be afraid to mention things that seem way over
their head. E.g. infinitesimals and surreal numbers. Be playful by
asking questions: does conditionalisation have an inverse? what if
there are infinitely many possibilities? Talk about history.

so then i started using a worksheet, like with some bits of the text
from the required reading on it, and some discussion questions
starting with comprehension and going into philosophical questions,
and then after the reading stuff on the previous week's lectures in
general. that worked a lot better because it meant that the small
groups were able to work independently by just working through the
sheet, and i could go around and join each group one by one and listen
in or help / redirect if necessary. so that was good, a lot less
effort for me and probably easier for them. next week i think i will
try it with the last ten or fifteen minutes with a whole-class
discussion maybe just having the groups report back to each other?

xxx exams are "tests" not exames, this is 100 percent coursework

xxx practicing and recalling are the best tools for learning, much
better than re-reading etc. Retrieval is key to learning.

}

%%% Local Variables: 
%%% mode: latex
%%% TeX-master: "logic2.tex"
%%% End:
