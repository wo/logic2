\documentclass{../wobook2018}

\definecolor{commentcol}{rgb}{0.0, 0.2, 0.1}
%\newcommand{\cmnt}[1]{{\small\color{commentcol}#1}}
\newcommand{\cmnt}[1]{{\iffalse #1 \fi}}


\usepackage[normalem]{ulem}
% \usepackage{logicproof}
% \usepackage{KMcalc}

\begin{document}

\section*{More tree exercises}

\renewcommand{\KMproveword}{Show}
\renewcommand{\KMProveword}{\sout{Show}}
\newlength{\proofwidth}
\setlength{\proofwidth}{6cm}
\newcommand{\ndline}[2]{\KMline{\;\makebox[\proofwidth][l]{$#1$}#2}{}}
\newcommand{\ndprove}[2]{\KMprove{\;\makebox[\proofwidth][l]{$#1$}#2}{}}

\cmnt{
  \textbf{ND in Parsons}

  https://sites.google.com/site/tparsons5555/home/logic-text
  
  Parsons introduces a Kalish-Montague-style ND system, similar to
  Fitch-style, but with boxes around subproofs.
  
  ch.1, p.11 lists the ``rules'' of repetition, MP, MT, and (two rules
  of) double negation. The rules state what ``you can infer'' from
  what.

  p.13 introduces ``direct derivations'' which consist of stringing
  together simple inferences. Example:

  \begin{logicproof}{1}
    \text{\sout{To show} $\neg W$:} & \\
    \begin{subproof}
     \neg T & pr\\
    \neg P \to T & pr\\
    \neg\neg P & 2\; 3\; mt\\ 
    P &  4\; dn\\
   \ldots & \\
    \neg W & 7\;8\; mp\; dd
    \end{subproof}
  \end{logicproof}

  The box contains the actual proof; `dd' indicates that it is a
  direct derivation. P.16 precisely defines direct derivations.

  p.19 introduces ``conditional derivations''. These look like so:

  \begin{logicproof}{1}
    \text{\sout{To show} $R \to \neg W$:} & \\
    \begin{subproof}
     R & ass cd\\
     R \to \neg S  & pr\\
     \ldots & \\
     \neg W & 7\;8\; mt\; cd
    \end{subproof}
  \end{logicproof}

  p.22 introduces ``indirect derivations'', for reductio arguments:
  
  \begin{logicproof}{1}
    \text{\sout{To show} $\neg(R \to W)$:} & \\
    \begin{subproof}
     R\to W & ass id\\
     R  & pr\\
     W & 2 3 mp\\
     \neg W & pr 4 id
    \end{subproof}
  \end{logicproof}

  p.27 introduces derivations within derivations. Here the striked-out
  ``show'' lines come in handy: after striking out, the line holds an
  ordinary sentence which can be used further down, and the box proof
  below it can be ignored.

  p.30 summarizes the derivation rules.

  p.43 adds another rule: ``any instance of any previously derived
  theorem may be entered on any line of a derivation. As
  justification, write the name of the theorem. (E.g. ‘T13’.)''

  ch.2 introduces '$\land$', `$\lor$', and `$\leftrightarrow$' into
  the language.

  p.11 introduces ``rules'' for these:
  \begin{itemize}
  \item rule s: $A \land B \therefore A$
    
  \item rule adj: $A, B \therefore A \land B$
    
  \item rule add: $A \therefore A \lor B$, $B \therefore A \lor B$

  \item rule mtp: $A \lor B, \neg A \therefore B$ and variants

  \item rule bc: $A \leftrightarrow B \therefore A \to B$ etc.

  \item rule cb: $A \to B, B \to A \therefore A \leftrightarrow B$
  \end{itemize}

  pp.25--31 introduces some ``derived rules'' such as de Morgan's.

  
}

Natural deduction proofs try to mirror intuitive (``natural'') ways of arguing
for a conclusion. For example, if you wanted to show that a conjunction
$p\land q$ is true, an intuitive approach would be to show that $p$ is true,
then show that $q$ is true, and then infer that $p\land q$ is true. Since people
disagree over what kinds of inference are natural, there are many styles of
natural deduction. I will not survey all the possibilities. Instead, I will
briefly explain how one particular style of natural deduction -- known as the
\emph{Kalish-Montague} style -- can be extended to modal logic.

% See
% https://sites.ualberta.ca/~francisp/papers/PellHazenSubmittedv2.pdf
% for a survey. KM is derived from an idea roposed by Stanisław
% Jaśkowski)

Let's say we want to prove $(p \land q) \to q$, in classical
propositional logic. In a Kalish-Montague proof, we'd start by writing
down our goal, like this.

\setlength{\proofwidth}{5cm}

\begin{KMcalc}[]
  \ndline{\text{Show\, } (p \land q) \to q}{}
\end{KMcalc}

\noindent%
A (supposedly) ``natural'' way to prove a conditional $A\to B$ is to assume the
antecedent $A$ and derive the consequent $B$. We might therefore start a
subproof in which we try to derive $q$ from $p \land q$.

\begin{KMcalc}[]
  \ndline{\text{Show\, } (p \land q) \to q}{}
  \ndline{\quad p \land q}{ass cd}
\end{KMcalc}

\noindent%
The annotation `ass cd' tells us that we're \emph{assuming} $p\land q$
for the purpose of a \emph{conditional derivation}. From $p\land q$ we
can directly infer $q$, by the rule of ``simplification'' (also known as
``conjunction elimination'').

\begin{KMcalc}[]
  \ndline{\text{Show\, } (p \land q) \to q}{}
  \ndline{\quad p \land q}{ass cd}
  \ndline{\quad q}{2, s}
\end{KMcalc}

\noindent%
Having derived $q$ from $p \land q$, we can infer $(p \land q) \to q$.
So we cross out `Show' from `Show $(p \land q) \to q$' and close off
the subproof by putting it in a box.
%
\begin{KMcalc}[]
  \ndprove{(p \land q) \to q}{}
  \ndline{\quad p \land q}{ass cd}
  \ndline{\quad q}{2 s}
  \ndline{}{2 3 cd}
  \KMclose[]
\end{KMcalc}
%
The empty last line indicates that the box was closed by the rule of
conditional derivation applied to lines 2 and 3.

A proof can contain several subproofs, and subsubproofs within
subproofs. Different subproofs are isolated from one another: if
you've introduced an assumption $A$ in one subproof, you can't draw on
$A$ in another subproof, except if the second subproof is embedded in
the first. Sentences from a higher-up level may be imported into a
subproof, by the rule of ``repetition''.

You can find a complete description of this proof method, with all its
rules, in Terence Parson's \emph{Exposition of Symbolic Logic}, which
is freely available at
\href{https://sites.google.com/site/tparsons5555/home/logic-text}{sites.google.com/site/tparsons5555/home/logic-text}.

The method is easily extended to a range of modal logics. To reflect
the duality of the box and the diamond, we need to add a ``modal
negation'' rule \emph{mn}. It is actually four rules:
%
\[ 
  \emph{mn:}\quad
  \neg \Box \neg A \therefore \Diamond A \qquad
  \neg \Diamond \neg A \therefore \Box A \qquad
  \neg \Box A \therefore \Diamond\neg A \qquad
  \neg \Diamond A \therefore \Box \neg A
\]
The three dots `$\therefore$' indicate that any instance of the schema
on the right can be inferred from the corresponding instance of the
schema on the left. So `$\neg \Box \neg A \therefore \Diamond A$'
states that one may infer, say, $\Diamond (p \to \Box p)$ from
$\neg \Box \neg (p \to \Box p)$.

We also need a new type of derivation, \emph{sd} (for ``strict
derivation''), to derive sentences of the form $\Box A$. Strict
derivations use a special kind of subproof that starts with no
assumption. Intuitively, the subproof takes you to an arbitrary new
world that is accessible from a world at which the sentences you have
previously proved (or assumed) are true. Your goal is to prove that
$A$ holds at this world. If that is done, the subproof can be closed
and $\Box A$ has been shown. In this kind of subproof, you are not
allowed to import sentences from outside the subproof by the
repetition rule. Instead, you have to use a \emph{modal importation
  rule}.

The basic importation rule, \emph{im}, says that if some boxed
sentence $\Box A$ has been established on a higher-up level in a
proof, then you may assume the corresponding sentence $A$ inside a
strict derivation.

Here is a proof of $(\Box p \land \Box q) \to \Box (p\land q)$, using
these resources.
 
\begin{KMcalc}[]
  \ndprove{(\Box p \land \Box q) \to \Box(p \land q)}{}
  \ndline{\Box p \land \Box q}{ass cd}
  \ndline{\Box p}{2, s}
  \ndline{\Box q}{2, s}
  \ndprove{\Box(p \land q)}{}
  \ndline{p}{3, im}
  \ndline{q}{4, im}
  \ndline{p \land q}{6, 7, adj}
  \ndline{}{8, sd}
  \KMclose[]
  \ndline{}{2, 5, cd}
  \KMclose[]
\end{KMcalc}

\noindent%
On line 6, the modal importation rule $im$ is used to import
assumption $p$, based on assumption $\Box p$ on line 3 (which is on a
higher-up level in the proof). Similarly for $q$ on line 7. Line 9
indicates that since $p \land q$ could be derived for an arbitrary
accessible world, we can infer $\Box(p \land q)$, by \emph{strict
  derivation}.

These rules suffice to prove every K-valid sentence. For stronger
systems of modal logic, we need further rules.

For example, for the system T we would add the rule
\[
  \emph{ni:} \quad \Box A \therefore A.
\]
For system D, we would instead add
\[
  \emph{bd:} \quad \Box A \therefore \Diamond A.
\]
For K4, we need another modal importation rule. This rule, \emph{im4},
allows you to import sentences of type $\Box A$ unchanged into a
strict derivation. The rule is used in the following proof of
$\Box p \to \Box \Box p$.

\begin{KMcalc}[]
  \ndprove{\Box p \to \Box\Box p}{}
  \ndline{\Box p}{ass cd}
  \ndprove{\Box\Box p}{}
  \ndline{\Box p}{2, im4}
  \ndline{}{4, sd}
  \KMclose[]
  \ndline{}{2, 5, cd}
  \KMclose[]
\end{KMcalc}
%

\noindent
K5 requires a similar modal repetition rule, \emph{im5}. This
one allows you to import sentences of type $\Diamond A$ unchanged into
strict derivations.

If both \emph{ni} and \emph{im4} are added to the natural deduction
rules for K, we get a natural deduction system for S4. \emph{ni} and
\emph{im5} together yield a natural deduction system for S5. For S4.2,
yet another rule, \emph{img}, is needed, which allows importing
sentences of type $\Diamond \Box A$ unchanged into strict derivations.

% This claim is from Hazen 1972, "Semantics for S4.2", p.527.

% \begin{exercise}
%   \beginwithlist
%   \begin{exlist}
%   \item Use the K5-rules to prove $\Diamond p \to \Box\Diamond p$.
%   \item Use the K-rules to prove $\Box(p \to q) \to (\Box p \to \Box q)$. 
%   \end{exlist}
  
% 1. Show <>p -> []<>p
% 2. <>p       ass cd
% 3. Show []<>p
% 4. <>p         2, mr5
% 5.             4, sd

% - Construct a T-derivation of [](p->q) -> ([]p ->[]q)

% 1. Show [](p->q) -> ([]p ->[]q)
% 2. [](p->q)    ass id
% 3. Show []p ->[]q
% 4. []p          ass id
% 5. Show []q
% 6. (p->q)        2, t rule
% 7. p             4, t rule
% 8. q             6,7,mp
% 9.               8, nec

% - Construct an S5-derivation of p -> []<>p
% - Construct an S4-derivation of <><>p -> <>p
% - Construct an S5-derivation of <>[]p -> []p

% \end{exercise}

% Completeness is easy to show by giving ND proofs of all axioms.

% Pelletier and Hazen: Not all modal logics have nice natural
% deduction formulations, but those that do include many of the most
% important for applications. Fitch (1966) gives natural deduction
% formulations for a number of alethic and deontic logics and systems
% with both alethic and deontic operators. Fitting (2002) describes
% natural deduction formulations of a large num- ber of logics, in
% addition to tableau versions. Several introductory logic texts (for
% example, Anderson and Johnstone, 1962; Iseminger, 1968; Purtill,
% 1971; Bonevac, 1987; Gamut, 1991; Forbes, 1994; Bessie and Glennan,
% 2000) include natural deduction formulations of one or more systems
% of modal logic, and Garson (2006), intended for students who have
% already done a bit of logic, introduces many modal logics, both
% propositional and quantified, with both natural deduction and
% tableau formulation.



\end{document}
