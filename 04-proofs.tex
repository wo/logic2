\chapter{Models and Proofs}\label{ch:proofs}

\section{Soundness and completeness}

You may find that this chapter is harder and more abstract than the previous
chapters. Feel free to skip or skim it if you're mostly interested in
philosophical applications.

We have introduced several kinds of validity: S5-validity, K-validity,
T-validity, and so on. All of these are defined in terms of models. K-validity
means truth at all worlds in all Kripke models. T-validity means truth at all
worlds in all reflexive Kripke models. S5-validity means truth at all worlds in
all universal Kripke models (equivalently, at all worlds in all ``basic''
models). And so on.

If you want to show that a sentence is, say, K-valid, you could directly work
through the clauses of definition \ref{def:kripkesemantics}, showing that there
is no world in any Kripke model in which the sentence is false. The tree method
regiments and simplifies this process. If you construct a tree for your sentence
in accordance with the K-rules and all branches close, then the sentence is
K-valid. If some branch remains open, the sentence isn't K-valid.

Or so I claimed. But these claims aren't obvious. The tree rule for the diamond,
for example, appears to assume that if $\Diamond A$ is true at a world then $A$
is true at some accessible world \emph{that does not yet occur on the branch}.
Couldn't $\Diamond A$ be true because $A$ is true at an accessible ``old'' world
instead? Also, why do we expand $\Diamond A$ nodes only once? Couldn't $A$ be
true at multiple accessible worlds?

In the next two sections, we are going to lay any such worries to rest. We are
going to prove that (1) if all branches on a K-tree close then the target
sentence is K-valid; conversely, (2) if some branch on a fully developed K-tree
remains open, then the target sentence is not K-valid. (1) establishes the
\emph{soundness} of the tree rules for K, (2) establishes their
\emph{completeness}.

When you use the tree method, you don't have to think of what you are doing as
exploring Kripke models. I could have introduced the method as a purely
syntactic game. You start the game by writing down the negation of the target
sentence, followed by `(w)' (and possibly `1.' to the left and `(Ass.)' to the
right, although in this chapter we will mostly ignore these book-keeping
annotations.) Then you repeatedly apply the tree rules until either all branches
are closed or no rule can be applied any more. At no point in the game do you
need to think about what any of the symbols you are writing might mean.

Soundness and completeness link this syntactic game with the ``model-theoretic''
concept of validity. Soundness says that if the game leads to a closed tree (a
tree in which all branches are closed) then the target sentence is true at all
worlds in all models. Completeness says that if the game doesn't lead to a
closed tree then the target sentence is not true at all worlds in all models.
This is called completeness because it implies that every valid sentence can be
shown to be valid with the tree method.

In general, a proof method is called \textbf{sound} if everything that is
provable with the method is valid. A method is \textbf{complete} if everything
that is valid is provable. Strictly speaking, we should say that a method is
sound or complete \emph{for a given concept of validity}. The tree rules for K
are sound and complete for \emph{K-validity}, but not for T-validity or
S5-validity.

The tree method is not the only method for showing that a sentence is K-valid
(or T-valid, or S5-valid). Instead of constructing a K-tree, you could construct
an axiomatic proof, trying to derive the target sentence from some instances of
\pr{Dual} and \pr{K} by \pr{Nec} and \pr{CPL}. This, too, can be done as a
purely syntactic exercise, without thinking about models and worlds. In section
\ref{sec:scaxiomatic}, we will show that the axiomatic calculus for K is indeed
sound and complete for K-validity: all and only the K-valid sentences can be
derived from \pr{Dual} and \pr{K} by \pr{Nec} and \pr{CPL}. The `all' part is
completeness, the `only' part soundness. Having shown soundness and completeness
for both the tree method and the axiomatic method, we will have shown that the
two methods are equivalent. Anything that can be shown with one method can also
be shown with the other.

There are other styles of proof besides the axiomatic and the tree format. Two
famous styles that we won't cover are ``natural deduction'' methods and
``sequence calculi''. Logicians are liberal about what qualifies as a proof
method. The only non-negotiable condition is that there must be a mechanical way
of checking whether something (usually, some configuration of symbols) is or is
not a proof of a given target sentence.

\begin{exercise}\label{ex:proofmethods}
  What do you think of the following proposals for new proof methods?
  \begin{exlist}
    \item In \emph{method A}, every $\L_{M}$-sentence is a proof of itself: To
    prove an $\L_{M}$-sentence with this method, you simply write down the
    sentence.
    \item In \emph{method B}, every $\L_{M}$-sentence that is an instance of
    $\Box(A \lor \neg A)$ is a proof of itself. Nothing else is a proof in
    method B.
    \item In \emph{method C}, a proof of a sentence $A$ is a list of
    $\L_{M}$-sentences terminating with $A$ and in which every sentence occurs
    in some logic textbook.
  \end{exlist}
  \medskip%
  
  Which of these qualify as genuine proof methods by the criterion I have
  described?
\end{exercise}
\begin{solution}
  Methods A and B are genuine proof methods. Method C is not because there is no
  simple mechanical check of whether a sentence occurs in some logic textbook.
\end{solution}

\begin{exercise}\label{ex:sillyproofmethod}
  Which, if any, of the methods from the previous exercise are sound for
  K-validity? Which, if any, are complete?
\end{exercise}
\begin{solution}
  Method A is complete, but not sound. Everything that's K-valid is provable
  with the method, but so is everything that's not K-valid.
  
  Method B is sound, but not complete. Since every instance of
  $\Box (A \lor \neg A)$ is K-valid, everything that is provable with method B
  is K-valid. But many K-valid sentences (e.g., $p \to p$) aren't provable with
  method B.

  Method C is neither sound nor complete. It is not sound because many K-invalid
  sentences figure in logic textbooks. It is not complete because there
  are infinitely many K-valid sentences almost all of which don't occur in any
  textbooks.
\end{solution}

% A set of sentences (system) is sometimes called complete iff there
% is a sound and complete proof method for it.

\section{Soundness for trees}%
\label{sec:soundnesstrees}

We are now going to show that the tree method for K is sound -- that every sentence
that can be proved with the method is K-valid. A proof in the tree method is a
tree in which all branches are closed. So this is what we have to show:
%
\begin{quote}
  Whenever all branches on a K-tree close then the target sentence is K-valid.
\end{quote}
%
By a \emph{K-tree} I mean a tree that conforms to the K-rules from the previous
chapter.

I'll first explain the proof idea, then I'll fill in the details. We will assume
that there is a K-tree for some target sentence $A$ on which all branches close.
We need to show that $A$ is K-valid. To this end, we suppose for reductio that
$A$ is \emph{not} K-valid. By definition \ref{def:kvalid}, a sentence is K-valid
iff it is true at all worlds in all Kripke models. Our supposition that $A$ is
not K-valid therefore means that $A$ is false at some world in some Kripke
model. Let's call that world `$w$' and the model `$M$'. Note that the closed
tree begins with

\begin{center}
  \tree{%
    \nnode{12}{1.}{$\neg A$}{w}{}%
  }
\end{center}

\noindent%
If we take the world variable `$w$' on the tree to pick out world $w$ in $M$,
then node 1 is a correct statement about $M$, insofar as $\neg A$ is indeed true
at $w$ in $M$. Now we can show the following:

\begin{quote}
    \emph{If all nodes on some branch of a tree are correct statements about
      $M$, and the branch is extended by the K-rules, then all nodes on at least
      one of the resulting branches are still correct statements about $M$.}
\end{quote}
%
Since our closed tree is constructed from node 1 by applying the K-rules, it
follows that all nodes on some branch of the tree are correct statements about
$M$. But every branch of a closed tree contains a pair of contradictory
statements, which can't both be correct statements about $M$. This completes the
reductio.

Let's fill in the details. We first define precisely what it means for the
nodes on a tree branch to be correct statements about a model.

\begin{definition}{}{correctstatement}
  A tree node is a \textbf{correct statement about} a Kripke model
  $M = \t{M,R,V}$ \textbf{under} a function $f$ that maps world variables to
  members of $W$ iff either the node has the form $\omega R \upsilon$ and
  $f(\omega)Rf(\upsilon)$, or the node has the form $A\; (\omega)$ and $A$ is
  true at $f(\omega)$ in $M$.

  A tree branch \textbf{correctly describes} a model $M$ iff there is a function
  $f$ under which all nodes on the branch are correct statements about $M$.
\end{definition}

% I don't like these labels, but can't think of better once.

We now prove the italicised statement above:

\begin{theorem}{Soundness Lemma}{soundnesslemma}
  If some branch $\beta$ on a tree correctly describes a Kripke model $M$, and
  the branch is extended by applying a K-rule, then at least one of the
  resulting branches correctly describes $M$.
\end{theorem}

\begin{proof}
  \emph{Proof:} We have to go through all the K-rules. In each case we assume
  that the rule is applied to some node(s) on a branch $\beta$ that correctly
  describes $M$, so that there is a function $f$ under which all nodes on
  the branch are correct statements about $M$. We show that once the rule has
  been applied, at least one of the resulting branches still correctly describes
  $M$.
  
  \medskip
  \begin{itemize}
  \itemsep1mm
  
    \item Suppose $\beta$ contains a node of the form $A \land B \;(\omega)$ and
          the branch is extended by two new nodes $A \;(\omega)$ and
          $B \;(\omega)$. Since $A \land B \; (\omega)$ is a correct statement
          about $M$ under $f$, we have $M,f(\omega) \models A \land B$. By
          clause (c) of definition \ref{def:kripkesemantics}, it follows that
          $M,f(\omega) \models A$ and $M,f(\omega) \models B$. So the extended
          branch still correctly describes $M$.

    \item Suppose $\beta$ contains a node of the form $A \lor B \;(\omega)$ and
          the branch is split into two, with $A \;(\omega)$ appended to one end
          and $B \;(\omega)$ to the other. Since the expanded node is a correct
          statement about $M$ under $f$, we have $M,f(\omega) \models A \lor B$.
          By clause (d) of definition \ref{def:kripkesemantics}, it follows that
          either $M,f(\omega) \models A$ or $M,f(\omega) \models B$. So at least
          one of the resulting branches also correctly describes $M$.
          
  \end{itemize}
  %
  The proof for the other non-modal rules is similar. Let's look at the rules
  for the modal operators.

  \begin{itemize}
    \itemsep1mm
    
    \item Suppose $\beta$ contains nodes of the form $\Box A \;(\omega)$ and
          $\omega R \upsilon$, and the branch is extended by adding
          $A \;(\upsilon)$. Since $\Box A \;(\omega)$ and $\omega R \upsilon$
          are correct statement about $M$ under $f$, we have
          $M,f(\omega) \models \Box A$ and $f(\omega)Rf(\upsilon)$. By clause
          (g) of definition \ref{def:kripkesemantics}, it follows that
          $M,f(\upsilon) \models A$. So the extended branch correctly describes
          $M$.

    \item Suppose $\beta$ contains a node of the form $\Diamond A \;(\omega)$
          and the branch is extended by adding nodes $\omega R \upsilon$ and
          $A \;(\upsilon)$, where $\upsilon$ is new on the branch. Since
          $\Diamond A \;(\omega)$ is a correct statement about $M$ under $f$, we
          have $M,f(\omega) \models \Diamond A$. By clause (h) of definition
          \ref{def:kripkesemantics}, it follows that $M,v \models A$ for some
          $v$ in $W$ such that $f(\omega)Rv$. Let $f'$ be the same as $f$ except
          that $f'(\upsilon) = v$. The newly added nodes are correct statements
          about $M$ under $f'$. Since $\upsilon$ is new on the branch, all
          earlier nodes on the branch are also correct statements about $M$
          under $f'$. So the expanded branch correctly describes $M$.

          % Note that f(w) may equal f(v). A reflexive model can be correctly
          % described by a tree branch that contains no "reflexive" statements
          % of the form wRw. As far as soundness is concerned, the rule for <>A
          % doesn't assume that A is true at a "new world". The rule merely
          % assumes that A is true at some world, for which we introduce a new
          % name. That the world is genuinely new is only assumed when we
          % read off countermodels.
           
  \end{itemize}
  %
  The cases for $\neg \Box$ and $\neg \Diamond$ are similar to the previous two
  cases. \qed

\end{proof}

\medskip

With the help of this lemma, we can prove that the method of K-trees is sound.

\begin{theorem}{Theorem: Soundness of K-trees}{soundness-tree-K}
  If a K-tree for a target sentence closes, then the target sentence is K-valid.
\end{theorem}

\begin{proof}
  \emph{Proof:} Suppose for reductio that some K-tree for some target sentence
  $A$ closes even though $A$ is not K-valid. Then $\neg A$ is true at some world
  $w$ in some Kripke model $M$. The first node on the tree, $\neg A \; (w)$, is
  a correct statement about $M$ under the function that maps the world variable
  `$w$' to $w$. Since the tree is created from the first node by applying the
  K-rules, the Soundness Lemma implies that some branch $\beta$ on the tree
  correctly describes $M$: all nodes on the tree are correct statements about
  $M$ under some function $f$. But the tree is closed. This means that $\beta$
  contains contradictory nodes of the form
  
  \begin{center}
    \tree{%
      \nnode{12}{n.}{$B$}{\upsilon}{}\\
      \nnode{12}{m.}{$\neg B$}{\upsilon}{}
    }
  \end{center}
  If both of these are correct statements about $M$ under $f$, then
  $M,f(\upsilon) \models B$ and also $M,f(\upsilon) \models \neg B$. This is
  impossible by definition \ref{def:kripkesemantics}. \qed
\end{proof}

\begin{exercise}
  Spell out the cases for $A \to B$ and $\neg\Diamond A$ in the proof of the
  Soundness Lemma.
\end{exercise}
\begin{solution}
  For $A\to B$: Suppose $\beta$ contains a node of the form $A \to B \;(\omega)$
  and the branch is split into two, with $\neg A \;(\omega)$ appended to one end
  and $B \;(\omega)$ to the other. Since the expanded node is a correct
  statement about $M$ under $f$, we have $M,f(\omega) \models A \to B$. By
  clause (e) of definition \ref{def:kripkesemantics}, it follows that either
  $M,f(\omega) \not\models A$ or $M,f(\omega) \models B$. By clause (b), this
  means that either $M,f(\omega) \models \neg A$ or $M,f(\omega) \models B$. So
  at least one of the resulting branches also correctly describes $M$.

  For $\neg\Diamond A$: Suppose $\beta$ contains nodes of the form
  $\neg\Diamond A \;(\omega)$ and $\omega R \upsilon$, and the branch is
  extended by adding $\neg A \;(\upsilon)$. Since $\neg\Diamond A \;(\omega)$
  and $\omega R \upsilon$ are correct statement about $M$ under $f$, we have
  $M,f(\omega) \models \neg\Diamond A$ and $f(\omega)Rf(\upsilon)$. By clause
  (b) of definition \ref{def:kripkesemantics},
  $M,f(\omega) \models \neg\Diamond A$ implies
  $M,f(\omega) \not\models \Diamond A$. By clause (h), it follows that
  $M,f(\upsilon) \models \neg A$. So the extended branch correctly describes
  $M$.
\end{solution}

\begin{exercise}
  Draw the K-tree for target sentence $\Box p$. The tree has a single open
  branch. Does this branch correctly describe the Kripke model in which there is
  just one world $w$, $w$ has access to itself, and all sentence letters are
  false at $w$?
\end{exercise}
\begin{solution}
  Yes. The function $f$ can map both `$w$' and `$v$' to $w$.
\end{solution}

The soundness proof for K-trees is easily adapted to other types of trees. The
tree rules for system T, for example, are all the K-rules plus the Reflexivity
rule, which allows adding $\omega R \omega$ for every world $\omega$ on the
branch. Suppose we want to show that everything that is provable with the
T-rules is T-valid -- true at every world in every reflexive Kripke model. All
the clauses in the Soundness Lemma still hold if we assume that the model $M$ is
reflexive. We only need to add a further clause for the Reflexivity rule, to
confirm that if a branch correctly describes a reflexive model $M$, and the
branch is extended by adding $\omega R \omega$, then the resulting branch also
correctly describes $M$. This is evidently the case.

\begin{exercise}
  How would we need to adjust the soundness proof to show that the tree rules
  for K4 are sound with respect to K4-validity?
\end{exercise}
\begin{solution}
  A sentence is K4-valid iff it is true at all worlds in all transitive Kripke
  models. We only need to check that the Transitivity rule is sound, in the
  sense that if a branch correctly describes a transitive model $M$, and the
  branch is extended by the Transitivity rule, then the resulting branch also
  correctly describes $M$. (The Transitivity rule allows adding a node
  $\omega R \upsilon$ to a branch that already contains nodes $\omega R \nu$ and
  $\nu R \upsilon$. If these nodes correctly describe a transitive model then so does $\omega R \upsilon$.)
\end{solution}

\section{Completeness for trees}%
\label{sec:completenesstrees}

Let's now show that the tree rules for K are complete -- that whenever a sentence is
K-valid then there is a closed K-tree for that sentence. In fact, we will show
something stronger:
\begin{quote}
  If a sentence is K-valid, then every fully developed K-tree for the sentence is closed.
\end{quote}
By a \emph{fully developed} tree, I mean a tree on which every node on any open
branch that can be expanded (in any way) has been expanded (in this way). A
fully developed tree may be infinite.

We will prove the displayed sentence by proving its contraposition:
\begin{quote}
  If a fully developed K-tree for a sentence does not close, then the sentence is
  not K-valid.
\end{quote}
Assume, then, that some fully developed K-tree for some target sentence has at
least one open branch. We want to show that the target sentence is false at some
world in some Kripke model.

We already know how to read off a countermodel from an open branch. All we need
to do is show that this method for generating countermodels really works. Let's
first define the method more precisely.

\begin{definition}{}{inducedmodel}
  The model \textbf{induced by} a tree branch is the Kripke model $(W,R,V)$
  where
  \begin{itemize}[leftmargin=12mm]
    \itemsep-1mm
    \item[(a)] $W$ is the set of world variables on the branch,
    \item[(b)] $\omega R \upsilon$ holds in the model iff a node
          $\omega R \upsilon$ occurs on the branch,
    \item[(c)] for any sentence letter $P$, $V(P)$ is the set of world variables
          $\omega$ for which a node $P \; (\omega)$ occurs on the branch.
  \end{itemize}
\end{definition}

Next we show that all nodes on any open branch on a fully developed tree are
correct statements about the Kripke model induced by the branch.

\begin{theorem}{Completeness Lemma}{completenesslemma}
  Let $\beta$ be an open branch on a fully developed K-tree, and let
  $M = \t{W,R,V}$ be the model induced by $\beta$. Then $M,\omega \models A$ for
  all sentences $A$ and world variables $\omega$ for which $A\; (\omega)$ is on
  $\beta$.
\end{theorem}

\begin{proof}
  We have to show that whenever $A\; (\omega)$ occurs on $\beta$ then
  $M,\omega \models A$. The proof is by induction on the length of $A$. We first
  show that the claim holds for sentence letters and negated sentence letters.
  Then we show that \emph{if} the claim holds for all sentences shorter than $A$
  (this is our induction hypothesis), \emph{then} it also holds for $A$ itself.
  
  \begin{itemize}
    
    \item If $A$ is a sentence letter then the claim is true by clause (c) of
          definition \ref{def:inducedmodel} and clause (a) of definition
          \ref{def:kripkesemantics}.

    \item If $A$ is the negation of a sentence letter $\rho$, then $rho\; (\omega)$
          does not occur on $\beta$, otherwise $\beta$ would be closed. By
          clause (c) of definition \ref{def:inducedmodel}, it follows that
          $\omega$ is not in $V(\rho)$, and so $M, \omega \models A$ by clauses (a)
          and (b) of definition \ref{def:kripkesemantics}.
          
    \item If $A$ is a doubly negated sentence $\neg\neg B$, then $\beta$
          contains a node $B \;(\omega)$, because the tree is fully developed.
          By induction hypothesis, $M,\omega \models B$. By clause (b) of
          definition \ref{def:kripkesemantics}, it follows that
          $M,\omega \models A$.
    
    \item If $A$ is a conjunction $B\land C$, then $\beta$ contains nodes
          $B \;(\omega)$ and $C \;(\omega)$. By induction hypothesis,
          $M,\omega \models B$ and $M,\omega \models C$. By clause (c) of
          definition \ref{def:kripkesemantics}, it follows that
          $M,\omega \models A$.

    \item If $A$ is a negated conjunction $\neg(B\land C)$, then $\beta$ contains
          either $\neg B \;(\omega)$ or $\neg C \;(\omega)$. By induction
          hypothesis, $M,\omega \models \neg B$ or $M,\omega \models \neg C$.
          Either way, clauses (b) and (c) of definition
          \ref{def:kripkesemantics} imply that $M,\omega \models A$.
    
          % \item If $A$ is a disjunction $B\lor C$, then branch also contains
          % either $B \;(\omega)$ or $C \;(\omega)$, because the tree is fully
          % developed. By induction hypothesis, $M,\omega \models B$ or
          % $M,\omega \models C$. So $M,\omega \models B\lor C$.
    
  \end{itemize} 
  
  I skip the cases where $A$ is a disjunction, a conditional, a
  biconditional, or a negated disjunction, conditional, or biconditional. The
  proofs are similar to one (or both) of the previous two cases.

  \begin{itemize}
    
    \item If $A$ is a box sentence $\Box B$, then $\beta$ contains a node
          $B \;(\upsilon)$ for each world variable $\upsilon$ for which
          $\omega R \upsilon$ is on $\beta$ (because the tree is fully
          developed). By induction hypothesis, $M, \upsilon \models B$, for each
          such $\upsilon$. By definition \ref{def:inducedmodel}, it follows that
          $M,\upsilon \models B$ for all worlds $\upsilon$ such that
          $\omega R \upsilon$. By clause (g) of definition
          \ref{def:kripkesemantics}, it follows that $M, \omega \models \Box B$.
    
    \item If $A$ is a diamond sentence $\Diamond B$, then there is a world
          variable $\upsilon$ for which $\omega R \upsilon$ and $B \;(\upsilon)$
          are on $\beta$. By induction hypothesis, $M, \upsilon \models B$. And
          by definition \ref{def:inducedmodel}, $\omega R\upsilon$. By clause
          (h) of definition \ref{def:kripkesemantics}, it follows that
          $M, \omega \models \Diamond B$.

  \end{itemize}

  For the case where $A$ has the form $\neg \Box B$ or $\neg \Diamond B$, the
  proof is similar to one of the previous two cases. \qed
  
\end{proof}
\medskip

To establish completeness, we need to verify one more point: that one can always
construct a fully developed tree for any invalid target sentence. Let's call a
K-tree \emph{regular} if it is constructed by (i) first applying all rules for
the truth-functional connectives until no more of them can be applied (without
adding only nodes to a branch that are already on the branch), then (ii)
applying the rules for $\Diamond$ and $\neg \Box$ until no more of them can be
applied, then (iii) applying the rules for $\Box$ and $\neg \Diamond$ until no
more of them can be applied, then starting over with (i), and so on.

\begin{observation}{regulartrees}
  Every regular open K-tree is fully developed.
\end{observation}
\begin{proof}
  \emph{Proof:} When constructing a regular tree, every iteration of (i), (ii),
  and (iii) only allows expanding finitely many nodes. So every node on every
  open branch that can be expanded in any way is eventually expanded in this way
  by some iteration of (i), (ii), and (iii). \qed
\end{proof}

Now we have all the ingredients to prove completeness.

\begin{theorem}{Theorem: Completeness of K-trees}{completeness-tree-k}
  If a sentence is K-valid, then there is a closed K-tree for that sentence.
\end{theorem}

\begin{proof}
  \emph{Proof:} Let $A$ be any K-valid sentence, and suppose for reductio that
  there is no closed K-tree for $A$. In particular, then, every regular K-tree
  for $A$ remains open. Take any such tree. By observation
  \ref{obs:regulartrees}, the tree is fully expanded. Choose any open branch on
  the tree. By the Completeness Lemma, $A$ is false at $w$ in the model induced
  by that branch. So $A$ is not true at all worlds in all Kripke models.
  Contradiction. \qed
\end{proof}

\begin{exercise}
  Fill in the cases for $B \to C$ and $\neg \Diamond B$ in the proof of the
  Completeness Lemma.
\end{exercise}
\begin{solution}
  For $B\to C$: If $A$ is a conditional $B\to C$, then $\beta$ contains either
  $\neg B \;(\omega)$ or $C \;(\omega)$. By induction hypothesis,
  $M,\omega \models \neg B$ or $M,\omega \models \neg C$. Either way, clauses
  (b) and (e) of definition \ref{def:kripkesemantics} imply that
  $M,\omega \models A$.

  For $\neg\Diamond B$: If $A$ is a negated diamond sentence $\neg\Diamond B$,
  then $\beta$ contains a node $\neg B \;(\upsilon)$ for each world variable
  $\upsilon$ for which $\omega R \upsilon$ is on $\beta$ (because the tree is
  fully developed). By induction hypothesis, $M, \upsilon \models \neg B$, for
  each such $\upsilon$. By definition \ref{def:inducedmodel}, it follows that
  $M,\upsilon \models \neg B$ for all worlds $\upsilon$ such that
  $\omega R \upsilon$. By clauses (b) and (g) of definition
  \ref{def:kripkesemantics}, it follows that $M, \omega \models A$.
\end{solution}

Like the soundness proof, the completeness proof for K is easily adapted to
other logics. To show that the T-rules are complete with respect to T-validity,
for example, we merely need check that the model induced by any open branch on a
fully developed T-tree is reflexive. It must be, because an open branch on a
fully developed T-tree contains $\omega R \omega$ for each world variable
$\omega$ on the branch.

% Strictly speaking, we also need to adjust the definition of regular trees.

\begin{exercise}
  What do we need to check to show that the K4-rules are complete with respect
  to K4-validity?
\end{exercise}
\begin{solution}
  We need to check that the model induced by an open branch on a fully developed
  K4-tree is transitive. (Suppose the model contains worlds $w$, $v$, $u$ for
  which $wRv$ and $vRu$. Then the Transitivity rule has been applied to the
  corresponding nodes on the branch, generating a node $wRu$. By definition
  \ref{def:inducedmodel}, $wRu$ holds in the induced model.)
\end{solution}

\begin{exercise}\label{ex:acyclical}
  A Kripke model is \emph{acyclical} if you can never return to the same world
  by following the accessibility relation. Show that if a sentence is true at
  some world in some Kripke model, then it is also true at some world in some
  acyclical Kripke model.

  (Hint: If $A$ is true at some world in some Kripke model then $\neg A$ is
  K-invalid. By the soundness theorem, there is a fully developed K-tree for
  $\neg A$ with an open branch. Now consider the model induced by this branch.)
\end{exercise}
\begin{solution}
  Suppose $A$ is true at some world in some Kripke model. Then $\neg A$ is
  K-invalid. Take any regular K-tree for $\neg A$. By observation
  \ref{obs:regulartrees}, that tree is fully developed. By the soundness theorem
  for K-trees, the tree has an open branch. Let $M$ be the model induced by some
  such branch $\beta$. Then $M$ is acyclical. This is because the only rules
  that allow adding a node $\omega R \upsilon$ to a branch of a K-tree are the
  rules for expanding $\Diamond A$ and $\neg\Box A$ nodes. In both cases, the
  rule requires that the relevant world variable $\upsilon$ is new on the
  branch. (Call this the \emph{novelty requirement}.) Now suppose the
  accessibility relation in $M$ has a cycle $\omega_{1} R \omega_{2}$,
  $\omega_{2} R \omega_{3}$, \ldots, $\omega_{n-1} R \omega_{n}$,
  $\omega_{n} R \omega_{1}$. Each of these facts about $R$ must correspond to a
  node on $\beta$. Of these nodes, the one that was added last (to $\beta$)
  violates the novelty requirement. So $M$ is acyclical.

  By the Completeness Lemma, the target sentence $\neg\neg A$ is true at world
  $w$ in $M$. So $A$ is true at $w$ in $M$. So $A$ is true at some world in some acyclical model.
\end{solution}

% This explains why it's OK to always introduce a new world when expanding <>A.
% Technically, we only introduce a new world /name/, and the completeness lemma
% exploits this technicality. But when we read off countermodels, we assume that
% different world names denote different worlds. If -- somehow -- there could be
% sentences C that are only true in cyclical models, our method might still be
% sound: we could not prove ~C. All ~C trees would remain open. But the models
% induced by these open branches would not be countermodels to C. The open
% branches might still "accurately describe" a countermodel, under a function
% that maps different world names to the same world. And then the method might
% even be complete.

% Why is it OK to only ever expand <>A once? What if the only countermodels to
% the target sentence have /two/ witnesses of a diamond sentence? If a tree
% closes with only one witness then we've already reached a contradiction
% assuming there is at least one witness. No point expanding <>A again. If a
% tree remains open, completeness says that we have a countermodel with one
% witness. This would fail if -- somehow -- there could be sentences that are
% true only in models with multiple witnesses. That is, the model induced by an
% open branch wouldn't be a countermodel.

% \begin{exercise}
%   When constructing a tree proof, one often has a choice of which rules to
%   apply in which order. Show that the choice doesn't matter, in the following
%   sense: If there is a K-tree for the target sentence in which all branches
%   close, then there is no fully developed K-tree for the target sentence in
%   which some branch remains open. xxx Well, if you develop []<>p & (q & ~q),
%   and you keep expanding []<>p without every dealing with (q & ~q) then the
%   tree never closes. so the choice /does/ matter.
% \end{exercise}
% \begin{solution}
%   We have to show that if there some K-tree for a given target sentence closes,
%   then there is no fully developed K-tree for the sentence that remains open.
%   So assume some K-tree for the target sentence closes. By the Completeness
%   Theorem, the target sentence is K-valid. If there were a fully developed but
%   open K-tree for the same target sentence, then by the Completeness Lemma the
%   negation of that sentence would be true at world $w$ in the model induced by
%   some open branch on that tree. This contradicts the fact that the target
%   sentence is K-valid.
% \end{solution}

\begin{exercise}
  The S5 tree rules from chapter \ref{ch:worlds} are sound and complete for
  S5-validity: all and only the S5-valid sentences can be proven. Are the rules
  sound for K-validity? Are they complete for K-validity?
\end{exercise}
\begin{solution}
  The S5 rules are not sound with respect to K-validity. For example,
  $\Box p \to p$ is provable with the S5 rules, but it isn't K-valid. The rules
  are, however, complete with respect to K-validity. This follows from the
  completeness of the S5 rules and the fact that every K-valid sentence is
  S5-valid (observation \ref{obs:kins5}).
\end{solution}

\iffalse
\section{Strong completeness and compactness}%
\label{sec:compactness}

% Greg Restall: trees are upside-down cut-free sequent calculus, rephrased for
% easy teaching. That’s it. If you rephrase them just a *tiny* bit (to make them
% +/- signed trees, instead of littering every second rule with negation) then
% you have an easy and direct proof of the subformula theorem: that any valid
% argument has a purely analytic proof—a proof using only subformulas of the
% premises and the conclusion. To show that for classical predicate logic in
% natural deduction by normalisation? No fun at all!

We have shown that every K-valid sentence is provable with the tree rules for K.
This kind of completeness -- linking validity with provability -- is sometimes
called \emph{weak} completeness. Strong completeness is concerned not with
validity, but with entailment: A proof method is \textbf{strongly complete} with
respect to a logic if, whenever a sentence is entailed (in the logic) by some
sentences $\Gamma$, then there is a proof (in the method) showing that the
sentence is entailed by $\Gamma$.

You will remember from observation \ref{obs:semantic-deduction-theorem} in
chapter 1 that claims about entailment can be converted into claims about
validity. $A$ entails $B$ iff $A \to B$ is valid; $A_{1}$ and $A_{2}$ together
entail $B$ iff $A_{1} \to (A_{2} \to B)$ is valid; and so on. We can therefore
show that, say, $A_{1}$ and $A_{2}$ entail $B$ by constructing a closed tree for
$A_{1} \to (A_{2} \to B)$. After two applications of the rule for negated
conditionals, this tree has three unexpanded nodes, corresponding to the
original premises and the negated conclusion: $A_{1}\; (w)$, $A_{2}\; (w)$, and
$\neg B\; (w)$. In general, to show that some premises $\Gamma$ and entail a
conclusion $B$, we can save a few steps by directly starting the tree with one
node $A_{i}\; (w)$ for each premise $A_{i}$ in $\Gamma$, and another node
$\neg B\; (w)$ for the negated conclusion. Together with observation
\ref{obs:semantic-deduction-theorem}, the soundness and completeness results
from the previous section immediately entail the following.

\begin{observation}{sckentailment}
  Some sentences $A_{1},\ldots,A_{n}$ K-entail a sentence $B$ iff there is a
  closed K-tree with starting nodes $A_{1}\; (w)$, \ldots, $A_{n}\; (w)$,
  $\neg B\; (w)$.
\end{observation}

But this isn't enough for strong completeness. It only shows that if a
conclusion is entailed by \emph{a finite collection} of premises, then the
entailment can be verified by the tree method. One may, however, reasonably ask
whether a sentence is entailed by some infinite collection of sentences. If
$\Gamma$ is infinite then the claim that $\Gamma$ entails $B$ can't be converted
into a claim about validity, because $L_{M}$-sentences are finite. Nor can we
start a tree with separate nodes for each premise followed by a node for the
negated conclusion: With infinitely many premises we would never get to the
negated conclusion, nor would we ever get to a stage where any of these nodes is
expanded.

Here is how we can nonetheless use the tree method do deal with infinitely many
premises. Instead of starting with all the premises
$A_{1}, A_{2}, A_{3}, \ldots$ at once, we proceed in stages. In the first stage,
we try to derive the conclusion $B$ from only the first premise $A_{1}$. If
that fails, or if the tree is getting too large, we move on to the second stage,
where we try to derive $B$ from the first two premises, $A_{1}$ and $A_{2}$,
allowing for somewhat larger trees. And so on. Even though we consider only
finitely many premises at each stage, every premise will eventually be
considered, unless the tree already closes without taking it into account.

Let me describe the method in a little more detail. Suppose we want to prove
that $B$ is K-entailed by infinitely many sentences
$A_{1}, A_{2}, A_{3},\ldots$. In the first stage, we develop a K-tree whose
starting assumptions are $A_{1}\; (w)$ and $\neg B\; (w)$. We don't develop that
tree fully, because this could take forever. Instead, we develop the tree by (i)
applying all rules for the truth-functional connectives until no more of them
can be applied (without repetition), then (ii) applying the rules for $\Diamond$
and $\neg \Box$ until no more of them can be applied, and finally (iii) applying
the rules for $\Box$ and $\neg \Diamond$ until no more of them can be applied.
If at that point the tree is closed, we're done: The proof is complete. If not,
we move on to stage 2. Here we construct another tree with starting assumptions
$A_{1}\; (w)$, $A_{2}\; (w)$ and $\neg B\; (w)$. To develop this tree, we first
repeat everything we did on the first tree, expanding $A_{1}\; (w)$ and
$\neg B\; (w)$. Then we go through another iteration of (i)--(iii). If the tree
is still open, we move on to stage 3, where we add $A_{3}\; (w)$ to the starting
assumptions, copy over all nodes from stage 2 and go through another iteration
of (i)--(iii). And so on.

This method is sound. For suppose at some stage $n$ the method yields a closed
tree. This tree is an ordinary K-tree with starting assumptions $A_{1}\; (w)$,
\ldots, $A_{n}\; (w)$, and $\neg B\; (w)$. By observation
\ref{obs:sckentailment}, it follows that $A_{1},\ldots,A_{n}$ K-entail $B$. And
then all of $A_{1},A_{2},A_{3},\ldots$ K-entail $B$.

More interestingly, the described method is complete. For suppose it never
yields a closed tree. At each stage, the tree has an open branch. Note that any
open branch at a stage after the first extends an open branch from the previous
stage. So there is a sequence of branches $b_{1}, b_{2}, b_{3}, \ldots$, one
from each stage, such that $b_{2}$ extends $b_{1}$, $b_{3}$ extends $b_{2}$, and
so on. Pick any such sequence. We may regard the set of nodes that occur
somewhere in the sequence as a single infinite branch. This infinite branch
contains a node $A_{i}\; (w)$ for every premise, as well as $\neg B\; (w)$. Just
as an ordinary open branch induces a Kripke model (as per definition
\ref{def:inducedmodel}), so does the infinite branch. And because every node on
the branch that can be expanded has been expanded, the Completeness Lemma goes
through just as in the previous section, showing that all nodes on the branch
are correct statements about the induced model. So there is a world in some
Kripke model at which all of $A_{1},A_{2},A_{3},\ldots$ are true and $B$ is
false.

\begin{theorem}{Theorem: Strong completeness of K-trees}{strongcompleteness}
  If a sentence $B$ is K-entailed by some sentences $\Gamma$ then there is a
  closed K-tree starting with $A_{1}\; (w)$, \ldots, $A_{n}\; (w)$ and
  $\neg B\; (w)$, where all of $A_{1},\ldots,A_{n}$ are in $\Gamma$.
\end{theorem}
\begin{proof}
  \emph{Proof:} Consider a sequence of K-trees, the first beginning with
  $A_{1}\; (w)$ and $\neg B\; (w)$, the second with $A_{1}\; (w)$,
  $A_{2}\; (w)$, and $\neg B\; (w)$, and so on -- each extending the previous
  one by an added premise from $\Gamma$ and one iteration of (i)--(iii). As
  we've just seen, if no tree in this sequence closes, then $B$ is not K-entailed
  by $\Gamma$. \qed
\end{proof}
%
Our proof easily carries over to other systems of modal logic. The only part
that needs adapting is the specification of (i)--(iii), where we must add the
extra tree rules in some way that ensures that all nodes are eventually
expanded.

Strong completeness has a noteworthy corollary:
%
\begin{theorem}{Theorem: Compactness of K}{compactness}
  If a sentence $B$ is K-entailed by some sentences $\Gamma$ then $B$ is
  K-entailed by a finite subset of $\Gamma$.
\end{theorem}
\begin{proof}
  \emph{Proof:} By strong completeness, if some sentences $\Gamma$ K-entail $B$,
  then there is a closed K-tree starting with $A_{1}\; (w)$, \ldots,
  $A_{n}\; (w)$ and $\neg B\; (w)$, where $A_{1},\ldots,A_{n}$ are finitely many
  sentences from $\Gamma$. By observation \ref{obs:sckentailment}, it follows
  that $B$ is entailed by $A_{1},\ldots,A_{n}$. \qed
\end{proof}

In general, a logic $S$ is called \textbf{compact} if any sentence that is
$S$-entailed by infinitely many sentences is also $S$-entailed by a
finite subset of these sentences.

Compactness is surprising. It is easy to think of cases in which a conclusion is
entailed by infinitely many premises, but not by any finite subset of these
premises. For example, suppose I like the number 0, I like the number 1, I like
the number 2, and so on, for all natural numbers 0,1,2,3,\ldots. Together, these
assumptions entail that I like all natural numbers. But no finite subset of the
assumptions entails that I like all natural numbers.

\begin{exercise}
  A set of sentences $\Gamma$ is called \emph{K-satisfiable} if there is a world
  in some Kripke model at which all members of $\Gamma$ are true. Show that an
  infinite set of sentences $\Gamma$ is K-satisfiable iff every finite subset of
  $\Gamma$ is K-satisfiable.
\end{exercise}
\begin{solution}
  Let $\Gamma$ is an infinite set of $\L_{M}$-sentences. If $\Gamma$ is
  K-satisfiable then obviously every finite subset of $\Gamma$ is satisfiable as
  well. For the converse direction, assume $\Gamma$ is not K-satisfiable: There
  is no world in any Kripke model at which all members of $\Gamma$ are true.
  Then there is no world in any Kripke model at which all members of $\Gamma$
  are true while $p \land \neg p$ is false. So $\Gamma \models p\land \neg p$.
  By the compactness theorem, it follows that there is a finite subset
  $\Gamma^{-}$ for which $\Gamma^{-} \models p \land \neg p$. If
  $\Gamma^{-} \models p \land \neg p$ then there is no world in any Kripke model
  at which all members of $\Gamma^{-}$ are true while $p \land \neg p$ is false.
  Since $p\land \neg p$ is false at every world in every Kripke model, it
  follows that there is no world in any Kripke model at which all members of
  $\Gamma^{-}$ are true. This shows that if $\Gamma$ is not K-satisfiable then
  there is a finite subset ($\Gamma^{-}$) of $\Gamma$ that is not K-satisfiable.
  Conversely, if every finite subset of $\Gamma$ is K-satisfiable then $\Gamma$
  is K-satisfiable.
\end{solution}

\fi 

\section{Soundness and completeness for axiomatic calculi}
\label{sec:scaxiomatic}

Next, we are going to show that the axiomatic calculus for system K is sound and
complete for K-validity. In the axiomatic calculus, a proof is a list of
sentences each of which is either an instance of \pr{Dual} or \pr{K} or can be
derived from earlier sentences on the list by application of \pr{CPL} or
\pr{Nec}. Expressed as a construction rule, \pr{Nec} says that whenever a list
contains a sentence $A$ then one may append $\Box A$. \pr{CPL} says that one may
append any truth-functional consequence of sentences that are already on the
list. (This is an acceptable rule because there is a simple mechanical test --
the truth-table method -- for checking whether a sentence is a truth-functional
consequence of finitely many other sentences.)

Soundness is easy. We want to show that everything that is derivable from some
instances of \pr{Dual} and \pr{K} by applications of \pr{CPL} and \pr{Nec} is
K-valid. We show this by showing that (1) every instance of \pr{Dual} and \pr{K}
is K-valid, and (2) every sentence that is derived from K-valid sentences by
\pr{CPL} or \pr{Nec} is itself K-valid.

\begin{theorem}{Theorem: Soundness of the axiomatic calculus for K}{soundnessK}
  Any sentence that is provable in the axiomatic calculus for K is K-valid.
\end{theorem}

\begin{proof}
  \emph{Proof:} We first show that every instance of \pr{Dual} and \pr{K} is
  K-valid.
  \begin{enumerate}[leftmargin=7mm]
  \itemsep0mm
  
    \item \pr{Dual} is the schema $\neg\Diamond A \leftrightarrow \Box\neg A$.
          By clauses (b), (g), and (h) of definition \ref{def:kripkesemantics},
          a sentence $\neg\Diamond A$ is true at a world $w$ in a Kripke model
          $M$ iff $\Box\neg A$ is true at $w$ in $M$. It follows by clauses (f)
          and (e) that all instances of \pr{Dual} are true at all worlds in all
          Kripke models.
          
    \item \pr{K} is the schema $\Box(A \to B) \to (\Box A \to \Box B)$. By
          clause (e) of definition \ref{def:kripkesemantics}, a sentence
          $\Box(A\to B) \to (\Box A \to \Box B)$ is false at a world $w$ in a
          Kripke model only if $\Box(A \to B)$ and $\Box A$ are both true at
          $w$ while $B$ is false. By clause (g) of definition
          \ref{def:kripkesemantics}, $\Box B$ is false at $w$ only if $B$ is
          false at some world $v$ accessible from $w$. But if $\Box(A \to B)$
          and $\Box A$ are both true at $w$, then $A\to B$ and $A$ are true at
          every world accessible from $w$, again by clause (g). And there can be
          no world at which $A\to B$ and $A$ are true while $B$ is false, by
          clause (e) of definition \ref{def:kripkesemantics}.
  \end{enumerate}
  %
  Next we show that \pr{CPL} and \pr{Nec} preserve K-validity.
  %
  \begin{enumerate}[leftmargin=9mm]
    
    \item By definition \ref{def:kripkesemantics}, the truth-functional
          operators have their standard truth-table meaning at every world in
          every Kripke model. It follows that all truth-functional consequences
          of sentences that are true at a world are themselves true at that
          world. In particular, if some sentences are true at every world in
          every Kripke model, then any truth-functional consequence of these
          sentences is also true at every world every Kripke model.

    \item Let $w$ be an arbitrary world in an arbitrary Kripke model. If $A$ is
          true at every world in every Kripke model, then $A$ is true at every
          world accessible from $w$, in which case $\Box A$ is true at $w$ by
          clause (g) of definition \ref{def:kripkesemantics}. So if $A$ is K-valid, then $\Box A$ is also K-valid. \qed
          
  \end{enumerate}
\end{proof}

The soundness proof for K is easily extended to other modal systems. Since all
instances of \pr{Dual} and \pr{K} are true at all worlds in all Kripke models,
they are also true at all worlds in any more restricted class of Kripke models.
The arguments for \pr{CPL} and \pr{Nec} also go through if we replace `every
Kripke model' by `every Kripke model of such-and-such type'. So if we want to
show that, say, the axiomatic calculus for T is sound with respect to the
concept of T-validity -- that is, if we want to show that anything that is
derivable from \pr{Dual}, \pr{K}, and \pr{T} by \pr{CPL} and \pr{Nec} is true at
all worlds in all reflexive Kripke models -- all that is left to do is to show
that every instance of the \pr{T}-schema is true at all worlds in all reflexive
Kripke model. (We've already shown this: see observation \ref{obs:treflexive}.)

% In general, if $\Gamma$ is any set of sentences that are valid in a given
% class C of frames, anything that's provable from $K+\Gamma$ is valid in that
% class. (HC.39)

\begin{exercise}
  Outline the soundness proof for the axiomatic calculus for S4, whose axiom
  schemas are \pr{Dual}, \pr{K}, \pr{T}, and \pr{4}.
\end{exercise}
\begin{solution}
  We need to show that everything that's derivable in the axiomatic calculus for
  S4 is true at every world in every transitive and reflexive Kripke model. From
  the soundness proof for K, we know that all instances of \pr{Dual} and \pr{K}
  are true at every world in every Kripke model. From observation
  \ref{obs:treflexive}, we know that all instances of \pr{T} are true at every
  world in every reflexive Kripke model. From observation \ref{obs:4trans}, we
  know that all instances of \pr{4} are true at every world in every transitive
  Kripke model. So all axioms in the S4-calculus are valid in the class of
  transitive and reflexive Kripke frames. Since \pr{CPL} and \pr{Nec} preserve
  validity in any class of Kripke frames, it follows that everything that's
  derivable in the S4-calculus is valid in the class of transitive and reflexive
  frames.
\end{solution}

Let's turn to completeness. We are going to show that every K-valid sentence is
derivable from some instances of \pr{Dual} and \pr{K} by \pr{CPL} and \pr{Nec}.
% You might want to grab a cup of tea before you continue.
As in section \ref{sec:completenesstrees}, we argue by contraposition. We will
show that any sentence that cannot be derived from \pr{Dual} and \pr{K} by
\pr{CPL} and \pr{Nec} is not K-valid. To show that a sentence is not K-valid, we
will give a countermodel -- a Kripke model in which the sentence is false at
some world. In fact, we will give the \emph{same} countermodel for every
sentence that isn't derivable in the calculus. You might think we need different
countermodels for different sentences, but it turns out that there is a
particular model in which every K-invalid sentence is false at some world. This
model is called the \emph{canonical model} for K.

In order to define the canonical model, let's introduce some shorthand
terminology. We'll say that an $\L_M$-sentence is \emph{K-provable} if it can be
proved in the axiomatic calculus for K. A set of $\L_M$-sentences is
\emph{K-inconsistent} if it contains a finite number of sentences
$A_1,\ldots,A_n$ such that $\neg (A_1 \land \ldots \land A_n)$ is K-provable. A
set is \emph{K-consistent} if it is not K-inconsistent.

(For example, the set $\{ \Box(p \land q), q \to p, \neg \Box q \}$ is
K-inconsistent, because it contains two sentences, $\Box(p \land q)$ and
$\neg \Box q$ whose conjunction is refutable in K, in the sense that the
negation $\neg(\Box(p \land q) \land \neg \Box q)$ of their conjunction is
derivable from some instances of \pr{Dual} and \pr{K} by \pr{CPL} and \pr{Nec}.)

A set of $\L_M$-sentences is called \emph{maximal} if it contains either $A$ or
$\neg A$ for every $\L_M$-sentence $A$. A set is \emph{maximal K-consistent} if
it is both maximal and K-consistent.

% Exercise (HC 114): Show that if $\Gamma$ is maximal consistent, then for any
% sentences $A,B$, (i) $A\in \Gamma$ iff $\neg A \not\in \Gamma$, (ii)
% $A \lor B \in \Gamma$ iff either $A \in \Gamma$ or $B \in \Gamma$, (iii) $A \land B$ ...

\begin{exercise}
  Which, if any, of these sets are K-consistent? (a) $\{ p \}$, (b)
  $\{ \neg p \}$, (c) the set of all sentence letters, (d) the set of all
  $\L_{M}$-sentences.
\end{exercise}
\begin{solution}
  (a), (b), and (c) are K-consistent, (d) is not.
\end{solution}

% The universe of the canonical model is always uncountable since there are
% uncountably many (2^\omega) sets of negated/unnegated sentence letters.

Now here's the canonical model for K.

\begin{definition}{}{cmk}
  %\leavevmode\vspace{-2em}
  
  The \textbf{canonical model} $M_K$ for K is the Kripke model $\t{W,R,V}$, where%
  \vspace{-2mm}
  \begin{itemize}[leftmargin=7mm]
    \itemsep-1mm
    \item $W$ is the set of all maximal K-consistent sets of $\L_M$-sentences,
    \item $wRv$ iff $v$ contains every sentence $A$ for which $w$ contains
          $\Box A$,
    \item for every sentence letter $P$, $V(P)$ is the set of all members of $W$
          that contain $P$.
  \end{itemize}
\end{definition}

% Exercise: show that for any consistent normal modal logic, the canonical
% accessibility relation R is such that wRv iff for all B\in v we have <>B\in w.

The ``worlds'' in the canonical model are sets of $\L_M$-sentences. The
interpretation function makes a sentence letter true at a world iff the letter
is a member of the world. As we are going to see, this generalizes to arbitrary
sentences:
\begin{enumerate}[leftmargin=10mm]
  \item[(1)] A world $w$ in $M_K$ contains all and only the sentences that are
        true at $w$ in $M_K$.
\end{enumerate}
We will also prove the following:
\begin{enumerate}[leftmargin=10mm]
  \item[(2)] If some sentence cannot be proved in the axiomatic calculus for K,
        then its negation is a member of some world in $M_K$.
\end{enumerate}

Together, these two lemmas will establish completeness for the axiomatic
calculus. Fact (2) tells us that if a sentence $A$ isn't K-provable, then
$\neg A$ is a member of some world $w$ in the canonical model $M_K$. By fact
(1), we can infer that $\neg A$ is true at $w$ in $M_K$, which means that $A$ is
false at $w$ in $M_K$. So any sentence that isn't K-provable isn't K-valid.

We are going to prove (2) first. We'll need the following observation. 

\begin{observation}{extensionconsistency}
  If a set $\Gamma$ is K-consistent, then for any sentence $A$, either
  $\Gamma \cup \{ A \}$ or $\Gamma \cup \{ \neg A \}$ is K-consistent.
\end{observation}
%
\noindent
($\Gamma \cup \{ A \}$, called the \emph{union} of $\Gamma$ and $\{ A \}$, is
the smallest set that contains all members of $\Gamma$ as well as $A$.)
%
\begin{proof}
  \emph{Proof}: Let $\Gamma$ be any K-consistent set and $A$ any sentence.
  Suppose for reductio that $\Gamma \cup \{ A \}$ and
  $\Gamma \cup \{ \neg A \}$ are both K-inconsistent.

  That $\Gamma \cup \{ A \}$ is K-inconsistent means there are sentences
  $A_1,\ldots,A_n$ in $\Gamma \cup \{ A \}$ such that
  $\neg (A_1\land\ldots\land A_n)$ is K-provable. Since $\Gamma$ itself is
  K-consistent, one of the sentences $A_1,\ldots,A_n$ must be $A$. Let $B$ be
  the conjunction of the other sentences in $A_1,\ldots,A_n$, all of which are
  in $\Gamma$. So $\neg(B \land A)$ is K-provable.

  That $\Gamma \cup \{ \neg A \}$ is K-inconsistent means that there are sentences $A_1,\ldots,A_n$ in $\Gamma \cup \{ \neg A \}$ such that $\neg (A_1\land\ldots\land A_n)$ is K-provable. As before, one of these sentences must be $\neg A$. Let $C$ be the conjunction of the others, all of which are in $\Gamma$. So $\neg(C \land \neg A)$ is K-provable.

  If $\neg(B \land A)$ and $\neg(C \land \neg A)$ are both K-provable, then so
  is $\neg(B \land C)$, because it is a truth-functional consequence of
  $\neg(B \land A)$ and $\neg(C \land \neg A)$. But $B \land C$ is a conjunction
  of sentences from $\Gamma$. So $\Gamma$ itself is K-inconsistent, contradicting our assumption. \qed
\end{proof}

Now we can prove fact (2).

\begin{theorem}{Lindenbaum's Lemma}{extensibility}
  Every K-consistent set is a subset of some maximal K-consistent set.
\end{theorem}
%
\begin{proof}
  \emph{Proof}: Let $S_0$ be some K-consistent set of sentences. Let
  $A_1,A_2,\ldots$ be a list of all $\L_M$-sentences in some arbitrary
  order. For every number $i\geq 0$, define
  \begin{gather*}
    S_{i+1} = \begin{cases} S_i \cup \{ A_i \} & \text{if $S_i \cup \{ A_i \}$ is K-consistent}\\
      S_i \cup \{ \neg A_i \} & \text{otherwise}.
    \end{cases}
  \end{gather*}
%
  This gives us an infinite list of sets $S_0,S_1,S_2,\ldots$. Each set in the
  list is K-consistent: $S_0$ is K-consistent by assumption. And if some set
  $S_i$ in the list is K-consistent, then either $S_i \cup \{ A_i \}$ is
  K-consistent, in which case $S_{i+1} = S_i \cup \{ A_i \}$ is K-consistent, or
  $S_i \cup \{ A_i \}$ is not K-consistent, in which case $S_{i+1}$ is
  $S_i \cup \{ \neg A_i \}$, which is K-consistent by observation
  \ref{obs:extensionconsistency}. So if any set in the list is K-consistent, then
  the next set in the list is also K-consistent. It follows that
  $S_0,S_1,S_2,\ldots$ are all K-consistent.

  Now let $S$ be the set of sentences that occur in at least one of the sets
  $S_{0},S_1, S_2,S_3\ldots$. (That is, let $S$ be the union of
  $S_{0},S_1,S_2,S_3,\ldots$.) Evidently, $S_0$ a subset of $S$. And $S$ is
  maximal. Moreover, $S$ is K-consistent. For if $S$ were not K-consistent, then
  it would contain some sentences $B_1,\ldots,B_n$ such that
  $\neg (B_1\land \ldots\land B_n)$ is K-provable. All of these sentences would
  have to occur somewhere on the list $A_1,A_2,\ldots$. Let $A_j$ be a sentence
  from $A_1,A_2,\ldots$ that occurs after all the $B_1,\ldots,B_n$. If
  $B_1,\ldots,B_n$ are in $S$, they would have to be in $S_j$ already, so $S_j$
  would be K-inconsistent. But we've seen that all of $S_0,S_1,S_2,\ldots$ are
  K-consistent. \qed
\end{proof}

Notice that the proof of Lindenbaum's Lemma does not turn on any assumptions
about the axiomatic calculus for K except that \pr{CPL} is one of its rules. The
lemma holds for every calculus with \pr{CPL} as a (possibly derived) rule.

To prove fact (1), we need another observation, which relies on the presence of
\pr{K} and \pr{Nec}, besides \pr{CPL}.

\begin{observation}{hcp117}
  If $\Gamma$ is a maximal K-consistent set of sentences that does not
  contain $\Box A$, and $\Gamma^{-}$ is the set of all sentences $B$ for which
  $\Box B$ is in $\Gamma$, then $\Gamma^{-} \cup \{ \neg A \}$ is
  K-consistent.
\end{observation}
\begin{proof}
  \emph{Proof:} We show that if $\Gamma^{-} \cup \{ \neg A \}$ is not
  K-consistent, then neither is $\Gamma$. If $\Gamma^- \cup \{ \neg A \}$ is not
  K-consistent, then there are sentences $B_1,\ldots,B_n$ in $\Gamma^{-}$ such
  that $\neg(B_1\land\ldots\land B_n \land \neg A)$ is K-provable. And then
  $(B_1\land\ldots\land B_n) \to A$ is K-provable, because it is a
  truth-functional consequence of $\neg(B_1\land\ldots\land B_n \land \neg A)$.
  By repeated application of \pr{Nec}, \pr{K}, and \pr{CPL}, one can derive
  $(\Box B_1\land\ldots\land \Box B_n) \to \Box A$ from
  $(B_1\land\ldots\land B_n) \to A$. Another application of \pr{CPL} yields
  $\neg (\Box B_1\land\ldots\land \Box B_n \land \neg\Box A)$. So
  $\{\Box B_1,\ldots,\Box B_n, \neg \Box A\}$ is K-inconsistent. But
  $\Box B_1,\ldots,\Box B_n$ are in $\Gamma$. And since $\Box A$ is not in
  $\Gamma$ and $\Gamma$ is maximal, $\neg \Box A$ is in $\Gamma$. So
  $\{\Box B_1,\ldots,\Box B_n, \neg \Box A\}$ is a subset of $\Gamma$. And so
  $\Gamma$ is K-inconsistent. \qed
\end{proof}

Here, then, is fact (1):

\begin{theorem}{Canonical Model Lemma}{lemma:cml}
  For any world $w$ in $M_K$ and any sentence $A$, $A$ is in $w$ iff
  $M_K,w \models A$.
\end{theorem}

\begin{proof}
  \emph{Proof:} The proof is by induction on complexity of $A$. We first show
  that the claim (that $A$ is in $w$ iff $M_{K},w\models A$) holds for sentence
  letters. Then we show that if the claim holds for the immediate parts of a
  complex sentence (this is our induction hypothesis), then the claim also
  holds for the sentence itself.%
  \vspace{-1mm}
  \begin{itemize}
    \itemsep0mm
    \item Suppose $A$ is a sentence letter. By definition \ref{def:cmk},
          $w\in V(A)$ iff $A\in w$. So by clause (a) of definition
          \ref{def:kripkesemantics}, $M_K,w \models A$ iff $A \in w$. (`$\in$'
          means `is a member of the set'.)
  
    \item Suppose $A$ is a negation $\neg B$. By clause (b) of definition
          \ref{def:kripkesemantics}, $M_K,w \models \neg B$ iff
          $M_K,w \not\models B$. By induction hypothesis, $M_K,w\not\models B$
          iff $B \not\in w$. Since $w$ is maximal K-consistent, $B \not\in w$
          iff $\neg B \in w$. So $M_K,w \models \neg B$ iff $\neg B \in w$.
          % I haven't actually shown that maximal K-consistency implies $B
          % \not\in w iff \neg B \in w$.
        
    \item Suppose $A$ is a conjunction $B\land C$. By clause (c) of definition
          \ref{def:kripkesemantics}, $M_K,w \models B\land C$ iff
          $M_K,w \models B$ and $M_K,w \models C$. By induction hypothesis,
          $M_K,w \models B$ iff $B\in w$, and $M_K,w \models C$ iff $C\in w$.
          Since $w$ is maximal K-consistent, $B$ and $C$ are in $w$ iff
          $B\land C$ is in $w$.
          % I haven't actually shown this.
          So $M_K,w \models B \land C$ iff $B \land C \in w$.
  \end{itemize}
  %
  The cases for the other truth-functional connectives are similar.
  \begin{itemize}
    \itemsep0mm
    \item Suppose $A$ is a box sentence $\Box B$, and that $\Box B \in w$. By
          definition \ref{def:cmk}, it follows that $B\in v$ for all $v$ with
          $wRv$. By induction hypothesis, this means that $M_K,v \models B$ for
          all $v$ with $wRv$. And then $M_K,w \models \Box B$, by clause (g) of
          definition \ref{def:kripkesemantics}.

          For the converse direction, suppose $\Box B \not\in w$. Let $\Gamma^-$
          be the set of all sentences $C$ for which $\Box C \in w$. By
          observation \ref{obs:hcp117}, $\Gamma^- \cup \{ \neg B \}$ is
          K-consistent. By definition \ref{def:cmk} and Lindenbaum's Lemma, it
          follows that there is some $v\in W$ such that $wRv$ and
          $\neg B \in v$. Since $v$ is K-consistent, $B \not\in v$. By induction
          hypothesis, it follows that $M_K,v \not\models B$. And so
          $M_K,w \not\models \Box B$, by clause (g) of definition
          \ref{def:kripkesemantics}.
          
    \item Suppose $A$ is a diamond sentence $\Diamond B$, and that
          $\Diamond B \in w$. By \pr{Dual} and \pr{CPL}, any set that contains
          both $\Diamond B$ and $\Box \neg B$ is K-inconsistent. So
          $\Box \neg B \not\in w$. By observation \ref{obs:hcp117} and
          Lindenbaum's Lemma (as in the previous case), it follows that there is
          some $v\in W$ such that $wRv$ and $B \in v$. By induction hypothesis,
          $M,v\models B$. So $M_{K},w\models \Diamond B$, by clause (h) of
          definition \ref{def:kripkesemantics}.

          For the converse direction, suppose $\Diamond B \not\in w$. Then
          $\Box \neg B \in w$, by \pr{Dual}, \pr{CPL}, and the fact that $w$ is
          maximal K-consistent. By definition \ref{def:cmk}, it follows that
          $\neg B\in v$ for all $v$ with $wRv$. Since all such $v$ are maximal
          K-consistent, none of them contain $B$. By induction hypothesis, $B$
          is not true at any of them. By clause (h) of definition
          \ref{def:kripkesemantics}, it follows that
          $M_{K}, w \not\models \Diamond B$. \qed
  \end{itemize}
\end{proof}

The completeness of the axiomatic calculus for K follows immediately from the
previous two lemmas, as foreshadowed above:

\begin{theorem}{Theorem: Completeness of the axiomatic calculus for K}{completenessK}
  If $A$ is K-valid, then $A$ is provable in the axiomatic calculus for K.
\end{theorem}
%
\begin{proof}
  \emph{Proof}: We show that if a sentence is not K-provable then it is not
  K-valid. Suppose $A$ is not K-provable. Then $\{ \neg A \}$ is K-consistent.
  It follows by Lindenbaum's Lemma that $\{ \neg A \}$ is included in some
  maximal K-consistent set $S$. By definition \ref{def:cmk}, that set is a world
  in $M_K$. Since $\neg A$ is in $S$, it follows from the Canonical Model Lemma
  that $M_K,S \models \neg A$. So $M_K,S \not\models A$. So $A$ is not true at
  all worlds in all Kripke models. \qed
\end{proof}

Done!

% We have shown that the axiomatic proof system for K is rightly
% called a proof system \emph{for K} because the sentences that are
% provable in this system are precisely the sentences that are K-valid.

Once again, the proof is easily adjusted to many axiomatic calculi for logics
stronger than K. All we have assumed about the K-calculus is that it contains
\pr{Dual}, \pr{K}, \pr{Nec}, and \pr{CPL}. So if we're interested in, say,
whether the axiomatic calculus for T is complete, we can simply replace
`K-consistent' by `T-consistent' throughout the proof, and almost everything
goes through as before. We only have to add a small step at the end.

% Did we also need the assumption that the calculus is consistent? Perhaps not.
% If the calculus is inconsistent and contains CPL then everything we said
% should hold vacuously.

By adapting the argument for K, we can show that if a sentence $A$ is not
T-provable then $A$ is false at some world in the canonical model for T. This
shows that $A$ is not K-valid. But we want to show that $A$ is not T-valid --
meaning that $A$ is not true at all worlds in all reflexive Kripke models. To
complete the proof, we need to show that the canonical model $M_{T}$ for T is
reflexive.

This isn't hard. Given how accessibility in canonical models is defined, a world
$w$ in a canonical model is accessible from itself iff whenever $\Box A \in w$
then $A \in w$. Since the worlds in $M_T$ are maximal T-consistent sets of
sentences, and every such set contains every instance of the \pr{T} schema
$\Box A \to A$, there is no world in $M_T$ that contains $\Box A$ but not $A$.
So every world in $M_T$ has access to itself.

In general, to show that a calculus that extends the K-calculus by further axiom
schemas is complete, we only need to show that the canonical model for the
calculus satisfies the frame conditions that correspond to the added axiom
schemas. This is usually the case. But not always. Sometimes, an axiomatic
calculus is sound and complete with respect to some class of Kripke models, but
the canonical model of the calculus is not a member of that class. (An example is
the calculus for the system GL, which I will describe at the very end of this
chapter.) Completeness must then be established by some other means.

% The present technique for proving completeness does not always work. In the next
% section, we will meet an axiomatic calculus \Ax{GL} that is
% characterised by a certain class of frames, but its canonical model doesn't
% belong to that class.  Worse, the axioms of \Ax{GL} are not valid on the frame
% of its canonical model. So the above technique can't be used to prove
% completeness. Other (normal) calculi are not even characterised by any class of
% frames. An example is the calculus \Ax{KH}, which results from \Ax{K} by adding
% the axiom schema
% %
% \principle{H}{\Box(\Box A \leftrightarrow A) \to \Box A}

\begin{exercise}
  Outline the completeness proof for the axiomatic calculus for S5.
\end{exercise}
\begin{solution}
  We have to show that all S5-valid sentences are provable in the axiomatic
  calculus for S5, which extends the calculus for T by the axiom schemas
  $\Box A \to \Box\Box A$ and $\Diamond A \to \Box\Diamond A$. (The second
  schema alone would be sufficient, as I mentioned in chapter 1, but it doesn't
  hurt to have the first.) The argument is by contraposition: We suppose that
  some sentence is not S5-provable and show that it is not S5-valid.

  Suppose $A$ is not S5-provable. Then $\{ \neg A \}$ is S5-consistent. It
  follows by Lindenbaum's Lemma that $\{ \neg A \}$ is included in some maximal
  S5-consistent set $\Gamma$. By definition of canonical models, this set is a
  world in the canonical model $M_{S5}$ for S5. Since $\neg A$ is in $\Gamma$,
  it follows from the Canonical Model Lemma that $M_{S5},\Gamma \models \neg A$.
  So $M_{S5},S \not\models A$.

  It remains to show that the accessibility relation in $M_{S5}$ has the right
  formal properties. We know that a sentence is S5-validity iff it is valid in
  the class of Kripke models whose accessibility relation is an equivalence
  relation. So we will show that the accessibility relation in $M_{S5}$ is reflexive, transitive, and symmetric.

  By definition, a world $v$ in a canonical model is accessible from $w$ iff
  whenever $\Box A \in w$ then $A \in v$. Since the worlds in $M_{S5}$ are
  maximal S5-consistent sets of sentences, and every such set contains every
  instance of the \pr{T}-schema $\Box A \to A$, there is no world in $M_{S5}$
  that contains $\Box A$ but not $A$. So every world in $M_{S5}$ has access to
  itself.

  For transitivity, suppose for some worlds $w,v,u$ in $M_{S5}$ we have $wRv$
  and $vRu$. We need to show that $wRu$. Given how $R$ is defined in $M_{S5}$,
  we have to show that $u$ contains all sentences $A$ for which $w$ contains
  $\Box A$. So let $A$ be an arbitrary sentence for which $w$ contains $\Box
  A$. Since every world in $M_{S5}$ contains every instance of
  $\Box A \to \Box\Box A$, we know that $w$ also contains $\Box\Box A$. From
  $wRv$, we can infer that $v$ contains $\Box A$. And from $vRu$, we can infer
  that $u$ contains $A$. 

  For symmetry, suppose for some worlds $w,v$ in $M_{S5}$ we have $wRv$ and not
  $vRw$. Given how $R$ is defined, this means that there is some sentence $A$
  for which $\Box A$ is in $v$ but $\neg A$ is in $w$. Since $w$ contains the
  T-provable sentence $\neg A \to \Diamond \neg A$ and the \pr{5}-instance
  $\Diamond \neg A \to \Box \Diamond \neg A$, it also contains
  $\Box \Diamond \neg A$. So $v$ contains $\Diamond \neg A$. This contradicts
  the assumption that $v$ is S5-consistent, given that $v$ contains $\Box A$. 
\end{solution}

% in ch.2 I say we will prove here that S5 is the logic of "basic" models. Do we?

% We can show that S5 is complete wrt the simple semantics as per Humberstone
% 72: if A is not in S5 then by completeness for equivalence frames there is a
% world in an equivalence model where A is false. Take the submodel generated by
% that world...

% Exercise (H65): show that if all instances of T are provable in a consistent
% normal logic, then the canonical acc rel is reflexive. (Similarly for
% transitive with 4 and serial with D)

\begin{exercise}
  The set of all $\L_{M}$-sentences is a system of modal logic. Let's call this
  system $X$ (for ``explosion''). (a) Describe a sound and complete proof method
  for $X$. (b) Explain why $X$ does not have a canonical model.
\end{exercise}
\begin{solution}
  (a) Method A from exercise \ref{ex:proofmethods} is sound and complete for
  $X$. (b) No set of $\L_{M}$-sentences is $X$-consistent, but every Kripke model
  must have at least one world.
\end{solution}

% \begin{exercise}
%   Suppose we add the (``McKinsey'') schema $\Box\Diamond A \to \Diamond \Box A$
%   to the axiomatic calculus for S5. Explain why we can then prove all instances
%   of the (``Triv'') schema $\Box A \leftrightarrow A$.
%    % 1. A -> <>A (T, PL)
%    % 2. <>A -> []<>A (p5)
%    % 3. []<>A -> <>[]A (M)
%    % 4. A -> <>[]A (1,2,3)
%    % 5. <>[]A -> []A (p5, PL)
%    % 6. A -> []A (1,5)
% \end{exercise}
% \begin{solution}
%   \begin{alignat*}{2}
%     1.\quad& A \to \Diamond A &\quad& \text{(\pr{T}, Prop.\ Logic)}\\
%     2.\quad& \Diamond A \to \Box \Diamond A &\quad& \text{(\pr{5})}\\
%     3.\quad& \Box\Diamond A \to \Diamond\Box A &\quad& \text{(\pr{M})}\\
%     4.\quad& \Diamond\Box A \to \Box A &\quad& \text{(\pr{5}, Prop.\ Logic)}\\
%     5.\quad& A \to \Box A &\quad& \text{(1, 2, 3, 4, Prop.\ Logic)}
%   \end{alignat*}
% \end{solution}


\section{Loose ends}\label{sec:looseends}

You will remember from observation \ref{obs:semantic-deduction-theorem} in
chapter 1 that claims about entailment can be converted into claims about
validity. $A$ entails $B$ iff $A \to B$ is valid; $A_{1}$ and $A_{2}$ together
entail $B$ iff $A_{1} \to (A_{2} \to B)$ -- equivalently,
$(A_{1}\land A_{2}) \to B$ -- is valid; and so on. But what if there are
infinitely many premises $A_{1},A_{2},A_{3},\ldots$? Sentences of $\L_{M}$ are
always finite, so we can't convert the claim that $A_{1},A_{2},A_{3},\ldots$
entail $B$ into a claim that some $\L_{M}$-sentence is valid.

We also can't use the tree method or the axiomatic method to directly show that
a conclusion follows from infinitely many premises. A proof in either method is
a finite object that can only invoke finitely many sentences.

As it turns out, this is not a serious limitation. In many logics -- including
classical propositional and predicate logic and all the modal logics we have so
far encountered -- a sentence is entailed by infinitely many premises only if it
is entailed by a finite subset of these premises. Logics with this property are
called \textbf{compact}.

Let's show that K is compact. To this end, I'll say that a sentence $B$ is
\emph{K-derivable} from a (possibly infinite) set of sentences $\Gamma$ if there
are finitely many members $A_{1},\ldots,A_{n}$ of $\Gamma$ for which
$(A_1 \land \ldots \land A_n) \to B$ is provable in the axiomatic calculus for
K. Now we first show that whenever $\Gamma \models_{K} B$ then $B$ is
K-derivable from $\Gamma$. This is called \emph{strong completeness} because it
is stronger than the (``weak'') kind of completeness that we have established in
the previous section.

\begin{theorem}{Theorem: Strong completeness of the axiomatic calculus for K}{strongcompletenessK}
  Whenever $\Gamma \models_{K} B$ then $B$ is K-derivable from $\Gamma$.
\end{theorem}
\begin{proof}
  \emph{Proof}: Suppose $B$ is not K-derivable from $\Gamma$. Then there are no
  $A_1, \ldots, A_n$ in $\Gamma$ such that $(A_1 \land \ldots \land A_n) \to B$
  is K-provable. This means that $\Gamma \cup \{ \neg B \}$ is K-consistent. By
  Lindenbaum's Lemma, it follows that $\Gamma \cup \{ \neg B \}$ is included in
  some maximal K-consistent set and thereby in some world in the canonical model
  $M_{K}$ for K. (Lindenbaum's lemma says that every K-consistent set of
  $\L_{M}$-sentences, even if it is infinite, is included in a maximal
  K-consistent set.) By the Canonical Model Lemma, $M_{K}, w \models_K A$ for
  all $A$ in $\Gamma$, and $M_{K}, w \not\models_K B$. Thus
  $\Gamma \not\models_K B$. \qed
\end{proof}

\begin{theorem}{Theorem: Compactness of K}{compactnessK}
  If a sentence $B$ is K-entailed by some sentences $\Gamma$, then $B$ is
  K-entailed by a finite subset of $\Gamma$.
\end{theorem}
\begin{proof}
  \emph{Proof:} Suppose $\Gamma \models_{K} B$. By strong completeness, it
  follows that there are finitely many sentences $A_{1},\ldots,A_{n}$ in
  $\Gamma$ for which $(A_{1}\land\ldots\land A_{n}) \to B$ is K-provable. By the
  soundness of the K-calculus, $(A_{1}\land\ldots\land A_{n}) \to B$ is valid.
  So $A_{1},\ldots,A_{n} \models_{K} B$, by observation
  \ref{obs:semantic-deduction-theorem}. \qed
\end{proof}

Compactness is surprising. It is easy to think of cases in which a conclusion is
entailed by infinitely many premises, but not by any finite subset of these
premises. For example, suppose I like the number 0, I like the number 1, I like
the number 2, and so on, for all natural numbers 0,1,2,3,\ldots. Together, these
assumptions entail that I like every natural number. But no finite subset of the
assumptions has this consequence.

\begin{exercise}
  A set of sentences $\Gamma$ is called \emph{K-satisfiable} if there is a world
  in some Kripke model at which all members of $\Gamma$ are true. Show that an
  infinite set of sentences $\Gamma$ is K-satisfiable iff every finite subset of
  $\Gamma$ is K-satisfiable.
\end{exercise}
\begin{solution}
  Let $\Gamma$ be an infinite set of $\L_{M}$-sentences. If $\Gamma$ is
  K-satisfiable then obviously every finite subset of $\Gamma$ is satisfiable as
  well. For the converse direction, assume $\Gamma$ is not K-satisfiable: There
  is no world in any Kripke model at which all members of $\Gamma$ are true.
  Then there is no world in any Kripke model at which all members of $\Gamma$
  are true while $p \land \neg p$ is false. So $\Gamma \models p\land \neg p$.
  By the compactness theorem, it follows that there is a finite subset
  $\Gamma^{-}$ for which $\Gamma^{-} \models p \land \neg p$. If
  $\Gamma^{-} \models p \land \neg p$ then there is no world in any Kripke model
  at which all members of $\Gamma^{-}$ are true while $p \land \neg p$ is false.
  Since $p\land \neg p$ is false at every world in every Kripke model, it
  follows that there is no world in any Kripke model at which all members of
  $\Gamma^{-}$ are true. This shows that if $\Gamma$ is not K-satisfiable then
  there is a finite subset ($\Gamma^{-}$) of $\Gamma$ that is not K-satisfiable.
  Conversely, if every finite subset of $\Gamma$ is K-satisfiable then $\Gamma$
  is K-satisfiable.
\end{solution}

To conclude this chapter, I want to take a quick look at the logic of
mathematical provability.

Our proofs of soundness, completeness, compactness, etc.\ were informal. We have
not translated the relevant claims into a formal language, nor have we used a
formal method of proof. In principle, however, this can be done. All our
proofs could be formalized in an axiomatic calculus for predicate logic with a
few additional axioms about sets. A well-known calculus of that kind is ZFC
(named after Ernst Zermelo, Abraham Fraenkel, and the Axiom of Choice). ZFC is
strong enough to prove not just soundness and completeness in modal logic, but
practically everything that can be proved in any branch of maths.

An interesting feature of ZFC is that it can not only prove facts about what's
provable in simpler axiomatic calculi; it can also prove facts about what's
provable in ZFC itself. For example, one can prove in ZFC that one can prove in
ZFC that 2+2=4.

This gives us  an interesting application of modal logic. Let's read the box as
`it is mathematically provable that', which we understand as provability in ZFC.
One can easily show (in ZFC) that this operator has all the properties of the
box in the basic logic K. For example, all instances of the \pr{K}-schema are
provable in ZFC. (The language of ZFC doesn't have a box symbol. But one can
encode the \pr{K}-schema into a schema of ZFC, given the present reading of the
box, and all instances of that schema are ZFC-provable.)

So the logic of mathematical provability is at least as strong as K. In fact, it
is stronger. One can prove in ZFC that whenever a sentence is ZFC-provable then
it is ZFC-provable that the sentence is ZFC-provable. This gives us the
\pr{4}-schema $\Box A \to \Box\Box A$.

You might expect that we also have the \pr{T}-schema $\Box A \to A$ or the
\pr{D}-schema $\Box A \to \Diamond A$. The latter says that if something is
provable then its negation isn't provable (since $\Diamond A$ means
$\neg\Box\neg A$). And surely ZFC can't prove both a sentence and its negation
-- which would make ZFC inconsistent. I say `surely', but can we
prove (in ZFC) that ZFC is consistent? The answer is no. More precisely, one can
prove that if one can prove that ZFC is consistent then ZFC is
\emph{in}consistent. This bizarre fact is a consequence of \emph{G\"odel's
  second incompleteness theorem}, established by Kurt G\"odel in 1931. It is
reflected by the following schema (named after G\"odel and Martin L\"ob), all
whose instances are provable in ZFC:
%
\principle{GL}{\Box(\Box A \to A) \to \Box A}
%
The system GL, which is axiomatized by \pr{K}, \pr{GL}, \pr{Nec}, and \pr{CPL},
completely captures what ZFC can prove about provability in ZFC. (Schema \pr{4}
isn't needed as a separate axiom schema because it can be derived.)%
% To show []p->[][]p you need the GL instance with A = □p∧p

\begin{exercise}
  Suppose ZFC can prove its own consistency, so that there is a proof of
  $\neg \Box (p \land \neg p)$. Explain how this proof could be extended to a
  proof of $\Box (p \land \neg p)$. You need
  each of \pr{GL}, \pr{Nec}, and \pr{CPL}.
\end{exercise}
\begin{solution}
  Suppose there is a proof of $\neg\Box (p \land \neg p)$. By \pr{CPL}, we can
  infer $\Box (p \land \neg p) \to (p \land \neg p)$, because $A \to B$ is a
  truth-functional consequence of $\neg A$. By \pr{Nec}, we get
  $\Box (\Box (p \land \neg p) \to (p \land \neg p))$. By \pr{GL} and
  \emph{modus ponens} (an instance of \pr{CPL}), we can derive
  $\Box (p \land \neg p)$.
\end{solution}

% \begin{exercise}
%   Let $p$ be the statement that 2+2=5. Properly encoded into the language of
%   ZFC, one can prove in ZFC that $p$ is false. And this fact itself is provable
%   in ZFC: One can prove in ZFC that $\Box\neg p$. Explain (informally) why it
%   follows that $\neg\Box p$ is not provable in ZFC, assuming that ZFC is
%   consistent and drawing on the fact that the logic of ZFC-provability is
%   axiomatized by \pr{K}, \pr{GL}, \pr{Nec}, and \pr{CPL}.
% \end{exercise}
% \begin{solution}
%   Suppose $\neg\Box p$ is ZFC-provable. By \pr{CPL}, any proof of $\neg\Box p$
%   could be extended to a proof of $\Box p \to p$, because $\Box p \to p$ is a
%   truth-functional consequence of $\neg p$. By \pr{Nec}, we would get
%   $\Box(\Box p \to p)$. By \pr{GL} and Modus Ponens (an instance of \pr{CPL}),
%   we would get $\Box p$. So if $\neg\Box p$ is ZFC-provable, then $\Box p$ is
%   ZFC-provable. And if both $\Box p$ and $\neg \Box p$ are ZFC-provable, then
%   ZFC is inconsistent.
% \end{solution}

% Surprisingly, Kripke models play an important role in the study of mathematical
% provability: the standard technique for proving that all instance of \pr{GL}
% are provable in ZFC draws on the fact that GL is sound and complete with respect
% to the class of finite, transitive, and irreflexive Kripke models.
% (As
% I said earlier, the completeness proof is non-trivial, because the canonical
% model of \Ax{GL} has infinitely many worlds and therefore does not itself belong
% to the relevant model class.)
%
% why irreflexive? can't we drop that? no; if we do so, we allow for
% infinite chains.
%
% This is surprising because intuitively, mathematical truths are true at all
% possible worlds, so it is hard to see how mathematical provability could be
% usefully analysed in terms of truth at accessible worlds.

% [Boolos p.xx (yes, xx in the introduction) says that this is the
% decisive scientific justification for modal logic and Kripke
% models.]

% So we should not think of logical necessity as provability. Otherwise
% the logic of logical necessity certainly isn't S5. Whatever logical
% possibility means, it can't be defined in terms of provability.


%%% Local Variables: 
%%% mode: latex
%%% TeX-master: "logic2.tex"
%%% End:
